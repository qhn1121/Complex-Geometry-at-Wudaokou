\chapter{$L^2$ Hodge理论}

\section{向量丛上的微分算子}
%Differential operators on vector bundles.
%Let $X$ is a (connected) smooth manifold of ($\bbR$-)dimension $n$.
%$E,F:\bbK$-vector bundle of rank $r,r'$ respectively.

现在,假设$X$为$n$维(连通)光滑流形,
$E,F\to X$为$X$上的光滑$\bbK$-向量丛,其秩分别为$r,r'$.
这里$\bbK=\bbR$或$\bbC$.

\begin{definition}(向量丛上的微分算子)

对于光滑流形$X$上的秩分别为$r,r'$的光滑$\bbK$-向量丛$E,F\to X$,
称$\bbK$-线性算子
\begin{eqnarray*}
  P: C^\infty(X,E)&\to& C^\infty(X,F)\\
  u&\mapsto& Pu
\end{eqnarray*}
为次数为$k$的\textbf{微分算子},
如果在局部平凡化坐标卡下,
截面的坐标函数满足
$$
  (Pu)^\mu
= \sum_{|\afa|\leq k}
  a^\mu_{\lmd,\afa}\tD^\afa u^\lmd
$$
其中$u$为$E$的某截面在某局部标架下的坐标函数,视为$r$维列向量;
$1\leq\lmd\leq r,\,1\leq\mu\leq s$,$\afa\in\bbN^r$为多重指标,
$\tD^\afa$为相应的偏微分算子。
\end{definition}

容易验证微分算子$P$的次数$k$与局部标架选取无关。
固定$E,F$上的局部标架,$x=(x^1,x^2,...,x^n)^T$为$X$上的
关于$E,F$的局部平凡化坐标卡,
若$X$上还有另一组局部平凡化坐标卡$\xtil=(\xtil^1,\xtil^2,...,\xtil^n)^T$,
则对于多重指标$|\afa|=k$,可以去验证$P$在$x,\xtil$
下的第$\afa$阶系数矩阵$a_\afa,\atil_\afa$的转换关系。
不过要注意,若$|\afa|<k$,则转换关系会异常复杂。

%\begin{definition}a linear differential operator of degree $k$ from $E$ to $F$
%is a $\bbK$-linear map$$P:C^\infty(M,E)\to C^\infty(M,F)$$
%$$u\mapsto Pu$$locally given by$$Pu(x)=\sum_{|\afa|\leq k}
%a_{\afa}(x)D^\afa u(x)$$where $a_\afa(x)=(a_{afa,\lmd\mu}(x))$
%be a $r'\times r$ matrix.$$u(x)=(u_1(x),...,u_r(x))^T$$\end{definition}

\begin{prop}(微分算子的主符号)

记号同上,则对于向量丛上的$k$阶微分算子
$P:C^\infty(X,E)\to C^\infty(X,F)$,
在$E$的局部平凡化坐标卡下,对于$x\in X$,
$u\in C^\infty(X,E)$视为$r$为列向量值函数,以及
$\xi=\xi_i\td x^i\in\Omg^1(X)=C^\infty(X,T^*X)$
为$X$上的$1$-形式,在坐标卡下视为$n$维行向量$(\xi_1,\xi_2,...,\xi_n)$,则
$$
  \sgm_P(x,\xi):u
  \to\sum_{|\afa|=k}
    \xi^\afa a_\afa u
$$
整体定义了一个$\bbK$-线性算子
$$\sgm_P:T^*X\to\Hom(E,F)$$
称该算子为微分算子$P$的\textbf{主符号}(principal symbol)。
其中$\xi^\afa=\xi_1^{\afa_1}\xi_2^{\afa_2}\cdots\xi_n^{\afa_n}$.
\end{prop}

\begin{proof}
需要验证$\sgm_P(x,\xi)$是在$X$上的整体定义的。

{\color{red}(待补)}
\end{proof}

\begin{rem}
要注意,对于余切向量$\xi\in T^*_xX$,
$\xi\to\sgm_{x,\xi}$一般\textbf{不是}关于$\xi$线性的,
而是$\xi$的$k$次函数,其中$k$是微分算子$P$的阶数。
若$k=1$,则$\sgm_P(x,-)$关于$\xi$线性。
\end{rem}

\begin{rem}设$f$为$X$上的在$x\in X$附近定义的光滑函数,
则对$E$的在$x$附近的截面$u$,容易验证
$$t\mapsto e^{-tf}P(e^{tf}u)$$
是关于$t$的$k$次多项式,并且首项$t^k$的系数正是
$\sgm(x,\td f)(u)$.
\end{rem}

%Let $t\in\bbK$, $f\in C^\infty(M,\bbK)$, $u\in C^\infty(M,E)$,
%then$$e^{-tf(x)}P(e^{tf(x)}u(x))=t^k\sgm_P(x,\td f(x))u(x)+
%\text{terms } c_j(x)^{t_j}\quad(j<k)$$\begin{definition}$$
%\sgm_P:T^*M\to\Hom(E,F)$$is called the principal symbol of $P$,
%which is a polynomial on $T^*M$.\end{definition}locally,
%$$\sgm_P(x,\xi)=\sum_{|\afa|=k}a_\afa(x)\xi^\afa$$
%$(\xi^\afa:=\xi_1^{\afa_1}...\xi_n^{\afa_n})$

\begin{example}(外微分算子$\td$)

对于光滑流形$X$,外微分算子$\td$自然视为$p$-形式丛
$\wedgeform{p}T^*X$到$(p+1)$-形式丛
$\wedgeform{p+1}T^*X$的一阶微分算子.
\end{example}
注意到对于$X$上的光滑函数$f$,以及$\omg\in\Omg^p(X)$,
$$e^{-tf}\td(e^{tf}\omg)=t\td f\wedge\omg+\td \omg$$
从而知$\td$的主符号$\sgm_{\td}(x,\xi)(\omg)=\xi\wedge\omg$.

%\begin{example}Consider $\td: C^\infty(M,\bbK)\to C^{\infty}(M,T^*M)$.
%then$$\td u=\sum_{j=1}^n\begin{pmatrix}0\\\vdots\\
%1(j^{th})\\\vdots\\0\end{pmatrix}\pfrac{u}{x^i}$$i.e.$$\sgm_d(x,\xi)
%=\sum_{j=1}^n\begin{pmatrix}0\\\vdots\\1(j^{th})\\\vdots\\0
%\end{pmatrix}\xi_j$$\end{example}

\begin{definition}(椭圆算子)
\index{elliptic operator\kong 椭圆算子}

设$E,F$为$X$的光滑向量丛,微分算子$P:C^\infty(X,E)\to\infty(X,F)$
称为椭圆算子,若$P$的主符号$\sgm_P$满足:对任意$x\in X$以及
$\xi\in T^*_xX\setminus\{0\}$,
$\sgm_P(x,\xi):E_x\to F_x$是单射。
\end{definition}

例如,外微分算子$\td:\Omg^0(X)\to\Omg^1(X)$是椭圆的。

%\begin{definition}
%$P$ is called elliptic, if $\forall x\in M,\xi\in T^*_xM\setminus\{0\}$,
%$$\sgm_P(x,\xi)\in \Hom(E_x,E_x)$$is injective.
%\end{definition}For example, $\td$ is elliptic.

接下来引入一些分析上的概念,首先是向量丛的$L^2$截面:

\begin{definition}
设$X$为定向的光滑流形,$\td\Vol$为$X$的一个体积形式,
$E\to X$为光滑丛,并配以欧氏结构或者Hermite结构$\pair{}{}$。
则对任意$u,v\in C^\infty(X,E)$,定义
$$
  \ppair{u}{v}
:=
  \int_M\pair{u}{v}\td\Vol
$$
定义$L^2(X,E):=\Bigset{u\in C^\infty(X,E)}{\ppair{u}{u}<+\infty}$
的(在$\ppair{}{}$诱导的范数下的)完备化。
\end{definition}

%\textbf{$L^2$-inner product}Let $M$ be an oriented $C^\infty$
%-manifold with a smooth volume form, locally
%$$\td V(x)=\gamma(x)\td x_1\wedge\cdots\wedge\td x_n$$
%$\gma(x)>0$. Assume $E$ has a Euclidean(or Hermitian) structure...
%Let $u,v\in C^{\infty}(M,E)$,define$$\langle\langle u,v \rangle\rangle:=
%\int_M\langle u , v \rangle\td V(x)$$define $L^2(M,E):=$
%space of sections with measurable coefficients with are $L^2$ w.r.t
%$\langle\langle,\rangle\rangle$.

$L^2(X,E)$配以内积$\ppair{}{}$,构成Hilbert空间。
对于$u\in L^2(X,E)$,局部标架下,$u$可视为向量值的$L^2$-可积的可测函数。
由分析学可知,$C^\infty(X,E)\cap L^2(X,E)$在$L^2(X,E)$中稠密。
以后我们在讨论$L^2(X,E)$时,不妨假定$E$是Hermite向量丛。

\begin{prop}(微分算子的形式伴随)
\index{formal adjoint\kong 形式伴随}

设$E,F\to X$为光滑定向流形$X$上的Hermite向量丛,
给定$X$的体积形式$\td\Vol$,则对任意的微分算子
$P:C^\infty(X,E)\to C^\infty(X,F)$,存在唯一的微分算子
$P^*:C^\infty(X,F)\to C^\infty(X,E)$,使得
对任意的$u\in C^\infty(X,E)$以及$v\in C^\infty(X,F)$,
若$\supp u\cap\supp v\ssubset X$,都有
$$
  \ppair{Pu}{v}
= \ppair{u}{P^*v}
$$
称算子$P^*$为$P$的\textbf{形式伴随}(formal adjoint)。
\end{prop}

\begin{proof}
取定Hermite丛$E,F$的幺正标架以及$X$的局部坐标卡。
记$\Omg:=\supp u\cap\supp v$,不妨$\Omg$
包含于该局部坐标卡。则在该幺正标架、局部坐标下,
记微分算子$P$与体积形式$\td\Vol$分别为
$$
  P=\sum_{|\afa|\leq k}a_\afa\tD^\afa,
\qquad
  \td\Vol=\gma(x)\td x^1\wedge\td x
$$
其中$\td x:=\td x^1\wedge\td x^2\wedge\cdots\wedge\td x^n$.因此有
\begin{eqnarray*}
     \ppair{Pu}{v}
&=&
     \sum_{|\afa|\leq k}
     \int_\Omg
       \pair{a_\afa\tD^\afa u}{v}
       \gma\td x
 =
     \sum_{|\afa|\leq k}
     \int_{\Omg}
       \tD^\afa u^\dag a_\afa^\dag v\gma\td x\\
&=&
     \sum_{|\afa|\leq k}
     \int_{\Omg}
       (-1)^{|\afa|}
       u^\dag\tD^\afa(\gma a_\afa^\dag v)\td x
 =
     \int_\Omg
       u^\dag
       \sum_{|\afa|\leq k}
       (-1)^{|\afa|}
       \gma^{-1}\tD^\afa(\gma a_\afa^\dag v)\cdot\gma\td x\\
&=&
     \ppair{u}
           {\gma^{-1}\sum_{|\afa|\leq k}
             (-1)^{|\afa|}
             \tD^\afa
             (\gma a_\afa^\dag v)
           }
\end{eqnarray*}
因此,只需取
$$
  P^*(v)
:=
  \gma^{-1}\sum_{|\afa|\leq k}
             (-1)^{|\afa|}
             \tD^\afa
             (\gma a_\afa^\dag v)
$$
\end{proof}

%\begin{definition}Let $P:C^\infty(M,E)\to C^\infty(M,F)$
%be a differential operator,$E,F$ have Euclidean (or Hermitian) structure,
%then there exists uniquedifferential operator
%$$P^*:C^{\infty}(M,F)\to C^\infty(M,E)$$s.t.
%$$\langle\langle Pu,v\rangle\rangle=\langle\langle u,P^*v\rangle\rangle$$
%for all $u,v$ s.t. $Supp u\cap Supp v\subset\subset M$(relative compact...)
%$P^*$ is called the formal adjoint of $P$.\end{definition}\begin{proof}
%Existence: Assume that $Supp U, Supp v\subset\subset$
%some coordinate chart $\Omg$ with coordinates $(x_1,...,x_n)$, then
%$$\ll Pv,u\gg=\int_\Omg\sum_{\afa,\lmd,\mu}a_{\afa,\lmd\mu}(x)
%D^\afa u_\mu(x)\overline{v_\lmd(x)}\gma(x)\td x_1\cdots\td x_n
%$$integration by parts, it$$=\int_\Omg\sum_{\afa,\lmd,\mu}
%(-1)^{|\afa|}u_\mu(x)\overline{D^\afa(\gma(x)\overline{a_{\afa,\lmd\mu}}v_\lmd(x))}
%\td x_1\...\td x_n$$Locally,$$P^*v=\sum_{|\afa|\leq k}
%(-1)^{|\afa|}\gma(x)^{-1}D^\afa(\gma(x)\overline{a_\afa(x)}^Tv(x))$$
%Uniqueness: use the density of $C^{\infty}$-section with compact
%support in $L^2(M,-)$.\end{proof}

\begin{cor}(形式伴随的主符号)

设$E,F\to X$为配以体积形式$\td\Vol$的光滑定向流形$X$上的Hermite向量丛,
$P:C^\infty(X,E)\to C^\infty(X,F)$为$k$阶微分算子,
则形式伴随$P^*$的主符号$\sgm_{P^*}$满足
$$
  \sgm_{P^*}(x,\xi)=(-1)^k[\sgm_P(x,\xi)]^*
$$
\end{cor}

\begin{proof}
  由$P^*$的局部表达式显然得到。
\end{proof}

\begin{cor}
条件记号同上,若$\rank E=\rank F$,则$P$为椭圆算子当且仅当$P^*$为椭圆算子。
\end{cor}

\begin{proof}
显然。
\end{proof}

%\begin{cor}If $\sgm_P(x,\xi)=\sum_{|\afa|=k}
%a_\afa(x)\xi^{\afa}$ ,then$$\sgm_{P^*}
%=(-1)^k\overline{\sgm_P(x,\xi)}^T$$\end{cor}
%\begin{cor}If rank $E=$ rank$F$, $P$ is differential operator, then
%$P^*$ is elliptic $\iff$ $P^*$ is elliptic.\end{cor}
%%%%%%%%%%%%%%%%%%2019.4.11第七周周四%%%%%%%%%%%%%%%%%%%%%%%%%%%%%%
%\textbf{Fundamental results of elliptic operators}

\section{椭圆算子的基本性质}

本节不加证明地陈述分析学当中的一些重要结果。

设$X$为紧致、定向光滑流形,配以体积形式$\td\Vol$,$\dim_{\bbR}X=n$;
$E\to X$为Hermite向量丛,$\rank_{\bbC}E=r$.
取$X$的有限的关于$E$的局部平凡化坐标覆盖$\Bigset{\Omg_j}{1\leq j\leq n}$,
以及$E$的在$\Omg_j$上的局部的幺正标架$e_j=(e_{j,1},e_{j,2},...,e_{j,r})$.
对于$u\in L^2(X,E)$,记$u$在$\Omg_j$的局部表达式为$u=u_j^\lmd e_{j,\lmd}$,
视$u_j$为$r$为列向量。从而对于$k\geq 0$,对于$u\in C^k(X,E)$,记
$$
  \norm{u}_k^2
:=
  \sum_{j=1}^{N}
    \int_{\Omg_j}
      \sum_{|\afa|\leq k}
        |\tD^\afa u_j|^2\td\Vol
$$

从而定义\textbf{索伯列夫空间}(Sobolev space)
$W^k(X,E):=\Bigset{u\in C^k(X,E)}{\norm{u}_k<+\infty}$
在范数$\norm{\cdot}_k$下的完备化。易知$W^k(X,E)$为Hilbert空间。
事实上对于紧流形$X$,取$X$的不同的开覆盖、坐标卡、局部幺正标架,
所得到的范数$\norm{\cdot}_k$是等价范数。

%$M$ is a compact (oriented) $C^\infty$-manifold, $\dim_{\bbR}M=n$,
%with a smooth volume form $\td V$.$E$ is an Hermite vector bundle,
%$rank_{\bbC}E=r$.Sobolev space:$W^k(M,E):=$ the space of section $s:M\to E$
%whose derivations up to order $=k$, $:=$ the completion of space
%of smooth sections w.r.t $W^k$-norm.$(\Omg_j)_{j\in I}$: a finite open
%covering of $M$, $E|_{\Omg_j}$ trivial,Let $(\rho_j)_{j\in I}$ be a
%partition of unity w.r.t. $(\Omg_j)_{j\in I}$,s.t. $\sum_{j}\rho^2_j=1$.
%locally, choose an orthonormal frame $(e_{j,\lmd})_{1\leq\lmd\leq r}$ on $\Omg_j$,
%then $u=\sum_{\lmd=1}^ru_{j,\lmd}e_{j,\lmd}$ on $\Omg_j$. Define
%$$||u||^2_k:=\sum_{j,\lmd}||e_ju_{j,\lmd}||_k^2$$where$$
%||e_ju_{j,\lmd}||_k^2:=\int_{\Omg_j}\sum_{|\afa|\leq k}
%|D^\afa(e_ju_{j,\lmd})|^2\td V(x)$$\begin{rem}On a compact manifold,
%the equivalence of class of $||\cdot||_k$ is independent of the choice of
%: partition of unity,local trivialization, holomorphic covering...\end{rem}
%(Sobolev lemma)For $k>l+\frac{n}{2}$, then we have$$W^k(M,E)\subseteq C^l(M,E)$$

接下来的一系列结果,述而不证:

\begin{lemma}(Sobolev引理)
若$k>l+\frac{n}{2}$,则有$W^k(X,E)\subseteq C^l(X,E)$.
\end{lemma}

\begin{lemma}(Rellich紧嵌入定理)

对任意$k\in\bbZ_{\geq 0}$,则$W^{k+1}(M,E)\inj W^k(M,E)$是紧算子。
\end{lemma}

%For any $k\in\bbZ_{\geq 0}$, the inclusion$$W^{k+1}(M,E)
%\inj W^k(M,E)$$is a compact operator.

\begin{lemma}(G{\aa}rding不等式)

设$E,F$为紧定向流形$X$上的Hermite向量丛,$\rank_{\bbC}E=\rank_{\bbC}F$,
$P:C^\infty(E)\to C^\infty(F)$为$d$次的椭圆算子,并
将$P$的定义域自然延拓到$E$的分布系数的截面上。
则对于任意$u\in W^0(X,E)$,如果$Pu\in W^k(X,F)$,
那么$u\in W^{k+d}(X,E)$,并且成立不等式
$$
  \norm{u}_{k+d}
\leq
  C_k\left(
    \norm{Pu}_k
   +\norm{u}_0
  \right)
$$
其中常数$C_k$只与$k$有关。
\end{lemma}

%If$$P:C^\infty(M,E)\to C^\infty(M,F)$$is elliptic, and rank$E=$rank $F$,
%$\tilde{P}:$the extension of $P$ to sections with distribution coefficients, then :
%for all $u\in W^0(M,E)$, if $\tilde{P}u\in W^k(M,F)$, then
%$u\in W^{k+d}(M,E)$, where $d=\deg P$, and$$||u||_{k+d}\leq C_k
%\left(||\tilde{P}u||_k+||u||_0\right)$$where $C_k$ depending on $k,M$.
%\begin{proof}Reference: \verb"Kodaira:
%deformation of complex structures (Appendix)"\end{proof}
%\begin{cor}If $u\in\ker\tilde{P}\cap W^0(M,E)$,
%then $u\in C^\infty(M,E)$.\end{cor}

特别地,我们有椭圆算子正则性:对于椭圆算子$P$,
若$u\in W^0(X,E)$满足$Pu=0$,
则必有$u\in C^\infty(X,E)$.

我们还有关于椭圆算子的如下的一般结果:

\begin{lemma}(有限性定理)
\label{椭圆算子-有限性定理-lemma}

设$X$为紧定向流形,$E,F\to X$为Hermite向量丛,并且
$\rank_{\bbC}E=\rank_{\bbC}F$,
$P:C^\infty(M,E)\to C^\infty(M,F)$为椭圆算子,则有:

(1)$\ker P$是有限维空间;

(2)$P(C^\infty(X,E))$是$C^\infty(X,F)$的闭子空间,
并且余维数有限;

(3)若$P^*$为$P$的形式伴随,则有($L^2(X,F)$上的)正交直和分解
$$
  C^\infty(X,F)
=
  P(C^\infty(X,E))\oplus\ker P^*
$$
\end{lemma}

%Setting $M$ be a compact manifold, rank$E=$rank$F$,
%$$P:C^\infty(M,E)\to C^\infty(M,F)$$elliptic,then:
%(1) $\ker P$ is of finite dimension(2) $P(C^\infty(M,E))$ is
%closed and of finite codimension in $C^\infty(M,F)$.
%If $P^*$ is the formal adjoint of $P$, then $\exists$ decomposition
%$$C^\infty(M,F)=P(C^\infty(M,E))\oplus\ker P^*$$
%which is orthogonal in $W^0(M,F)=L^2(M,F)$\end{lemma}
%\begin{proof}椭圆算子的一般结果,分析的东西233333333.
%可以参考小平邦彦复流形与复结构形变的附录。\end{proof}
%\textbf{Hodge theory in compact Riemannian manifold}
%Hodge star operator.

\section{Hodge $\star$ 算子与Laplace算子}

我们先来介绍黎曼流形上的Hodge理论。
在本节,假设$X$为$n$维紧定向黎曼流形,$E\to X$为Herimite向量丛,
$\rank_{\bbC}E=r$.局部上,取切丛$TX$的(关于黎曼度量的)幺正标架
$\xi=(\xi_1,\xi_2,...,\xi_n)$以及$E$的(关于Hermite结构的)幺正标架
$e=(e_1,e_2,...,e_r)$.记$\xi^*:=(\xi_1^*,\xi_2^*,...,\xi_n^*)$
以及$e^*:=(e_1^*,e_2^*,...,e_r^*)$为相应的对偶标架。

%$M$ compact Riemannian manifold,
%$\dim_{\bbR}=n$, $E$ is a Hermitian vector bundle.
%Assume $(\xi_1,...,\xi_n),(e_1,...,e_n)$ be orthonormal frame of $TM, E$
%on some local chart $\Omg$, denote $(\xi_1^*,...,\xi_n^*),
%(e_1^*,...,e_n^*)$ be the co-frame of$T^*M,T^*E$.

众所周知,对于$x\in X$,
$X$上的黎曼度量诱导了微分形式丛$\wedgeform{\bullet}T^*_xM$上的内积结构:
对于任意$u_1,...,u_p;v_1,...,v_p\in T^*_xX$,有
$$
  \pair{u_1\wedge\cdots\wedge u_p}
       {v_1\wedge\cdots\wedge v_p}
:=
  \det(\langle u_i,v_j\rangle)
$$

%$\wedge\updot T^*M$ is endowed with an inner product frame from $TM$.
%locally,$$\langle u_1\wedge\cdots\wedge u_p,u_1\wedge\cdots\wedge u_p\rangle
%:=\det(\langle u_i,v_j\rangle)$$
%for $u_i,v_j\in T^*M$. Then , get an inner product on $\wedge^pT^*M$.
%Assume$$U=\sum_{|I|=p\atop i_1\leq...\leq i_p}u_I\xi_I^*$$
%$$V=\sum_{|I|=p\atop i_1\leq...\leq i_p}
%v_I\xi_I^*$$be $p$-forms, then
%$$\langle u,v\rangle=\sum_{|I|=p}u_Iv_I$$
%i.e. $\{\xi^*_T\}$ is an orthonormal basis of $\wedge^pT^*M$.

事实上,对于$T^*_xX$的幺正基$\xi^*$,则
$
  \Bigset{\xi_{i_1}^*\wedge\xi_{i_2}^*\wedge\cdots\wedge\xi_{i_p}^*}
  {p\geq 0,1\leq i_1<\cdots<i_p\leq n}
$为$\wedgeform{\bullet}T^*_xX$的一组幺正基。
用多重指标记号,将$\xi_{i_1}^*\wedge\xi_{i_2}^*\wedge\cdots\wedge\xi_{i_p}^*$
简记为$\xi_I^*$,其中$I:=(i_1,i_2,...,i_p)$.

$\wedgeform{\bullet}T^*M\ten E$的纤维上
也自然地有由$\wedgeform{\bullet}T^*M$
以及$E$诱导的Hermite内积结构。
此外,也自然有$\bbC$-共轭双线性型
\begin{eqnarray*}
  \bair{}{}:
  \left(
  \wedgeform{p}T^*X\ten E\right)
  \times
  \left(
  \wedgeform{q}T^*X\ten E\right)
  &\to&
  \wedgeform{p+q}T^*X
\end{eqnarray*}
它由$\wedgeform{\bullet}T^*X$中的外积,
以及$E$上的Hermite内积所诱导。

\begin{definition}(Hodge $\star$ 算子)

记号同之前,则定义\textbf{Hodge $\star$ 算子}
$$\star:\wedgeform{p}T^*X\to\wedgeform{n-p}T^*X$$
使得对任意$u,v\in \wedgeform{p}T^*X$,成立%the Hodge star operator is defined by
$$u\wedge\star v=\pair{u}{v}\td\Vol$$
\end{definition}

良定性容易验证,并且$\star$为线性算子。
容易验证,在$T^*X$的幺正标架$\xi^*=(\xi^*_1,\xi^*_2,...,\xi^*_n)$下,
对于指标$I=(i_1,i_2,...,i_p)$,其中
$1\leq i_1<1_2<\cdots<i_p$,
则有
$$\star\xi^*_I=\sgn(I,I^c)\xi^*_{I^c}$$
其中$(n-p)$重指标$I^c=(j_1,j_2,...,j_{n-p})$
为$\{1,2,...,n\}\setminus I$中元素的从小到大排列,
$\sgn(I,I^c)$为排列
$
  \begin{pmatrix}
    1 & \cdots & p & p+1 & \cdots & n \\
    i_1 & \cdots & i_p & j_1 & \cdots & j_{n-p}
  \end{pmatrix}
$的符号。

%Locally, let$$U=\sum_{|I|=p}u_I\xi_I^*,\,V=\sum_{|I|=p}v_I\xi_I^*$$
%assume$$*V=\sum_{|J|=n-p}a_J\xi_J^*$$then$$U\wedge*\sum u_I a_{I^c}\xi_I^*
%\wedge\xi_{I^c}^*=\sum u_Ia_{I^c}\veps(I,I^c)\xi_1^*\wedge\cdots\wedge\xi_n^*
%$$$$\langle u,v\rangle\td V=\sum_{|I|=p}u_Iv_I\xi_1^*\wedge\cdots\wedge\xi_n^*
%$$so, we have$$*V=\sum_{|I|=p}\veps(I,I^c)V_I\xi_{I^c}^*\in \wedgeform{n-p}T^*M$$

\begin{definition}(向量丛上的$\star$算子与$\#$算子)

设$E\to X$为黎曼流形上的Hermite向量丛,则定义
\begin{eqnarray*}
* :\wedgeform{p}T^*X\ten E&\to&\wedgeform{n-p}X^*M\ten E \\
\#:\wedgeform{p}T^*X\ten E&\to&\wedgeform{n-p}X^*M\ten E^*
\end{eqnarray*}
使得对任意$u,v\in \wedgeform{p}T^*X\ten E$,成立
\begin{eqnarray*}
\bair{u}{\star v}&=&\pair{u}{v}\td\Vol\\
u\wedge\# v      &=&\pair{u}{v}\td\Vol
\end{eqnarray*}
\end{definition}

其中双线性型$\bair{}{}$由$\wedgeform{\bullet}T^*X$
中外积$\wedge$与$E$的Hermite结构诱导,
$u\wedge\# v$当中的“$\wedge$”则由通常的外积与$E^*,E$的配对所诱导。
容易验证,在$\wedgeform{\bullet}T^*X$的幺正标架$\{\xi_I^*\}$与$E$
的幺正标架$e=(e_1,...,e_r)$下,成立
\begin{eqnarray*}
  \star (\xi_I^*\ten e_i) &=& \sgn(I,I^c)\xi_{I^c}^*\ten e_i\\
  \#    (\xi_I^*\ten e_i) &=& \sgn(I,I^c)\xi_{I^c}^*\ten e_i^*
\end{eqnarray*}
以及容易验证,在$\wedgeform{p}T^*X\ten E$上,成立
$$
  \star^2=\#^2=(-1)^{p(n-p)}\id
$$

%\begin{definition}$$*:\wedgeform{p}T^*M\ten E\to\wedgeform{n-p}T^*M\ten E$$
%is defined by$$\{s,*t\}:=\langle s,t\rangle\td V$$\end{definition}
%Locally, assume$$t=\sum_{|I|=p\atop1\leq\lmd\leq r}t_{I,\lmd}\xi_I^*\ten e_{\lmd}
%$$then$$*t=\sum_{|I|=p\atop 1\leq\lmd\leq r}\veps(I,I^c)
%t_{I,\lmd}\xi_{I^c}^*\ten e_{\lmd}$$\begin{definition}
%$$\#:\wedgeform{p}T^*M\ten E\to\wedgeform{n-p}T^*M\ten E^*$$
%is defined by: for any $s,t\in \wedgeform{p}T^*M\ten E$,such that
%$$s\wedge\# t:=\langle s,t\rangle\td V$$wedge product$+$ pairing of
%$E^*\times E\to\bbC$.\end{definition}Locally: assume$$t=\sum_
%{|I|=p\atop 1\leq\lmd r}t_{I,\lmd}\xi_T^*\ten e_{\lmd}$$then,$$\#t=\sum_{|I|=p,\lmd}
%\veps(I,I^c)t_{I,\lmd}\xi_c^*I\ten e_\lmd^*$$
%\begin{prop}$$*^2=(-1)^{p(n-1)}\quad\text{on } \wedgeform{p}T^*M\ten E$$
%$$\#^2=(-1)^{p(n-1)}\quad\text{on } \wedgeform{p}T^*M\ten E$$
%\end{prop}(正负号对吗?)

回顾$\Omg^p(X,E):=C^\infty(X,\wedgeform{p}T^*X\ten E)$上有Hermite内积结构
$$\ppair{s}{t}:=\int_X\pair{u}{v}\td\Vol$$
在此意义下,我们可谈论相关微分算子的形式伴随:

%Recall: For all $s,t\in C^\infty(M,\wedgeform{p}T^*M\ten E)$,
%we have an inner product$$\langle\langle s,t\rangle\rangle
%:=\int_M\langle s,t\rangle\td V$$\begin{thm}
%Let $D_E$ be an Hermite connection on $E$,acting on $\wedgeform{p}T^*M\ten E$, then
%$$D_E^*:=(-1)^{np+1}*D_E*$$where $D^*_E$ is the formal adjoint of $D_E$.\end{thm}
%\begin{proof}Let $s\in C^\infty(M,\wedgeform{p}T^*M\ten E)$ and
%$t\in C^\infty(M,\wedgeform{p+1}T^*M\ten E)$. then$$\langle\langle D_Es,t\rangle\rangle
%=\int_M\langleD_Es,t\rangle\td V=\int_M\{D_Es,*t\}$$Since $D_E$ is Hermitian ,by definetion ,
%$$\td\{s,*t\}=\{D_Es,t\}+(-1)^p\{s,D_E(*t)\}$$so,$$\langle\langle D_Es,t\rangle\rangle=
%\int_M\td\{s,*t\}+(-1)^{p+1}\{s,D_E*t\}=(-1)^{p+1}(-1)^{p(n_1)}\int_M\{s,*(*D_E*t)\}
%=\langle\langle s,D_E^*t\rangle\rangle$$so,$$D_E^*t=(-1)^{np+1}*D_E*$$\end{proof}

\begin{prop}(Hermite联络的形式伴随)

设$E\to X$为紧定向黎曼流形上的Hermite向量丛,
$\tD_E:\Omg^p(X,E)\to\Omg^{p+1}(X,E)$为Hermite联络,则有
$$\tD_E^*=(-1)^{np+1}\star\tD_E\star$$
\end{prop}

\begin{proof}
对任意的$s\in\Omg^p(X,E)$以及$t\in\Omg^{p+1}(X,E)$,只需注意
\begin{eqnarray*}
     \ppair{s}{\tD_E^*t}
&=&
     \ppair{\tD_Es}{t}
 =
     \int_X\pair{\tD_Es}{t}\td\Vol
 =
     \int_X
       \bair{\tD_Es}{\star t}
\end{eqnarray*}
再注意$\tD_E$为Hermite联络,从而
$
  \td\bair{s}{\star t}
= \bair{\tD_Es}{\star t}+(-1)^p\bair{s}{\tD_E\star t}
$;再注意$X$为紧流形,从而由Stokes定理知
$\int_X\td\bair{s}{\star t}=0$.因此
\begin{eqnarray*}
  \ppair{s}{\tD_E^*t}
&=&
  \int_X\bair{\tD_Es}{\star t}
=
  (-1)^{p+1}\int_X
    \bair{s}{\tD_E\star t}
=
  (-1)^{p+1}(-1)^{p(n-p)}\int_X
    \bair{s}{\star\star\tD_E\star t}\\
&=&
  (-1)^{np+1}\int_X
    \pair{s}{\star\tD_E\star t}\td\Vol
=
  (-1)^{np+1}\ppair{s}{\star\tD_E\star t}
\end{eqnarray*}
从而由$s,t$的任意性,有$\tD_E^*=(-1)^{np+1}\star\tD_E\star$.
\end{proof}

\begin{lemma}($\#$算子的性质)
\label{井算子与Hermite联络}

设$E\to X$为紧定向黎曼流形$X$上的Hermite向量丛,$\tD_E$
为$E$上的Hermite联络,则在空间$\Omg^p(X,E)$上,成立
\begin{eqnarray*}
\tD_{E^*}\#   &=&  (-1)^p\#\tD_E^*\\
\tD_{E^*}^*\# &=&  (-1)^{p+1}\#\tD_E
\end{eqnarray*}
\end{lemma}

\begin{proof}
对于任意$s\in\Omg^p(X,E)$以及$t\in\Omg^{n-p+1}(X,E^*)$,注意到
对偶联络$\tD_{E^*}$满足
$$
  \td(\#s\wedge\#t)
=
  \tD_{E^*}\#s\wedge\#t
  +(-1)^{n-p}\#s\wedge\tD_E\#t
$$
再注意$X$的紧性,使用Stokes定理,因此有
\begin{eqnarray*}
     \ppair{\tD_{E^*}\#s}{t}
&=&
     \int_X\pair{\tD_{E^*}\#s}{t}\td\Vol
 =
     \int_X
       \tD_{E^*}\#s\wedge\#t
 =
     (-1)^{n-p+1}\int_X
       \#s\wedge\tD_E\#t\\
&=&
     (-1)^{n-p+1}(-1)^{(n-p)p}
     \int_X
       \tD_E\#t\wedge\#s
 =
     (-1)^{np+n+1}
     \int_X
       \pair{\tD_E\#t}{s}\td\Vol\\
&=&
     (-1)^{np+n+1}
     \int_X
       \pair{\#t}{\tD_E^*s}\td\Vol
 =
     (-1)^{np+n+1}
     \int_X
       \#t\wedge\#\tD_E^*s\\
&=&
     (-1)^{np+n+1}(-1)^{(p-1)(n-p+1)}
     \int_X
       \#\tD_E^*s\wedge\#t
 =
     (-1)^p\int_X
       \pair{\#\tD_E^*s}{t}\td\Vol\\
&=&
     (-1)^p\ppair{\#\tD_E^*s}{t}
\end{eqnarray*}
由$s,t$的任意性,可得$\tD_{E^*}\#=(-1)^p\#\tD_E^*$.
类似地对任意$s\in\Omg^p(X,E)$以及$t\in\Omg^{n-p-1}(X,E)$,
我们还可以如下验证之(利用$\#\#=(-1)^{p(n-p)}\id$):
\begin{eqnarray*}
     \ppair{\tD_{E^*}^*\#s}{t}
&=&
     \int_X\pair{\tD_{E^*}^*\#s}{t}\td\Vol
 =
     \int_X\pair{\#s}{\tD_{E^*}t}\td\Vol
 =
     \int_X
     \overline{\pair{\tD_{E^*}t}{\#s}}\td\Vol
 =
     \int_X
     \overline{\tD_{E^*}t\wedge\#\#s}\\
&=&
     (-1)^{p(n-p)}\int_X
     \overline{\tD_{E^*}t\wedge s}
 =
     (-1)^{(p-1)(n-p)}
     \int_X
     \overline{t\wedge\tD_E s}\\
&=&
     (-1)^{(p-1)(n-p)}(-1)^{(p+1)(n-p-1)}
     \int_X\overline{t\wedge\#\#\tD_Es}
 =
     (-1)^{p-1}\int_X
     \overline{\pair{t}{\#\tD_Es}}\td\Vol\\
&=&
     (-1)^{p-1}\int_X
     \pair{\#\tD_Es}{t}\td\Vol
 =
      (-1)^{p-1}
      \ppair{\#\tD_Es}{t}
\end{eqnarray*}
从而$\tD_{E^*}^*\#=(-1)^{p-1}\#\tD_E$.
\end{proof}

\begin{definition}(向量丛上的Laplace算子)

设$E\to X$为紧定向黎曼流形$X$上的Hermite向量丛,
$\tD_E$为$E$的一个Hermite联络,则定义关于联络$\tD_E$的\textbf{Laplace算子}
$\yc_E:\Omg^p(X,E)\to\Omg^p(X,E)$为:
$$\yc_E:=\tD_E\tD_E^*+\tD_E^*\tD_E$$
\end{definition}

%\begin{definition}$$\yc_E=D_ED_E^*+D_E^*D_E:C^\infty
%(M,\wedgeform{p}T^*M\ten E)\to C^\infty
%(M,\wedgeform{p}T^*M\ten E)$$\end{definition}

为说明如此定义的Laplace算子与古典情形
$\yc:=\sum\limits_{i=1}^{n}\frac{\p^2}{\p x_i^2}$
之间的关系,我们来看特殊例子。
考虑$X=\bbR^n$配以标准黎曼度量$g=\sum\limits_{i=1}^n\td x_i\ten\td x_i$,
以及平凡Hermite线丛$E:=X\times\bbC$。对于外微分算子(平凡线丛$E$上的联络)
$$\td:\Omg^{p-1}(X,E)\to\Omg^p(X,E)$$
则对任意$u\in\Omg^p(X,E)$以及$v\in\Omg^{p-1}(X,E)$,
若$\supp u\cap\supp v\ssubset X=\bbR^n$,则
\begin{eqnarray*}
     \ppair{\td^*u}{v}
&=&
     \ppair{u}{\td v}
 =
     \int_X\pair{u}{\td v}\td\Vol
 =
     \sum_{i=1}^{n}\int_X
     \pair{u}{\td x_i\wedge\pfrac{v}{x_i}}\td\Vol\\
&=&
     \sum_{i=1}^{n}\int_X
     \pair{\pp{x_i}\suobing u}{\pfrac{v}{x_i}}\td\Vol
 =
     \sum_{i=1}^{n}\int_X
     -\pair{\pp{x_i}\suobing\frac{u}{x_i}}{v}\td\Vol
 =
     -\ppair{\sum_{i=1}^{n}\pp{x_i}\suobing\pfrac{u}{x_i}}
           {v}
\end{eqnarray*}
其中“$\suobing$”为切向量场与微分形式之间的缩并运算。
因此对于$u=\sum\limits_{|I|=p}u_I\td x_I\in\Omg^p(X,E)$,成立
$$
  \td^* u
=
  -\sum_{i=1}^{n}
  \pp{x_i}\suobing\pfrac{u}{x_i}
=
  -\sum_{i=1}^{n}
   \sum_{|I|=p}
   \pfrac{u_I}{x_i}
   \pp{x_i}\suobing\td x_I
$$
因此对于$u\in\Omg^p(X,E)$,成立
\begin{eqnarray*}
     \td\td^*u
&=&
     -\sum_{i,j=1}^{n}
        \td x_j\wedge\pp{x_j}
        \left(
          \pp{x_i}\suobing\pfrac{u}{x^i}
        \right)
 =
     -\sum_{i,j=1}^{n}
        \pp{x_j}
        \left(
          \td x_j\wedge
          \left(
          \pp{x_i}\suobing\pfrac{u}{x^i}
          \right)
        \right)
\\
     \td^*\td u
&=&
     -\sum_{i,j=1}^{n}
        \pp{x_j}\suobing\pp{x_j}
        \left(
          \td x_i\wedge\pfrac{u}{x_i}
        \right)
 =
     -\sum_{i,j=1}^n
        \delta_{ij}\pmfrac{u}{x_i}{x_j}
       +\pp{x_j}
        \left(
          \td x_i\wedge
          \left(
            \pp{x_j}\suobing\pfrac{u}{x_j}
          \right)
        \right)
\end{eqnarray*}
因此有
$$
  \yc u=(\td\td^*+\td^*\td)u
       =-\sum_{i=1}^{n}
         \frac{\p^2 u}{\p x_i^2}
$$

%\begin{example}Let $M=\bbR^n$, $g=\sum\limits_{i=1}^n\td x_i^2$,
%$E=M\times \bbC$ trivial line bundle with $D_E=\td$.then
%$$\yc_Eu=(\td\td^*+\td^*\td)u=-\sum_{i=1}^n\left(\sum_{|I|=p}
%\ppfrac{u_I}{x_I^2}\td x_I\right)$$where$$u=\sum_{|I|=p}u_I\td x_I$$\end{example}

\begin{thm}
设$E\to X$为紧定向黎曼流形$X$上的Hermite向量丛,
$\tD_E$为$E$的一个Hermite联络,则关于$\tD_E$的Laplace算子
$\yc_E:=\tD_E\tD_E^*+\tD_E^*\tD_E$是自伴的椭圆算子。
\end{thm}

\begin{proof}
显然$\yc_E^*=(\tD_E\tD_E^*+\tD_E^*\tD_E)^*=\tD_E^*\tD_E+\tD_E\tD_E^*=\yc_E$,
从而 $\yc_E$是自伴的。再注意容易验证$\tD_E$与$\tD_E^*$
的主符号$\sgm_{\tD_E},\sgm_{\tD_{E}^*}$满足
\begin{eqnarray*}
\sgm_{\tD  }(x,\xi)(u)&=&\xi\wedge u\\
\sgm_{\tD^*}(x,\xi)(u)&=&-\xi^*\suobing u
\end{eqnarray*}
因此$\yc_E$的主符号$\sgm_{\yc_E}$满足
$$
  \sgm_{\yc_E}(x,\xi)(u)
=
  -\xi^*\suobing(\xi\wedge u)
  +\xi\wedge(-\xi^*\suobing u)
=
  -(\xi^*\suobing\xi)\wedge u
=
  -u
$$
即$\sgm_{\yc_E}(x,\xi)\equiv-\id:
(\wedgeform{p}T^*X\ten E)_x\to(\wedgeform{p}T^*X\ten E)_x$.
特别地,任意$\xi\neq 0\in T^*_xX$,$\sgm_{\yc_E}(x,\xi)$
都为单射,从而$\yc_E$为椭圆算子。
\end{proof}

%\begin{prop}$\yc_E$ is a self-adjoint elliptic operator.
%(i.e. $\yc_E^*=\yc_E$)\end{prop}\begin{proof}
%$\yc_E^*=\yc_E$ be definition.note that
%$$e^{-tf}D_E(e^{tf}s)=t\td f\wedge s+D_Es$$
%so,$$\sgm_{D_E}(x,\xi)s=\xi\wedge s$$
%$$\sum_{D^*_E}=-\overline{\sgm_{D_E}}^T$$
%$$\sgm_{D^*_E}(x,\xi)s=-\tilde{\xi}\suobing s$$
%where $\tilde{\xi}$ be the vector field dual to $\xi$.
%\end{proof}\begin{definition}$$\yc_E=D_ED_E^*+D_ED_E^*:
%C^\infty(M,\wedgeform{p}T^*M\ten E)\toC^\infty(M,\wedgeform{p}T^*M\ten E)$$
%so,$$\sgm_{\yc_E}(x,\xi)s=\left(\sgm_{D_E}\sgm_{D_E^*}(x,\xi)
%+\sgm_{D_E^*}\sgm_{D_E}(x,\xi)\right)s$$so,
%$\sgm_{\yc_E}$ is injective if $\xi\neq 0$, so $\yc_E$ is elliptic.
%\end{definition}Harmonic forms and Hodge isomorphism.
%$u$ is called harmonic if $\yc_{\td}u=0$.

\begin{lemma}(Laplace算子与$\#$算子)
\label{井算子与Lapalce算子-lemma}

设$E\to X$为紧定向黎曼流形$X$上的Hermite向量丛,
$\tD_E$为$X$上的一个Hermite联络,
$\yc_E$为关于联络$\tD_E$的Laplace算子,$\yc_{E^*}$为
对偶联络$\tD_{E^*}$的Laplace算子,则成立
$$\yc_{E^*}\#=\#\yc_E$$
\end{lemma}

\begin{proof}
考虑它们在$\Omg^p(X,E)$上的作用,
利用引理\ref{井算子与Hermite联络}可得
\begin{eqnarray*}
     \yc_{E^*}\#
&=&
     (\tD_{E^*}\tD_{E^*}^*+\tD_{E^*}^*\tD_{E^*})\#\\
&=&
     (-1)^{p-1}
     \tD_{E^*}\#\tD_E
    +(-1)^p
     \tD_{E^*}^*\#\tD_E^*\\
&=&
     (-1)^{p-1}(-1)^{p+1}
     \#\tD_E^*\tD_E
    +(-1)^p(-1)^{p-1-1}
     \#\tD_E\tD_E^*\\
&=&
     \#\yc_E
\end{eqnarray*}
从而证毕。
\end{proof}

\section{紧黎曼流形上Hodge理论}

\begin{definition}(平坦联络)
\index{flat connection\kong 平坦联络}

设$E\to X$为光滑流形上的光滑向量丛,$\tD_E$为$E$上的联络。
称$\tD_E$为\textbf{平坦联络}(flat connection),
若曲率$\Theta:=\tD_E^2=0$.
\end{definition}

例如,光滑流形$X$上的外微分算子$\td$
视为平凡线丛$X\times\bbC$上的联络,是平坦联络。
注意到对于平坦联络$\tD_E:\Omg^p(X,E)\to\Omg^{p+1}(X,E)$,
则有上链复形$(\Omg\updot(X,E),\tD_E)$,该上链复形的上同调
称为$E$关于平坦联络$\tD_E$的\textbf{de Rham上同调},记作
$$
  H\updot_{\DR}(X,E)
:=H\updot(\Omg\updot(X,E),\tD_E)
$$

\begin{definition}(调和形式)
\index{harmonic form\kong 调和形式}

对于紧定向黎曼流形$X$上的Hermite向量丛$E\to X$,
$\tD_E$为$E$的一个Hermite联络。称
$u\in\Omg^p(X,E)$为(关于$\tD_E$的)\textbf{调和形式}(harmonic form),
若$\yc_{\tD_E}u=0$.
其中$\yc_{E}$为关于$\tD_E$的Laplace算子。
\end{definition}

容易知道调和形式构成的空间
$$
  \mcalH^p(X,E):=\Bigset{u\in\Omg^p(X,E)}{\yc_{E}u=0}
$$
为$\Omg^p(X,E)$的线性子空间。

\begin{lemma}\label{Hodge分解-引理-lem}
条件、记号承上,则对于$s\in\Omg^p(X,E)$,
$s\in\mcalH^p(X,E)$当且仅当$\tD_E s=\tD_{E}^*s=0$.
\end{lemma}

\begin{proof}
充分性显然;再注意到
$$
  \ppair{\yc_E s}{s}
=
  \ppair{\tD_E\tD_E^*s}{s}
+ \ppair{\tD_E^*\tD_Es}{s}
= \norm{\tD_E^*s}^2+\norm{\tD_Es}^2
$$
从而必要性也显然。
\end{proof}

\begin{thm}(Hodge分解定理)

设$E\to X$是紧定向黎曼流形$X$上的Hermite向量丛,
$\tD_E$为$E$上的一个平坦联络,则有正交直和分解
$$
  \Omg^p(X,E)
=
  \mcalH^p(X,E)\oplus\im\tD_E\oplus\im\tD^*_E
$$
\end{thm}

\begin{proof}
$\mcalH^p(X,E)$与$\im\tD_E,\im\tD_E^*$的正交性由
引理\ref{Hodge分解-引理-lem}易得;
$\im\tD_E$与$\im\tD_E^*$正交,是因为$\tD_E$的平坦性,这都容易验证。
由于$\yc_E$为椭圆算子,从而由椭圆算子的一般结果
(引理\ref{椭圆算子-有限性定理-lemma}),有
$$
  \Omg^p(X,E)=\im\yc_E\oplus\ker\yc_E^*
  =\im\yc_E\oplus\ker\yc_E
  =\mcalH^p(X,E)\oplus\im\yc_E
$$
之后再注意到
$$
  \im\yc_E
=
  \im(\tD_E\tD_E^*+\tD_E^*\tD_E)
\subseteq
  \im\tD_E\oplus\im\tD_E^*
$$
从而证毕。
\end{proof}

\begin{thm}(Hodge同构定理)

设$E\to X$是紧定向黎曼流形$X$上的Hermite向量丛,
$\tD_E$为$E$上的一个平坦联络,则有同构
$$\mcalH^p(X,E)\cong H^p_{\DR}(X,E)$$
\end{thm}

\begin{proof}
注意到Hodge分解$\Omg^p(X,E)=\mcalH^p(X,E)\oplus\im\tD_E\oplus\im\td_E^*$,
从而有$\ker\tD_E=(\im\tD_E^*)^{\bot}=\mcalH^p(X,E)\oplus\im\tD_E$,因此
$$
  H^p_{\DR}(X,E)
=
  \frac{\ker\tD_E}{\im\tD_E}
\cong
  \frac{\mcalH^p(X,E)\oplus\im\tD_E}{\im\tD_E}
\cong
  \mcalH^p(X,E)
$$
证毕。
\end{proof}

%\begin{thm}$M$ is a compact Riemannian manifold,then de Rham cohomology$$H^p_{DR}(M,\bbR)\cong
%\ker(\yc_d:C^\infty(M,\wedgeform{p}T^*M))$$\end{thm}\begin{proof}
%$\yc_d$ self-adjoint elliptic, so by general result for elliptic operator,
%$$C^\infty(M,\wedgeform{p}T^*M)=\im\yc_d\oplus\ker\yc_d^*=\im\yc_d\oplus\ker\yc_d$$Claim:
%$$\im\yc_{\td}=\in \td\oplus\im\td^*$$Recall $\yc_{\td}=\td\td^*+\td^*\td$, so
%$$\im\yc_\td\subseteq\im\td\oplus\in\td^*$$on the other hand,
%$$\im\td\oplus\im\td^*\subseteq(\ker\yc_\td)^\bot=\im\yc_\td$$so,
%$$\im\yc_\td=\im\td\oplus\im\td^*$$so,$$C^\infty(M,\wedgeform{p}T^*M)=
%\im\td\oplus\im\td^*\oplus\ker\yc_\td$$so,$$H^p_{DR}(M,\bbR)
%=\frac{\im\td\oplus\ker\yc_\td}{\im\td}=\ker\yc_\td$$\end{proof}

特别地,取$E=X\times\bbC$为平凡线丛,$\tD_E=\td$为外微分,
则得到:紧定向流形(注意一定有黎曼结构)的de Rham上同调群一定是有限维的。
以及我们有同构
$$
  H^p_{\sing}(X,\bbC)
\cong H^p(X,\bbC)
\cong H^p_{\DR}(X,\bbC)
\cong \mcalH^p(X,\bbC)
$$
即:(复系数)奇异上同调、层上同调、de Rham上同调、调和形式四者同构。

\begin{cor}条件、记号承上,则有
$$\dim H^{p}_{\DR}(X,E)<+\infty$$
\end{cor}

\begin{proof}
利用Hodge同构定理,以及椭圆算子的一般性质
(引理\ref{椭圆算子-有限性定理-lemma})。
\end{proof}

\begin{thm}(庞加莱对偶)

设$X$为紧定向$n$维光滑流形,$E\to X$为$X$上的Hermite向量丛,
$\tD_E$为$E$上的平坦Hermite联络,
则对任意$0\leq p\leq n$,双线性型
\begin{eqnarray*}
     H^p_{\DR}(X,E)\times H^{n-p}_{\DR}(X,E^*)&\to&\bbC\\
     ([s],[t])&\mapsto&\int_Xs\wedge t
\end{eqnarray*}
是良定的,并且非退化。
\end{thm}

\begin{proof}
对于任意$s\in\Omg^{p-1}(X,E)$,以及任意$t\in\Omg^{n-p}(X,E^*)$
使得$[t]\in H^{n-p}(X,E^*)$,则$\tD_{E^*}t=0$.注意到
$$
  \td(s\wedge t)
=
  \tD_Es\wedge t+(-1)^{p-1}s\wedge\tD_{E^*t}
=
  \tD_Es\wedge t
$$
因此有
$$
  \int_X\tD_Es\wedge t=-\int_X\td(s\wedge t)=0
$$
这就证明了$([s],[t])\to\bbC$与$[s]$的代表元选取无关,
类似地也可证明与$[t]$的代表元选取无关,从而良定。

在看非退化性。给定$X$上的一个黎曼度量$g$,
于是可谈论关于$g$与$\tD_E$的Laplace算子$\yc_E$.
注意Hodge同构定理$H^p_{\DR}(X,E)\cong\mcalH^p(X,E)$,
从而对任意$s\in H^p(X,E)$,不妨取代表元$s$为调和形式。
则有引理\ref{井算子与Lapalce算子-lemma}容易得到,
$\#s\in\mcalH^{n-p}(E^*)$,从而$[\#s]\in H^{n-p}_{\DR}(X,E^*)$.因此有
$$
  \int_Xs\wedge\#s
=
  \int_X\pair{s}{s}\td\Vol
=
  \norm{s}^2
$$
这就证明了对任意$[s]\neq 0\in H^p_{\DR}(X,E)$,
$[t]\mapsto\int_Xs\wedge t$是非退化的。
同理该双线性型对另一边也非退化。
\end{proof}

\begin{rem}
事实上,以上结果对于赋以欧氏内积(而不是Hermite内积)
的向量丛也成立。证明也完全类似。
\end{rem}

%\begin{rem}Consider$$u\mapsto \int_M(\langle u,u \rangle+
%\langle \td u,\td u \rangle+\langle \td^*u,\td^*u \rangle)\td V$$
%这个泛函的变分是什么鬼?\end{rem}%%%%%%%%%%%%%2019.4.16第八周周二
%%%%%%%%%%%%%%%%%%%%%%%%%%Harmonic forms and Hodge isomorphismRecall:
%$M$ is a compact Riemann manifold,$$\td:C^\infty(M,\wedgeform{*}T^*M)\to
%C^\infty(M,\wedgeform{*+1}T^*M)$$adjoint $\td^*$,$$\yc_d=dd^*+d^*d$$is a
%self-adjoint elliptic operator.Hodge decomposition:$$C^\infty(M,\wedgeform{p}T^*M)
%=\ker\yc_d\oplus\im \td\oplus\im\td^*$$$$\mcalH^p(M,\bbR):=\ker\yc_d
%\quad\text{finite dimension}$$$$\mcalH^p(M,\bbR)\cong H^p_{DR}
%\cong H^p(M,\bbR)$$(Hodge isomorphism, and, de Rham-Weil)
%\textbf{Poincare duality}\begin{thm}The pairing$$H^p_{DR}(M,\bbR)\times H^{n-p}_{DR}
%(M,\bbR)\to\bbR$$$$(s,t)\mapsto\int_M s\wedge t$$
%(is well defined) is non-degenerated. In particular,
%$H^p_{DR}(M,\bbR)^*\cong H^{n-p}_{DR}(M,\bbR)$\end{thm}
%\begin{proof}the pairing factors through the pairing on
%$$\mcalH^p(M,\bbR)\times\mcalH^{n-p}(M,\bbR)\to\bbR$$
%$$(s,t)\mapsto\int_Ms\wedge t$$need to verify:(1)
%it is independent of the choice of representations.(Easy, check)
%(2) Pairing $\mcalH...\times\mcalH...$ is non-degenerated..
%claim(Exercise): Hodge star $*$ s.t. $*\yc_\td=\yc_\td*$.
%so, $s$ is a harmonic $p$-form$\iff *s$ is a harmonic $(n-p)$-form.
%note that$$s\wedge *s=\langle s,s\rangle \td V
%=\int_Ms\wedge *s=\int_M\langle s,s\rangle\td V=||s||^2$$\end{proof}
%Generalization to flat bundle. $M$ is a compact Riemannian manifold,
%$\dim_{\bbR}M=n$, $E\to M$ is a complex Hermitian vector bundle.
%\begin{definition}$E\to X$ is called flat, if it admit a connection $D_E$ s.t.
%$$D^2_E=0$$\end{definition}\begin{rem}$E$ is flat $\iff E$ is given by a representation
%$$\pi_1(M)\to GL(r,\bbC)$$\end{rem}(我们不证)Consider the complex :
%$$(C^\infty(M,\wedgeform*T^*M\ten E),D_E)$$$$\rightsquigarrow H^p_{DR}(M,E)
%:=\frac{\ker D_E}{\im D_E}$$Exercise: we have decomposition
%$$C^\infty(M,\wedgeform{p}T^*M\ten E)=\ker\yc_{D_E}\oplus\im D_E\oplus \im D_E^*$$
%$$H^p_{DR}(M,E)\cong \ker\yc_{D_E}$$and the pairing$$H^p_{DR}(M,E)\times 
%H^{n-p}_{DR}(M,E^*)\to\bbC$$$$(s,t)\mapsto \int_M s\wedge t$$
%is non-degenerate..以上是实的Hodge理论。

\begin{cor}
对于$n$维紧致定向光滑流形$X$,则其de Rham上同调满足
$$
  \dim_{\bbC}H^p_{\DR}(X,\bbC)
= \dim_{\bbC}H^{n-p}_{\DR}(X,\bbC)
$$
\end{cor}

\begin{proof}
取$X$上的平凡线丛$E\cong X\times\bbC$,配以标准的Hermite内积,则外微分$\td$
为$E$上的平坦Hermite联络。注意到$E^*\cong E$,之后庞加莱对偶即可。
\end{proof}

%\begin{cor}$$\dim\mcalH^p(M,\bbR)=\dim\mcalH^{n-p}(M,\bbR)$$\end{cor}

\section{Hermite流形与\Kahler 流形}

\begin{definition}
对于复流形$X$,称$X$为Hermite流形,若$X$配以张量$h$,局部上形如
$$h=h_{ij}\td z^i\ten\td\zbar^j$$
并且矩阵$(h_{ij})_{1\leq 1,j\leq n}$是正定Hermite矩阵。
\end{definition}

无非是全纯切丛$TX$上的Hermite结构。
但是要注意这里的Hermite结构关于右边的变量(而不是左边)是共轭线性的,
与之前介绍Hermite向量丛时有区别。
容易验证该定义的良定性。此外,与黎曼度量存在性类似,
用单位分解定理可易证任何复流形都可配以Hermite流形结构。

%\begin{definition}Let $X$ be a complex manifold, $\dim_\bbC X=n$,
%$X$ is called a Hermitian manifold, if $X$ has a Hermitian metric,
%i.e. locally $h(z):=\sum_{1\leq j,k\leq n}h_{jk}(z)\td z_j\ten\td \zbar_k$,
%where $(h_{jk})$ is positive definition Hermitian matrix.\end{definition}
%Check:the positivity of $h$ is independent of the choice of holomorphic local coordinate
%Rmk: Any complex manifold has a Hermitian metric...(Exercise)
%Fundamental $(1,1)$-form associated to $h(z)$ is defined by
%$$\omg:=-\im h=\frac{\sqrt{-1}}{2}\sum_{j,k}h_{jk}\td z_j\td \zbar_k$$
%we also call $\omg$ is the Hermitian metric on $X$
%Fact: $\omg$ is real (i.e. $\overline{\omg}=\omg$).\begin{rem}
%$h$ is a Hermite structure on $TX$(holomorphic tangent bundle of $X$).
%locally,$$\langle \pp{z_i},\pp{z_j}\rangle(z)=h_{ij}(z)$$\end{rem}

对于Hermite张量$h=h_{ij}\td z^i\ten \td\zbar^j$,
$h$诱导了$(1,1)$-形式$\omg$:
\begin{eqnarray*}
  h=h_{ij}\td z^i\ten\td\zbar^j
&\mapsto&
  \omg=\frac{\ii}{2}
  h_{ij}\td z^i\wedge\td\zbar^j
\end{eqnarray*}
容易验证良定性,事实上对于$X$上的任意全纯切向量场$\xi,\eta$,成立
$$
  \omg(\xi,\eta)=-\ii\im h(\xi,\eta)
$$
即$\omg$是$h$的虚部的相反数。显然$\omg$是实的$(1,1)$-形式,
即$\omg=\omgbar$,也就是说$\omg\in\Omg^2(X,\bbR)$.

\begin{definition}(\Kahler 度量与\Kahler 流形)
\index{\Kahler 流形}

对于Hermite流形$(X,h)$,称Hermite度量$h$为\textbf{\Kahler 度量},
若关于$h$的微分形式$\omg$满足$\td\omg=0$,
此时也称$\omg$为\textbf{\Kahler 形式};
称复流形$X$为\textbf{\Kahler 流形},若$X$上存在\Kahler 度量。
\end{definition}

%\begin{definition}$(X,\omg)$ is an Hermitian manifold, 
%$X$ is K\"{a}hler if $\td\omg=0$.\end{definition}

注意到$\td\omg=0$当且仅当$\p\omg=\pbar\omg=0$,而Hermite形式
$\omg$总是实形式,从而$\overline{\p\omg}=\pbar\omg$.
因此$\td\omg=0\iff\p\omg=0$.在局部坐标下,
$\omg=\frac{\ii}{2}h_{jk}\td z^j\wedge\td\zbar^k$,则
$$
  \p\omg
=
  \frac{\ii}{2}
  \pfrac{h_{jk}}{z^l}
    \td z^l\wedge\td z^j\wedge\td\zbar^k
=
  \frac{\ii}{2}
  \sum_{l<j}
    \left(
      \pfrac{h_{jk}}{z^l}
     -\pfrac{h_{lk}}{z^j}
    \right)
    \td z^l\wedge\td z^j\wedge\td\zbar^k
$$
$$
  \td\omg=0\iff
  \pfrac{h_{jk}}{z^l} = \pfrac{h_{lk}}{z^j}
$$

\begin{prop}(紧\Kahler 流形的拓扑限制)
\label{紧Kahler流形的拓扑限制-prop}

设$X$为紧\Kahler 流形,则对任意$k\geq 0$,必有
$$H^{2k}(X,\bbR)\neq 0$$
\end{prop}

\begin{proof}
设$\omg=\frac{\ii}{2}h_{ij}\td z^i\wedge\td\zbar^j$
为$X$的一个\Kahler 形式,则$\td\omg=0$.注意到
\begin{eqnarray*}
     \omg^n
&=&
     \left(\frac{\ii}{2}\right)^n
     n!\det(h_{ij})
     \td z^1\wedge\td\zbar^1
     \wedge\cdots\wedge
     \td z^n\wedge\td\zbar^n\\
&=&
     n!\det(h_{ij})
     \td x^1\wedge\td y^1
     \wedge\cdots\wedge
     \td x^n\wedge\td y^n
\end{eqnarray*}
从而$\omg^n$为$X$上的一个体积形式,$\int_X\omg^n\neq 0$.
因此对于$k\geq 0$,考虑
$$
  ([\omg^k],[\omg^{n-k}])
\mapsto
  \int_X\omg^k\wedge\omg^{n-k}
=
  \int_X\omg^n\neq 0
$$
从而由庞加莱对偶,知$0\neq[\omg^k]\in H^{2k}(X,\bbR)$.
\end{proof}

%\begin{prop}Locally, $\omg=\frac{\sqrt{-1}}{2}\sum_{jk} h_{jk}
%\td z_j\wedge\td \zbar_k$is Kaehler, $\iff \p\omg=0$ and $\pbar\omg=0$, i.e.
%$$\pfrac{h_{jk}}{z_l}=\pfrac{h_{lk}}{z_j}$$If $(X,\omg)$ 
%is a compact Kaehler manifold, then$$H^{2k}(X,\bbR)\neq 0$$
%\end{prop}\begin{proof}$\td\omg=0$, so $\omg\in H^2(M,\bbR)$. Claim:
%$$0\neq \omg^k\in H^{2k}(M,\bbR)$$proof of the claim:
%$$[\omg^k][\omg^{n-k}]=\int_X\omg^k\wedge\omg^{n-k}=\int_X\omg^n$$
%Since $\omg$ is positive,  locally$$\omg^n=n!\det(h_{jk})\bigwedge_{l=1}^n\left(
%\frac{\sqrt{-1}}{2}\td zl\wedge\td\zbar_l\right)>0$$is a volume form. So,
%$$[\omg^k][\omg^{n-k}]=\int_{X}\omg^n>0$$(Using Poincare dual)\end{proof}

\begin{example}(紧复流形但非\Kahler 的例子:Hopf曲面)

给定$0<\lmd<1$,考虑离散群$\Gma:=\Bigset{\lmd^n}{n\in\bbZ}\cong\bbZ$.
则$\Gma$中的元素作为标量乘积,在$\bbC^2$上有自然的作用。考虑
$$X:=(\bbC^2\setminus\{0\})/\Gma$$
则$X$具有紧复流形结构,但非\Kahler .
\end{example}

\begin{proof}
容易验证在拓扑同胚意义下$X\cong S^1\times S^3$,
从而紧致。其复流形结构由$\bbC^2\setminus\{0\}$的商流形给出。
由代数拓扑(K\"{u}nneth 公式)可知
$H^2(X,\bbR)=H^2(S^1\times S^3,\bbR)=0$,
从而由性质\ref{紧Kahler流形的拓扑限制-prop}可知$X$不是\Kahler 流形。
\end{proof}

%\begin{example}(Exists a complex manifold NOT Kaehler)
%(Hopf Surface)$$X=(\bbC^2\setminus\{0\})\big/\Gma$$
%where discrete group $\Gma:=\{\lmd^n|n\in \bbZ\}$,
%$0<\lmd<1$ fixed.\end{example}Exercise: $X\cong S^1\times S^3$ 
%$C^\infty$ homeomorphism..and $X$ is compact complex manifold.
%and $H^2(X,\bbR)=H^2(S^1\times S^3,\bbR)=0$by K\"{u}nneth Formula...
%So, $X$ is non-Kahler...

\begin{example}
黎曼曲面一定是\Kahler 流形。
\end{example}

这是平凡的例子。

\begin{example}
复环面$X=\bbC^n/\Gma$为\Kahler 流形。其中$\Gma$为格点子群。
\end{example}
因为$\bbC^n$上具有\Kahler 形式
$\omg=\frac{\ii}{2}\td z^i\wedge\td\zbar^j$.
但要注意这样的流形未必是紧的。

\begin{example}
复射影空间$\bbC\bbP^n$是\Kahler 流形。
\end{example}

\begin{proof}
回顾例子\ref{复射影空间上的超平面丛-example}当中的曲率形式
$$\Theta(\mcalO(1))=\ii\p\pbar\log(1+|z_1|^2+\cdots+|z_n|^2)$$
事实上,可以验证该形式的系数矩阵正定;并且它作为线丛的曲率形式,必然是$\td$-闭的,
因此$\Theta(\mcalO(1))$是$\bbC\bbP^n$上的\Kahler 形式。
\end{proof}

%\begin{example}Examples of Kaehler manifold)(1)Riemann surface 
%must be Kaehler...(trivial)(2)(complex torus) $X=C^n\big/\Gma$, 
%$\Gma$ is a lattice.(this manifold may not compact...)
%$$\omg=\sqrt{-1}\sum_{j,k}h_{jk}\td z_j\wedge\td\zbar_k$$
%is a Kahler metric on $X$ if $(H_{jk})>0$, $h_{jk}$ are constant.
%(3) Projective space $\bbC P^n$.$$\omg:=\sqrt{-1}\Theta_h(\mcalO(1))$$
%locally,$$\omg=\sqrt{-1}\p\pbar\log(1+|z_1|^2+...+|z_n|^2)$$
%on $\Omg$. This $\omg$ is a Kahler metric,\end{example}

\begin{example}
Let $(X,\omg)$ is a Kahler manifold,
then  any complex submanifold $Y\subseteq X$ is also Kahler.
$$i:Y\inj X$$
with the Kahler metric $i^*\omg$.
\end{example}
Exercise: Let $f:Y\to X$ be a holomorphic immersion, and assume $X$ is Kahler,
then $Y$ is Kahler.
\begin{cor}
Any projective manifold (i.e. $X\inj \bbC P^N$) is K\"{a}hler.
\end{cor}
(Algebraic Geometry.....)

\begin{prop}(Equivalent definition of Kaehler metrics)
a Hermitian metric $\omg$ is Kahler, $iff$
for all $x_0\in X$, there exists a holomorphic chart $(z_1,...,z_n)$
centered at $x_0$, s.t.
$$\omg(z)=\sqrt{-1}\sgm_{jk}\delta_{jk}\td z_j\wedge\td\zbar_k
+O(|z|^2)
$$
\end{prop}

($\Leftarrow$ is trivial...)
(left to HW)

\begin{thm}(Exercise)

If $(X,\omg)$ is Kahler, then for all $x_0\in X$,
$\exists$ holomorphic chart $z_1,...,z_n$ centered at $x_0$, s.t.
assume
$$\omg=\sqrt{-1}h_{jk}\td z_j\wedge\td\zbar_k$$
then

$$h_{lm}(z)=\delta_{lm}-\sum_{j,k}c_{jk,lm}z_j\zbar_k+O(|z|^3)$$
where $c_{jk,lm}$ is the coefficients of the Chern curvature tensor,
$$\Theta(TX)_{x}:=\sum c_{jk,lm}\td z_j\wedge\td\zbar_k\ten(\pp{z_l})^*\ten
\pp{z_m}$$
\end{thm}
(查书)



\section{紧复流形上的Hodge理论}

$(X,\omg)$ is a compact Hermitian manifold,
$E\to X$ is a homomorphic Hermitian vector bundle.
$$D_E:=D_E'+D_E''$$
Chern connection, $D_E''=\pbar$.

\begin{definition}
$$\yc_E:=D_ED_E^*+D_E^*D_E$$
\end{definition}
$$(D_E')^*=-*D_E''*$$
$$(D_E'')^*=-*D_E'*$$
$$\yc_E'=D_E'(D_E')^*+...$$
$$\yc_E''=...$$

Note that $(D_E'')^2=0$, consider the complex
$$C^\infty(X,\wedgeform{p,q}\ten E)\xra{D_E''}C^\infty(X,\wedgeform{p,q+1}\ten E)$$
$$\rightsquigarrow H^{p,q}_{D_E''}(X,E)$$
Dolbeaut cohomology...
it isom to $\ker\yc_E''$

%下次课仔细再讲
%%%%%%%%%%%%2019.4.18第八周周四%%%%%%%%%%%%%%%%%%%%%%%

Hodge theory in compact complex manifold.

Let $(X,\omg)$ be a compact complex manifold of dimension $n$.
$E\to X$ holomorphic Hermitian vector bundle, with Chern connection $D_E$,
$D_E=D_E'+D_E''$ where $D_E''=\pbar$.

Recall: $L^2$ inner product: $u\in C^\infty(X\wedgeform{p,q}\ten E)$,
$$\langle\langle u,v\rangle\rangle:=
\int_X\langle u,v\rangle
\td\text{vol}
$$

Hodge star operator $*$:
$u,v\in C^\infty(X,\wedgeform{p,q}\ten E)$,
\begin{definition}
$$*:\wedgeform{p,q}\ten E\to\wedgeform{n-q,n-p}\ten E$$
s.t.
$$u\wedge *v=\langle u,v\rangle\td\text{vol}$$
(wedge product from $\wedgeform{p,q}$, with inner product from $E$)
\end{definition}

Exercise: Take a holomorphic chart $(z_1,...,z_n)$
s.t.
$$\omg=\sqrt{-1}\sum_j \td z_j\wedge\td\zbar_j$$
at some point $p$. An orthonormal frame
$\{e_1,...,e_r\}$, Let
$$u=\sum_{|I|=p\atop |J|=q}
\sum_{\lmd=1}^r
u_{IJ}\td z_I\wedge\td\zbar_j\ten e_\lmd\in\wedgeform{p,q}\ten E$$
WHAT IS $*u$?

Formal adjoint of $D_E,D_E',D_E''$?
\begin{prop}
$$D_E^*=-*D_E*$$
$$(D_E')^*=-*D_E''*$$
$$(D_E'')^*=-*D_E'*$$
\end{prop}

\begin{definition}
$$\yc_E:=D_ED_E^*+D_E^*D_E$$
$$\yc_E':=D_E'D_E'^*+D_E'^*D_E'$$
$$\yc_E'':=\cdots$$
\end{definition}

Check: $\yc_E,\yc_E',\yc_E''$ are self adjoint, elliptic operators.

Hodge theory w.r.t. $\yc_E''$.

\begin{thm}
We have a decomposition
$$C^\infty(X,\wedgeform{p,q}\ten E)=
\ker\yc_E''\oplus\im D_E''\oplus \im D_E''^*$$

As a consequence, Dolbeault cohomology
$$H^{p,q}_{D_E''}(X,\bbC)\cong
\ker\yc_E''$$
\end{thm}

\begin{cor}
$$\dim_{\bbC}H^{p,q}_{D_E''}(X,\bbC)<+\infty$$
\end{cor}

Cohomology group
$$H^{p,q}_{D_E''}(X,\bbC)$$
$\Omg^p$:sheaf of holomorphic $p$-forms on $X$
(i.e. a $(p,0)$-form $\fai$ is holomorphic if $\pbar\fai=0$).

$\mcalE^{p,q}$:Sheaf of smooth $(p,q)$-forms on $X$.

Similarly, we have $\Omg^p(E)$ the sheaf of holomorphic $p$-forms
with values in $E$,and $\mcalE^{p,q}(E)$ the sheaf...smooth $(p,q)$-forms ...

we have an acyclic resolutions
$$0\to\Omg^p(E)\xra{D_E''}
\mcalE^{p,1}(E)\xra{D_E''}
\mcalE^{p,2}(E)\xra{D_E''}\cdots$$
(check, it is a resolution)

By de Rham-Weil theorem,
$$H^q(X,\Omg^p(E))\cong D^{p,q}_{D_E''}(X,\bbC)
\cong \mcalH^{p,q}_{D_E''}(X,\bbC):=\ker\yc_E''$$

\begin{thm}(Serre duality)

The pairing
$$H^{p,q}_{D_E''}(X,E)
\times H^{n-p,n-q}_{D_E''}(X,E^*)\to\bbC$$
$$(s,t)\mapsto\int_X s\wedge t$$
is non-degenerate
\end{thm}

\begin{proof}Define
$$\#:\wedgeform{p,q}\ten E\to \wedgeform{n-p,n-q}\ten E^*$$
by: for $u,v\in\wedgeform{p,q}\ten E$,
$$u\wedge\# v:=\langle u,v \rangle\td\text{vol}$$

Fact:
$$\yc_{E^*}''\#=\#\yc_E''$$
\end{proof}

Remark: take $E=X\times \bbC,D_E=\td=\td'+\td''$,($\td'=\p,\td''=\pbar$)
 then we have
 $$\yc'=\td'\td'^*+\td'^*\td'$$
 $$\yc''=\cdots$$
then
$$H^{p,q}_{\td''}(X,\bbC)\cong\ker\yc''\curvearrowright C^\infty(X,\wedgeform{p,q})$$
the pairing
$$H^{p,q}(X,\bbC)\times H^{n-p,n-q}(X,\bbC)\to\bbC$$
is non-degenerate.



















