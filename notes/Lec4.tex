\chapter{$L^2$ Hodge理论}

\section{向量丛上的微分算子}
Differential operators on vector bundles.

Let $X$ is a (connected) smooth manifold of ($\bbR$-)dimension $n$.
$E,F:\bbK$-vector bundle of rank $r,r'$ respectively.

\begin{definition}
a linear differential operator of degree $k$ from $E$ to $F$ is a $\bbK$-linear map
$$P:C^\infty(M,E)\to C^\infty(M,F)$$
$$u\mapsto Pu$$
locally given by
$$Pu(x)=\sum_{|\afa|\leq k}a_{\afa}(x)D^\afa u(x)$$
where $a_\afa(x)=(a_{afa,\lmd\mu}(x))$ be a $r'\times r$ matrix.
$$
u(x)=(u_1(x),...,u_r(x))^T
$$
\end{definition}

Let $t\in\bbK$, $f\in C^\infty(M,\bbK)$, $u\in C^\infty(M,E)$,
then
$$e^{-tf(x)}P(e^{tf(x)}u(x))=
t^k\sgm_P(x,\td f(x))u(x)+\text{terms } c_j(x)^{t_j}\quad(j<k)$$
\begin{definition}
$$\sgm_P:T^*M\to\Hom(E,F)$$
is called the principal symbol of $P$, which is a polynomial on $T^*M$.
\end{definition}
locally,
$$\sgm_P(x,\xi)=\sum_{|\afa|=k}a_\afa(x)\xi^\afa$$
$(\xi^\afa:=\xi_1^{\afa_1}...\xi_n^{\afa_n})$

\begin{example}
Consider $\td: C^\infty(M,\bbK)\to C^{\infty}(M,T^*M)$.
then
$$\td u=
\sum_{j=1}^n
\begin{pmatrix}
0\\
\vdots\\
1(j^{th})\\
\vdots\\
0
\end{pmatrix}
\pfrac{u}{x^i}$$
i.e.
$$\sgm_d(x,\xi)
=\sum_{j=1}^n
\begin{pmatrix}
0\\
\vdots\\
1(j^{th})\\
\vdots\\
0
\end{pmatrix}\xi_j$$
\end{example}
\begin{definition}
$P$ is called elliptic, if $\forall x\in M,\xi\in T^*_xM\setminus\{0\}$,
$$\sgm_P(x,\xi)\in \Hom(E_x,E_x)$$
is injective.
\end{definition}
For example, $\td$ is elliptic.

\textbf{$L^2$-inner product}

Let $M$ be an oriented $C^\infty$-manifold with a smooth volume form, locally
$$\td V(x)=\gamma(x)\td x_1\wedge\cdots\wedge\td x_n$$
$\gma(x)>0$. Assume $E$ has a Euclidean(or Hermitian) structure...

Let $u,v\in C^{\infty}(M,E)$,define
$$\langle\langle u,v \rangle\rangle:=
\int_M\langle u , v \rangle\td V(x)$$

define $L^2(M,E):=$ space of sections with measurable coefficients with are $L^2$ w.r.t
$\langle\langle,\rangle\rangle$.

\begin{definition}
Let $P:C^\infty(M,E)\to C^\infty(M,F)$ be a differential operator,
$E,F$ have Euclidean (or Hermitian) structure, then there exists unique
differential operator
$$P^*:C^{\infty}(M,F)\to C^\infty(M,E)$$
s.t.
$$\langle\langle Pu,v\rangle\rangle=\langle\langle u,P^*v\rangle\rangle$$
for all $u,v$ s.t. $Supp u\cap Supp v\subset\subset M$(relative compact...)

$P^*$ is called the formal adjoint of $P$.
\end{definition}

\begin{proof}
Existence: Assume that $Supp U, Supp v\subset\subset$
some coordinate chart $\Omg$ with coordinates $(x_1,...,x_n)$, then
$$\ll Pv,u\gg=\int_\Omg
\sum_{\afa,\lmd,\mu}a_{\afa,\lmd\mu}(x)
D^\afa u_\mu(x)\overline{v_\lmd(x)}\gma(x)\td x_1\cdots\td x_n
$$
integration by parts, it
$$
=\int_\Omg
   \sum_{\afa,\lmd,\mu}
     (-1)^{|\afa|}
     u_\mu(x)
     \overline{
       D^\afa(\gma(x)\overline{a_{\afa,\lmd\mu}}v_\lmd(x))
     }
     \td x_1\...\td x_n
$$
Locally,
$$
  P^*v=\sum_{|\afa|\leq k}
         (-1)^{|\afa|}
         \gma(x)^{-1}
         D^\afa
         (\gma(x)\overline{a_\afa(x)}^Tv(x))
$$

Uniqueness: use the density of $C^{\infty}$-section with compact support in $L^2(M,-)$.
\end{proof}
\begin{cor}
If $\sgm_P(x,\xi)=\sum_{|\afa|=k}a_\afa(x)\xi^{\afa}$ ,then
$$\sgm_{P^*}
=(-1)^k\overline{\sgm_P(x,\xi)}^T
$$
\end{cor}
\begin{cor}
If rank $E=$ rank$F$, $P$ is differential operator, then
$P^*$ is elliptic $\iff$ $P^*$ is elliptic.
\end{cor}


%%%%%%%%%%%%%%%%%%2019.4.11第七周周四%%%%%%%%%%%%%%%%%%%%%%%%%%%%%%

\section{椭圆算子的基本性质}
\textbf{Fundamental results of elliptic operators}

$M$ is a compact (oriented) $C^\infty$-manifold, $\dim_{\bbR}M=n$,
with a smooth volume form $\td V$.

$E$ is an Hermite vector bundle, $rank_{\bbC}E=r$.

Sobolev space:$W^k(M,E):=$ the space of section $s:M\to E$
whose derivations up to order $=k$, $:=$ the completion of space
of smooth sections w.r.t $W^k$-norm.

$(\Omg_j)_{j\in I}$: a finite open covering of $M$, $E|_{\Omg_j}$ trivial,
Let $(\rho_j)_{j\in I}$ be a partition of unity w.r.t. $(\Omg_j)_{j\in I}$,
s.t. $\sum_{j}\rho^2_j=1$.
locally, choose an orhtonormal frame $(e_{j,\lmd})_{1\leq\lmd\leq r}$ on $\Omg_j$,
then $u=\sum_{\lmd=1}^ru_{j,\lmd}e_{j,\lmd}$ on $\Omg_j$. Define
$$||u||^2_k:=
  \sum_{j,\lmd}
    ||e_ju_{j,\lmd}||_k^2
$$
where
$$
  ||e_ju_{j,\lmd}||_k^2
:=
  \int_{\Omg_j}
    \sum_{|\afa|\leq k}
      |D^\afa(e_ju_{j,\lmd})|^2\td V(x)
$$

\begin{rem}
On a compact manifold,
the equivalence of class of $||\cdot||_k$ is independent of
the choice of : partition of unity,
local trivialization, holomorphic covering...
\end{rem}

\begin{lemma}(Sobolev lemma)

For $k>l+\frac{n}{2}$, then we have
$$W^k(M,E)\subseteq C^l(M,E)$$
\end{lemma}

\begin{lemma}(Rellich lemma)

For any $k\in\bbZ_{\geq 0}$, the inclusion
$$W^{k+1}(M,E)\inj W^k(M,E)$$
is a compact operator.
\end{lemma}

\begin{lemma}(Garding inequality)

If
$$P:C^\infty(M,E)\to C^\infty(M,F)$$
is elliptic, and rank$E=$rank $F$,
$\tilde{P}:$the extension of $P$ to sections with distribution coefficients, then :
for all $u\in W^0(M,E)$, if $\tilde{P}u\in W^k(M,F)$, then
$u\in W^{k+d}(M,E)$, where $d=\deg P$, and
$$||u||_{k+d}\leq C_k
\left(
  ||\tilde{P}u||_k+||u||_0
\right)
$$
where $C_k$ depending on $k,M$.
\end{lemma}

\begin{proof}
Reference: \verb"Kodaira: deformation of complex structures (Appendix)"
\end{proof}

\begin{cor}
If $u\in\ker\tilde{P}\cap W^0(M,E)$, then $u\in C^\infty(M,E)$.
\end{cor}

\begin{lemma}(Finiteness theorem)

Setting $M$ be a compact manifold, rank$E=$rank$F$,
$$P:C^\infty(M,E)\to C^\infty(M,F)$$
elliptic,then:

(1) $\ker P$ is of finite dimension

(2) $P(C^\infty(M,E))$ is closed and of finite codimension in $C^\infty(M,F)$.
If $P^*$ is the formal adjoint of $P$, then $\exists$ decomposition
$$C^\infty(M,F)=P(C^\infty(M,E))\oplus\ker P^*$$
which is orthogonal in $W^0(M,F)=L^2(M,F)$
\end{lemma}
\begin{proof}
椭圆算子的一般结果,分析的东西233333333.
可以参考小平邦彦复流形与复结构形变的附录。
\end{proof}

\section{紧黎曼流形的Hodge理论}
\textbf{Hodge theory in compact Riemannian manifold}

Hodge star operator.

$M$ compact Riemannian manifold, $\dim_{\bbR}=n$, $E$ is a Hermitian vector bundle.
Assume $(\xi_1,...,\xi_n),(e_1,...,e_n)$ be orthonormal frame of $TM, E$
on some local chart $\Omg$, denote
$(\xi_1^*,...,\xi_n^*),(e_1^*,...,e_n^*)$ be the co-frame of
$T^*M,T^*E$.

$\wedge\updot T^*M$ is endowed with an inner product frame from $TM$.
locally,
$$\langle u_1\wedge\cdots\wedge u_p,
          u_1\wedge\cdots\wedge u_p\rangle
:=\det(\langle u_i,v_j\rangle)
$$
for $u_i,v_j\in T^*M$. Then , get an inner product on $\wedge^pT^*M$.

Assume
$$U=\sum_{|I|=p\atop i_1\leq...\leq i_p}
  u_I\xi_I^*
$$
$$V=\sum_{|I|=p\atop i_1\leq...\leq i_p}
  v_I\xi_I^*
$$
be $p$-forms, then
$$\langle u,v\rangle=\sum_{|I|=p}u_Iv_I$$
i.e. $\{\xi^*_T\}$ is an orthonormal basis of $\wedge^pT^*M$.

$\wedge^*T^*M\ten E$ has an inner product induced from $\wedge^*T^*M,E$,

\begin{definition}
the Hodge star operator
$$^*:\wedge^pT^*M\to\wedge^{n-p}T^*M$$
is defined by
$$u\wedge*v=\langle u,v\rangle\td V$$
\end{definition}
Locally, let
$$U=\sum_{|I|=p}u_I\xi_I^*,\,V=\sum_{|I|=p}v_I\xi_I^*$$
assume
$$*V=\sum_{|J|=n-p}a_J\xi_J^*$$
then
$$U\wedge*\sum u_I a_{I^c}\xi_I^*\wedge\xi_{I^c}^*
=\sum u_Ia_{I^c}\veps(I,I^c)\xi_1^*\wedge\cdots\wedge\xi_n^*
$$
$$
  \langle u,v\rangle\td V=\sum_{|I|=p}u_Iv_I\xi_1^*\wedge\cdots\wedge\xi_n^*
$$
so, we have
$$
  *V=\sum_{|I|=p}\veps(I,I^c)V_I\xi_{I^c}^*
\in \wedgeform{n-p}T^*M
$$

\begin{definition}
$$*:\wedgeform{p}T^*M\ten E\to\wedgeform{n-p}T^*M\ten E$$
is defined by
$$\{s,*t\}:=\langle s,t\rangle\td V$$
\end{definition}

Locally, assume
$$t=\sum_{|I|=p\atop1\leq\lmd\leq r}
t_{I,\lmd}\xi_I^*\ten e_{\lmd}
$$
then
$$
  *t=
  \sum_{|I|=p\atop 1\leq\lmd\leq r}
  \veps(I,I^c)
  t_{I,\lmd}\xi_{I^c}^*\ten e_{\lmd}
$$

\begin{definition}
$$\#:\wedgeform{p}T^*M\ten E\to\wedgeform{n-p}T^*M\ten E^*$$
is defined by: for any $s,t\in \wedgeform{p}T^*M\ten E$,such that
$$s\wedge\# t:=\langle s,t\rangle\td V$$
wedge product$+$ pairing of $E^*\times E\to\bbC$.
\end{definition}
Locally: assume
$$t=\sum_{|I|=p\atop 1\leq\lmd r}
t_{I,\lmd}\xi_T^*\ten e_{\lmd}
$$
then,
$$\#t=\sum_{|I|=p,\lmd}
\veps(I,I^c)t_{I,\lmd}\xi_c^*I\ten e_\lmd^*
$$

\begin{prop}
$$*^2=(-1)^{p(n-1)}\quad\text{on } \wedgeform{p}T^*M\ten E$$
$$\#^2=(-1)^{p(n-1)}\quad\text{on } \wedgeform{p}T^*M\ten E$$
\end{prop}
(正负号对吗?)

Recall: For all $s,t\in C^\infty(M,\wedgeform{p}T^*M\ten E)$,
we have an inner product
$$\langle\langle s,t\rangle\rangle
:=\int_M\langle s,t\rangle\td V
$$

\begin{thm}
Let $D_E$ be an Hermite connection on $E$,
acting on $\wedgeform{p}T^*M\ten E$, then
$$D_E^*:=(-1)^{np+1}*D_E*$$
where $D^*_E$ is the formal adjoint of $D_E$.
\end{thm}

\begin{proof}
Let $s\in C^\infty(M,\wedgeform{p}T^*M\ten E)$ and
$t\in C^\infty(M,\wedgeform{p+1}T^*M\ten E)$. then
$$
  \langle\langle D_Es,t\rangle\rangle
 =\int_M
    \langle
      D_Es,t
    \rangle
    \td V
 =
  \int_M
    \{D_Es,*t\}
$$
Since $D_E$ is Hermitian ,by definetion ,
$$\td\{s,*t\}=\{D_Es,t\}+(-1)^p\{s,D_E(*t)\}$$
so,
$$
  \langle\langle D_Es,t\rangle\rangle
=
  \int_M\td\{s,*t\}
  +(-1)^{p+1}
   \{s,D_E*t\}
=
  (-1)^{p+1}(-1)^{p(n_1)}
  \int_M
    \{s,*(*D_E*t)\}
=
  \langle\langle s,D_E^*t\rangle\rangle
$$
so,
$$D_E^*t=(-1)^{np+1}*D_E*$$
\end{proof}

\begin{definition}
$$\yc_E=D_ED_E^*+D_E^*D_E:
C^\infty(M,\wedgeform{p}T^*M\ten E)
\to
C^\infty(M,\wedgeform{p}T^*M\ten E)
$$
\end{definition}

\begin{example}
Let $M=\bbR^n$, $g=\sum\limits_{i=1}^n\td x_i^2$,
$E=M\times \bbC$ trivial line bundle with $D_E=\td$.
then
$$\yc_Eu=(\td\td^*+\td^*\td)u=
-\sum_{i=1}^n
  \left(
    \sum_{|I|=p}
    \ppfrac{u_I}{x_I^2}
    \td x_I
  \right)
$$
where
$$u=\sum_{|I|=p}u_I\td x_I$$
\end{example}

\begin{prop}
$\yc_E$ is a self-adjoint elliptic operator.
(i.e. $\yc_E^*=\yc_E$)
\end{prop}
\begin{proof}
$\yc_E^*=\yc_E$ be definition.

note that
$$e^{-tf}D_E(e^{tf}s)=t\td f\wedge s+D_Es$$
so,
$$\sgm_{D_E}(x,\xi)s=\xi\wedge s$$
$$\sum_{D^*_E}=-\overline{\sgm_{D_E}}^T$$
$$\sgm_{D^*_E}(x,\xi)s=-\tilde{\xi}\suobing s$$
where $\tilde{\xi}$ be the vector field dual to $\xi$.
\end{proof}

\begin{definition}
$$\yc_E=D_ED_E^*+D_ED_E^*:
C^\infty(M,\wedgeform{p}T^*M\ten E)\to
C^\infty(M,\wedgeform{p}T^*M\ten E)$$
so,
$$\sgm_{\yc_E}(x,\xi)s=
 \left(
   \sgm_{D_E}\sgm_{D_E^*}(x,\xi)
   +\sgm_{D_E^*}\sgm_{D_E}(x,\xi)
 \right)s
$$
so,
$\sgm_{\yc_E}$ is injective if $\xi\neq 0$, so $\yc_E$ is elliptic.
\end{definition}

Harmonic forms and Hodge isomorphism.
\begin{definition}
$u$ is called harmonic if $\yc_{\td}u=0$.
\end{definition}

\begin{thm}
$M$ is a compact Riemannian manifold,
then de Rham cohomology
$$H^p_{DR}(M,\bbR)\cong
\ker(\yc_d:C^\infty(M,\wedgeform{p}T^*M))
$$
\end{thm}
\begin{proof}
$\yc_d$ self-adjoint elliptic, so by general result for elliptic operator,
$$C^\infty(M,\wedgeform{p}T^*M)=\im\yc_d\oplus\ker\yc_d^*
=\im\yc_d\oplus\ker\yc_d$$
Claim:
$$\im\yc_{\td}=\in \td\oplus\im\td^*$$
Recall $\yc_{\td}=\td\td^*+\td^*\td$, so
$$\im\yc_\td\subseteq\im\td\oplus\in\td^*$$
on the other hand,
$$\im\td\oplus\im\td^*\subseteq(\ker\yc_\td)^\bot=\im\yc_\td$$
so,
$$\im\yc_\td=\im\td\oplus\im\td^*$$
so,
$$C^\infty(M,\wedgeform{p}T^*M)=
\im\td\oplus\im\td^*\oplus\ker\yc_\td
$$
so,
$$
  H^p_{DR}(M,\bbR)
=\frac{\im\td\oplus\ker\yc_\td}{\im\td}=\ker\yc_\td
$$
\end{proof}

\begin{cor}
$$\dim H^{p}_{DR}(M,\bbR)=\dim\ker\yc_\td<+\infty$$
\end{cor}

\begin{rem}Consider
$$u\mapsto \int_M(
  \langle u,u \rangle+
  \langle \td u,\td u \rangle+
  \langle \td^*u,\td^*u \rangle
)\td V$$
这个泛函的变分是什么鬼?
\end{rem}

%%%%%%%%%%%%%2019.4.16第八周周二%%%%%%%%%%%%%%%%%%%%%%%%%%
Harmonic forms and Hodge isomorphism

Recall: $M$ is a compact Riemann manifold, 
$$\td:C^\infty(M,\wedgeform{*}T^*M)\to C^\infty(M,\wedgeform{*+1}T^*M)$$
adjoint $\td^*$,
$$\yc_d=dd^*+d^*d$$
is a self-adjoint elliptic operator. 

Hodge decomposition:
$$C^\infty(M,\wedgeform{p}T^*M)=\ker\yc_d\oplus\im \td\oplus\im\td^*$$
$$\mcalH^p(M,\bbR):=\ker\yc_d \quad\text{finite dimension}$$
$$\mcalH^p(M,\bbR)\cong H^p_{DR}\cong H^p(M,\bbR)$$
(Hodge isomorphism, and, de Rham-Weil)

\textbf{Poincare duality}

\begin{thm}
The pairing 
$$H^p_{DR}(M,\bbR)\times H^{n-p}_{DR}(M,\bbR)\to\bbR$$
$$(s,t)\mapsto\int_M s\wedge t$$
(is well defined) is non-degenerated. In particular,
$H^p_{DR}(M,\bbR)^*\cong H^{n-p}_{DR}(M,\bbR)$
\end{thm}

\begin{proof}
the pairing factors through the pairing on 
$$\mcalH^p(M,\bbR)\times\mcalH^{n-p}(M,\bbR)\to\bbR$$
$$(s,t)\mapsto\int_Ms\wedge t$$

need to verify:(1) 
it is independent of the choice of representations.(Easy, check)
(2) Pairing $\mcalH...\times\mcalH...$ is non-degenerated..

claim(Exercise): Hodge star $*$ s.t. $*\yc_\td=\yc_\td*$.

so, $s$ is a harmonic $p$-form$\iff *s$ is a harmonic $(n-p)$-form. 

note that 
$$s\wedge *s=\langle s,s\rangle \td V
=\int_Ms\wedge *s=\int_M\langle s,s\rangle\td V=||s||^2
$$
\end{proof}

\begin{cor}
$$\dim\mcalH^p(M,\bbR)=\dim\mcalH^{n-p}(M,\bbR)$$
\end{cor}

Generalization to flat bundle. $M$ is a compact Riemannian manifold,
$\dim_{\bbR}M=n$, $E\to M$ is a complex Hermitian vector bundle.

\begin{definition}
$E\to X$ is called flat, if it admit a connection $D_E$ s.t.
$$D^2_E=0$$
\end{definition}
\begin{rem}
$E$ is flat $\iff E$ is given by a representation 
$$\pi_1(M)\to GL(r,\bbC)$$
\end{rem}
(我们不证)

Consider the complex :
$$(C^\infty(M,\wedgeform*T^*M\ten E),D_E)$$
$$\rightsquigarrow H^p_{DR}(M,E):=\frac{\ker D_E}{\im D_E}$$

Exercise: we have decomposition
$$C^\infty(M,\wedgeform{p}T^*M\ten E)=
\ker\yc_{D_E}\oplus\im D_E\oplus \im D_E^*$$
$$H^p_{DR}(M,E)\cong \ker\yc_{D_E}$$
and the pairing 
$$H^p_{DR}(M,E)\times H^{n-p}_{DR}(M,E^*)\to\bbC$$
$$(s,t)\mapsto \int_M s\wedge t$$
is non-degenerate..

以上是实的Hodge理论。

\section{K\"{a}hler流形}
\begin{definition}
Let $X$ be a complex manifold, $\dim_\bbC X=n$, 
$X$ is called a Hermitian manifold, if $X$ has a Hermitian metric, 
i.e. locally $h(z):=\sum_{1\leq j,k\leq n}h_{jk}(z)\td z_j\ten\td \zbar_k$,
where $(h_{jk})$ is positive definition Hermitian matrix.
\end{definition}

Check:the positivity of $h$ is independent of the choice of holomorphic local coordinate

Rmk: Any complex manifold has a Hermitian metric...(Exercise)

Fundamental $(1,1)$-form associated to $h(z)$ is defined by 
$$\omg:=-\im h=\frac{\sqrt{-1}}{2}\sum_{j,k}h_{jk}\td z_j\td \zbar_k$$
we also call $\omg$ is the Hermitian metric on $X$

Fact: $\omg$ is real (i.e. $\overline{\omg}=\omg$).

\begin{rem}
$h$ is a Hermite structure on $TX$(holomorphic tangent bundle of $X$).
locally, 
$$\langle \pp{z_i},\pp{z_j}\rangle(z)=h_{ij}(z)$$
\end{rem}

\begin{definition}
$(X,\omg)$ is an Hermitian manifold, $X$ is K\"{a}hler if $\td\omg=0$.
\end{definition}

\begin{prop}
Locally, $\omg=\frac{\sqrt{-1}}{2}\sum_{jk} h_{jk}\td z_j\wedge\td \zbar_k$
is Kaehler, $\iff \p\omg=0$ and $\pbar\omg=0$, i.e. 
$$\pfrac{h_{jk}}{z_l}=\pfrac{h_{lk}}{z_j}$$

If $(X,\omg)$ is a compact Kaehler manifold, then 
$$H^{2k}(X,\bbR)\neq 0$$
\end{prop}

\begin{proof}
$\td\omg=0$, so $\omg\in H^2(M,\bbR)$. Claim: 
$$0\neq \omg^k\in H^{2k}(M,\bbR)$$

proof of the claim: 

$$[\omg^k][\omg^{n-k}]=\int_X\omg^k\wedge\omg^{n-k}=\int_X\omg^n$$
Since $\omg$ is positive,  locally 
$$\omg^n=n!\det(h_{jk})
\bigwedge_{l=1}^n\left(
\frac{\sqrt{-1}}{2}\td zl\wedge\td\zbar_l
\right)>0
$$
is a volume form. So, 
$$[\omg^k][\omg^{n-k}]=\int_{X}\omg^n>0$$
(Using Poincare dual)
\end{proof}

\begin{example}(Exists a complex manifold NOT Kaehler)
(Hopf Surface)

$$X=(\bbC^2\setminus\{0\})\big/\Gma$$
where discrete group $\Gma:=\{\lmd^n|n\in \bbZ\}$, 
$0<\lmd<1$ fixed.
\end{example}

Exercise: $X\cong S^1\times S^3$ $C^\infty$ homeomorphism..
and $X$ is compact complex manifold.

and $H^2(X,\bbR)=H^2(S^1\times S^3,\bbR)=0$
by K\"{u}nneth Formula...

So, $X$ is non-Kahler...

\begin{example}Examples of Kaehler manifold)

(1)Riemann surface must be Kaehler...(trivial)

(2)(complex torus) $X=C^n\big/\Gma$, $\Gma$ is a lattice.
(this manifold may not compact...)

$$\omg=\sqrt{-1}\sum_{j,k}h_{jk}\td z_j\wedge\td\zbar_k$$
is a Kahler metric on $X$ if $(H_{jk})>0$, $h_{jk}$ are constant.

(3) Projective space $\bbC P^n$.
$$\omg:=\sqrt{-1}\Theta_h(\mcalO(1))$$
locally, 
$$\omg=\sqrt{-1}\p\pbar\log(1+|z_1|^2+...+|z_n|^2)$$
on $\Omg$. This $\omg$ is a Kahler metric,
\end{example}

\begin{example}
Let $(X,\omg)$ is a Kahler manifold, 
then  any complex submanifold $Y\subseteq X$ is also Kahler.
$$i:Y\inj X$$
with the Kahler metric $i^*\omg$.
\end{example}
Exercise: Let $f:Y\to X$ be a holomorphic immersion, and assume $X$ is Kahler, 
then $Y$ is Kahler.
\begin{cor}
Any projective manifold (i.e. $X\inj \bbC P^N$) is K\"{a}hler.
\end{cor}
(Algebraic Geometry.....)

\begin{prop}(Equivalent definition of Kaehler metrics)
a Hermitian metric $\omg$ is Kahler, $iff$
for all $x_0\in X$, there exists a holomorphic chart $(z_1,...,z_n)$ 
centered at $x_0$, s.t. 
$$\omg(z)=\sqrt{-1}\sgm_{jk}\delta_{jk}\td z_j\wedge\td\zbar_k
+O(|z|^2)
$$
\end{prop}

($\Leftarrow$ is trivial...)
(left to HW)

\begin{thm}(Exercise)

If $(X,\omg)$ is Kahler, then for all $x_0\in X$, 
$\exists$ holomorphic chart $z_1,...,z_n$ centered at $x_0$, s.t.
assume
$$\omg=\sqrt{-1}h_{jk}\td z_j\wedge\td\zbar_k$$
then

$$h_{lm}(z)=\delta_{lm}-\sum_{j,k}c_{jk,lm}z_j\zbar_k+O(|z|^3)$$
where $c_{jk,lm}$ is the coefficients of the Chern curvature tensor,
$$\Theta(TX)_{x}:=\sum c_{jk,lm}\td z_j\wedge\td\zbar_k\ten(\pp{z_l})^*\ten
\pp{z_m}$$
\end{thm}
(查书)

\section{Hodge theory on compact compact manifold}

$(X,\omg)$ is a compact Hermitian manifold, 
$E\to X$ is a homomorphic Hermitian vector bundle.
$$D_E:=D_E'+D_E''$$
Chern connection, $D_E''=\pbar$.

\begin{definition}
$$\yc_E:=D_ED_E^*+D_E^*D_E$$
\end{definition}
$$(D_E')^*=-*D_E''*$$
$$(D_E'')^*=-*D_E'*$$
$$\yc_E'=D_E'(D_E')^*+...$$
$$\yc_E''=...$$

Note that $(D_E'')^2=0$, consider the complex 
$$C^\infty(X,\wedgeform{p,q}\ten E)\xra{D_E''}C^\infty(X,\wedgeform{p,q+1}\ten E)$$
$$\rightsquigarrow H^{p,q}_{D_E''}(X,E)$$
Dolbeaut cohomology...
it isom to $\ker\yc_E''$

%下次课仔细再讲










