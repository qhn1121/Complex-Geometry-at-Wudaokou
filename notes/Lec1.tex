\chapter{多复变函数}
\section{多元全纯函数}
首先快速回顾单复变函数的知识。
我们通常用$\Omg$来表示$\bbC$的开子集,
$z=x+iy$为$\bbC$的坐标。对于$z\in\bbC$以及实数$R>0$,我们令
$$\bbD(z,R):=\{w\in\bbC|\,|w-z|< R\}$$
为以$z$为圆心$R$为半径的开圆盘。

此外,我们有如下常用记号:
$$\left\{\begin{array}{l}
\td z:=\td x + i\td y\\
\td \bar{z} := \td x- i\td y
\end{array}\right.\quad
\left\{\begin{array}{l}
\pp{z}:=\frac{1}{2}
\left(\pp{x}-i\pp{y}\right)\\
\pp{\bar{z}} := \frac{1}{2}
\left(\pp{x}+i\pp{y}\right)
\end{array}\right.
$$
对于函数$f:\Omg\to\bbC$,
称$f$是\textbf{全纯}(holomorphic)的,
\index{holomorphic function\kong 全纯函数}
若在$\Omg$中成立
$$\pbar f:=\pfrac{f}{\bar{z}}\td\zbar=0$$
我们知道,$f$是全纯的当且仅当$f$在$\Omg$
处处能够局部地展开为收敛幂级数。

对于$\bbC$中的紧致集$K$,称函数$f:K\to\bbC$是全纯的,
如果存在$K$的开邻域$\Omg\supseteq K$,
使得$f$可延拓为$\Omg$上的全纯函数。

单复变函数论中有如下重要结果:

\begin{thm}(柯西积分公式)
设$\bbD\subseteq\bbC$为$\bbC$中的开圆盘,$f:\bbD\to\bbC$为
$\bbD$上的全纯函数,且在$\p\bbD$连续,
则对于任意$w\in\bbD$,成立
$$f(w)=\frac{1}{2\pi i}\int_{\p\bbD}\frac{f(z)}{z-w}\td z$$
\end{thm}

此定理能推导出单变量全纯函数理论的“almost everything”.
这里不再赘述。

我们开始考虑多变量全纯函数。

\begin{definition}
设$\Omg\subseteq\bbC^n$为$\bbC^n$的开子集,函数$f:\Omg\to\bbC$
称为(多变量)\textbf{全纯函数},如果满足以下条件:

(1)$f$是连续函数;

(2)对任意$1\leq j\leq n$,以及任意固定的
$z_1,...,z_{j-1};z_{j+1},...,z_n\in\bbC$,关于$z_j$的单变量函数
$$z_j\mapsto f(z_1,...,z_{j-1};z_j;z_{j+1},...,z_n)$$
是(单变量)全纯函数。
\end{definition}

事实上,如果该定义中的(2)成立,那么能推出(1)成立,
也就是说此定义中的(1)可以去掉。其证明比较复杂,我们承认之。

\begin{notation}
对于$\bbC^n$的开子集$\Omg$,我们记
$$\mcalO(\Omg):=\{f:\Omg\to\bbC|f\text{是$\Omg$上的全纯函数}\}$$
\end{notation}
容易知道$\mcalO(\Omg)$有显然的$\bbC$-代数结构。\vs

本节将说明,多变量全纯函数具有一些与单变量全纯函数类似的性质。

\begin{notation}
对于$z=(z_1,z_2,...,z_n)\in\bbC^n$以及$R=(R_1,R_2,...,R_n)\in\bbR^n$,
并且$R_j>0\,\,(\forall 1\leq j\leq n)$,则我们记
$$\bbD(z,R):=\bbD(z_1,R_1)\times\bbD(z_2,R_2)\times
\cdots\times\bbD(z_n,R_n)$$
称为以$z$为中心,$R$为半径的\textbf{多圆柱}(polydisk)。
\index{polydisk\kong 多圆柱}

对于多圆柱$\bbD(z,R)$,我们记
$$\Gamma(z,R):=\p\bbD(z_1,R_1)\times\p\bbD(z_2,R_2)\times
\cdots\times\p\bbD(z_n,R_n)$$
称为$\bbD(z,R)$的\textbf{特征边界}(distinguished boundary)。
\index{distinguished boundary\kong 特征边界}
\end{notation}
特别注意特征边界$\Gamma(z,R)$
并不等于该多圆柱的边界$\p\bbD(z,R)$.

\begin{thm}(多变量全纯函数的柯西积分公式)

设$f:\overline{\bbD(z,R)}\to\bbC$为全纯函数,
则对任意的$w\in\bbD(z,R)$,成立
$$
  f(w)=
       \frac{1}{(2\pi i)^n}\int_{\Gamma(z,R)}
         \frac{f(\xi)\td\xi_1\td\xi_2\cdots\td\xi_n}
              {(\xi_1-w_1)(\xi_2-w_2)\cdots(\xi_n-w_n)}
$$
\end{thm}
\begin{proof}
由多变量全纯函数的定义,
反复使用单变量全纯函数的柯西积分公式即可。这是容易的。
\end{proof}

与单复变函数完全类似,我们也有泰勒展开:

\begin{cor}(多元全纯函数的泰勒展开公式)

对于$f\in\mcalO(\Omg)$,其中$\Omg\subseteq\bbC^n$为开子集,则
对于任何多圆柱$\bbD(z_0,R)$,如果
$\overline{\bbD(z_0,R)}\subseteq\Omg$,则对于任意$w\in\bbD(z_0,R)$,成立
$$
  f(w)=
       \sum_{\afa\in\bbN^n}a_\afa(w-z_0)^\afa
$$
其中
$$
  a_\afa=\frac{1}{(2\pi i)^n}
           \int_{\Gamma(z_0,R)}
             \frac{f(z)}
                  {(z-z_0)^{\afa+1}}
           \td z_1\td z_2\cdots\td z_n
  =\frac{f^{(\afa)}(z_0)}{\afa!}
$$
\label{多元泰勒-cor}
\end{cor}
注意这里的$\afa$为多重指标,即$\afa=(\afa_1,...,\afa_n)$,
其中每个$\afa_i$都为非负整数。
我们记
\begin{eqnarray*}
z^{\afa}&:=&z_1^{\afa_1}z_2^{\afa_2}\cdots z_n^{\afa_n}\\
\afa!&:=&\afa_1!\afa_2!\cdots\afa_n!\\
f^{(\afa)}&:=&(\p_{z_1})^{\afa_1}(\p_{z_2})^{\afa_2}\cdots(\p_{z_n})^{\afa_n}f\\
\afa+1&:=&(\afa_1+1,\afa_2+1,...,\afa_n+1)
\end{eqnarray*}

其中$z=(z_1,...,z_n)\in\bbC^n$,$f$为$n$元全纯函数。
\begin{proof}
与单复变函数的情形完全类似,可由柯西积分公式得到。
\end{proof}

\begin{thm}(柯西不等式)对于$\bbC^n$的开子集$\Omg$,
若$f\in\mcalO(\Omg)$,多圆柱$\overline{\bbD(z_0,R)}\subseteq\Omg$,
则对任意多重指标$\afa\in\bbN^n$,成立
$$\left|f^{(\afa)}(z_0)\right|\leq
\frac{\afa!}{R^\afa}
\sup_{z\in\Gamma(z_0,R)}|f(z)|$$
\end{thm}
\begin{proof}
与单复变函数的情形完全类似。
利用多元泰勒展开(推论\ref{多元泰勒-cor})即可。
\end{proof}

\begin{cor}设$\Omg\subseteq \bbC^n$为\textbf{连通}开集,
$f\in\mcalO(\Omg)$满足$\forall 1\leq k\leq n$,
$\pfrac{f}{z_k}$在$\Omg$上恒为$0$,则$f$在$\Omg$上为常值函数。
\end{cor}

\begin{cor}(刘维尔定理)
设$f\in\mcalO(\bbC^n)$,并且满足
$$|f(z)|\leq A(1+|z|)^B$$
其中$A,B$为正实数,那么$f$必为次数不超过$B$的多项式函数。
\end{cor}

这些性质于单变量全纯函数雷同,证明也是类似的。

\begin{cor}(Montel定理)

设$\Omg$为$\bbC^n$的开子集,则$\mcalO(\Omg)$
中的任何局部一致有界的全纯函数列都存在一致收敛的子列。
\end{cor}
\begin{proof}
仍类似于单复变全纯函数的情形。使用柯西积分公式,再配合
Arzela-Ascoli定理即可。从略。
\end{proof}







