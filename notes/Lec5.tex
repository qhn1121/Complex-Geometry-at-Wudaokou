\chapter{\Kahler 流形}

\section{线性代数版本的Lefschitz算子}

%Three goals:\textbf{Kahler package}\textbf{Lefschetz decomposition}
%\textbf{Hodge-Riemann bilinear relations}Linear algebra(baby representation
%theory)(local case)$\bbC^n$,$$\omg=\sqrt{-1}\sum_{i,j}h_{ij}
%\td z_i\wedge\td\zbar_j$$Kahler metric with constant coefficients.(i.e.
%$h_{ij}$ is constant, $(h_{ij})$ is positive Hermite matrix)
%W.L.O.G, by taking a linear transformation, we can assume
%$$\omg=\sqrt{-1}\sum_{j=1}^n\td z_j\wedge\td\zbar_j$$

我们考虑复向量空间$V:=\bbC^n$,配以Hermite内积$h$,其Hermite形式为
$\omg=\ii h_{ij}\td z^i\wedge\td\zbar^j$
(不妨相差$\frac{1}{2}$倍)其中$(h_{ij})$为正定Hermite矩阵。
适当选取$\bbC^n$的基,我们不妨
$$\omg=\ii\sum_{k=1}^n\td z_k\wedge\td\zbar_k$$
注意到空间$V$的Hermite内积自然诱导了
$\wedgeform{\bullet}(V^*)$上的Hermite内积。

注意线性空间$\wedgeform{\bullet}(V^*)$的自然分次
$
  \wedgeform{\bullet}(V^*)
=\bigoplus\limits_{k\geq 0}\wedgeform{k}(V^*)
$,其中$
  \wedgeform{k}(V^*)
=
  \bigoplus_{p+q=k}\wedgeform{p,q}(V^*)
$.在此意义下,$\wedgeform{\bullet}(V^*)$配以外积$\wedge$,
构成\textbf{分次交换代数}。对于线性算子
$A:\wedgeform{\bullet}(V^*)\to \wedgeform{\bullet}(V^*)$,
称$A$的\textbf{分次}为$d$,若对任意$k\geq 0$都有
$A:\wedgeform{k}(V^*)\to \wedgeform{k+d}(V^*)$;
称$A$的\textbf{双分次}为$(p,q)$,若对任意$p',q'$,都有
$A:\wedgeform{p',q'}(V^*)\to \wedgeform{p'+p,q'+q}(V^*)$.
设算子$A,B$的分次分别为$d_1,d_2$,则定义\textbf{超对易子}
$$[A,B]=AB-(-1)^{d_1d_2}BA$$

%\begin{notation} An operator is of pure degree $r$
%if it transform a form of $\deg=k$to as form of degree $k+r$.
%An operator ..of bi-degree $(p,q)$ if ...$(s,t)\to (s+p,t+q)$
%(in this case, degree $=p+q$)if $A,B$ with degree $\deg A,\deg B$,define
%$$[A,B]:=AB-(-1)^{\deg A\deg B}BA$$\end{notation}

\begin{definition}(Lefschetz算子及其伴随)
\index{Lefschetz 算子}

对于$n$维酉空间$(V,h)$,定义双分次$(1,1)$的线性算子
\begin{eqnarray*}
  L:\wedgeform{p,q}(V^*)&\to&\wedgeform{p+1,q+1}(V^*)\\
                    \eta&\mapsto&\omg\wedge\eta
\end{eqnarray*}
其中$\omg$为$h$的Hermite形式。称算子$L$为\textbf{Lefschetz算子}。
在记其(关于$\wedgeform{\bullet}(V^*)$的Hermite内积的)伴随算子为$\Lmd:=L^*$.
\end{definition}

%\begin{definition}$$L:\wedgeform{p,q}\to\wedgeform{p+1,q+1}$$
%$$u\mapsto \omg\wedge u$$is called Lefschetz operator.
%Denote $\Lmd$ to be the adjoint of $L$, adjointed by :
%Let $v\in\wedgeform{p-1,q-1}$ and $u\in\wedgeform{p,q}$
%$$\langle Lv,u\rangle:=\langle u,\Lmd u\rangle$$
%\end{definition}The operator $\Lmd$ is of bi-degree $(-1,-1)$.

对于$\omg=\ii\sum\limits_{k=1}^n\td z_k\wedge\td\zbar_k$的情形,
若$u=\sum\limits_{|I|=p\atop |J|=q}i_{IJ}\td z_I\wedge\td\zbar_J
\in\wedgeform{p,q}(V^*)$,则有
\begin{eqnarray*}
   Lu
&=&
   \ii(-1)^p
   \sum_{|I|=p\atop|J|=q}
     \sum_{k=1}^{n}
       u_{IJ}
       \left(
         \td z_k\wedge\td z_I
       \right)
       \wedge
       \left(
         \td\zbar_k\wedge\td\zbar_J
       \right)
\\
  \Lmd u
&=&
   \ii(-1)^p
   \sum_{|I|=p\atop|J|=q}
     \sum_{k=1}^{n}
       \left(
         \pp{z_k}\suobing \td z_I
       \right)\wedge
       \left(
         \pp{\zbar_k}\suobing \td\zbar_J
       \right)
\end{eqnarray*}

%\begin{prop}If$$u=\sum_{|I|=p\atop |J|=q}u_{IJ}\td z_I\wedge\td\zbar_j$$
%then$$Lu=\sqrt{-1}\sum_{|I|=p\atop |J|=q}\sum_{m=1}^n
%u_{IJ}\td z_m\wedge\td\zbar_m\wedge\td z_I\wedge\td\zbar_J$$$$\Lmd u
%= \sqrt{-1}(-1)^p\sum_{|I|=p\atop|J|=q}\sum_{m=1}^nu_{IJ}\left(
%\pp{z_m}\suobing\td z_I\right)\wedge\left(\pp{\zbar_m}\suobing\td\zbar_J
%\right)$$where "$\suobing$" is contraction.\end{prop}
%\begin{cor}(Exercise)Let$$\afa=\sqrt{-1}\sum_{j=1}^n\afa_j
%\td z_j\wedge\zbar_j$$then,($\afa$ is a operator of bi-degree $(1,1)$)
%$$[\afa,\Lmd]u=\sum_{|I|=p\atop |J|=q}\left(\sum_{i\in I}\afa_i
%+\sum_{j\in J}\afa_j-\sum_{k=1}^n\afa_k\right)u_{IJ}\td z_I\wedge\td\zbar_J
%$$where$$u=\sum_{|I|=p\atop |J|=q}u_{IJ}\td z_I\wedge\td\zbar_J$$\end{cor}
%\begin{cor}if $u\in\wedgeform{p,q}$,then$$[L,\Lmd]u=(p+q-n)u$$\end{cor}

\begin{prop}
对于$n$维酉空间$(V,h)$,则在$\wedgeform{p,q}(V^*)$上成立超对易关系
$$[L,\Lmd]=(p+q-n)\id$$
\end{prop}

\begin{proof}在正交基下暴力验证即可,
即$\omg=\ii\sum\limits_{k=1}^n\td z_k\wedge\td\zbar_k$.

对于任意$u=\sum\limits_{|I|=p\atop|J|=q}u_{IJ}
\td z_I\wedge\td\zbar_J\in\wedgeform{p,q}(V^*)$,有
\begin{eqnarray*}
     L\Lmd u
&=&
      \sum_{|I|=p\atop|J|=q}
       u_{IJ}\sum_{k,l=1}^{n}
         \left(
           \td z_k\wedge
           (\pp{z_l}\suobing\td z_I)
         \right)\wedge
         \left(
           \td\zbar_k\wedge
           (\pp{\zbar_l}\suobing\td\zbar_j)
         \right)
\end{eqnarray*}
\begin{eqnarray*}
     \Lmd Lu
&=&
      \sum_{|I|=p\atop|J|=q}
       u_{IJ}\sum_{k,l=1}^{n}
         \left(
           \pp{z_l}\suobing
           (\td z_k\wedge\td z_I)
         \right)\wedge
         \left(
           \pp{\zbar_l}\suobing
           (\td\zbar_k\wedge\td\zbar_j)
         \right)\\
&=&
      \sum_{|I|=p\atop|J|=q}
       u_{IJ}\sum_{k,l=1}^{n}
         \left(
           \delta_{kl}\td z_I
          -\td z_k\wedge(\pp{z_l}\suobing\td z_I)
         \right)\wedge
         \left(
           \delta_{kl}\td\zbar_j
          -\td\zbar_k\wedge(\pp{\zbar_l}\suobing\td\zbar_J)
         \right)\\
&=&
      \sum_{|I|=p\atop|J|=q}
       u_{IJ}
         \left(
           (n-p-q)\td z_I\wedge\td\zbar_J
          +\sum_{k,l=1}^{n}
         \left(
           \td z_k\wedge
           (\pp{z_l}\suobing\td z_I)
         \right)\wedge
         \left(
           \td\zbar_k\wedge
           (\pp{\zbar_l}\suobing\td\zbar_j)
         \right)
         \right)\\
&=&
     L\Lmd u
    +(n-p-q)u
\end{eqnarray*}
因此$[L,\Lmd]u=(p+q-n)u$.
\end{proof}

\begin{cor}若记算子$B:=[L,\Lmd]$,则有对易关系
$$
  [B,L]=2L\qqquad
  [B,\Lmd]=-2\Lmd\qqquad
  [L,\Lmd]=B
$$
\end{cor}

\begin{proof}
只需注意$B$在$\wedgeform{p,q}(V^*)$上为标量作用$(p+q-n)$,因此有
$$[B,L]=BL-LB=[(p+1)+(q+1)-n]L-L(p+q-n)=2L$$
第二式同理。第三式为$B$的定义。
\end{proof}

%\begin{cor}Denote $B:=[L,\lmd]$, then$$[B,L]=2L$$$$[B,\Lmd]=-2\Lmd$$\end{cor}
%\begin{proof}Take $u\in\wedgeform{p,q}$, then$$[B,L]=BLu-LBu=(p+q-n+2)
%Lu-(p+q-n)Lu=2Lu$$the second is similar..\end{proof}
%\textbf{$\mfksl(2,\bbC)$-representation}
%$$\mfksl(2,\bbC)=\Span_\bbC{l,\lmd,b}$$where

\begin{rem}(李代数表示论)

上述结果表明,空间$\wedgeform{\bullet}(V^*)$自然视为
李代数$\mfksl(2,\bbC):=\Span_{\bbC}\{l,b,\lmd\}$的一个表示,其中
$$
  l=\begin{pmatrix}
      0 & 0\\
      1 & 0
    \end{pmatrix}
,\qquad
  \lmd=\begin{pmatrix}
      0 & 1\\
      0 & 0
    \end{pmatrix}
,\qquad
  b=\begin{pmatrix}
      1 & 0\\
      0 & -1
    \end{pmatrix}
$$
该表示为$l\mapsto L,\,\lmd\mapsto\Lmd,\,b\mapsto B$.
\end{rem}

注意到$\wedgeform{\bullet}(V^*)$作为$\mfksl(2,\bbC)$-模,
刚好具有\textbf{权空间分解}
$$
  \wedgeform{\bullet}(V^*)
=
  \bigoplus_{k\geq 0}
    \wedgeform{k}(V^*)
$$
其中$\wedgeform{k}(V^*)$的权为$k-n$
(即算子$B$的属于本征值$(p+q-n)$的本征子空间)。
李代数$\mfksl(2,\bbC)$的有限维模结构我们众所周知,
由此立刻得到:

%we have$$[l,\lmd]=b\quad[b,l]=2l\quad[b,\lmd]=-2\lmd$$
%\begin{prop}There exists a natural action
%$$\rho:\mfksl(2,\bbC)\to\End(\bigoplus_{p,q}\wedgeform{p,q})$$
%with$$\rho(l)=L$$$$\rho(\lmd)=\Lmd$$$$\rho(b)=B$$\end{prop}
%$$L^{n-k}$$$$u\to\omg^{n-k}\wedge u$$is an isomorphism.
%$$L^{n-k}:\wedgeform{p,q}\to \wedgeform{n-k+p,n-k+q}$$
%is also an isomorphism.

\begin{thm}(Hard Lefschitz定理)
\label{HL-局部-thm}

条件、记号承上,则对任意$0\leq k\leq n$,算子
$$L^{n-k}:\wedgeform{k}(V^*)\to\wedgeform{2n-k}(V^*)$$
为线性同构。此外,事实上对于$p+q=k$,算子
$$L^{n-k}:\wedgeform{p,q}(V^*)\to\wedgeform{n-q,n-p}(V^*)$$
也为线性同构。
\end{thm}

\begin{proof}
第一式由李代数$\mfksl(2,\bbC)$的表示论立刻得到。
再注意到$L$的$(1,1)$-双分次性,比较分次立刻得第二式。
\end{proof}

%\begin{proof}Lemma:$$[L^r,\Lmd]u=r(k-n+r-1)L^{r-1}u$$
%(induction, omit)Assume $\afa\in\wedgeform{k}_\bbC$, $L^{n-k}\afa=0$,
%need to verify $\afa=0$.Claim:$$L^r:\wedgeform{k}\to\wedgeform{k+2r}$$
%is injective whenever $r\leq n-k$.proof of the claim:
%claim is true when $k=0$ or $k=1$.(check)
%Let $\afa\in \wedgeform{k}$ s.t. $L^r\afa=0(r\leq n-k)$.
%By the lemma,$$L^r\Lmd\afa-\lmd L^r\afa=r(k-n+r-1)L^{r-1}\afa$$
%so,$$L^{r-1}(L\Lmd\afa-r(k-n+r-1)\afa)=0$$by the induction on $r$,
%$$L\Lmd\afa=r(k-n+r-1)\afa$$since $r(k-n+r-1)\neq 0$, $\afa=L\beta$
%for some $\beta\in\wedgeform{k-2}$.so,$L^r\afa=L^{r+1}\beta=0$,
%by induction on $k$, we have$\beta=0$,so $\afa=0$.The claim is proved.\end{proof}

\begin{definition}(本原形式)

条件、记号承上。则对于$0\leq k\leq n$,若$L^{n-k+1}\afa=0$,则
称$\afa\in\wedgeform{k}(V^*)$为\textbf{本原形式}
(primitive form)。
\index{primitive form\kong 本原形式}
\end{definition}

我们将本原$(p,q)$-形式之全体记作$\wedgeform{p,q}_{\prim}(V^*)$.
表示论的解释:考虑$\mfksl(2,\bbC)$-模$\wedgeform{\bullet}(V^*)$
的不可约模分解。易知本原形式就是表示论当中的\textbf{最低权向量}。
从而由表示论立刻得到:

\begin{thm}(Lefschitz分解定理)
\label{LD-局部-thm}

条件记号承上,则对任意的$\afa\in\wedgeform{\bullet}(V^*)$,
则存在唯一的形如下述的分解
$$
  \afa=\sum_{i=1}^{N}
    L^{s_i}\afa_i
$$
其中$\afa_1,\afa_2,...,\afa_N$为本原形式,$s_i\geq 0$.
\end{thm}
\begin{proof}
只需在$\wedgeform{\bullet}(V^*)$的每个不可约子模的最低权空间
之中适当选取本原形式(最低权向量)即可。
\end{proof}

%\begin{definition}(Primitive form)$\afa\in\wedgeform{k}
%(k\leq n)$ is called primitive form, if\end{definition}
%\begin{cor}(Lefischtz Decomposition)(LD)For any $\afa\in\wedgeform{k},
%(1\leq k\leq 2n)$,we have a unique decomposition:$$\afa=\sum_{\gma\geq(k-n)_+}
%L^\gma\afa_r$$($(k-n)_+:=\max\{k-n,0\}$)with $\afa_r\in\wedgeform{k-2r}$
%is primitive\end{cor}\begin{proof}Existence: assume $k\leq n$, consider
%$$L^{n-k+1}\afa\in\wedgeform{2n-k+2}$$by HL, $\exists !\,\beta\in\wedgeform{k-2}$ s.t.
%$L^{n-k+2}\beta=L^{n-k+1}\afa$, so$L^{n-k+1}(\afa-L\beta)=0$, i.e. $\afa_0=\afa-L\beta$
%is primitive. $\afa=\afa_0+L\beta$, then induction on degrees,we get the
%decomposition for $\afa$.If $k>n$, we apply HL to reduce it to case 1.
%Uniqueness: Next time..\end{proof}%%%%%%%%%%%%%%%2019.4.23第九周周二%%%%%%%%%
%Today: Continuous to Hard Lef decomposition, Hodge-Riemann
%bilinear relations.Hard-Lefschitz: HL Lefschitz decomposition :LD
%Hodge-Riemann bilinear relations :HRR
%Recall: $\bbC^n,\wedgeform{k}=\bigoplus\limits_{p+q=k}\wedgeform{p,q}$,
%$\omg$: a Kahler metric on $\bbC^n$with constant coefficient $\in\wedgeform{1,1}_\bbR$.
%Lefschitz operator : $Lu=\omg\wedge u$.\begin{thm}(HL)Assume $k\leq n,
%p+q\leq n$, then$$L^{n-k}:\wedgeform{k}\to \wedgeform{2n-k}$$
%is a linear isomorphism.$$L^{n-k}:\wedgeform{p,q}\to\wedgeform{p+n-k,q+n-k}$$
%is also a linear isomorphism.\end{thm}Linear algebra..
%\begin{thm}(LD)for any $u\in\wedgeform{k}$,we have a unique decomposition
%$$u=\sum_{r\geq(k-n)_+}L^ru_r$$where $u_r\in\wedgeform{k-2r}_{prim}$ is a
%primitive form.\end{thm}Recall: a $k$-form $u\in\wedgeform{k}(k\leq n)$
%is called primitive,if $L^{n-k+1}(u)=0$. When $k>n$, $u$ is called primitive,
%$\Lmd(u)=0$, where $\Lmd$ is the adjoint of $L$.\begin{proof}
%Existence: application of $HL$.Uniqueness: Omit.\end{proof}

%听到一首歌:《Two Days》   -  2019-06-09

对于酉空间$(V,h)$,回顾Hodge$\star$算子
$
  \star:\wedgeform{p,q}(V^*)\to
  \wedgeform{n-q,n-p}(V^*)
$
使得对任意$\afa,\beta\in\wedgeform{p,q}(V^*)$,成立
$$\afa\wedge\star\betabar=\pair{\afa}{\beta}\td\Vol$$
容易验证$\star$是实算子,即对任意的$\beta\in\wedgeform{p,q}(V^*)$,
成立$\overline{\star\beta}=\star\betabar$.
这是因为对任意$\afa\in\wedgeform{p,q}(V^*)$,
$$
  \afa\wedge\overline{\star\beta}
=
  \overline{\afabar\wedge\star\beta}
=
  \overline{\pair{\afabar}{\betabar}}\td V
=
  \pair{\afa}{\beta}\td V
=
  \afa\wedge\star\betabar
$$
之后由$\afa$的任意性即可。

%%%%  正确性待验证     %%%%

%容易验证在幺正标架下,Hodge$\star$算子有显式表达式
%$$
%  \star:\td z_I\wedge\td\zbar_J
%  \mapsto(-1)^{n(n-p)}
%  \sgn(I,I^c)\sgn(J,J^c)
%  \td z_{J^c}\wedge\td\zbar_{I^c}
%$$
%由此还可以推出在$\wedgeform{p,q}(V^*)$
%上成立$\star^2=(-1)^{p+q-n}\id$.

%歌曲: Midsummer Madness

\begin{lemma}(Hodge$\star$算子与Lefschitz算子的基本关系)
$$\star\Lmd=L\star$$
\end{lemma}

\begin{proof}
对于任意$\afa\in\wedgeform{p+1,q+1}(V^*)$
以及$\beta\in\wedgeform{p,q}(V)$,注意到
\begin{eqnarray*}
     \pair{\afa}{L\beta}\td\Vol
&=&
     \overline{\pair{L\beta}{\afa}}\td\Vol
 =
     \overline{L\beta\wedge\star\afabar}
 =
     \overline{\omg\wedge\beta\wedge\star\afabar}
 =
     \overline{\beta\wedge L\star\afabar}\\
&=&
     \overline{\beta\wedge\star\star^{-1}L\star\afabar}
 =
     \overline{\beta\wedge\star\overline{\star^{-1}L\star\afa}}
 =
     \overline{\pair{\beta}{\star^{-1}L\star\afa}}\td\Vol
 =
     \pair{\star^{-1}L\star\afa}{\beta}\td\Vol
\end{eqnarray*}
因此$\Lmd=\star^{-1}L\star$,也就是说$\star\Lmd=L\star$.
\end{proof}

\begin{thm}(Hodge$\star$算子在$\wedgeform{\bullet}(V^*)$
的$\mfksl(2,\bbC)$不可约子模上的作用)
\label{Hodge star与L的关系 - thm}

设$\afa\in\wedgeform{p,q}_{\prim}(V^*)$为本原$(p,q)$-形式,
记$k:=p+q$,则成立
$$
  \star L^j\afa
=
  (-1)^{\frac{k(k+1)}{2}}
  \sqrt{-1}^{p-q}
  \frac{j!}{(n-k-j)!}
  L^{n-k-j}\afa
$$
\end{thm}

%\begin{prop} Assume $\afa\in\wedgeform{p,q}_{prim}$, and $p+q\leq n$.
%(i.e. $L^{n-p-q+1}\afa=0$),then$$*\afa=(-1)^{\frac{(p+q)(p+q-1)}{2}}
%(\sqrt{-1})^{p-q}\frac{1}{(n-p-q)!}L^{n-p-q}\afa$$\end{prop}

\begin{proof}暴力计算,异常复杂。{\color{red}\textbf{(待补)}}

[详见\verb"Huybrechts, Prop 1.2.31"]
\end{proof}
特别地,此定理表明,$\wedgeform{\bullet}(V^*)$作为
$\mfksl(2,\bbC)=\Span_{\bbC}\{L,B,\Lmd\}$-模,每个不可约子模都是
$\star$-不变子空间。

%Define the bilinear form $Q$ on $\wedgeform{k}\,\,(k\leq n)$ as follows:
%for any $u\in \wedgeform{p,q}_{prim},p+q=k\leq n$,
%and equal holds $$\iff u=0$$
\begin{thm}(Hodge-Riemann双线性关系)
\label{HRR-局部-thm}

对于$k\leq n$,定义$\wedgeform{k}(V^*)$上的共轭双线性型$Q$如下:
$$Q(\afa,\beta):=L^{n-k}\afa\wedge\overline{\beta}$$
那么对任意$u\in \wedgeform{p,q}_{\prim}(V^*)$,若$k:=p+q\leq n$,则
$$(\sqrt{-1})^{p-q}(-1)^{\frac{(p+q)(p+q-1)}{2}}Q(u,u)\geq 0$$
且等号成立当且仅当$u=0$.
\end{thm}

\begin{proof}利用定理\ref{Hodge star与L的关系 - thm},
容易验证在相差常数倍意义下,有%Take $u\in\wedgeform{p,q}_{prim}$,
$$Q(u,u)=L^{n-k}\wedge u\wedge\overline{u}
=*u\wedge\overline{u}=\langle\overline{u},\overline{u}\rangle\td\Vol
=|u|^2\td\Vol\geq 0$$%(up to a factor!)
\end{proof}

{\color{blue}(细节待补,可能有小错,但基本精神如此)}

%Summary: $\wedgeform{\bullet}=\bigoplus\limits_{1\leq k\leq n}\wedgeform{k}_\bbC$,
%where$\wedgeform{k}_{\bbC}=\bigoplus\limits_{p+q=k}\wedgeform{p,q}_\bbC$.
%Lefschitz operator $L\rightsquigarrow$ HL,LD,HRR.


\section{\Kahler 流形上的算子对易关系}

%The analogue of compact Kahler manifolds,
%$$H^k_{DR}(X,\bbC)\cong\bigoplus_{p+q=k}H^{p,q}_{Dol}(X,\bbC)$$
%$\omg$: A Kahler metric $\in H^{1,1}_{Dol}(X,\bbR)$.
%Denote $L\curvearrowright H^k_{DR}(X,\bbC)$,
%$$L(u)=[\omg,u]=[\omg]\wedge u$$

现在,设$(X,\omg)$为\Kahler 流形,考虑$\Omg^{p,q}(X,\bbC)$为
$X$上的光滑$(p,q)$-形式之全体.对于$u\in\Omg^{p,q}(X,\bbC)$,局部上
$$
  u=\sum_{|I|=p\atop |J|=q}
    u_{IJ}\td z_I\wedge\td\zbar_J
$$
再回顾微分算子$\td,\td':=\p$以及$\td'':=\pbar$.
回顾我们之前介绍的Lefschitz算子$L$及其伴随$\Lmd$,
我们在Hermite流形上也可定义之,只需在每点处的切空间上考虑即可。
即有$L:\Omg^{p,q}(X,\bbC)\to\Omg^{p+1,q+1}(X,\bbC)$,以及
$\Lmd:\Omg^{p,q}(X,\bbC)\to\Omg^{p-1,q-1}(X,\bbC)$.

%$$(\bbC^n,\omg=\sqrt{-1}\sum_{j=1}^n\td z_j\wedge\td\overline{z}_j)$$
%$u\in C^\infty(\bbC^n,\wedgeform{p,q})$, locally
%$$u=\sum_{|I|=p,|J|=q}u_{I,J}\td z_I\wedge \td z_j,\quad
%v=\sum_{|I|=p,|J|=q}v_{I,J}\td z_I\wedge \td z_j
%$$$$\langle\langle u,v\rangle\rangle=\int_{\bbC^n}\sum_{|I|=p,|J|=q}
%u_{I,J}\overline{V_{I,J}}\td Vol$$$\td=\td'+\td''$, $\td'=\p,\td''=\pbar$.
%$$\td' u=\sum_{I,J}\sum_k\pfrac{u_{I,J}}{z_k}\td z_k\wedge\td z_I
%\wedge\td z_J$$$$\td''u=\cdots$$

回顾$\Omg^{p,q}(X,\bbC)$上的内积结构$\ppair{}{}$
(视$\bbC$为$X$上的平凡线丛,配以标准Hermite度量)。
在此意义下可谈论微分算子的形式伴随$\td^*,\td'^*,\td''^*$.
容易验证,在$X=\bbC^n,\omg=\ii\sum\limits_{k=1}^n\td z_k\wedge\td\zbar_k$
的经典情况下,对于$u=\sum\limits_{|I|=p\atop|J|=q}u_{IJ}
\td z_I\wedge\td\zbar_J\in\Omg^{p,q}(X,\bbC)$,则有
\begin{eqnarray*}
  \td'^*u
&=&
  -\sum_{|I|=p\atop |J|=q}
    \sum_{k=1}^n
    \pfrac{u_{I,J}}{\zbar_k}
    \pp{z_k}
    \suobing
    (\td z_I\wedge\td\zbar_J)
\\
  \td''^*u
&=&
  -\sum_{|I|=p\atop |J|=q}
    \sum_{k=1}^n
      \pfrac{u_{I,J}}{z_k}
      \pp{\zbar_k}
      \suobing
      (\td z_I\wedge\td\zbar_J)
\end{eqnarray*}

%\begin{prop}$$[(\td'')^*,L]=\sqrt{-1}\td'$$
%\end{prop}\begin{proof}Exercise.\end{proof}

\begin{lemma}
\label{Kahler对易关系-基本引理-lem}
对于$X=\bbC^n$,$\omg=\ii\sum\limits_{k=1}^n\td z_k\wedge\td\zbar_k$,
则微分算子$\td'$与Lefshictz算子$L$满足对易关系
$$[\td''^*,L]=\ii\td'$$
\end{lemma}

\begin{proof}
暴力验证即可。对于$u\in\Omg^{p,q}(X,\bbC)$,不妨$u$为单项式
$u=u_{IJ}\td z_I\wedge\td\zbar_J$,其中多重指标$|I|=p,|J|=q$.则有
\begin{eqnarray*}
     L\td''^*u
&=&
     \omg\wedge
     \left(
       -\sum_{k=1}^{n}
         \pfrac{u_{IJ}}{z_k}
         \pp{\zbar_k}\suobing
         \left(
           \td z_I\wedge\td\zbar_J
         \right)
     \right)
=
     -\sqrt{-1}
     \sum_{k,l=1}^{n}
     \td z_I\wedge
     \left(
       \td z_l\wedge
       \left(
         \pp{\zbar_k}\suobing\td\zbar_J
       \right)
     \right)
\end{eqnarray*}
\begin{eqnarray*}
     \td''^*Lu
&=&
     \sqrt{-1}(-1)^{p+1}
     \sum_{k,l=1}^{n}
       \pfrac{u_{IJ}}{z_l}
       \pp{\zbar_l}\suobing
       \Big[
         (\td z_k\wedge\td z_I)\wedge
         (\td\zbar_k\wedge\td\zbar_J)
       \Big]\\
&=&
     \sqrt{-1}
     \sum_{k,l=1}^{n}
       \pfrac{u_{IJ}}{z_l}
         (\td z_k\wedge\td z_I)\wedge
         \left(
           \delta_{lk}\td\zbar_J-\td\zbar_k\wedge
           \left(
             \pp{\zbar_l}\suobing\td\zbar_J
           \right)
         \right)\\
&=&
     \ii\sum_{k=1}^{n}
     \pfrac{u_{IJ}}{z_k}
     \td z_k\wedge\td z_I\wedge\td\zbar_J
    -\ii\sum_{k,l=1}^{n}
       \pfrac{u_{IJ}}{z_l}
       \td\zbar_k\wedge
       \left(
         \pp{\zbar_l}\suobing\td\zbar_J
       \right)\\
&=&
     \ii\td' u+L\td''^*u
\end{eqnarray*}
因此有$[\td''^*,L]=\ii\td'$.
\end{proof}

%\begin{thm}Let $X$ be a Kahler manifold (may not compact), with Kahler metric
%$\omg$,then we have$$[(\td'')^*,L]=\sqrt{-1}\td'$$\end{thm}
%\begin{proof}Only need to verify $u\in  C^\infty_c(X,\wedgeform{p,q})$
%with compact support in a holomorphic chart at $x$.
%Assume the holomorphic chart near $x$ is choosen s.t.
%$$\omg(z)=\sqrt{-1}\sum_{1\leq j\leq n}\td z_j\wedge\td\zbar_j+O(|z|^2)$$
%$$u\in\sum_{I,J}u_{I,J}\td z_I\wedge\zbar_J$$is a $(p,q)$-form, $v$
%is also...$$\langle u,q\rangle= u_{I,J}\overline{v_{M,N}}
%\langle \td z_I,\td z_M\rangle\langle\td\zbar_J,\td\zbar_N\rangle
%=u_{IJ}\overline{V_{ij}}+a_{IJMN}(z)u_{IJ}\overline{V_{MN}}$$
%where $a_{IJMN}=O(|z|^2)$.So,$$(\td'')^*u=-\sum_{IJk}\pfrac{u_{IJ}}
%{z_k}\pp{\zbar_k}\suobing(\td z_I\wedge\td\zbar_J)+\sum_{IJMN}b_{IJMN}u_{IJ}
%\td z_M\wedge\td\zbar_N$$where $b_{IJMN}(z)=O(|z|)$. So,
%$$[(\td'')^*,L]u(x)=\sqrt{-1}\td'u(x)$$$$\Longrightarrow [(\td'')^*,L]=\sqrt{-1}\td'$$
%\end{proof}\begin{prop}In Kahler manifold,$$[(\td')^*,L]=-\sqrt{-1}\td''$$
%$$[\Lmd,\td'']=-\sqrt{-1}(\td')^*$$$$[\Lmd,\td']=\sqrt{-1}(\td'')^*$$\end{prop}



\begin{thm}(\Kahler 流形上的基本对易关系)
\label{Kahler 流形上的基本对易关系-thm}

设$(X,\omg)$为\Kahler 流形(可以非紧),则成立算子对易关系
\begin{eqnarray*}
  [\td''^*,L]=\ii\td'
&\qquad&
  [\td'^*,L]=-\ii\td''
\\
  \,[\Lmd,\td'']=-\ii\td'^*
&\qquad&
  \,[\Lmd,\td']=\ii\td''^*
\end{eqnarray*}
\end{thm}


\begin{proof}由于$\omg$为\Kahler 形式,从而由
性质\ref{Kahler流形的测地坐标-简单版本-prop}可知
对任意$x\in X$,存在$x$的全纯坐标邻域$(z_1,z_2,...,z_n)$,
使得$x$位于该坐标卡原点,并且
$$
  \omg=\ii\sum_{k=1}^{n}\td z_k\wedge\td\zbar_k
  +O(|z|^2)
$$
从而对于$u=\sum\limits_{|I|=p\atop|J|=q}
u_{IJ}\td z_I\wedge\td\zbar_J$,容易验证成立
\begin{eqnarray*}
     \td'^*u
&=&
     -\sum_{|I|=p\atop|J|=q}
      \sum_{k=1}^{n}
        \pfrac{u_{IJ}}{\zbar_k}
        \pp{z_k}\suobing
        \left(\td z_I\wedge\td\zbar_J\right)+O(|z|)
\\
     \td''^*u
&=&
     -\sum_{|I|=p\atop|J|=q}
      \sum_{k=1}^{n}
        \pfrac{u_{IJ}}{z_k}
        \pp{\zbar_k}\suobing
        \left(\td z_I\wedge\td\zbar_J\right)+O(|z|)
\end{eqnarray*}
之后与引理\ref{Kahler对易关系-基本引理-lem}的证明完全类似,
可知$[\td''^*,L]=\ii\td'$.对此式取共轭,即得$[\td'^*,L]=-\ii\td''$.
再取形式伴随,即可得到余下的两个关于$\Lmd$的对易关系。
\end{proof}

\begin{lemma}
\label{Kahler流形微分算子对易-lem}
设$(X,\omg)$为\Kahler 流形(可以非紧),则成立
$$[\td',\td''^*]=[\td'',\td'^*]=0$$
\end{lemma}

\begin{proof}注意利用对易关系(定理\ref{Kahler 流形上的基本对易关系-thm})

$$
  [\td',\td''^*]
=
  -\ii[\td',[\Lmd,\td']]
$$
而再注意超对易子$[,]$的超雅可比恒等式,以及$(\td')^2=0$,从而有
$$
  [\td',[\Lmd,\td']]
=
  [[\td',\Lmd],\td']+[\Lmd,[\td',\td']]
=
  -[\td',[\Lmd,\td']]
$$
因此$[\td',[\Lmd,\td']]=0$,从而$[\td',\td''^*]=0$.
另一式也类似。
\end{proof}

\begin{prop}(\Kahler 流形上的Laplace算子)

设$(X,\omg)$为\Kahler 流形(可以非紧),则成立
$$
  \yc'=\yc''=\frac{1}{2}\yc
$$
其中$\yc:=[\td,\td^*]$,$\yc':=[\td',\td'^*]$,
$\yc'':=[\td'',\td''^*]$为Laplace算子。
\end{prop}

\begin{proof}
注意利用对易关系(定理\ref{Kahler 流形上的基本对易关系-thm})
以及超对易子$[,]$的超雅可比恒等式,再注意到$[\td',\td'']=0$,从而
\begin{eqnarray*}
     \yc''
&=&
     [\td'',\td''^*]
=
     -\ii[\td'',[\Lmd,\td']]
=
     -\ii\left(
       [[\td'',\Lmd],\td']+[\Lmd,[\td'',\td']]
     \right)\\
&=&
     -\ii[[\td'',\Lmd],\td']
=
     -\ii\cdot\ii[\td'^*,\td']
=
     \yc'
\end{eqnarray*}
从而再注意引理\ref{Kahler流形微分算子对易-lem},可知
\begin{eqnarray*}
     \yc
&=&
     [\td'+\td'',\td'^*+\td''^*]\\
&=&
     [\td',\td'^*]
    +[\td',\td''^*]
    +[\td'',\td'^*]
    +[\td'',\td''^*]\\
&=&
    \yc'+\yc''
\end{eqnarray*}
从而证毕。
\end{proof}

%\begin{cor}$(X,\omg)$ is a Kahler manifold, then
%$$\yc_{\td}=2\yc_{\td'}=2\yc_{\td''}$$\end{cor}
%\begin{proof}For example, $\yc_\td=2\yc_{\td''}$,
%$$\yc_\td=(\td'+\td'')(\td'+\td'')^*+(\td'+\td'')^*(\td'+\td'')
%=(\td'+\td'')(\td'^*-\sqrt{-1}[\Lmd,\td'])
%+(\td'^*-\sqrt{-1}[\Lmd,\td'])(\td'+\td'')$$
%然后暴力展开, 12项???$\cdots$从略。\end{proof}

\begin{cor}
\label{Kahler-Laplace 保持双分次-cor}
设$(X,\omg)$为\Kahler 流形,则$\yc$保持双分次,即
$$
  \yc:\Omg^{p,q}(X,\bbC)\to\Omg^{p,q}(X,\bbC)
$$
\end{cor}

\begin{proof}
只需注意$\yc=2\yc'$,而$\yc'$保持双分次。
\end{proof}
由此可得,若$\afa\in\Omg\updot(X,E)$使得$\yc\afa=0$,
则$\afa$在每个$\Omg^{p,q}(X,\bbC)$的分量也是$\yc$-调和的。

\begin{thm}(Laplace算子的对易性)
\label{紧Kahler-Laplace算子对易性}

设$(X,\omg)$为\Kahler 流形(可以非紧),则
算子$\yc$与以下算子都对易:
$$
  \star\quad
  \td'\quad
  \td''\quad
  \td'^*\quad
  \td''^*\quad
  L\quad
  \Lmd
$$
\end{thm}

\begin{proof}
$[\yc,\star]=0$与黎曼流形的情形类似;直接验证$[\yc',\td']=[\yc',\td'^*]=0$
以及$[\yc'',\td'']=[\yc'',\td''^*]=0$,而$\yc=2\yc''=2\yc''^*$,从而
$\yc$与$\td',\td'',\td'^*,\td''^*$对易。

而注意$\omg$为\Kahler 形式,$\td\omg=0$,比较分次得$\td'\omg=\td''\omg=0$,从而
对任意$u\in\Omg\updot(X,\bbC)$,
$$
  [\td,L]u
=
  \td(\omg\wedge u)-\omg\wedge\td u
=
  \td\omg\wedge u=0
$$
从而$[\td,L]=0$.比较分次可知$[\td',L]=[\td'',L]=0$.
从而由超雅可比恒等式,
$$
  [\yc',L]=[[\td',\td'^*],L]
=
  [\td',[\td'^*,L]]+[[\td',L],\td'^*]
=
  -\ii[\td',\td'']
=0
$$
再注意到$\yc=2\yc'$,从而$\yc$与$L$对易。
再取伴随,注意$\yc$是自伴的,从而易知$\yc$也与$\Lmd$对易。
\end{proof}

%\begin{cor}If $(X,\omg)$ is a Kahler manifold, then
%$$\yc_\td:C^\infty(C,\wedgeform{p,q})\to C^\infty(C,\wedgeform{p,q})$$
%\end{cor}\begin{proof}Since $\yc_\td=2\yc_{\td'}$,
%$\yc_{td'}$ preserves the bi-degree.\end{proof}
%\begin{cor}If $(X,\omg)$ is a compact Kahler manifold,
%$u$ is a $\yc_\td$-harmonic $k$-form. Assume
%$$u=\sum_{p+q=k}u^{p,q}$$$$u^{p,q}\in C^\infty(X,\wedgeform{p,q})$$
%then each $u^{p,q}$ is also harmonic.\end{cor}

\section{紧\Kahler 流形的上同调群}

对于一般的紧Hermite流形$(X,h)$,回顾Hodge分解定理
(定理\ref{紧复流形的Hodge理论-thm}),考虑$E=X\times\bbC$为平凡线丛
并配以标准Hermite度量的情形,有Hodge分解
$$
  \Omg^{p,q}(X,\bbC)=
  \mcalH^{p,q}_{\td''}(X,\bbC)+\im\td''+\im\td''^*
$$
在平凡线丛的特殊情形下,注意到$(\td')^2=0$,
我们亦可类似得到关于$\yc'=[\td',\td'^*]$的Hodge分解
$$
  \Omg^{p,q}(X,\bbC)=
  \mcalH^{p,q}_{\td'}(X,\bbC)+\im\td'+\im\td'^*
$$

而对于\Kahler 流形$(X,\omg)$,注意到$\yc=2\yc'=2\yc''$,从而有
$$
     \mcalH^{p,q}_{\td}(X,\bbC)
\cong\mcalH^{p,q}_{\td'}(X,\bbC)
\cong\mcalH^{p,q}_{\td''}(X,\bbC)
$$
于是为了方便,我们将其记作$\mcalH^{p,q}(X,\bbC)$,省略下标不会有歧义。
由于$\yc=2\yc'=2\yc''$,从而推论\ref{Kahler-Laplace 保持双分次-cor}表明
$\mcalH^{k}(X,\bbC)=\bigoplus\limits_{p+q=k}\mcalH^{p,q}(X,\bbC)$.
再注意$\yc,\yc',\yc''$都为实算子,从而立刻得到:

\begin{thm}(紧\Kahler 流形的Hodge分解定理)
\label{紧Kahler流形的Hodge分解-thm}

设$X$为紧\Kahler 流形,则其de Rham上同调与Dolbeault上同调满足
\begin{eqnarray*}
  H^k_{\DR}(X,\bbC)&\cong&
  \bigoplus_{p,q=k}H^{p,q}_{\td''}(X,\bbC)
\\
  H^{p,q}_{\td''}(X,\bbC)&\cong&
  \overline{
  H^{q,p}_{\td''}(X,\bbC)
  }
\end{eqnarray*}
\end{thm}

\begin{proof}
任取\Kahler 度量$\omg$,
只需注意到Hodge同构$H^k(X,\bbC)\cong\mcalH^k(X,\bbC)$
以及$H^{p,q}_{\td''}(X,\bbC)=\mcalH^{p,q}(X,\bbC)$.
\end{proof}

事实上,上述同构是典范的,与$X$上的\Kahler 度量选取无关。
为说明这一点,我们需要一个引理:

\begin{lemma}(紧\Kahler 流形的$\p\pbar$-引理)
\label{紧Kahler流形的ppbar引理}

设$(X,\omg)$为紧\Kahler 流形,
$\afa\in\Omg^{p,q}(X,\bbC)$满足$\td\afa=0$,
则以下等价:

(1)$\afa$是$\td$-恰当的;

(2)$\afa$是$\td'$-恰当的;

(3)$\afa$是$\td''$-恰当的;

(4)$\afa$是$\td'\td''$-恰当的;

(5)$\afa$与空间$\mcalH^{p,q}(X,\bbC)$正交。
\end{lemma}

\begin{proof}分以下几步:

\textbf{(1)(2)(3)(4)$\to$(5):}对于任意$\beta\in\mcalH^{p,q}(X,\bbC)$,
则$\yc'\beta=\yc''\beta=0$.类似于引理\ref{Hodge分解-引理-lem}可知
$\td'^*\beta=\td''^*\beta=0$,从而易证(5)成立。

\textbf{(4)$\to$(1)(2)(3):}若(4)成立,则(2)(3)显然。
令$\afa=\td'\td''\beta$,则$\afa=\frac{1}{2}(\td'+\td'')(\td''-\td')\beta=
\td\frac{(\td''-\td')\beta}{2}$,从而(1)成立。

\textbf{最后只需再证(5)$\to$(4):}若$\afa\bot\mcalH^{p,q}(X,\bbC)$,
则注意到关于$\yc'$的Hodge分解,必有
$$\afa=\td'\gma+\td'^*\gma'$$
而由$\td\afa=0$比较分次可得$\td'\afa=\td''\afa=0$,因此
$$0=\td'\afa=\td'\td'^*\gma'$$
因此$0=\ppair{\td'\td'^*\gma'}{\gma'}=\ppair{\td'^*\gma'}{\td'^*\gma'}
=\norm{\td'^*\gma'}^2$,迫使$\td'^*\gma'=0$.因此$\afa=\td'\gma$.

再对$\gma$考虑关于$\yc''$的Hodge分解,令
$$
  \gma=\td''\beta+\td''^*\beta'+\beta''
$$
其中$\yc''\beta''=0$,从而$\td''\beta''=0$
(类似于引理\ref{Hodge分解-引理-lem}).于是
$$
  \afa=\td'\gma
=
  \td'\td''\beta+\td'\td''^*\beta'
=
  \td'\td''\beta-\td''^*\td'\beta'
$$
其中第二个等号利用了引理\ref{Kahler流形微分算子对易-lem}.
又因为$\td''\afa=0$,从而得到$\td''\td''^*\td'\beta'=0$,因此
$0=\ppair{\td''\td''^*\td'\beta'}{\td'\beta'}
=\norm{\td''^*\td'\beta'}^2$,迫使$\td''^*\td'\beta'=0$,因此
$\afa=\td'\td''\beta$证毕。
\end{proof}

%\begin{thm}(Hodge decomposition)$X$ is a compact Kahler manifold, 
%then we have a decomposition$$H_{\td}^k(X,\bbC)=\bigoplus_{p+q=k}H^{p,q}_{\td''}(X,\bbC)$$
%Equivalently, (sheaf cohomology)$$H^k(X,\bbC)\cong\bigoplus_{p+q=k}H^q(X,\Omg^p)$$\end{thm}
%\begin{proof}take a Kahler metric $\omg$, we can define $\yc_d,\yc_{td'}
%,\yc_{\td''}$,then\ker\yc_{\td}:=\mcalH^k(X,\bbC)
%\cong\bigoplus\limits_{p+q=k}\mcalH^{p,q}_{\td''}(X,\bbC)$$
%then $\Longrightarrow$ the decomposition for $H_\td^k(X,\bbC)$
%the decomposition for $H^k_\td(X,\bbC)$ is independent of the choice of $\omg$
%(Next time)\end{proof}%%%%%%%%%%%%%%2019.4.25第九周周四%%%%%%%%%%%%%%%%%%%%%
%Recall: Hodge decomposition,$X$ compact Kahler manifold, 
%$\dim_\bbC X=n$,Thm:(Hodge decomposition)
%$$H^k_{DR}(X,\bbC)=\bigoplus_{p+q=k}H^{p,q}(X,\bbC)
%\cong\bigoplus_{p+q=n}H^{p,q}_{\td''}(X,\bbC)$$where
%$$H^{p,q}(X,\bbC)=\{[\afa]\in H_{DR}^k(X,\bbC)|\afa\text{is a $\td$-closed s.m. $(p,q)$-form}\}$$
%Proof: take a Kahler metric $\omg$,$$H^k_{DR}(X,\bbC)\cong
%\mcalH_{\td}^k(X,\bbC)=\bigoplus\mcalH_\td^{p,q}(X,\bbC)
%=\bigoplus\mcalH_{\td''}^{p,q}(X,\bbC)$$

从而我们有以下结果,由此可知定理\ref{紧Kahler流形的Hodge分解-thm}
当中的同构是典范的:

\begin{prop}设$X$为紧\Kahler 流形,则存在典范同态
$$
  H^{p,q}_{\td''}(X,\bbC)\to H^{p+q}_{\td}(X,\bbC)
$$
\end{prop}

\begin{proof}
任取$X$的\Kahler 度量$\omg$,即相应的Laplace算子为$\yc=2\yc_{td''}$.
对任意$[\afa]_{\td''}\in H^{p,q}_{\td''}(X,\bbC)$,
注意Hodge同构,可取$[\afa]_{\tdd''}$的代表元$\afa$,
使得$\yc_{\td''}\afa=0$,从而$\yc\afa=0$,从而$\td\afa=0$.
这就给出了$[\afa]_{\td}\in H^{p+q}_{\td}(X,\bbC)$.

我们需要验证此映射的良定性,即与\Kahler 度量选取无关。
现在,若$\omg$与 $\omgtil$都为$X$上的\Kahler 度量,
记相应的Laplace算子为$\yc,\yctil$.则对于
$[\afa]\in H^{p,q}_{\td''}(X,\bbC)$,取调和代表元
$
  \left\{
    \begin{array}{l}
      \afa_1\in\mcalH^{p,q}_{\yc}(X,\bbC)\\
      \afa_2\in\mcalH^{p,q}_{\yctil}(X,\bbC)
    \end{array}
  \right.
$,则$\td\afa_1=\td\afa_2=0$.注意$[\afa_1]_{\td''}=[\afa_2]_{\td''}$,
从而存在$\gma$使得$\afa_1-\afa_2=\td''\gma$.则
$$\td\td''\gma=\td(\afa_1-\afa_2)=0$$
又容易直接验证$\td''\gma\bot\mcalH^{p,q}_{\yc}(X,\bbC)$,从而
由$\p\pbar$-引理可知,$\td''\gma$是$\td$-正合的。
因此$[\afa_1]_{\td}=[\afa_2]_{\td}$.证毕。
\end{proof}

%\begin{prop}There is a canonical isomorphism$$H^{p,q}_{\td}(X,\bbC)
%\xra{\sim}H^{p,q}_{\td''}(X,\bbC)$$$$[\afa]_{\td}\mapsto[\afa]_{\td''}$$
%where $\td\afa=0$,$\afa$ is a $(p,q)$-form.$\Rightarrow \td''\afa=0$
%\end{prop}\begin{proof}Check: this map is well defined. Need to verify:
%if $\afa=\td\beta$ is a $(p,q)$-form, then $[\afa]_{\td''}=0$,
%i.e. $\afa$ is also $\td''$-exact.$\afa$ is a $(p,q)$-form,
%$$\Rightarrow\afa=\td'\beta^{p-1,q}+\td''\beta^{p,q-1}$$
%we have $\td''\td'\beta^{p-1,q}=0$, $\td'\td''\beta^{p,q-1}=0$\end{proof}
%We need a very important lemma:\begin{lemma}($\p\pbar$-lemma)
%Let $X$ is a Kahler manifold, $\afa$ is a smooth form which is
%$\td'$ and $\td''$ closed. Then, if $\afa$ is
%$\td$ or $\td'$ or $\td''$-exact, then $\afa=\td'\td''\gma$
%for some $\gma$.\end{lemma}Using $\p\pbar$-lemma, this map is well-defined.
%Now, notice that the two space has the same dimension. So,
%we need to show the map is injective(or, surjective).
%Claim : this map is injective. If $\afa$ is a $\td$-closed
%with $[\afa]_{\td''}=0$, i.e. $\afa=\td''\beta^{p,q-1}$.
%$\afa$ is $\td$-closed$\Rightarrow\td'\td''\beta^{p,q-1}=0$,
%$\p\pbar$-lemma applying to $\td''\beta^{p,q-1}$, we have
%$$\td''\beta^{p,q-1}=\td'\td''\gma=\td(\td''\gma)$$for some $\gma$.
%Proof of $\p\pbar$-lemma:\begin{proof}Assume $\afa$ is $\td''$ exact, 
%1.e. $\afa=\td''\beta$,write$$\beta=H(\beta)+\yc_\td\gma$$
%where $H(\beta)$ is $\yc_\td$-harmonic, so
%$$\afa=\td'' H(\beta)+\td''\yc_\td\gma-2\td''\yc_{\td'}\gma$$
%(Since $\yc_\td=2\yc_{\td''}$)$$\Rightarrow\afa=2\td''(\td'\td'^*+\td'^*\td')
%=2\td''\td;\td'^*\gma-2\td'^*\td''\td'\gma$$
%By the assumption, $\td'\afa=0$, so$\td'^*\td''\td'\gma=0$
%$$\afa=-2\td'\td''\td'^*\gma$$\end{proof}

\begin{rem}(Bott-Chern 上同调)\index{Bott-Chern 上同调}

对于复流形$X$,定义$X$的\textbf{Bott-Chern 上同调}
$$
  H^{p,q}_{\BC}(X,\bbC)
:=
  \frac{\ker\td\cap\Omg^{p,q}(X,\bbC)}
       {\im(\td'\td'')\cap\Omg^{p,q}(X,\bbC)}
$$
\end{rem}

则由$\p\pbar$-引理(引理\ref{紧Kahler流形的ppbar引理})
可知,若$X$为紧\Kahler 流形,则有同构
$$
  H^{p,q}_{\BC}(X,\bbC)
\cong
  H^{p,q}_{\td}(X,\bbC)
:=\frac{(\ker\td)\cap\Omg^{p,q}(X,\bbC)}
       {(\im\td)\cap\Omg^{p,q}(X,\bbC)}
\cong
  H^{p,q}_{\td''}(X,\bbC)
$$

%\begin{rem}(Deligne-Griffiths-Morrora)If $\widehat{X}$ is bimeromapic 
%to $X$, where $X$ isacompact Kahler, then$\hat{X}$ is also satisfys the 
%$\p\pbar$-lemma.$X$ is a kahler manifold, then
%$$H^{p,q}_\td(X,\bbC)\cong H^{p,q}_{\td''}(C,\bbC)\cong
%H^{p,q}——{X,\bbC}$$\end{rem}$X$ us a compact complex manifold,define
%$$H_{BC}^{p,q}:=\frac{\td\text{-closed $(p,q)$)}}{\td'\td;;\text{exact}}$$
%Bott-Chern cohomologyExercise" If $X$ is Kahler , then $H^{p,q}_{BC}=H^{p,q}_\td$

\begin{rem}(Appeli上同调)
\index{Appeli上同调}

对于复流形$X$,还可以定义$X$的\textbf{Appeli}上同调如下:
$$
  K^{p,q}_{\tA}(X,\bbC)
:=
  \frac{\ker(\td'\td'')\cap\Omg^{p,q}(X,\bbC)}
       {(\im\td'+\im\td'')\cap\Omg^{p,q}(X,\bbC)}
$$
\end{rem}

%$$H^{p,q}_A(X,\bbC):=\frac{\td'\td''\text{closed}}{(\td')
%\text{-exact}+\{\td''\text{exact}\}}$$(Appeli cohomology)
%denote$$h_{BC}^k:=\sum_{p+q=k}\dim_{\bbC}H_{BC}^{p,q}$$
%$$h_{A}^k:=\sum_{p+q=k}\dim_{\bbC}H_{A}^{p,q}$$
%\begin{thm}$X$ satisfies $\p\pbar$-lemma $\iff$
%$$h_B^k+h_A^k=2b_k$$where$$b_k=\dim_\bbC H_{DR}^k(X,\bbC)$$
%\end{thm}%这是比较新的结果,仅供了解

\begin{Example}(复环面的Dolbeault上同调)

考虑$X:=\bbC^n/\Gma$,其中$\Gma:=\bbZ^{2n}$使得
$\Gma=\Span_{\bbZ}\{e_1,e_2,...,e_n;\ii e_1,\ii e_2,...,\ii e_n\}$,
其中$\{e_1,e_2,...,e_n\}$为$\bbC^n$的一组$\bbC$-基。
则$X$为\Kahler 流形,并且其Dolbeault上同调满足
$$H^{p,q}_{\td''}(X,\bbC)\cong\wedgeform{p,q}(\bbC^n)$$
\end{Example}

\begin{proof}首先在拓扑上,$X\cong (S^1)^n$.
由代数拓扑中的K\"{u}nneth公式可知
$$
  H^{k}_{\DR}(X,\bbC)
=
  \bigoplus_{a_1+a_2+\cdots+a_{2n}=k}
    H^{a_1}_{\DR}(S^1)\ten
    H^{a_2}_{\DR}(S^1)\ten\cdots\ten
    H^{a_{2n}}_{\DR}(S^1)
$$
注意到$H^0_{\DR}(S^1,\bbC)\cong\bbC$
以及$H^1_{\DR}(S^1)\cong\bbC\td x$.从而可知De Rham上同调满足
$$
  H^k_{\DR}(X,\bbC)
\cong
  \bigoplus_{|I|=k}\bbC\td x_I
$$
其中$I=(a_1,a_2,...,a_k)$满足$1\leq a_1<a_2<\cdots< a_k\leq 2n$.
事实上,对任意$[u]\in H^{k}_{\DR}(X,\bbC)$,该同调类中存在形如
$u=\sum\limits_{|I|=k}u_I\td x_I$的代表元,使得$u_I\in\bbC$为常数。

注意到$X$为\Kahler 流形,从而由Hodge分解
$$H^k_{\DR}(X,\bbC)\cong\bigoplus_{p+q=k}H^{p,q}_{\td''}(X,\bbC)$$
立刻得到
$$
  H^{p,q}_{\td''}(X,\bbC)
=
  \Bigset{\Big[\sum_{|I|=p\atop|J|=q}u_{IJ}\td z_I\wedge\td\zbar_J\Big]}
         {u_{IJ}\in\bbC}
$$
\end{proof}

\begin{rem}对于该复环面$X$,存在一一对应
$$
  \Bigset{[\omg]\in H^{1,1}_{\td''}(X,\bbC)\cap H^2_{\DR}(X,\bbR)}
         {\text{$\omg$为$X$上的\Kahler 度量}}
\Leftrightarrow
  \Big\{\text{$n$阶正定Hermite方阵}\Big\}
$$
\end{rem}

%\textbf{Exercise}: Complex tori$$\bbT^n:=\bbC^n/\Gma$$
%where $\Gma=\bbZ^n$. $\bbT^n$ is a compact Kahler manifold.
%Then$$H^{1,1}(\bbT^n,\bbC)\cong\wedgeform{1,1}_{\bbC}$$
%the space of $(1,1)$-forms on $\bbC^n$ with constant coefficient,
%in particular,$$\dim_\bbC H^{1,1}(\bbT^n,\bbC)=n^2$$
%\textbf{Exercise}: the set of all the Kahler class on $\bbT^n
%\subseteq H^{1,1}(X,\bbC)\cap H^2(X,\bbR)$ is equal to
%the set of $n\times n$ positive definite Hermitian metrics.
%(Hint: using Hodge theory)

\begin{thm}(Hard Lefschitz定理)

设$X$为紧\Kahler 流形,$\dim_\bbC X=n$,则对任意
$k:=p+q\leq n$,Lefschitz算子$L^{n-k}$诱导同构
\begin{eqnarray*}
L^{n-k}: H_{\DR}^k(X,\bbC)&\cong& H_{\DR}^{2n-k}(X,\bbC)\\
H^{p,q}_{\td''}(X,\bbC)&\cong& H^{n-q,n-p}_{\td''}(X,\bbC)
\end{eqnarray*}
\end{thm}

\begin{proof}
任取$X$上的\Kahler 形式$\omg$.
由Hodge同构,$H^{p,q}_{\td''}(X,\bbC)\cong\mcalH^{p,q}_{\td}(X,\bbC)$.
由定理\ref{HL-局部-thm}知,$L^{n-k}$在每一点$x\in X$处都给出了同构
$$
  L^{n-k}:\wedgeform{p,q}T^*_xX
\cong
  \wedgeform{n-q,n-p}T^*_xX
$$
再注意定理\ref{紧Kahler-Laplace算子对易性},$[\yc,L]=0$,
从而有同构
$$
  L^{n-k}:\mcalH^{p,q}_{\td}(X,\bbC)
\cong     \mcalH^{n-q,n-p}_{\td}(X,\bbC)
$$
再利用Hodge分解$H^k_{\DR}(X,\bbC)\cong\bigoplus\limits_{p+q=k}
H^{p,q}_{\td''}(X,\bbC)$即可。
\end{proof}

%$X$ is a compact Kahler, $\dim_\bbC X=n$, denote$L=\{\omg\}\curvearrowright 
%H_{DR}^k(X,\bbC)$, $\omg$ is a Kahler metric,Then we have:$$L^{n-k}: H_{DR}^k
%(X,\bbC)\cong H_{DR}^{2n-k}(X,\bbC)$$$$H^{p,q}(X,\bbC)
%\cong H^{p+n-k,q+n-k}(X,\bbC)$$where $k\leq n$, $p+q\leq n$.
%\begin{proof}Fox a Kahler metric $\omg$,$$L^{n-k}:H_{DR}^k
%\to H_{DR}^{2n-k}$$($\cong \mcalH_\td^k$, $\cong\mcalH_{\td}^{2n-k}$ 
%respectively)(there is a commutative diagram...)need to proof: 
%For any $\fai\in\mcalH_\td^k$, then$$L^{n-k}(\fai)=\omg^{n-k}\wedge\fai$$
%is also harmonic.\end{proof}\begin{lemma}$$[\yc_{\td},L]=0$$\end{lemma}\begin{proof}$
%$[\yc_\td,L]=2[\yc_{\td'},L]=2\left([\td'\td'^*,L]+[\td'^*\td',L]\right)
%=2\left(\td'[\td'*,L]+[\td'^*,L]\td'\right)$$
%(check: $[L,\td']=0$)So,$$=-2\sqrt{-1}(\td'\td''+\td''\td')=0$$\end{proof}

回顾Lefschitz分解当中的\textbf{本原形式},对于复流形$X$,
记$\Omg^{p,q}_{\prim}(X,\bbC):=
\Bigset{\afa\in\Omg^{p,q}(X,\bbC)}
{\forall\, x\in X,\,\afa(x)\in\wedgeform{p,q}_{\prim}T^*_xX}$.
而对于Hermite流形$X$,令$\mcalH^{p,q}_{\prim}(X,\bbC):=
\Bigset{\afa\in\Omg^{p,q}_{\prim}(X,\bbC)}{\yc\afa=0}$,
称其中元素为\textbf{本原调和形式}。
类似地,我们有如下定理:

%Define a class $\afa\in H^{k}_{DR}(X,\bbC)$to be positive if
%$$L^{n-k+1}(\afa)=0$$if $k\leq n$.
%(When $\afa\in H^k_{DR}(X,\bbC)$, $k>n$, we call $\afa$ positive)
%Then $\forall\fai\in H^k_{DR}(X,\bbC)$, exist unique decomposition
%$$\fai=\sum_{\gma\geq(k-n)_+}L^\gma\fai_\gma$$
%where $\fai_\gma\in H_{prim}^{k-2\gma}(X,\bbC)$.Similarly,
%$$H^{p,q}(X,\bbC)=\bigoplus_{r\geq(p+q-n)_+} H_{prim}^{p-r,q-r}(X,\bbC)$$

\begin{thm}(Lefschitz分解定理)

设$X$为紧\Kahler 流形,则成立
$$
  \mcalH^{p,q}(X,\bbC)
=
  \bigoplus_{r\geq(p+q-n)_+}
  L^r\mcalH^{p-r,q-r}_{\prim}(X,\bbC)
$$
\end{thm}
这里的$(p+q-n)_+:=\max\{p+q-n,0\}$.

\begin{proof}
对于任意$u\in\Omg^{p,q}(X,\bbC)$,
由之前结果(定理\ref{LD-局部-thm})易知存在唯一的分解
$$
  u=\sum_{r\geq(p+q-n)_+}L^r\fai_r
$$
其中$\fai_r\in\Omg^{p-r,q-r}_{\prim}(X,\bbC)$
为本原形式。注意到$[\yc,L]=0$,以及Lefshcitz分解的唯一性,易知
$\yc u=0$当且仅当$\yc\fai_r=0$对每个$r$都成立。从而证毕。
\end{proof}

%$X$ compact Kahler, $\dim_\bbC X=n$, $\omg$ is Kahler metric, define
%$$Q(\afa,\beta)=L^{n-k}\afa\wedge\overline{\beta}$$
%where $\afa,\beta\in H^{p,q}(X,\bbC)$, and $p+q=k$.
%Then $Q|_{H_{prim}^{p,q}}$ is positive defined (up to a factor).

\begin{thm}(Hodge-Riemann双线性关系)

设$X$为紧\Kahler 流形,若$p+q\leq n$,
则$\mcalH_{\prim}^{p,q}(X,\bbC)$上的双线性型
$$
  Q(\afa,\beta)
:=
  \lmd_{p,q}\int_XL^{n-p-q}\afa\wedge\betabar
$$
是严格正定的。其中$\lmd_{p,q}$为某常数。
\end{thm}

\begin{proof}
只需注意定理\ref{HRR-局部-thm}.
\end{proof}

%要求:对这三个定理非常熟悉!!!\begin{prop}$X$ is a compact Kahler, then
%$$\overline{H^{p,q}(X,\bbC)}=H^{q,p}(X,\bbC)$$
%\end{prop}\begin{proof}Use harmonic form.. 
%and $\yc_\td$ is a real operator...\end{proof}
%\textbf{Summary} $X$-compact Kahler with a Kahler metric $\omg$, then
%define Lefschitz operator $L=[\omg]\wedge$, then:Hodge decomposition:
%$$H^k=\bigoplus_{p+q=k}H^{p,q}$$$$\overline{H^{p,q}}=H^{q,p}$$
%Hard Lefschitz:$$L^{n-k}:H^{p,q}\cong H^{p+n-k,q+n-k}$$where $p+q=k$
%Lefschitz decomposition:$$H^{p,q}=\bigoplus_{r\geq(p+q-1)_+} L^r H^{p-r,q-r}_{prim}$$HRR:...
%\textbf{References}Kahler pairing in other settings..
%\verb"Adiprusito-Huh-Katz: Hodge theory in combinatorial geometries"
%\verb"McMullen: On simple polytopes"\verb"Deligne: Weil II"
%\verb"Beillinson-Bernstein-Deligne-Gabber: Faisceaux Pervers"
%\verb"Adiprasito: Combinatorial Lefschetz theorem beyond positivity, 2018"
%%%%%%%%%%2019.5.7%%%%%%%%%%%%%%%%%%%%%%%%%%%%%
%Recall: Kahler pairing:$X$-compact Kahler manifold of complex 
%dimension $n$, $\omg$-Kahler metric.Lefischitz operator
%$$L=\{\omg\}\curvearrowright H\updot$$Hodge decomposition
%$$H^k=\bigoplus_{p+q=k}H^{p,q},\qquad \overline{H^{p,q}}=H^{q,p}$$

\begin{notation}(Betti数与Hodge数)

对于紧\Kahler 流形$X$,记其Dolbeault上同调、de Rham上同调的维数
\begin{eqnarray*}
  h^{p,q}&:=&\dim_{\bbC}H^{p,q}_{\td''}(X,\bbC)\\
  b^{k}&:=&\dim_{\bbC}H^{k}_{\DR}(X,\bbC)
\end{eqnarray*}
分别称为$X$的\textbf{Hodge 数}、\textbf{Betti 数}.
\end{notation}
则对于紧\Kahler 流形,我们有以下关系:
$$
  \left\{
    \begin{array}{lcl}
      h^{p,q}=h^{q,p}     && \text{共轭}           \\
      h^{p,q}=h^{n-q,n-p} && \text{Serre对偶}      \\
      \,\,\,\,
       b^k=\sum\limits_{p+q=k}h^{p,q}&&\text{Hodge 分解}
    \end{array}
  \right.
$$
特别地,\textbf{当$k$为奇数时,$b^k$为偶数},
这是紧Kahler流形的又一拓扑限制。

若再令$h^{p,q}_{\prim}:=\dim_{\bbC}\mcalH^{p,q}_{\prim}(X,\bbC)$,
则Lefschitz分解定理与Serre对偶定理表明
$$
  h^{p,q}
=
  \left\{
    \begin{array}{ll}
      h^{p,q}_{\prim}
     +h^{p-1,q-1}_{\prim}
     +h^{p-2,q-2}_{\prim}+\cdots
    &p+q\leq n
    \\
      h^{n-q,n-p}_{\prim}
     +h^{n-q-1,n-p-1}_{\prim}
     +h^{n-q-2,n-p-2}_{\prim}+\cdots
    &p+q>n
    \end{array}
  \right.
$$
由此得到Betti数与Hodge数满足不等式
$$
  \left\{
    \begin{array}{ll}
      \text{如果$p+q\leq n$,}
     &h^{p,q}\geq h^{p-1,q-1},b^k\geq b^{k-2}\\
       \text{如果$p+q\geq n$,}
     &h^{p,q}\geq h^{p+1,q+1},b^k\geq b^{k+2}
    \end{array}
  \right.
$$

%(Corollary: if $k$ is odd, then $b_k:=\dim_{\bbC}H^k(X,\bbC)$ is even.)
%Rmk: if $X$ is compact complex surface($\dim_{\bbC}=2$),$X$ is Kahler $\iff b_1$ is even.
%(The proof of "$\Leftarrow$" we not given...Ref: Kodaira\&Siu,Lamari 1999)

\textbf{Exercise}:Consider $X$-compact Kahler,$\dim_\bbC X=n$, $\omg$-Kahler metric,
Then $\forall\afa,\beta\in H^{1,1}(X,\bbR)=H^{1,1}(X,\bbC)\cap H^2(X,\bbR)$,
Then
$$
  \left(
    \{\omg^{n-2}\}\cdot\afa\cdot\beta
  \right)^2
\geq
  \left(
    \{\omg^{n-2}\}\cdot\afa^2
  \right)
  \left(
    \{\omg^{n-2}\}\cdot\beta^2
  \right)
$$
with equality if and only if $\afa=\lmd\beta$ for some $\lmd\in \bbR$

Eg: $\bbC^2$, $\afa,\beta$ real $(1,1)$-forms,
$$(\afa,\beta)^2\geq \afa^2\beta^2$$

Hint: Using HRR, and Lefschitz decomposition...
"Alg-Geom-inequality over Kahler manifold".

%Hard Lef. ($p+q=k$)$$L^{n-k}:H^{p,q}\xra{\sim}H^{p+n-k,q+n-k}$$
%Lef. decomposition:$$H^{p,q}=\bigoplus_{r\geq (k-n)_+} L^rH_{prim}^{p-r,q-r}$$
%Denote $h^{p,q}:=\dim_{\bbC}H^{p,q}$,"Hodge number".Cor:$$h^{p,q}=
%\left\{\begin{array}{lc}h_{prim}^{p,q}+h_{prim}^{p-1,q-1}+\cdots 
%& p+q\leq n\\h_{prim}^{n-q,n-p}+h_{prim}^{n-q-1,n-p-1}+\cdots & p+q\geq n
%\end{array}\right.$$(Using the property of $L^r$)If $p+q\leq n$, $h^{p,q}
%\geq h^{p-1,q-1}\Rightarrowb_k\geq b_{k-2}$ if $k\leq n$.
%If $p+q\geq n$, $h^{p,q}\leq h^{p-1,q-1}\Rightarrow b_k\leq b_{k-2}$if $k\geq n$.

\section{Hodge-Frolicher谱序列}
%spectral sequence
$X$-compact Kahler, then Hodge decomposition
$$\Rightarrow b_k=\sum_{p+q=k}h^{p,q}$$

Question: $X$ compact complex manifold, relation between
$b_k$ and $\sum\limits_{p+q=k}h^{p,q}$?

\begin{thm}(Hodge-Frolicher inequality)
$X$ compact complex manifold, then
$$b_k\leq\sum_{p+q=k} h^{p,q}$$
\end{thm}

Spectral sequence:
$(K^{p,q},\td=\td'+\td'')$ a double complex of modules.
$$K^{p,q}\xra{\td'}K^{p+1,q}\quad
K^{p,q}\xra{\td''}K^{p,q+1}$$
with $\td'^2=0,\td''^2=0,\td^2=0$.

Assume $K^{p,q}=0$ if $p\leq 0$ or $q\leq 0$.

$\rightsquigarrow$ total complex $(K\updot,\td)$where
$$K^l:=\bigoplus_{p+q=l}K^{p,q}$$
$\exists$ a natural filtration
$$F_pK^l:=\bigoplus_{l\geq i\geq p}K^{i,l-i}$$

$F$ induces a filtration on $H\updot(K\updot)$.
$$F_pH^l(K\updot)=\im(H^l(F_pK\updot)\to H^l(K\updot))
=\frac{F_p Z^l}{F_p B^l}$$
where $Z^l=\ker \td\curvearrowright K^l$ and $B^l=\im\td\curvearrowright K^{l-1}$

Denote $G_pH^l(K\updot)=F_pH^l/F_{p+1}H^l$.

\begin{thm}There exists a sequence
$$\{E_r,\td_r\}_{r\geq 0}$$
satisfying:

(1) $E_r=\bigoplus\limits_{p,q\geq 0}E_r^{p,q}$

(2)$\td_r:E_r^{p,q}\to E_r^{p+r,q+r-1}$, $\td_r^2=0$.

(3) $E_{r+1}=H\updot((E_r,\td_r))$.

\end{thm}

$$E_0^{p,q}=\frac{F_p K^{p+q}}{F_{p+1}K^{p+q}}=K^{p,q}$$
$\td_0$ induced by $\td$.
$$E_1^{p,q}=H^q((K^{p,\bullet},\td''))$$
$\td_1$ induced by $\td$.

查任何一本同调代数的书。

\begin{definition}
We call the sequence ${E_r}$ converges at $E_{r_0}$,
if $E_{r+1}=E_r$ for any $r\geq r_0$,
($\iff\td_r=0$for any $r\geq r_0$)
then we denote $E_\infty=E_{r_0}$
\end{definition}

In our setting, $E_\infty^{p,q}=G_pH^{p+q}(K\updot)$

\textbf{Application:}$X$ compact complex manifold,
$$K^{p,q}=C^\infty(X,\wedgeform{p,q})\quad \td=\td'+\td''$$
$\rightsquigarrow E_0^{p,q}=K^{p,q}$,
$E_1^{p,q}=H^{p,q}(X,\bbC)$.

\begin{cor}
$$E^{p,q}_\infty=G_pH^{p+q}(X,\bbC)$$
\end{cor}

\begin{thm}
$X$ is a compact complex manifold of complex dimension $n$,then
$$b_l=\dim_{\bbC}H^l(X,\bbC)=\sum_{p+q=l}\dim_{\bbC}E_\infty^{p,q}\leq
\sum_{p+q=l}\dim_{\bbC}E_1^{p,q}=\sum_{p+q=l}h^{p,q}$$
with equality holds if and only if $\td_1=0$
(i.e $\{E_r\}$ converges at $E_1$.)
\end{thm}

\begin{thm} $X$ compact Kahler $\Rightarrow\{E_r\}$ converges at $E_1$
($\iff b_l=\sum\limits_{p+q=l}h^{p,q}$)
\end{thm}

Remark: algebraic proof by Deligne-Illusive 1987.

Rel\`{e}vement module $p^2$ et d\'{e}composition du complexe de de Rham

remark: Assume $X$ is bimeromorphic to a compact Kahler manifold,
then we still have the convergence of $\{E_r\}$
($\iff$ Hodge decomposition)

(Deligne-Griffiths-Morgan)

\textbf{Picard group $H^1(X,\mcalO^*)$}.

Recall:
$$\{\text{isomorphic class of holomorphic line bundle}\}
\xra{1-1} H^1(X,\mcalO^*)$$
Consider the sequence
$$0\to\bbZ\to\mcalO\xra{e^{2\pi\sqrt{-1}}}\mcalO^*\to 0$$
$$
  \rightsquigarrow
  0\to H^0(X,\bbZ)\to H^0(X,\mcalO)\to H^0(X,\mcalO^*)
  \to H^1(X,\bbZ)\to H^1(X,\mcalO)\to H^1(X,\mcalO^*)\to\cdots
$$

Assume $X$ is a compact complex manifold, then
$$H^0(X,\mcalO)=\bbC$$
$$H^0(X,\mcalO^*)=\bbC^*$$
$\Rightarrow H^0(X,\mcalO)\to H^0(X,\mcalO^*)$ is surjective,

$\Rightarrow H^1(X,\bbZ)\to H^1(X,\mcalO)$ is injective.

So we have an exact sequence
$$
  0\to H^1(X,\bbZ)\to H^1(X,\mcalO)
  \to H^1(X,\mcalO^*)\xra{c_1}H^2(X,\bbZ)
$$
so we have an isomorphism
$$\ker\{c_1:H^1(X,\mcalO^*)\to H^2(X,\bbZ)\}\cong H^1(X,\mcalO)/H^1(X,\bbZ)$$

\begin{definition}(Irregularity of $X$)
$$q(X)=\dim_{\bbC}H^1(X,\mcalO)
=h^{0,1}$$
if $X$ is also complex Kahler, then $h^{0,1}=h^{1,0}$.
\end{definition}

Assume $X$ is compact Kahler:

\begin{lemma}
$H^1(X,\bbZ)$ is also a lattice in $H^1(X,\mcalO)$of
$$rank_{\bbZ}H^1(X,\bbZ)=2q$$
$\Rightarrow H^1(X,\mcalO)/H^1(X,\bbZ)$ is a
compact torus of $\dim_{\bbC}=q$.
\end{lemma}
$$H^1(C,\mcalO)/H^1(X,\bbZ):=
\ker\{c_1:H^1(X,\mcalO^*)to H^2(X,\bbZ)\}$$
is called \textbf{Jacobian variety}($Jac(X)$) or \textbf{Picard variety}
($Pic^\circ(X)$)

Denote $NS(X)_{\bbZ}=\im(c_1:H^1(X,\mcalO^*)to H^2(X,\bbZ))$
the Neron-Severi group of $X$,
$$
  \rightsquigarrow\quad
  0\to Pic^\circ(X)\to H^1(X,\mcalO^*)\xra{c_1}NS(X,\bbZ)\to 0
$$

\begin{proof}[proof of the lemma]

$\bbZ\to\mcalO$ can be decomposed : $\bbZ\to\bbR\to\bbC\to\mcalO$.
It induces a sequence
$$H^1(X,\bbZ)\to H^1(X,\bbR)\to H^1(X,\bbC)\to H^1(X,\mcalO)$$

$H^1(X,\bbR)\to H^1(X,\mcalO)$ is an isomorphism.

Consider the diagram
%%%%%%biubiu%%%%%%%%%%%%%%

then $H^1(X,\bbR)\to H^1(X,\mcalO)$ corresponds to
$$
  H^1_{DR}(X,\bbR)\inj H^1_{DR}(X,\bbC)\surj H^{0,1}(X,\bbC)
$$

$H^1(X,\bbZ)$ is a lattice in $H^1(X,\bbR)$ of $rank_{\bbZ}=2q$
\end{proof}

\textbf{Albanese map, Albanese torus}

$X$-compact Kahler $\Rightarrow$ any holomorphic $p$-forms are $\td$-closed.

(Exercise!!)

Special case: holo 1-forms is $\td$-closed.

$$Alb(X):=H^0(X,\Omg^1)^*/\im(H_1(X,\bbZ))$$
where $H^1(X,\bbZ)$ is mapped to
$H^0(X,\Omg^1)^*$ in the following way:
$$[\gma]\mapsto(\afa\in H^0(X,\Omg^1)\mapsto\int_\gma\afa)$$

(Fact: $\int_\gma\afa$ depends only on the class on $[\gma]$)

Then $Alb(X)$ is compact complex of $\dim_\bbC=q(X)$.
More precisely, we have a map:
$$alb: X\to Alb(X)$$
Fix a base point $x_0\in X$, then
$$alb(x)=
\left(
  u\mapsto
  \int_{x_0}^x u
\right)\mod\Lmd
$$
where
$$\Lmd:=\Bigset{(\int_\gma u_1,...,\int_\gma u_q)}
{[\gma]\in H_1(X,\bbZ)}$$
$\{u_1,...,u_q\}$ is a basis of $H^0(X,\Omg^1)$.
Then $\Lmd$ is a lattice of $rank_\bbZ=2q$.

The map
$$alb: X\to Alb(X)$$
is holomorphic.

%%%%%%%%%%期末考试:最后一堂课随堂考%%%%%%%%%%%%%5

%%%%%%%%%%%2019.5.9%%%%%%%%%%%%%%%%%%%%%%%%%%%%%%%%%%





