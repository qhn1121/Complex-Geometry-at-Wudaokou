\chapter{Lefschitz分解}

\section{线性代数版本的Lefschitz算子}

Three goals:

\textbf{Kahler package}

\textbf{Lefschetz decomposition}

\textbf{Hodge-Riemann bilinear relations}

Linear algebra(baby representation theory)(local case)

$\bbC^n$,
$$\omg=\sqrt{-1}\sum_{i,j}h_{ij}\td z_i\wedge\td\zbar_j$$
Kahler metric with constant coefficients.(i.e.
$h_{ij}$ is constant, $(h_{ij})$ is positive Hermite matrix)

W.L.O.G, by taking a linear transformation, we can assume
$$\omg=\sqrt{-1}\sum_{j=1}^n\td z_j\wedge\td\zbar_j$$

\begin{notation}
An operator is of pure degree $r$ if it transform a form of $\deg=k$
to as form of degree $k+r$.

An operator ..of bi-degree $(p,q)$ if ...$(s,t)\to (s+p,t+q)$
(in this case, degree $=p+q$)

if $A,B$ with degree $\deg A,\deg B$,define
$$[A,B]:=AB-(-1)^{\deg A\deg B}BA$$
\end{notation}

\begin{definition}
$$L:\wedgeform{p,q}\to\wedgeform{p+1,q+1}$$
$$u\mapsto \omg\wedge u$$
is called Lefschetz operator.

Denote $\Lmd$ to be the adjoint of $L$, adjointed by :
Let $v\in\wedgeform{p-1,q-1}$ and $u\in\wedgeform{p,q}$
$$\langle Lv,u\rangle:=\langle u,\Lmd u\rangle$$
\end{definition}
The operator $\Lmd$ is of bi-degree $(-1,-1)$.

\begin{prop}
If
$$u=\sum_{|I|=p\atop |J|=q}u_{IJ}\td z_I\wedge\td\zbar_j$$
then
$$Lu=\sqrt{-1}\sum_{|I|=p\atop |J|=q}
 \sum_{m=1}^n
 u_{IJ}\td z_m\wedge\td\zbar_m\wedge\td z_I\wedge\td\zbar_J$$

$$
  \Lmd u
= \sqrt{-1}(-1)^p
  \sum_{|I|=p\atop|J|=q}
    \sum_{m=1}^n
      u_{IJ}
      \left(
        \pp{z_m}\suobing\td z_I
      \right)\wedge
      \left(
        \pp{\zbar_m}\suobing\td\zbar_J
      \right)
$$
where "$\suobing$" is contraction.
\end{prop}

\begin{cor}(Exercise)
Let
$$
  \afa=\sqrt{-1}\sum_{j=1}^n
       \afa_j\td z_j\wedge\zbar_j
$$
then,($\afa$ is a operator of bi-degree $(1,1)$)
$$
  [\afa,\Lmd]u
=
  \sum_{|I|=p\atop |J|=q}
    \left(
      \sum_{i\in I}
        \afa_i
     +\sum_{j\in J}
        \afa_j
     -\sum_{k=1}^n
        \afa_k
    \right)
    u_{IJ}\td z_I\wedge\td\zbar_J
$$
where
$$u=\sum_{|I|=p\atop |J|=q}
u_{IJ}\td z_I\wedge\td\zbar_J$$
\end{cor}

\begin{cor}
if $u\in\wedgeform{p,q}$,then
$$[L,\Lmd]u=(p+q-n)u$$
\end{cor}

\begin{cor}
Denote $B:=[L,\lmd]$, then
$$[B,L]=2L$$
$$[B,\Lmd]=-2\Lmd$$
\end{cor}

\begin{proof}
Take $u\in\wedgeform{p,q}$, then
$$[B,L]=BLu-LBu=(p+q-n+2)Lu-(p+q-n)Lu=2Lu$$
the second is similar..
\end{proof}


\textbf{$\mfksl(2,\bbC)$-representation}
$$\mfksl(2,\bbC)=\Span_\bbC{l,\lmd,b}$$
where
$$
  l=\begin{pmatrix}
      0 & 0\\
      1 & 0
    \end{pmatrix}
,\qquad
  \lmd=\begin{pmatrix}
      0 & 1\\
      0 & 0
    \end{pmatrix}
,\qquad
  b=\begin{pmatrix}
      1 & 0\\
      0 & -1
    \end{pmatrix}
$$
we have
$$[l,\lmd]=b\quad
  [b,l]=2l\quad
  [b,\lmd]=-2\lmd
$$

\begin{prop}
There exists a natural action
$$\rho:\mfksl(2,\bbC)\to\End(\bigoplus_{p,q}\wedgeform{p,q})$$
with
$$\rho(l)=L$$
$$\rho(\lmd)=\Lmd$$
$$\rho(b)=B$$
\end{prop}

\begin{thm}(HL)
$$L^{n-k}:\wedgeform{k}\to\wedgeform{2n-k}$$
$$u\to\omg^{n-k}\wedge u$$
is an isomorphism.

$$L^{n-k}:\wedgeform{p,q}\to \wedgeform{n-k+p,n-k+q}$$
is also an isomorphism.
\end{thm}

\begin{proof}

Lemma:
$$[L^r,\Lmd]u=r(k-n+r-1)L^{r-1}u$$
(induction, omit)

Assume $\afa\in\wedgeform{k}_\bbC$, $L^{n-k}\afa=0$,
need to verify $\afa=0$.

Claim:
$$L^r:\wedgeform{k}\to\wedgeform{k+2r}$$
is injective whenever $r\leq n-k$.

proof of the claim:

claim is true when $k=0$ or $k=1$.(check)

Let $\afa\in \wedgeform{k}$ s.t. $L^r\afa=0(r\leq n-k)$.
By the lemma,
$$L^r\Lmd\afa-\lmd L^r\afa=r(k-n+r-1)L^{r-1}\afa$$
so,
$$L^{r-1}(L\Lmd\afa-r(k-n+r-1)\afa)=0$$
by the induction on $r$,
$$L\Lmd\afa=r(k-n+r-1)\afa$$
since $r(k-n+r-1)\neq 0$, $\afa=L\beta$ for some $\beta\in\wedgeform{k-2}$.
so,$L^r\afa=L^{r+1}\beta=0$, by induction on $k$, we have
$\beta=0$,so $\afa=0$.

The claim is proved.
\end{proof}

\begin{definition}(Primitive form)

$\afa\in\wedgeform{k}(k\leq n)$ is called primitive form, if
$$L^{n-k+1}\afa=0$$
\end{definition}

\begin{cor}(Lefischtz Decomposition)(LD)

For any $\afa\in\wedgeform{k},(1\leq k\leq 2n)$,
we have a unique decomposition:
$$
  \afa
=
  \sum_{\gma\geq(k-n)_+}
    L^\gma\afa_r
$$
($(k-n)_+:=\max\{k-n,0\}$)
with $\afa_r\in\wedgeform{k-2r}$ is primitive
\end{cor}

\begin{proof}
Existence: assume $k\leq n$, consider
$$L^{n-k+1}\afa\in\wedgeform{2n-k+2}$$
by HL, $\exists !\,\beta\in\wedgeform{k-2}$ s.t.
$L^{n-k+2}\beta=L^{n-k+1}\afa$, so
$L^{n-k+1}(\afa-L\beta)=0$, i.e. $\afa_0=\afa-L\beta$
is primitive. $\afa=\afa_0+L\beta$, then induction on degrees,
we get the decomposition for $\afa$.

If $k>n$, we apply HL to reduce it to case 1.

Uniqueness: Next time..
\end{proof}

%%%%%%%%%%%%%%%2019.4.23第九周周二%%%%%%%%%%%%%%%%%%%%%%

Today: Continuous to Hard Lef decomposition, Hodge-Riemann bilinear relations.

Hard-Lefschitz: HL

Lefschitz decomposition :LD

Hodge-Riemann bilinear relations :HRR

Recall: $\bbC^n,\wedgeform{k}=\bigoplus\limits_{p+q=k}\wedgeform{p,q}$,
$\omg$: a Kahler metric on $\bbC^n$
with constant coefficient $\in\wedgeform{1,1}_\bbR$.

Lefschitz operator : $Lu=\omg\wedge u$.

\begin{thm}(HL)

Assume $k\leq n,p+q\leq n$, then
$$L^{n-k}:\wedgeform{k}\to \wedgeform{2n-k}$$
is a linear isomorphism.

$$L^{n-k}:\wedgeform{p,q}\to\wedgeform{p+n-k,q+n-k}$$
is also a linear isomorphism.
\end{thm}

Linear algebra..

\begin{thm}(LD)
for any $u\in\wedgeform{k}$,we have a unique decomposition
$$u=\sum_{r\geq(k-n)_+}L^ru_r$$
where $u_r\in\wedgeform{k-2r}_{prim}$ is a primitive form.
\end{thm}
Recall: a $k$-form $u\in\wedgeform{k}(k\leq n)$ is called primitive,
if $L^{n-k+1}(u)=0$. When $k>n$, $u$ is called primitive,
$\Lmd(u)=0$, where $\Lmd$ is the adjoint of $L$.

\begin{proof}

Existence: application of $HL$.

Uniqueness: Omit.
\end{proof}

\begin{prop} Assume $\afa\in\wedgeform{p,q}_{prim}$, and $p+q\leq n$.
(i.e. $L^{n-p-q+1}\afa=0$),then
$$
  *\afa
=
  (-1)^{\frac{(p+q)(p+q-1)}{2}}
  (\sqrt{-1})^{p-q}
  \frac{1}{(n-p-q)!}
  L^{n-p-q}\afa
$$
\end{prop}

\begin{proof}
  See [Humphreys, Prop 1.2.31]
\end{proof}

\begin{thm}(HRR)
Define the bilinear form $Q$ on $\wedgeform{k}\,\,(k\leq n)$ as follows:
$$Q(\afa,\beta):=L^{n-k}\wedge\afa\wedge\overline{\beta}$$
Then
$$(\sqrt{-1})^{p-q}(-1)^{\frac{(p+q)(p+q-1)}{2}}Q(u,u)\geq 0$$
for any $u\in \wedgeform{p,q}_{prim},p+q=k\leq n$,
and equal holds $$\iff u=0$$

(i.e. $Q|_{\wedgeform{p,q}_{prim}}$ is positive definite up to a factor)
\end{thm}

\begin{proof}
Take $u\in\wedgeform{p,q}_{prim}$,
$$Q(u,u)=L^{n-k}\wedge u\wedge\overline{u}
=*u\wedge\overline{u}=\langle\overline{u},\overline{u}\rangle\td Vol
=|u|^2\td Vol\geq 0$$
(up to a factor!)

(We apply the following result: $\overline{*\fai}=*\overline{\fai}$,
i.e. $*$ is a real operator)
\end{proof}

Summary: $\wedgeform{\bullet}=\bigoplus\limits_{1\leq k\leq n}\wedgeform{k}_\bbC$,
where$\wedgeform{k}_{\bbC}=\bigoplus\limits_{p+q=k}\wedgeform{p,q}_\bbC$.

Lefschitz operator $L\rightsquigarrow$ HL,LD,HRR.


\section{紧Kahler流形的上同调群}

The analogue of compact Kahler manifolds,

$$H^k_{DR}(X,\bbC)\cong\bigoplus_{p+q=k}H^{p,q}_{Dol}(X,\bbC)$$

$\omg$: A Kahler metric $\in H^{1,1}_{Dol}(X,\bbR)$.

Denote $L\curvearrowright H^k_{DR}(X,\bbC)$,
$$L(u)=[\omg,u]=[\omg]\wedge u$$

\textbf{Commutative relations on Kahler manifolds}

$$(\bbC^n,\omg=\sqrt{-1}\sum_{j=1}^n\td z_j\wedge\td\overline{z}_j)$$
$u\in C^\infty(\bbC^n,\wedgeform{p,q})$, locally
$$u=\sum_{|I|=p,|J|=q}u_{I,J}\td z_I\wedge \td z_j,\quad
v=\sum_{|I|=p,|J|=q}v_{I,J}\td z_I\wedge \td z_j
$$

$$\langle\langle u,v\rangle\rangle=
\int_{\bbC^n}\sum_{|I|=p,|J|=q}
u_{I,J}\overline{V_{I,J}}\td Vol$$

$\td=\td'+\td''$, $\td'=\p,\td''=\pbar$.

$$\td' u=\sum_{I,J}\sum_k\pfrac{u_{I,J}}{z_k}\td z_k\wedge\td z_I\wedge\td z_J$$
$$\td''u=\cdots$$

\begin{thm}
$$
  (\td'')^*u
=
  -\sum_{I,J}
    \sum_k
    \pfrac{u_{I,J}}{\zbar_k}
    \pp{\zbar_k}
    \suobing
    (\td z_I\wedge\td\zbar_J)
$$

$$
  (\td')^*u
=
  -\sum_{I,J}
    \sum_k
      \pfrac{u_{I,J}}{\zbar_k}
      \pp{z_k}
      \suobing
      (\td z_I\wedge\td\zbar_J)
$$
\end{thm}

\begin{prop}
$$[(\td'')^*,L]=\sqrt{-1}\td'$$
\end{prop}
\begin{proof}
Exercise.
\end{proof}

\begin{thm}
Let $X$ be a Kahler manifold (may not compact), with Kahler metric $\omg$,
then we have
$$[(\td'')^*,L]=\sqrt{-1}\td'$$
\end{thm}

\begin{proof}
Only need to verify $u\in  C^\infty_c(X,\wedgeform{p,q})$
with compact support in a holomorphic chart at $x$.

Assume the holomorphic chart near $x$ is choosen s.t.
$$\omg(z)=\sqrt{-1}\sum_{1\leq j\leq n}\td z_j\wedge\td\zbar_j+O(|z|^2)$$

$$u\in\sum_{I,J}u_{I,J}\td z_I\wedge\zbar_J$$
is a $(p,q)$-form, $v$ is also...

$$\langle u,q\rangle= u_{I,J}\overline{v_{M,N}}
\langle \td z_I,\td z_M\rangle\langle\td\zbar_J,\td\zbar_N\rangle
=u_{IJ}\overline{V_{ij}}+a_{IJMN}(z)u_{IJ}\overline{V_{MN}}$$
where $a_{IJMN}=O(|z|^2)$.

So,
$$(\td'')^*u=-\sum_{IJk}\pfrac{u_{IJ}}{z_k}\pp{\zbar_k}
\suobing(\td z_I\wedge\td\zbar_J)+
\sum_{IJMN}b_{IJMN}u_{IJ}\td z_M\wedge\td\zbar_N
$$
where $b_{IJMN}(z)=O(|z|)$. So,

$$[(\td'')^*,L]u(x)=
\sqrt{-1}\td'u(x)$$

$$\Longrightarrow [(\td'')^*,L]=\sqrt{-1}\td'$$
\end{proof}

\begin{prop}In Kahler manifold,
$$[(\td')^*,L]=-\sqrt{-1}\td''$$
$$[\Lmd,\td'']=-\sqrt{-1}(\td')^*$$
$$[\Lmd,\td']=\sqrt{-1}(\td'')^*$$
\end{prop}

\begin{cor}$(X,\omg)$ is a Kahler manifold, then
$$\yc_{\td}=2\yc_{\td'}=2\yc_{\td''}$$
\end{cor}
\begin{proof}
For example, $\yc_\td=2\yc_{\td''}$,
$$\yc_\td=(\td'+\td'')(\td'+\td'')^*+(\td'+\td'')^*(\td'+\td'')
=(\td'+\td'')(\td'^*-\sqrt{-1}[\Lmd,\td'])
+(\td'^*-\sqrt{-1}[\Lmd,\td'])(\td'+\td'')$$
然后暴力展开, 12项???$\cdots$

从略。
\end{proof}

\begin{cor}
If $(X,\omg)$ is a Kahler manifold, then
$$\yc_\td:C^\infty(C,\wedgeform{p,q})\to C^\infty(C,\wedgeform{p,q})$$
\end{cor}
\begin{proof}
Since $\yc_\td=2\yc_{\td'}$,
$\yc_{td'}$ preserves the bi-degree.
\end{proof}

\begin{cor}
If $(X,\omg)$ is a compact Kahler manifold,
$u$ is a $\yc_\td$-harmonic $k$-form. Assume
$$u=\sum_{p+q=k}u^{p,q}$$
$$u^{p,q}\in C^\infty(X,\wedgeform{p,q})$$
then each $u^{p,q}$ is also harmonic.
\end{cor}

\begin{thm}(Hodge decomposition)

$X$ is a compact Kahler manifold, then we have a decomposition
$$H_{\td}^k(X,\bbC)=\bigoplus_{p+q=k}H^{p,q}_{\td''}(X,\bbC)$$

Equivalently, (sheaf cohomology)
$$H^k(X,\bbC)\cong\bigoplus_{p+q=k}H^q(X,\Omg^p)$$
\end{thm}
\begin{proof}
take a Kahler metric $\omg$, we can define $\yc_d,\yc_{td'},\yc_{\td''}$,
then
$$\ker\yc_{\td}:=\mcalH^k(X,\bbC)
\cong\bigoplus\limits_{p+q=k}\mcalH^{p,q}_{\td''}(X,\bbC)$$

then $\Longrightarrow$ the decomposition for $H_\td^k(X,\bbC)$

the decomposition for $H^k_\td(X,\bbC)$ is independent of the choice of $\omg$
(Next time)
\end{proof}

%%%%%%%%%%%%%%%2019.4.25第九周周四%%%%%%%%%%%%%%%%%%%%%

Recall: Hodge decomposition,

$X$ compact Kahler manifold, $\dim_\bbC X=n$,

Thm:(Hodge decomposition)
$$H^k_{DR}(X,\bbC)=\bigoplus_{p+q=k}H^{p,q}(X,\bbC)
\cong\bigoplus_{p+q=n}H^{p,q}_{\td''}(X,\bbC)$$
where

$$H^{p,q}(X,\bbC)=\{[\afa]\in H_{DR}^k(X,\bbC)|\afa\text{is a $\td$-closed s.m. $(p,q)$-form}\}$$

Proof: take a Kahler metric $\omg$,
$$
H^k_{DR}(X,\bbC)
\cong
\mcalH_{\td}^k(X,\bbC)=\bigoplus\mcalH_\td^{p,q}(X,\bbC)
=\bigoplus\mcalH_{\td''}^{p,q}(X,\bbC)$$

\begin{prop}There is a canonical isomorphism
$$H^{p,q}_{\td}(X,\bbC)\xra{\sim}H^{p,q}_{\td''}(X,\bbC)$$
$$[\afa]_{\td}\mapsto[\afa]_{\td''}$$
where $\td\afa=0$,$\afa$ is a $(p,q)$-form.
$\Rightarrow \td''\afa=0$
\end{prop}
\begin{proof}
  Check: this map is well defined. Need to verify:
if $\afa=\td\beta$ is a $(p,q)$-form, then $[\afa]_{\td''}=0$,
i.e. $\afa$ is also $\td''$-exact.

$\afa$ is a $(p,q)$-form,
$$\Rightarrow\afa=\td'\beta^{p-1,q}+\td''\beta^{p,q-1}$$
we have $\td''\td'\beta^{p-1,q}=0$, $\td'\td''\beta^{p,q-1}=0$

We need a very important lemma:
\end{proof}

\begin{lemma}($\p\pbar$-lemma)

Let $X$ is a Kahler manifold, $\afa$ is a smooth form which is
$\td'$ and $\td''$ closed. Then, if $\afa$ is
$\td$ or $\td'$ or $\td''$-exact, then $\afa=\td'\td''\gma$
for some $\gma$.
\end{lemma}

Using $\p\pbar$-lemma, this map is well-defined.

Now, notice that the two space has the same dimension. So,
we need to show the map is injective(or, surjective).
Claim : this map is injective. If $\afa$ is a $\td$-closed
with $[\afa]_{\td''}=0$, i.e. $\afa=\td''\beta^{p,q-1}$.
$\afa$ is $\td$-closed$\Rightarrow\td'\td''\beta^{p,q-1}=0$,
$\p\pbar$-lemma applying to $\td''\beta^{p,q-1}$, we have
$$\td''\beta^{p,q-1}=\td'\td''\gma=\td(\td''\gma)$$
for some $\gma$.

\vs

Proof of $\p\pbar$-lemma:

\begin{proof}
Assume $\afa$ is $\td''$ exact, 1.e. $\afa=\td''\beta$,
write
$$\beta=H(\beta)+\yc_\td\gma$$
where $H(\beta)$ is $\yc_\td$-harmonic, so
$$\afa=\td'' H(\beta)+\td''\yc_\td\gma-2\td''\yc_{\td'}\gma$$
(Since $\yc_\td=2\yc_{\td''}$)
$$\Rightarrow\afa=2\td''(\td'\td'^*+\td'^*\td')
=2\td''\td;\td'^*\gma-2\td'^*\td''\td'\gma$$

By the assumption, $\td'\afa=0$, so
$\td'^*\td''\td'\gma=0$
$$\afa=-2\td'\td''\td'^*\gma$$
\end{proof}

\begin{rem}(Deligne-Griffiths-Morrora)

If $\widehat{X}$ is bimeromapic to $X$, where $X$ is
acompact Kahler, then$\hat{X}$ is also satisfys the $\p\pbar$-lemma.

$X$ is a kahler manifold, then
$$H^{p,q}_\td(X,\bbC)\cong H^{p,q}_{\td''}(C,\bbC)\cong
H^{p,q}——{X,\bbC}$$
\end{rem}

$X$ us a compact complex manifold,define
$$H_{BC}^{p,q}:=
\frac{\td\text{-closed $(p,q)$)}}{\td'\td;;\text{exact}}$$
Bott-Chern cohomology

Exercise" If $X$ is Kahler , then $H^{p,q}_{BC}=H^{p,q}_\td$

$$H^{p,q}_A(X,\bbC):=
\frac{\td'\td''\text{closed}}{(\td')\text{-exact}+\{\td''\text{exact}\}}$$
(Appeli cohomology)

denote
$$h_{BC}^k:=\sum_{p+q=k}\dim_{\bbC}H_{BC}^{p,q}$$
$$h_{A}^k:=\sum_{p+q=k}\dim_{\bbC}H_{A}^{p,q}$$

\begin{thm}
$X$ satisfies $\p\pbar$-lemma $\iff$
$$h_B^k+h_A^k=2b_k$$
where
$$b_k=\dim_\bbC H_{DR}^k(X,\bbC)$$
\end{thm}
%这是比较新的结果,仅供了解

\begin{thm}(Hard Lef)

$X$ is a compact Kahler, $\dim_\bbC X=n$, denote
$L=\{\omg\}\curvearrowright H_{DR}^k(X,\bbC)$, $\omg$ is a Kahler metric,
Then we have:
$$L^{n-k}: H_{DR}^k(X,\bbC)\cong H_{DR}^{2n-k}(X,\bbC)$$
$$H^{p,q}(X,\bbC)\cong H^{p+n-k,q+n-k}(X,\bbC)$$
where $k\leq n$, $p+q\leq n$.
\end{thm}

\begin{proof}Fox a Kahler metric $\omg$,
$$L^{n-k}:H_{DR}^k\to H_{DR}^{2n-k}$$
($\cong \mcalH_\td^k$, $\cong\mcalH_{\td}^{2n-k}$ respectively)
(there is a commutative diagram...)

need to proof: For any $\fai\in\mcalH_\td^k$, then
$$L^{n-k}(\fai)=\omg^{n-k}\wedge\fai$$
is also harmonic.
\end{proof}

\begin{lemma}
$$[\yc_{\td},L]=0$$
\end{lemma}

\begin{proof}
$$[\yc_\td,L]=2[\yc_{\td'},L]
=2\left([\td'\td'^*,L]+[\td'^*\td',L]\right)
=2\left(\td'[\td'*,L]+[\td'^*,L]\td'\right)$$
(check: $[L,\td']=0$)
So,
$$=-2\sqrt{-1}(\td'\td''+\td''\td')=0$$
\end{proof}

\textbf{Exercise}: Complex tori
$$\bbT^n:=\bbC^n/\Gma$$
where $\Gma=\bbZ^n$. $\bbT^n$ is a compact Kahler manifold.
Then
$$H^{1,1}(\bbT^n,\bbC)\cong\wedgeform{1,1}_{\bbC}$$
the space of $(1,1)$-forms on $\bbC^n$ with constant coefficient,
in particular,
$$\dim_\bbC H^{1,1}(\bbT^n,\bbC)=n^2$$

\textbf{Exercise}: the set of all the Kahler class on $\bbT^n
\subseteq H^{1,1}(X,\bbC)\cap H^2(X,\bbR)$ is equal to
the set of $n\times n$ positive definite Hermitian metrics.

(Hint: using Hodge theory)

\begin{thm}(Lefschitz decomposition)

Define a class $\afa\in H^{k}_{DR}(X,\bbC)$
to be positive if
$$L^{n-k+1}(\afa)=0$$
if $k\leq n$.

(When $\afa\in H^k_{DR}(X,\bbC)$, $k>n$, we call $\afa$ positive)

Then $\forall\fai\in H^k_{DR}(X,\bbC)$, exist unique decomposition
$$\fai=\sum_{\gma\geq(k-n)_+}
L^\gma\fai_\gma$$
where $\fai_\gma\in H_{prim}^{k-2\gma}(X,\bbC)$.

Similarly,
$$H^{p,q}(X,\bbC)=\bigoplus_{r\geq(p+q-n)_+} H_{prim}^{p-r,q-r}(X,\bbC)$$
\end{thm}

\begin{proof}
Exercise.
\end{proof}

\begin{thm}(HRR)

$X$ compact Kahler, $\dim_\bbC X=n$, $\omg$ is Kahler metric, define
$$Q(\afa,\beta)=L^{n-k}\afa\wedge\overline{\beta}$$
where $\afa,\beta\in H^{p,q}(X,\bbC)$, and $p+q=k$.

Then $Q|_{H_{prim}^{p,q}}$ is positive defined (up to a factor).
\end{thm}

\begin{proof}
Exercise.
\end{proof}
%要求:对这三个定理非常熟悉!!!

\textbf{Exercise}:Consider $X$-compact Kahler,$\dim_\bbC X=n$, $\omg$-Kahler metric,
Then $\forall\afa,\beta\in H^{1,1}(X,\bbR)=H^{1,1}(X,\bbC)\cap H^2(X,\bbR)$,
Then
$$
  \left(
    \{\omg^{n-2}\}\cdot\afa\cdot\beta
  \right)^2
\geq
  \left(
    \{\omg^{n-2}\}\cdot\afa^2
  \right)
  \left(
    \{\omg^{n-2}\}\cdot\beta^2
  \right)
$$
with equality if and only if $\afa=\lmd\beta$ for some $\lmd\in \bbR$

Eg: $\bbC^2$, $\afa,\beta$ real $(1,1)$-forms,
$$(\afa,\beta)^2\geq \afa^2\beta^2$$

Hint: Using HRR, and Lefschitz decomposition...
"Alg-Geom-inequality over Kahler manifold".

\begin{prop}
$X$ is a compact Kahler, then
$$\overline{H^{p,q}(X,\bbC)}=H^{q,p}(X,\bbC)$$
\end{prop}
\begin{proof}
Use harmonic form.. and $\yc_\td$ is a real operator...
\end{proof}

\textbf{Summary} $X$-compact Kahler with a Kahler metric $\omg$, then
define Lefschitz operator $L=[\omg]\wedge$, then:

Hodge decomposition:
$$H^k=\bigoplus_{p+q=k}H^{p,q}$$
$$\overline{H^{p,q}}=H^{q,p}$$

Hard Lefschitz:
$$L^{n-k}:H^{p,q}\cong H^{p+n-k,q+n-k}$$
where $p+q=k$

Lefschitz decomposition:
$$H^{p,q}=\bigoplus_{r\geq(p+q-1)_+} L^r H^{p-r,q-r}_{prim}$$

HRR:...

\textbf{References}
Kahler pairing in other settings..

\verb"Adiprusito-Huh-Katz: Hodge theory in combinatorial geometries"

\verb"McMullen: On simple polytopes"

\verb"Deligne: Weil II"

\verb"Beillinson-Bernstein-Deligne-Gabber: Faisceaux Pervers"

\verb"Adiprasito: Combinatorial Lefschetz theorem beyond positivity, 2018"

%%%%%%%%%%2019.5.7%%%%%%%%%%%%%%%%%%%%%%%%%%%%%


Recall: Kahler pairing:
$X$-compact Kahler manifold of complex dimension $n$, $\omg$-Kahler metric.
Lefischitz operator
$$L=\{\omg\}\curvearrowright H\updot$$

Hodge decomposition
$$H^k=\bigoplus_{p+q=k}H^{p,q},\qquad \overline{H^{p,q}}=H^{q,p}$$
(Corollary: if $k$ is odd, then $b_k:=\dim_{\bbC}H^k(X,\bbC)$ is even.)

Rmk: if $X$ is compact complex surface($\dim_{\bbC}=2$),$X$ is Kahler $\iff b_1$ is even.
(The proof of "$\Leftarrow$" we not given...Ref: Kodaira\&Siu,Lamari 1999)

Hard Lef. ($p+q=k$)
$$L^{n-k}:H^{p,q}\xra{\sim}H^{p+n-k,q+n-k}$$

Lef. decomposition:
$$H^{p,q}=\bigoplus_{r\geq (k-n)_+} L^rH_{prim}^{p-r,q-r}$$
Denote $h^{p,q}:=\dim_{\bbC}H^{p,q}$,"Hodge number".
Cor:
$$
  h^{p,q}=
  \left\{
    \begin{array}{lc}
      h_{prim}^{p,q}+h_{prim}^{p-1,q-1}+\cdots & p+q\leq n\\
      h_{prim}^{n-q,n-p}
       +h_{prim}^{n-q-1,n-p-1}+\cdots & p+q\geq n
    \end{array}
  \right.
$$
(Using the property of $L^r$)

If $p+q\leq n$, $h^{p,q}\geq h^{p-1,q-1}\Rightarrow
b_k\geq b_{k-2}$ if $k\leq n$.

If $p+q\geq n$, $h^{p,q}\leq h^{p-1,q-1}\Rightarrow b_k\leq b_{k-2}$
if $k\geq n$.

\textbf{(Hodge-Frolicher spectral sequence)}

$X$-compact Kahler, then Hodge decomposition
$$\Rightarrow b_k=\sum_{p+q=k}h^{p,q}$$

Question: $X$ compact complex manifold, relation between
$b_k$ and $\sum\limits_{p+q=k}h^{p,q}$?

\begin{thm}(Hodge-Frolicher inequality)
$X$ compact complex manifold, then
$$b_k\leq\sum_{p+q=k} h^{p,q}$$
\end{thm}

Spectral sequence:
$(K^{p,q},\td=\td'+\td'')$ a double complex of modules.
$$K^{p,q}\xra{\td'}K^{p+1,q}\quad
K^{p,q}\xra{\td''}K^{p,q+1}$$
with $\td'^2=0,\td''^2=0,\td^2=0$.

Assume $K^{p,q}=0$ if $p\leq 0$ or $q\leq 0$.

$\rightsquigarrow$ total complex $(K\updot,\td)$where
$$K^l:=\bigoplus_{p+q=l}K^{p,q}$$
$\exists$ a natural filtration
$$F_pK^l:=\bigoplus_{l\geq i\geq p}K^{i,l-i}$$

$F$ induces a filtration on $H\updot(K\updot)$.
$$F_pH^l(K\updot)=\im(H^l(F_pK\updot)\to H^l(K\updot))
=\frac{F_p Z^l}{F_p B^l}$$
where $Z^l=\ker \td\curvearrowright K^l$ and $B^l=\im\td\curvearrowright K^{l-1}$

Denote $G_pH^l(K\updot)=F_pH^l/F_{p+1}H^l$.

\begin{thm}There exists a sequence
$$\{E_r,\td_r\}_{r\geq 0}$$
satisfying:

(1) $E_r=\bigoplus\limits_{p,q\geq 0}E_r^{p,q}$

(2)$\td_r:E_r^{p,q}\to E_r^{p+r,q+r-1}$, $\td_r^2=0$.

(3) $E_{r+1}=H\updot((E_r,\td_r))$.

\end{thm}

$$E_0^{p,q}=\frac{F_p K^{p+q}}{F_{p+1}K^{p+q}}=K^{p,q}$$
$\td_0$ induced by $\td$.
$$E_1^{p,q}=H^q((K^{p,\bullet},\td''))$$
$\td_1$ induced by $\td$.

查任何一本同调代数的书。

\begin{definition}
We call the sequence ${E_r}$ converges at $E_{r_0}$,
if $E_{r+1}=E_r$ for any $r\geq r_0$,
($\iff\td_r=0$for any $r\geq r_0$)
then we denote $E_\infty=E_{r_0}$
\end{definition}

In our setting, $E_\infty^{p,q}=G_pH^{p+q}(K\updot)$

\textbf{Application:}$X$ compact complex manifold,
$$K^{p,q}=C^\infty(X,\wedgeform{p,q})\quad \td=\td'+\td''$$
$\rightsquigarrow E_0^{p,q}=K^{p,q}$,
$E_1^{p,q}=H^{p,q}(X,\bbC)$.

\begin{cor}
$$E^{p,q}_\infty=G_pH^{p+q}(X,\bbC)$$
\end{cor}

\begin{thm}
$X$ is a compact complex manifold of complex dimension $n$,then
$$b_l=\dim_{\bbC}H^l(X,\bbC)=\sum_{p+q=l}\dim_{\bbC}E_\infty^{p,q}\leq
\sum_{p+q=l}\dim_{\bbC}E_1^{p,q}=\sum_{p+q=l}h^{p,q}$$
with equality holds if and only if $\td_1=0$
(i.e $\{E_r\}$ converges at $E_1$.)
\end{thm}

\begin{thm} $X$ compact Kahler $\Rightarrow\{E_r\}$ converges at $E_1$
($\iff b_l=\sum\limits_{p+q=l}h^{p,q}$)
\end{thm}

Remark: algebraic proof by Deligne-Illusive 1987.

Rel\`{e}vement module $p^2$ et d\'{e}composition du complexe de de Rham

remark: Assume $X$ is bimeromorphic to a compact Kahler manifold,
then we still have the convergence of $\{E_r\}$
($\iff$ Hodge decomposition)

(Deligne-Griffiths-Morgan)

\textbf{Picard group $H^1(X,\mcalO^*)$}.

Recall:
$$\{\text{isomorphic class of holomorphic line bundle}\}
\xra{1-1} H^1(X,\mcalO^*)$$
Consider the sequence
$$0\to\bbZ\to\mcalO\xra{e^{2\pi\sqrt{-1}}}\mcalO^*\to 0$$
$$
  \rightsquigarrow
  0\to H^0(X,\bbZ)\to H^0(X,\mcalO)\to H^0(X,\mcalO^*)
  \to H^1(X,\bbZ)\to H^1(X,\mcalO)\to H^1(X,\mcalO^*)\to\cdots
$$

Assume $X$ is a compact complex manifold, then
$$H^0(X,\mcalO)=\bbC$$
$$H^0(X,\mcalO^*)=\bbC^*$$
$\Rightarrow H^0(X,\mcalO)\to H^0(X,\mcalO^*)$ is surjective,

$\Rightarrow H^1(X,\bbZ)\to H^1(X,\mcalO)$ is injective.

So we have an exact sequence
$$
  0\to H^1(X,\bbZ)\to H^1(X,\mcalO)
  \to H^1(X,\mcalO^*)\xra{c_1}H^2(X,\bbZ)
$$
so we have an isomorphism
$$\ker\{c_1:H^1(X,\mcalO^*)\to H^2(X,\bbZ)\}\cong H^1(X,\mcalO)/H^1(X,\bbZ)$$

\begin{definition}(Irregularity of $X$)
$$q(X)=\dim_{\bbC}H^1(X,\mcalO)
=h^{0,1}$$
if $X$ is also complex Kahler, then $h^{0,1}=h^{1,0}$.
\end{definition}

Assume $X$ is compact Kahler:

\begin{lemma}
$H^1(X,\bbZ)$ is also a lattice in $H^1(X,\mcalO)$of
$$rank_{\bbZ}H^1(X,\bbZ)=2q$$
$\Rightarrow H^1(X,\mcalO)/H^1(X,\bbZ)$ is a
compact torus of $\dim_{\bbC}=q$.
\end{lemma}
$$H^1(C,\mcalO)/H^1(X,\bbZ):=
\ker\{c_1:H^1(X,\mcalO^*)to H^2(X,\bbZ)\}$$
is called \textbf{Jacobian variety}($Jac(X)$) or \textbf{Picard variety}
($Pic^\circ(X)$)

Denote $NS(X)_{\bbZ}=\im(c_1:H^1(X,\mcalO^*)to H^2(X,\bbZ))$
the Neron-Severi group of $X$,
$$
  \rightsquigarrow\quad
  0\to Pic^\circ(X)\to H^1(X,\mcalO^*)\xra{c_1}NS(X,\bbZ)\to 0
$$

\begin{proof}[proof of the lemma]

$\bbZ\to\mcalO$ can be decomposed : $\bbZ\to\bbR\to\bbC\to\mcalO$.
It induces a sequence
$$H^1(X,\bbZ)\to H^1(X,\bbR)\to H^1(X,\bbC)\to H^1(X,\mcalO)$$

$H^1(X,\bbR)\to H^1(X,\mcalO)$ is an isomorphism.

Consider the diagram
%%%%%%biubiu%%%%%%%%%%%%%%

then $H^1(X,\bbR)\to H^1(X,\mcalO)$ corresponds to
$$
  H^1_{DR}(X,\bbR)\inj H^1_{DR}(X,\bbC)\surj H^{0,1}(X,\bbC)
$$

$H^1(X,\bbZ)$ is a lattice in $H^1(X,\bbR)$ of $rank_{\bbZ}=2q$
\end{proof}

\textbf{Albanese map, Albanese torus}

$X$-compact Kahler $\Rightarrow$ any holomorphic $p$-forms are $\td$-closed.

(Exercise!!)

Special case: holo 1-forms is $\td$-closed.

$$Alb(X):=H^0(X,\Omg^1)^*/\im(H_1(X,\bbZ))$$
where $H^1(X,\bbZ)$ is mapped to
$H^0(X,\Omg^1)^*$ in the following way:
$$[\gma]\mapsto(\afa\in H^0(X,\Omg^1)\mapsto\int_\gma\afa)$$

(Fact: $\int_\gma\afa$ depends only on the class on $[\gma]$)

Then $Alb(X)$ is compact complex of $\dim_\bbC=q(X)$.
More precisely, we have a map:
$$alb: X\to Alb(X)$$
Fix a base point $x_0\in X$, then
$$alb(x)=
\left(
  u\mapsto
  \int_{x_0}^x u
\right)\mod\Lmd
$$
where
$$\Lmd:=\Bigset{(\int_\gma u_1,...,\int_\gma u_q)}
{[\gma]\in H_1(X,\bbZ)}$$
$\{u_1,...,u_q\}$ is a basis of $H^0(X,\Omg^1)$.
Then $\Lmd$ is a lattice of $rank_\bbZ=2q$.

The map
$$alb: X\to Alb(X)$$
is holomorphic.

%%%%%%%%%%期末考试:最后一堂课随堂考%%%%%%%%%%%%%5

%%%%%%%%%%%2019.5.9%%%%%%%%%%%%%%%%%%%%%%%%%%%%%%%%%%

\section{正性与消灭定理}
positivity and vanishing theorem

$X$-Kahler manifold, i.e. $\exists$ Hermitian metric $\omg$ s.t. $\td\omg=0$,
$\td=\td'+\td''$, $\td'=\p,\td''=\pbar$.
$$\yc_\td=[\td,\td^*]=\td\td^*+\td^*\td$$
$$\yc_{\td'}=[\td',\td'^*]$$
$$\yc_{\td''}=[\td'',\td''^*]$$
$\td\curvearrowright C^{\infty}(X,\wedgeform{p,q})$.

Fact: $\omg$ is Kahler $\iff\yc_{\td'}=\yc_{\td''}=\frac{1}{2}\yc_\td$.

Let $\underline{\bbC}:=X\times\bbC$ be the trivial line bundle,
$\td$ can be regraded as the Chern connection on $\underline{\bbC}$.

$(E,h)$-Hermitian holomorphic vector bundle over $(X,\omg)$,
with Chern connection $D_E=D_E'+D_E''$. $(D_E''=\pbar)$.

$$C^\infty(X,\wedgeform{p,q}\ten E)$$
has an inner product induced by $\omg,h$.
$\rightsquigarrow$ adjoint operators $D_E^*=D_E'^*+D_E''^*$.

$\rightsquigarrow\yc_E=[D_E,D_E^*]=D_ED_E^*+D_E^*D_E$, and
$\yc_E'\,,\,\yc_E''$. (self adjoint, elliptic operators)

Question: relation between $\yc_E'$ and $\yc_E''$?

\begin{thm}(Bochner-Kodaira-Nakaino identity)
$$
  \yc_E''-\yc_E'
=
  \left[
    \sqrt{-1}
    \Theta_E
  ,
    \Lmd
  \right]
$$
where $\Theta_E$ is the Chern curvature of $D_E$.
\end{thm}

Recall: $\Theta_E=D_E^2$, when $D_E$ is Chern connectoin, we have
$$D_E'^2=0\qquad D_E''^2=0$$
i.e. $\Theta_E=[D_E',D_E'']$.

Remark: $E$ is flat(i.e. $D_E^2=0$)$\iff \yc_E'=\yc_E''$.

\begin{proof}
based on following identities:
$$[D_E''^*,L]=\sqrt{-1}D_E'$$
$$[D_E'^*,L]=-\sqrt{-1}D_E''$$
$$[\Lmd,D_E']=-\sqrt{-1}D_E'^*$$
$$[\Lmd,D_E'']=\sqrt{-1}D_E''^*$$

then (by super Jacobi identity):
\begin{eqnarray*}
  \yc_E''=[D_E'',D_E''^*]
&=&
  -\sqrt{-1}
  \left[
    D_E''
  ,
    [\Lmd,D_E']
  \right]
=
  -\sqrt{-1}
  \left(
    [\Lmd,[D_E',D_E'']]
   +[D_E',[D_E'',\Lmd]]
  \right)\\
&=&
  -\sqrt{-1}
  \left(
    [\Lmd,\Theta_E]
   +[D_E',\sqrt{-1}D_E'^*]
  \right)
\end{eqnarray*}
so,
$$\yc_E''-\yc_E'=[\sqrt{-1}\Theta_E,\Lmd]$$
\end{proof}

\begin{lemma}(normal frame)

Let $X$ be a complex manifold, then for any $x_0\in X$,
and any holomorphic chart $(z_1,...,z_n)$ centered at $x_0$,
there exists a holomorphic frame $\{e_\lmd\}_{\lmd=1}^{r:=rank E}$
of $E$ near $x_0$ such that
$$
  \left\langle
    e_{\lmd}(z),e_{\mu}(z)
  \right\rangle
=
  \delta_{\lmd,\mu}-
  \sum_{1\leq j,k \leq n}
  C_{jk\lmd\mu}
  z_j\zbar_k
+
 O(|z|^3)
$$
where $(C_{jk\lmd\mu})$ are the coefficients of the Chern curvature
$$\Theta_E(x_0)=
  \sum_{1\leq j,k\leq n\atop 1\leq\lmd,\mu\leq r}
    C_{jk\lmd\mu}
    \td z_j\wedge\td \zbar_{k}\ten e_\lmd^*\ten e_\mu
$$
\end{lemma}


need to verify: $\forall s\in C^\infty(X,\wedgeform{p,q}\ten E),x_0\in X$,
$$[D_E''^*,L]s(x_0)=\sqrt{-1}D_E's(x_0)$$
w.r.t the normal frame $(e_\lmd)_{\lmd=1}^r$ near $x_0$, assume
$$s=\sum_{\lmd=1}^{n}\sgm_\lmd\ten e_\lmd$$
then
$$
  D_Es(z)=\sum_{\lmd=1}^{n}
    \td\sgm_\lmd\ten e_\lmd +O(|z|)
$$
$$
  D_E^*s(z)=\sum_{\lmd=1}^{n}
    \td^*\sgm_\lmd\ten e_\lmd +O(|z|)
$$

$$D_E''^*=\sum_{\lmd=1}^{r} \td''^*\sgm_\lmd\ten e_\lmd+O(|z|)$$
$$
  \Rightarrow
  [D_E''^*,L]s=
  D_E''^*(\sum\omg\wedge\sgm_\lmd\ten e_\lmd)
 -\omg\wedge
 \left(
   \sum_{\lmd=1}^{r}
   \td''^*\sgm_\lmd\ten e_\lmd+O(|z|)
 \right)
=
  \sum_{\lmd=1}^{r}
    [\td''^*,L]\sgm_\lmd\ten e_\lmd+O(|z|)
$$
Similarly,
$$
  D_E's
=
  \sum_{\lmd=1}^{r}
    \td'\sgm_\lmd\ten e_\lmd+O(|z|)
$$
we have:
$$
  [d''^*,L]=\sqrt{-1}\td'
$$
(because $\omg$ is Kahler)

...

$(E,h)$ hermitian holomorphic vector bundle over Kahler manifold $(X,\omg)$.
we have BKN identity
$$
  \yc_E''-\yc_E'=[\sqrt{-1}\Theta_E,\Lmd]
$$

Recall: $L^2$-Hodge theory. $X$ compact manifold,then
$$
  H^{p,q}(X,E):=
  \frac{\ker D_E''}{\im D_E''}
\cong
  \ker\yc_E''
$$
(harmonic form)

Take $u\in C^\infty(X,\wedgeform(p,q)\ten E)$,applying BKN identity to $u$,
$$\yc_E''u-\yc_E'u=[\sqrt{-1}\Theta_E,\Lmd]u$$

note that
$$\ppair{\yc_E'u}{u}=\norm{D_E'u}^2+\norm{D_E'^*u}^2\geq 0$$
$$
  \Rightarrow
  \norm{D_E''u}^2+\norm{D_E''^*u}^2
\geq
  \ppair{[\sqrt{-1}\Theta_E,\Lmd]}{u}
$$
i.e.
$$
  \norm{D_E''u}^2+\norm{D_E''^*u}^2
\geq
  \int_X\pair{[\sqrt{-1}\Theta_E,\Lmd]}{u}
  \td Vol
$$

Observation: if $u\in\ker\yc_E''$, and $[\sqrt{-1}\Theta_E,\Lmd]$
has "positivity",
then $LHS=0$. So, $H^{p,q}(X,E)=0$.

\begin{definition}(Positivity)

We call $[\sqrt{-1}\Theta_E,\Lmd]$ is positive at $x_0\in X$, if
for any $0\neq v\in\left(\wedgeform{p,q}\ten E\right)_{x_0}$, we have
$$\pair{[\sqrt{-1}\Theta_E,\Lmd]v}{v}>0$$

....positive on $X$, if ... at each point
\end{definition}

\begin{thm}
If $[\sqrt{-1}\Theta_E,\Lmd]$ is positive on $X$, then
$$H^{p,q}(X,E)=0$$
\end{thm}

Special case: $E$ is a holomorphic line bundle, with Hermitian metric $h$,
$$\Theta_E=-\td'\td''\log h$$
$\Rightarrow\sqrt{-1}\Theta_E$ is a real $\td$-closed $(1,1)$-form on $X$.

locally,
$$
     \afa=
\sqrt{-1}\sum_{1\leq i,j\leq n}
  a_{ij}\td z_i\wedge\td\zbar_j
$$

$\afa$ is real $\iff$ $\afa=\overline{\afa}$,
(i.e. locally $(a_{ij})$ is an hermitian matrix)

\begin{definition}
a real $(1,1)$-form $\afa$ is called positive, if
$(a_{ij})_{ij}$ is positive definite.
\end{definition}

\begin{lemma}
If $\sqrt{-1}\Theta_E$ is positive, then $\omg:=\sqrt{-1}\Theta_E$ gives a
Kahler metric on $X$.
\end{lemma}

\begin{lemma}
If $\omg=\sqrt{-1}\Theta_E>0$, and $\Lmd$ is the adjoint of $L=\omg\wedge$,
then
$$[\sqrt{-1}\Theta_E,\Lmd]$$
is positive on $\wedgeform{p,q}\ten E$
whenever $p+q\geq n+1$.
\end{lemma}

\begin{lemma}
Let $\afa$ be a real $(1,1)$-form, $\omg$ a Kahler metric, assume
the eigenvalue of $\afa$ at $x_0$ is $\afa_1\leq\afa_2\leq\cdots\leq\afa_n$,
then (in the coordinate chart $(z_1,z_2,...,z_n)$, and
$u=\sum\limits_{|I|=p\atop|J|=q}u_{IJ}\td z_I\wedge\td\zbar_J$)
$$
  [\afa,L]
=
  \sum_{I,J}
    \left(
      \sum_{i\in I}\afa_i
    + \sum_{j\in J}\afa_j
    - \sum_{k=1}^n\afa_k
    \right)
    u_{IJ}\td z_I\wedge\td\zbar_J
$$
\end{lemma}

\begin{cor}
$
  \afa=\omg
$, then
$$[\omg,\Lmd]u=(p+q-n)u$$
\end{cor}

\begin{cor}
Take an orthonormal frame $e$ of $E$, then
for any $u=\sum \limits_{|I|=p\atop|J|=q}u_{IJ}\td z_I\wedge\td\zbar_J\ten e$, we have
$$
  \pair{
  [\sqrt{-1}\Theta_E,\Lmd
  ]u
  }{u}
=(p+q-n)|u|^2
$$
\end{cor}

\begin{thm}
If $[\sqrt{-1}\Theta_E,\Lmd]$ is positive on $X$,
then
$$H^{p,q}(X,E)=0$$
\end{thm}

\begin{thm}If $E$ is a holomorphic line bundle
with a smooth hermitian metric $h$ s.t. $\sqrt{-1}\Theta_{(E,h)}\geq 0$,
then$H^{p,q}(X,E)=0$ whenever $p+q\geq n+1$.
\end{thm}
de Rham-Weil...$\cong H^q(X,\Omg^p\ten E)$.

\begin{definition}(canonical bundle)
$$K_X=\det T^*X$$
determinate bundle of cotangent bundle,
is called canonical bundle.
($\mcalO(K_X)=\Omg_X^n$)
\end{definition}


\begin{definition}
$X$ is called Fano, if $K_X^*=\det(TX)$ has a matric with positive curvature.

$X$ is called Calabi-Yau,if $K_X$ has a metric with vanishing curvature.

$X$ is of general type, if $K_X$ has a metric with positive curvature.
\end{definition}


\begin{cor}(Kodaira vanishing theorem)
$E$ is a positive line bundle, then
$$H^q(X,K_X\ten E)=0$$
for any $q\geq 1$.
\end{cor}

So, if $X$ is Fano, $(\iff K_X^*)$ positive,
$K_X\ten K_X^*=\underline{\bbC}$,
$\Rightarrow H^1(X,\mcalO)=0,\Rightarrow H^1(X,\bbR)=0$,

%%%%%%%%%%%%%%%%2019.5.14%%%%%%%%%%%%%%%%%%%%%%%%%5

Recall: BKN-inequality.

holomorphic Hermitian vector bundle $(E,h)\to (X,\omg)$, $\omg$ is Kahler.
For any $u\in C^\infty(X,\wedgeform{p,q}\ten E)$, we have
$$
  \norm{D''u}^2+\norm{D''^*u}^2
\geq
  \int_X
    \pair{[\sqrt{-1}\Theta_E,\Lmd_\omg]u}{u}
    \td Vol
$$

Recall: If $[\sqrt{-1}\Theta_E,\Lmd_\omg]$ is positive on
$C^\infty(X,\wedgeform{p,q}\ten E)$, then
$H^{p,q}(X,E)=0$.

\begin{thm}(Kodaira-Nakano vanishing theorem)

If $E$ is a holomorphic line bundle with a smooth metric $h$ s.t.
$\sqrt{-1}\Theta_{(E,h)}>0$,then
$[\sqrt{-1}\Theta_E,\Lmd_\omg]$ is positive on
$C^\infty(X,\wedgeform{p,q}\ten E)$ whenever $p+q\geq n+1$.

$\Rightarrow H^{p,q}(X,E)=0$ when $p+q\geq n+1$.
\end{thm}
(Last time)

Today:
\begin{thm}(Girbau vanishing theorem, 1976)

$E$ is a holomorphic line bundle over compact Kahler manifold, 
with smooth metric $h$ s.t. $\sqrt{-1}\Theta_{(E,h)}\geq 0$,
and has at least $n-s+1$ positive eigenvalues at every points of $X$, then
$$H^{p,q}(X,E)=0$$
if $p+q\geq n+s$.
\end{thm}

$\afa$: a \textbf{real} $(1,1)$-form on $X$, 
locally $\afa=\sqrt{-1}\sum\afa_{ij}\td z_i\wedge\td\zbar_j$.
then we have a matrix $M(\afa)=(\afa_{ij})_{n\times n}$, 
($\afa$ is real $\Rightarrow$)a hermite matrix.

we call $\afa$ has at least $k$ positive eigenvalues at $x$,
if $M(\afa)(x)$ has $k$ positive eigenvalues.
(Remark: It is well defined)

\begin{proof}
Claim: there exists some Kahler metric $\omg$ 
s.t. $[\sqrt{-1}\Theta,\Lmd]$ is positive.

Fix a Kahler metric $\omg$,
for $p\in X$, choose a holomorphic chart $(z_1,...,z_n)$,
s.t. $\omg(p)=\sqrt{-1}\sum\td z_j\wedge\td\zbar_j$ and 
$\sqrt{-1}\Theta_E(p)=\sqrt{-1}\sum\limits_{j=1}^{n}\gma_j\td z_j\wedge\td\zbar_j$.
WLOG, $0\leq\gma_1\leq\gma_2\leq\cdots\leq\gma_n$,
and for any $j\geq s$, $\gma_j>0$.

Consider 
$$\omg_\veps:=\veps\omg+\sqrt{-1}\Theta_E$$
for $\veps>0$, then $\omg_\veps$ is a Kahler metric.
$\omg_\veps(p)=\sqrt{-1}\sum\limits_{j}(\veps+\gma_j)\td z_j\wedge\td\zbar_j$.

$\Rightarrow$ the eigenvalue of $\sqrt{-1}\Theta$ with respective to
$\omg_\veps(p)$ is given by
$$\gma_{j,\veps}=\frac{\gma_j}{\veps+\gma_j}=
\frac{1}{1+\frac{\veps}{\gma_j}}$$

Claim: $[\sqrt{-1}\Theta,\Lmd_{\omg_\veps}]$ is positive on 
$\wedgeform{p,q}\ten E$ when $p+q\geq n+s$, $0<\veps<<1$.

Take $u=\sum\limits u_{IJ}\td w_T\wedge\td\wbar_J\ten e$,then
$$
  \pair{[\sqrt{-1}\Theta_E,\Lmd_{\omg_\veps}]}{u}
=
  \sum_{|I|=p\atop|J|=q}
    \left(
      \sum_{i\in I}
        \gma_{i,\veps}
     +\sum_{j\in J}
       \gma_{j,\veps}
     +\sum_{k=1}^{n}
       \gma_{k,\veps}
    \right)
    |u_{IJ}|^2
\geq
  (\gma_{1,\veps}+...+\gma_{p,\veps}-\gma_{q+1,\veps}-...-\gma_{n,\veps})
  |u|^2
$$
note that $\gma_{j,\veps}\geq 1-\frac{\veps}{\gma_s}$
if $j\geq s$, $\gma_{j,\veps}\in[0,1)$ for all $j$. it 
$$
\geq
  \left(
    (q+s-1)(1-\frac{\veps}{\gma_s})
   -(n-p)
  \right)
  |u|^2
>0
$$
if $p+q\geq n+s$ and $0<\veps<<1$.
\end{proof}

\begin{rem}(Kawamata-Viewheg vanishing theorem)

$E\to (X,\omg)$ is a holomorphic line bundle over a compact Kahler manifold.

Definition: $E$ is called positive, ...(positive="ample" in AG).
numerically effective(nef) if for any $\veps>0$, there is a smooth metric $h_\veps$
s.t. $\sqrt{-1}\Theta_{h_\veps}\geq -\veps\omg$.

Theorem: If $E$ is nef, and $\int_X c_1(E)^n>0$, then 
$H^{q}(X,K_X\ten E)=0$ for $q\geq 1$.
\end{rem}

\textbf{Positivity concept of vector bundles(rank $>1$)}

$(E,h)\to (X,\omg)$ Hermitian vector bundle of rank $r$,
over a complex manifold(may not Kahler).

Denote $(e_1,...,e_r)$ a local orthonormal frame of $E$,
$(z_1,...,z_n)$ local holomorphic chart,
Chern curvature of $(E,h)$:
$$
  \Theta_{(E,h)}
= \sum_{1\leq j,k\leq n\atop 1\lmd,\mu\leq r} 
    c_{ik\lmd\mu}\td z_j\wedge\td\zbar_k\ten e_\lmd^*\ten e_\mu
$$ 

Fact: $\sqrt{-1}\Theta_E$ induces a Hermitian operator $\theta_E$
on $TX\ten E$.

Let $u,v$ be local sections of $TX\ten E$,
$$u=\sum_{1\leq j\leq n\atop1\leq\lmd\leq r}
u_{k\mu}\pp{z_k}\ten e_\mu
$$ 
$$
  \theta_{E}(u,v)
:=
  \sum_{1\leq j,k\leq n\atop 1\leq\lmd,\mu\leq r}
    c_{jk\lmd\mu} u_{j\lmd}\overline{v_{k\mu}}
$$

\begin{definition}
We call $E$ Nakano positive, if $\theta_E$ is positive.
(i.e for any non-zero local section $u\in TX\ten E$, $\theta_E(u,u)>0$)

We call $E$ Griffith positive, if for any 
$0\neq \xi\in T_xX$, $s\in E_x,s\neq 0$,
$$\theta_E(\xi\ten s,\xi\ten s)>0$$
\end{definition}

\begin{rem}
By definition, Nakano positivity $\Rightarrow$ Griffith positivity.

If $E$ is line bundle, 
Nakano positivity $\iff$ Griffith positivity.
(and $\iff$ positivity of lines bundles)
\end{rem}

\begin{thm}(Demailly-Skota, 1979)

$E$ is Griffith positive $\Rightarrow$ $E\ten \det E$ is Nakano positive.
\end{thm}
\begin{proof}
  Omit. Non-trivial.
\end{proof}

Notation: $E>_{Nak}0$ ($E$ is Nakano positive).
Similarly, $E>_{Giff}0$...

\begin{prop}

(1)$E$ is Griffith positive if and only if $E^*$ is Griffith negative.

(2) Consider an exact sequence of holomorphic vector bundles:
$$0\to S\to E\to Q\to 0$$
then if $E$ is Griffith positive, then $Q$ is Griffith positive.
If $E$ is Griffith negative, then $S$ is Griffith negative.
If $E$ is Nakano negative, then $S$ is Nakano negative.
\end{prop}

\begin{proof}
Omit. Compute curvature...
\end{proof}

Remark: In general, $E$ is Nakano positive, $\not\Rightarrow$
$Q$ is Nakano positive.

\begin{thm}(Nakano vanishing theorem)

$(X,\omg)$ is compact Kahler of dimension $n$,
$(E,h)$ is a Nakano positive holomorphic Hermitian vector bundle, then
$$H^{n,q}(X,E)=0\qquad \forall q\geq 1$$
\end{thm}

\begin{proof}
$E$ is Nakano positive, check:
$$[\sqrt{-1}\Theta_E,\Lmd_\omg]$$
is positive on $\wedgeform{n,q}\ten E$ for $(q\geq 1)$
\end{proof}

\textbf{Ampleness}

$E\to X$, $E$: holomorphic line bundle of rank $r$, 
$X$:complex manifold.

\begin{definition}(Jet vector bundle)
$$J^kE=\bigcup_{x\in X}(J^kE)_x$$
where 
$$(J^kE)_x=\mcalO_x(E)\Big/\mfkm_x^{k+1}\mcalO_x(E)$$
$\mfkm_x\subseteq\mcalO_x$ be the maximal ideal of $\mcalO_x$.
\end{definition}

In local coordinate, 
$$
  (J^kE)_x
=
  \Big\{
    \sum_{1\leq\lmd\leq r \atop |\afa|\leq k}
      C_{\lmd\afa}(z-x)^{\afa} e_{\lmd}(z)
  \Big\}         
$$

\begin{prop}
$J^kE$ is a holomorphic vector bundle of rank $=r{n+k\choose n}$.
\end{prop}

\begin{proof}
  Exercise.
\end{proof}

\begin{definition}
$E$ is called very ample, if the following maps:
$$H^0(X,E)\to (J^1E)_x$$
$$H^0(X,E)\to E_x\oplus E_y$$
are surjective, for all $x,y\in X$, $x\neq y$. 

$E$ is called ample, if $S^mE:=\Sym^m E$ is very ample for some $m\in\bbN$.
\end{definition}

(ample: "足够多的全纯截面")

\begin{thm}(Kodaira)

$L$-holomorphic line bundle, $X$ is a compact complex manifold.
Then $L$ is positive if and only if $L$ is ample.
\end{thm}

$\mcalF_x\ten_{\mcalA_x}\mfkg^\vee$






