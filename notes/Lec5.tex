\chapter{Lefschitz分解}

\section{线性代数版本的Lefschitz算子}

Three goals:

\textbf{Kahler package}

\textbf{Lefschetz decomposition}

\textbf{Hodge-Riemann bilinear relations}

Linear algebra(baby representation theory)(local case)

$\bbC^n$,
$$\omg=\sqrt{-1}\sum_{i,j}h_{ij}\td z_i\wedge\td\zbar_j$$
Kahler metric with constant coefficients.(i.e.
$h_{ij}$ is constant, $(h_{ij})$ is positive Hermite matrix)

W.L.O.G, by taking a linear transformation, we can assume
$$\omg=\sqrt{-1}\sum_{j=1}^n\td z_j\wedge\td\zbar_j$$

\begin{notation}
An operator is of pure degree $r$ if it transform a form of $\deg=k$
to as form of degree $k+r$.

An operator ..of bi-degree $(p,q)$ if ...$(s,t)\to (s+p,t+q)$
(in this case, degree $=p+q$)

if $A,B$ with degree $\deg A,\deg B$,define
$$[A,B]:=AB-(-1)^{\deg A\deg B}BA$$
\end{notation}

\begin{definition}
$$L:\wedgeform{p,q}\to\wedgeform{p+1,q+1}$$
$$u\mapsto \omg\wedge u$$
is called Lefschetz operator.

Denote $\Lmd$ to be the adjoint of $L$, adjointed by :
Let $v\in\wedgeform{p-1,q-1}$ and $u\in\wedgeform{p,q}$
$$\langle Lv,u\rangle:=\langle u,\Lmd u\rangle$$
\end{definition}
The operator $\Lmd$ is of bi-degree $(-1,-1)$.

\begin{prop}
If
$$u=\sum_{|I|=p\atop |J|=q}u_{IJ}\td z_I\wedge\td\zbar_j$$
then
$$Lu=\sqrt{-1}\sum_{|I|=p\atop |J|=q}
 \sum_{m=1}^n
 u_{IJ}\td z_m\wedge\td\zbar_m\wedge\td z_I\wedge\td\zbar_J$$

$$
  \Lmd u
= \sqrt{-1}(-1)^p
  \sum_{|I|=p\atop|J|=q}
    \sum_{m=1}^n
      u_{IJ}
      \left(
        \pp{z_m}\suobing\td z_I
      \right)\wedge
      \left(
        \pp{\zbar_m}\suobing\td\zbar_J
      \right)
$$
where "$\suobing$" is contraction.
\end{prop}

\begin{cor}(Exercise)
Let
$$
  \afa=\sqrt{-1}\sum_{j=1}^n
       \afa_j\td z_j\wedge\zbar_j
$$
then,($\afa$ is a operator of bi-degree $(1,1)$)
$$
  [\afa,\Lmd]u
=
  \sum_{|I|=p\atop |J|=q}
    \left(
      \sum_{i\in I}
        \afa_i
     +\sum_{j\in J}
        \afa_j
     -\sum_{k=1}^n
        \afa_k
    \right)
    u_{IJ}\td z_I\wedge\td\zbar_J
$$
where
$$u=\sum_{|I|=p\atop |J|=q}
u_{IJ}\td z_I\wedge\td\zbar_J$$
\end{cor}

\begin{cor}
if $u\in\wedgeform{p,q}$,then
$$[L,\Lmd]u=(p+q-n)u$$
\end{cor}

\begin{cor}
Denote $B:=[L,\lmd]$, then
$$[B,L]=2L$$
$$[B,\Lmd]=-2\Lmd$$
\end{cor}

\begin{proof}
Take $u\in\wedgeform{p,q}$, then
$$[B,L]=BLu-LBu=(p+q-n+2)Lu-(p+q-n)Lu=2Lu$$
the second is similar..
\end{proof}


\textbf{$\mfksl(2,\bbC)$-representation}
$$\mfksl(2,\bbC)=\Span_\bbC{l,\lmd,b}$$
where
$$
  l=\begin{pmatrix}
      0 & 0\\
      1 & 0
    \end{pmatrix}
,\qquad
  \lmd=\begin{pmatrix}
      0 & 1\\
      0 & 0
    \end{pmatrix}
,\qquad
  b=\begin{pmatrix}
      1 & 0\\
      0 & -1
    \end{pmatrix}
$$
we have
$$[l,\lmd]=b\quad
  [b,l]=2l\quad
  [b,\lmd]=-2\lmd
$$

\begin{prop}
There exists a natural action
$$\rho:\mfksl(2,\bbC)\to\End(\bigoplus_{p,q}\wedgeform{p,q})$$
with
$$\rho(l)=L$$
$$\rho(\lmd)=\Lmd$$
$$\rho(b)=B$$
\end{prop}

\begin{thm}(HL)
$$L^{n-k}:\wedgeform{k}\to\wedgeform{2n-k}$$
$$u\to\omg^{n-k}\wedge u$$
is an isomorphism.

$$L^{n-k}:\wedgeform{p,q}\to \wedgeform{n-k+p,n-k+q}$$
is also an isomorphism.
\end{thm}

\begin{proof}

Lemma:
$$[L^r,\Lmd]u=r(k-n+r-1)L^{r-1}u$$
(induction, omit)

Assume $\afa\in\wedgeform{k}_\bbC$, $L^{n-k}\afa=0$,
need to verify $\afa=0$.

Claim:
$$L^r:\wedgeform{k}\to\wedgeform{k+2r}$$
is injective whenever $r\leq n-k$.

proof of the claim:

claim is true when $k=0$ or $k=1$.(check)

Let $\afa\in \wedgeform{k}$ s.t. $L^r\afa=0(r\leq n-k)$.
By the lemma,
$$L^r\Lmd\afa-\lmd L^r\afa=r(k-n+r-1)L^{r-1}\afa$$
so,
$$L^{r-1}(L\Lmd\afa-r(k-n+r-1)\afa)=0$$
by the induction on $r$,
$$L\Lmd\afa=r(k-n+r-1)\afa$$
since $r(k-n+r-1)\neq 0$, $\afa=L\beta$ for some $\beta\in\wedgeform{k-2}$.
so,$L^r\afa=L^{r+1}\beta=0$, by induction on $k$, we have
$\beta=0$,so $\afa=0$.

The claim is proved.
\end{proof}

\begin{definition}(Primitive form)

$\afa\in\wedgeform{k}(k\leq n)$ is called primitive form, if
$$L^{n-k+1}\afa=0$$
\end{definition}

\begin{cor}(Lefischtz Decomposition)(LD)

For any $\afa\in\wedgeform{k},(1\leq k\leq 2n)$,
we have a unique decomposition:
$$
  \afa
=
  \sum_{\gma\geq(k-n)_+}
    L^\gma\afa_r
$$
($(k-n)_+:=\max\{k-n,0\}$)
with $\afa_r\in\wedgeform{k-2r}$ is primitive
\end{cor}

\begin{proof}
Existence: assume $k\leq n$, consider
$$L^{n-k+1}\afa\in\wedgeform{2n-k+2}$$
by HL, $\exists !\,\beta\in\wedgeform{k-2}$ s.t.
$L^{n-k+2}\beta=L^{n-k+1}\afa$, so
$L^{n-k+1}(\afa-L\beta)=0$, i.e. $\afa_0=\afa-L\beta$
is primitive. $\afa=\afa_0+L\beta$, then induction on degrees,
we get the decomposition for $\afa$.

If $k>n$, we apply HL to reduce it to case 1.

Uniqueness: Next time..
\end{proof}

%%%%%%%%%%%%%%%2019.4.23第九周周二%%%%%%%%%%%%%%%%%%%%%%

Today: Continuous to Hard Lef decomposition, Hodge-Riemann bilinear relations.

Hard-Lefschitz: HL

Lefschitz decomposition :LD

Hodge-Riemann bilinear relations :HRR

Recall: $\bbC^n,\wedgeform{k}=\bigoplus\limits_{p+q=k}\wedgeform{p,q}$,
$\omg$: a Kahler metric on $\bbC^n$
with constant coefficient $\in\wedgeform{1,1}_\bbR$.

Lefschitz operator : $Lu=\omg\wedge u$.

\begin{thm}(HL)

Assume $k\leq n,p+q\leq n$, then
$$L^{n-k}:\wedgeform{k}\to \wedgeform{2n-k}$$
is a linear isomorphism.

$$L^{n-k}:\wedgeform{p,q}\to\wedgeform{p+n-k,q+n-k}$$
is also a linear isomorphism.
\end{thm}

Linear algebra..

\begin{thm}(LD)
for any $u\in\wedgeform{k}$,we have a unique decomposition
$$u=\sum_{r\geq(k-n)_+}L^ru_r$$
where $u_r\in\wedgeform{k-2r}_{prim}$ is a primitive form.
\end{thm}
Recall: a $k$-form $u\in\wedgeform{k}(k\leq n)$ is called primitive,
if $L^{n-k+1}(u)=0$. When $k>n$, $u$ is called primitive,
$\Lmd(u)=0$, where $\Lmd$ is the adjoint of $L$.

\begin{proof}

Existence: application of $HL$.

Uniqueness: Omit.
\end{proof}

\begin{prop} Assume $\afa\in\wedgeform{p,q}_{prim}$, and $p+q\leq n$.
(i.e. $L^{n-p-q+1}\afa=0$),then
$$
  *\afa
=
  (-1)^{\frac{(p+q)(p+q-1)}{2}}
  (\sqrt{-1})^{p-q}
  \frac{1}{(n-p-q)!}
  L^{n-p-q}\afa
$$
\end{prop}

\begin{proof}
  See [Humphreys, Prop 1.2.31]
\end{proof}

\begin{thm}(HRR)
Define the bilinear form $Q$ on $\wedgeform{k}\,\,(k\leq n)$ as follows:
$$Q(\afa,\beta):=L^{n-k}\wedge\afa\wedge\overline{\beta}$$
Then
$$(\sqrt{-1})^{p-q}(-1)^{\frac{(p+q)(p+q-1)}{2}}Q(u,u)\geq 0$$
for any $u\in \wedgeform{p,q}_{prim},p+q=k\leq n$,
and equal holds $$\iff u=0$$

(i.e. $Q|_{\wedgeform{p,q}_{prim}}$ is positive definite up to a factor)
\end{thm}

\begin{proof}
Take $u\in\wedgeform{p,q}_{prim}$,
$$Q(u,u)=L^{n-k}\wedge u\wedge\overline{u}
=*u\wedge\overline{u}=\langle\overline{u},\overline{u}\rangle\td Vol
=|u|^2\td Vol\geq 0$$
(up to a factor!)

(We apply the following result: $\overline{*\fai}=*\overline{\fai}$,
i.e. $*$ is a real operator)
\end{proof}

Summary: $\wedgeform{\bullet}=\bigoplus\limits_{1\leq k\leq n}\wedgeform{k}_\bbC$,
where$\wedgeform{k}_{\bbC}=\bigoplus\limits_{p+q=k}\wedgeform{p,q}_\bbC$.

Lefschitz operator $L\rightsquigarrow$ HL,LD,HRR.


\section{紧Kahler流形的上同调群}

The analogue of compact Kahler manifolds,

$$H^k_{DR}(X,\bbC)\cong\bigoplus_{p+q=k}H^{p,q}_{Dol}(X,\bbC)$$

$\omg$: A Kahler metric $\in H^{1,1}_{Dol}(X,\bbR)$.

Denote $L\curvearrowright H^k_{DR}(X,\bbC)$,
$$L(u)=[\omg,u]=[\omg]\wedge u$$

\textbf{Commutative relations on Kahler manifolds}

$$(\bbC^n,\omg=\sqrt{-1}\sum_{j=1}^n\td z_j\wedge\td\overline{z}_j)$$
$u\in C^\infty(\bbC^n,\wedgeform{p,q})$, locally
$$u=\sum_{|I|=p,|J|=q}u_{I,J}\td z_I\wedge \td z_j,\quad
v=\sum_{|I|=p,|J|=q}v_{I,J}\td z_I\wedge \td z_j
$$

$$\langle\langle u,v\rangle\rangle=
\int_{\bbC^n}\sum_{|I|=p,|J|=q}
u_{I,J}\overline{V_{I,J}}\td Vol$$

$\td=\td'+\td''$, $\td'=\p,\td''=\pbar$.

$$\td' u=\sum_{I,J}\sum_k\pfrac{u_{I,J}}{z_k}\td z_k\wedge\td z_I\wedge\td z_J$$
$$\td''u=\cdots$$

\begin{thm}
$$
  (\td'')^*u
=
  -\sum_{I,J}
    \sum_k
    \pfrac{u_{I,J}}{\zbar_k}
    \pp{\zbar_k}
    \suobing
    (\td z_I\wedge\td\zbar_J)    
$$

$$
  (\td')^*u
=
  -\sum_{I,J}
    \sum_k
      \pfrac{u_{I,J}}{\zbar_k}
      \pp{z_k}
      \suobing
      (\td z_I\wedge\td\zbar_J)
$$
\end{thm}

\begin{prop}
$$[(\td'')^*,L]=\sqrt{-1}\td'$$
\end{prop}
\begin{proof}
Exercise.
\end{proof}

\begin{thm}
Let $X$ be a Kahler manifold (may not compact), with Kahler metric $\omg$, 
then we have
$$[(\td'')^*,L]=\sqrt{-1}\td'$$
\end{thm}

\begin{proof}
Only need to verify $u\in  C^\infty_c(X,\wedgeform{p,q})$
with compact support in a holomorphic chart at $x$.

Assume the holomorphic chart near $x$ is choosen s.t. 
$$\omg(z)=\sqrt{-1}\sum_{1\leq j\leq n}\td z_j\wedge\td\zbar_j+O(|z|^2)$$

$$u\in\sum_{I,J}u_{I,J}\td z_I\wedge\zbar_J$$
is a $(p,q)$-form, $v$ is also...

$$\langle u,q\rangle= u_{I,J}\overline{v_{M,N}}
\langle \td z_I,\td z_M\rangle\langle\td\zbar_J,\td\zbar_N\rangle
=u_{IJ}\overline{V_{ij}}+a_{IJMN}(z)u_{IJ}\overline{V_{MN}}$$
where $a_{IJMN}=O(|z|^2)$.

So, 
$$(\td'')^*u=-\sum_{IJk}\pfrac{u_{IJ}}{z_k}\pp{\zbar_k}
\suobing(\td z_I\wedge\td\zbar_J)+
\sum_{IJMN}b_{IJMN}u_{IJ}\td z_M\wedge\td\zbar_N
$$
where $b_{IJMN}(z)=O(|z|)$. So,

$$[(\td'')^*,L]u(x)=
\sqrt{-1}\td'u(x)$$

$$\Longrightarrow [(\td'')^*,L]=\sqrt{-1}\td'$$
\end{proof}

\begin{prop}In Kahler manifold,
$$[(\td')^*,L]=-\sqrt{-1}\td''$$
$$[\Lmd,\td'']=-\sqrt{-1}(\td')^*$$
$$[\Lmd,\td']=\sqrt{-1}(\td'')^*$$
\end{prop}

\begin{cor}$(X,\omg)$ is a Kahler manifold, then 
$$\yc_{\td}=2\yc_{\td'}=2\yc_{\td''}$$
\end{cor}
\begin{proof}
For example, $\yc_\td=2\yc_{\td''}$,
$$\yc_\td=(\td'+\td'')(\td'+\td'')^*+(\td'+\td'')^*(\td'+\td'')
=(\td'+\td'')(\td'^*-\sqrt{-1}[\Lmd,\td'])
+(\td'^*-\sqrt{-1}[\Lmd,\td'])(\td'+\td'')$$
然后暴力展开, 12项???$\cdots$

从略。
\end{proof}

\begin{cor}
If $(X,\omg)$ is a Kahler manifold, then
$$\yc_\td:C^\infty(C,\wedgeform{p,q})\to C^\infty(C,\wedgeform{p,q})$$
\end{cor}
\begin{proof}
Since $\yc_\td=2\yc_{\td'}$, 
$\yc_{td'}$ preserves the bi-degree.
\end{proof}

\begin{cor}
If $(X,\omg)$ is a compact Kahler manifold,
$u$ is a $\yc_\td$-harmonic $k$-form. Assume 
$$u=\sum_{p+q=k}u^{p,q}$$
$$u^{p,q}\in C^\infty(X,\wedgeform{p,q})$$
then each $u^{p,q}$ is also harmonic. 
\end{cor}

\begin{thm}(Hodge decomposition)

$X$ is a compact Kahler manifold, then we have a decomposition 
$$H_{\td}^k(X,\bbC)=\bigoplus_{p+q=k}H^{p,q}_{\td''}(X,\bbC)$$

Equivalently, (sheaf cohomology)
$$H^k(X,\bbC)\cong\bigoplus_{p+q=k}H^q(X,\Omg^p)$$
\end{thm}
\begin{proof}
take a Kahler metric $\omg$, we can define $\yc_d,\yc_{td'},\yc_{\td''}$, 
then 
$$\ker\yc_{\td}:=\mcalH^k(X,\bbC)
\cong\bigoplus\limits_{p+q=k}\mcalH^{p,q}_{\td''}(X,\bbC)$$

then $\Longrightarrow$ the decomposition for $H_\td^k(X,\bbC)$

the decomposition for $H^k_\td(X,\bbC)$ is independent of the choice of $\omg$
(Next time)
\end{proof}








