%%%%%%%%%%%%%%%%%%%%%%%%%%%%%%%%%%%%%%%%%%%%%%%%%
%%%%%%%%%%%%%%%%%%%%%%%%%%%%%%%%%%%%%%%%%%%%%%%%%
%       2019.3.21启用电子稿                     %
%%%%%%%%%%%%%%%%%%%%%%%%%%%%%%%%%%%%%%%%%%%%%%%%%


\chapter{层与层上同调}

本章介绍层论、层上同调的语言。这套理论是J-Leray 于1945-1946年
在监狱中创立的。在正式介绍这套抽象的理论之前,先通过一个例子来大致了解
引入此理论的动机。

\textbf{问题:}设$S$为一个黎曼曲面,$\{p_n\}\subseteq S$为$S$的一个离散点集,
我们希望找一个$S$上的亚纯函数$f$,使得$f$在$S\setminus\{p_n\}$全纯,
并且在每个$p_i$处具有事先给定的主部。\vs

这样的函数$f$在局部上的存在性是显然的;而在$S$上的整体存在性并不平凡。

\begin{proof}[思路(\Cech)]

取$S$的一族开覆盖$\mcalU:=\Bigset{U_\afa}{\afa\in\mcalI}$,使得
每个$U_{\afa}$均为局部坐标卡,并且至多包含$\{p_n\}$中的一个点,
则局部地,可在每个$U_\afa$上找到满足要求的亚纯函数$f_\afa$.

之后我们希望找到$g_\afa\in\mcalO(U_\afa)$,使得对任意$\afa,\beta\in\mcalI$,
在$U_\afa\cap U_\beta$上成立$f_\afa-g_\afa=f_\beta-g_\beta$.于是我们可定义$S$
上的亚纯函数$f=f_\afa-g_\afa$.易知$f$良定,且满足要求。

令$f_{\afa\beta}\in\mcalO(U_\afa\cap U_\beta)$为
$$f_{\afa\beta}:=f_\afa-f_\beta$$
则显然对于任意指标$\afa,\beta,\gma$,在公共部分$U_\afa\cap U_\beta\cap U_\gma$上成立
$$f_{\afa\beta}+f_{\beta\gma}+f_{\gma\afa}=0\eqno{(*)}$$
而如果存在上述$g_\afa\in U_\afa$,则有$f_{\afa}=g_\afa-g_\beta$.现在,令
\begin{eqnarray*}
  Z^1(\mcalU,\mcalO)&:=&\Span
  \Bigset{f_{\afa\beta}\in\mcalO(U_{\afa}\cap U_\beta)}
  {f_{\afa\beta}\text{满足$(*)$}}
\\
  B^1(\mcalU,\mcalO)&:=&\Span
  \Bigset{f_{\afa\beta}\in\mcalO(U_{\afa}\cap U_\beta)}
  {\exists g_\afa\in\mcalO(U_\afa)\,,\,f_{\afa\beta}=g_\afa-g_\beta}
\end{eqnarray*}
显然$B^1(\mcalU,\mcalO)$为$Z^1(\mcalU,\mcalO)$的子空间。如果这两者相等,
则满足题设的解存在。
\end{proof}

我们记$H^1(\mcalU,\mcalO):=\frac{Z^1(\mcalU,\mcalO)}{B^1(\mcalU,\mcalO)}$
为$X$上的全纯函数“层”(sheaf)关于开覆盖$\mcalU$的第$1$个\textbf{\Cech 上同调}.
我们将了解到,\Cech 上同调与$S$的拓扑有密切关系。\vs

{\color{blue}
本章需要一定的范畴论准备。由于这不是专门介绍层论的讲义,
我们会省略很多论证细节,只介绍主要结果。
}

\section{预层与层的概念}

\begin{definition}(集值预层)
\index{presheaf\kong 预层}
\index{section\kong 截面}

设$X$为拓扑空间,$X$上的\textbf{预层}(presheaf)$\mcalF$是指以下资料:

(1)对任意$X$中的开集$U$,给定集合$\mcalF(U)$,
称$\mcalF(U)$为$\mcalF$在$U$上的\textbf{截面空间},其中的元素称为
$\mcalF$在$\mcalU$上的一个\textbf{截面}(section).

(2)对于$X$的任意开子集$U,V$,若$U\subseteq V$,则配以\textbf{限制映射}
\begin{eqnarray*}
\rho_{UV}:\mcalF(V)&\to&\mcalF(U)\\
s&\mapsto&s|_U
\end{eqnarray*}
并且对$X$的任意开子集$W\subseteq U\subseteq V$成立:
\begin{eqnarray*}
\rho_{UU}&=&\id_{\mcalF(U)} \\
\rho_{WV}&=&\rho_{WU}\circ\rho_{UV}
\end{eqnarray*}
\end{definition}

最典型的例子是,拓扑空间$X$上的函数之全体函数构成预层$\mcalC$.
具体地,对$X$的开子集$U$,$\mcalC(U):=C(U)$为定义在$U$上的连续函数之全体;
对于$V\subseteq U$,则限制映射$\rho_{UV}$为通常的函数定义域的限制。

\begin{rem}通常来说,预层$\mcalF$被假定具有代数结构。
具体地,对于$X$的开集$U$,$\mcalF(U)$被假定具有Abel群结构、交换环结构或者
$A$-模结构等等,此时分别称作取值于Abel群范畴、交换环范畴、$A$-模范畴的预层。
\end{rem}
当然,若$\mcalF(U)$具有上述代数结构,则我们也要求限制映射$\rho_{VU}$
为相应范畴中的态射,并且规定$\mcalF(\vkong)=\{0\}$为相应范畴中的零对象。

\begin{example}(常值预层)
\label{常值预层-def}

对于拓扑空间$X$,定义$X$上的集值预层$\bbC_X$如下:对于任意开子集$U$,
$\bbC_X(U):=\bbC$;对于$U\subseteq V$,限制映射
$
  \rho_{UV}:=
  \left\{
    \begin{array}{ll}
      \id_\bbC & U\neq\vkong\\
      0        & U=\vkong
    \end{array}
  \right.
$,则容易验证这是$X$上的预层,称为\textbf{常值预层}.
\end{example}

\begin{example}(全纯函数预层)

设$X$为复流形,则$\mcalO_X:U\mapsto\mcalO(U)$,配以通常的函数限制,
构成$X$上的预层,称为\textbf{全纯函数预层}。
\end{example}

\begin{example}(微分形式预层)

设$X$为光滑流形,对$X$的任意开子集$U$,考虑$U$上的光滑$k$形式之全体$\wedgeform{k}(U)$,
配以通常的限制映射,则$\wedgeform{k}$构成预层,称为
\textbf{光滑$k$-形式预层}。
\end{example}

\begin{definition}(层)
\index{sheaf\kong 层}

设$\mcalF$为拓扑空间$X$上的预层,称$\mcalF$为层(sheaf),若以下成立:

(1)(粘合公理)若$U$与$U_\afa(\afa\in\mcalI)$均为$X$的开子集,并且
$U=\bigcup\limits_{\afa\in\mcalI}U_\afa$,
则对于任何$s_\afa\in\mcalF(U_\afa)$,如果
$s_\afa|_{U_\afa\cap U_\beta}
=s_\beta|_{U_\afa\cap U_\beta}$
对任意$\afa,\beta\in\mcalI$成立,则存在$s\in\mcalF(U)$,使得
$s|_{U_\afa}=s_\afa$对任意$\afa\in\mcalI$成立。\vs

(2)(唯一性公理)条件同上,则对于任意$s,t\in\mcalF(U)$,
若对任意$\afa\in\mcalI$,$s|_{U_\afa}=t|_{U_\afa}$,则$s=t$.
\end{definition}

类似地也可以定义取值于Abel范畴上的层。此时,容易验证唯一性公理等价于:
($U=\bigcup\limits_{\afa\in\mcalI}U_\afa$)
对于$s\in\mcalF(U)$,若$s|_{U_\afa}=0$对任意$\afa\in\mcalI$成立,则$s=0$.

\begin{example}若拓扑空间$X$包含至少两个不交的开集,
则常值预层(例子\ref{常值预层-def})$\bbC_X$\textbf{不是}层,因为不满足粘合公理。
\end{example}
具体地,若$U,V$为$X$的两个不交的开子集,考虑$1\in\bbC_X(U)$以及$2\in\bbC_{X}(V)$,
则显然不存在$z\in\bbC_X(U\cup V)$使得$1=z|_{U}$以及$2=z|_{V}$.

\begin{example}(向量丛是层)设$E\to X$为光滑流形$X$上的向量丛,
则$E$自然视为$X$上的层$\Gma(-,E)$:对任意$U\subseteq X$, 考虑丛$E$在$U$
上的截面之全体$\Gma(U,E)$。易验证其满足层的公理。
\end{example}
类似地,复流形上的全纯函数预层是层,光滑$k$-形式预层也是层。

\begin{definition}(预层的同态)

设$\mcalF$与$\mcalG$为拓扑空间$X$上的(取值于同一个Abel范畴的)预层,
预层同态$\fai:\mcalF\to\mcalG$是指以下资料:对任意开集$U\subseteq X$,
配以(相应Abel范畴中的)态射$\fai_U:\mcalF(U)\to\mcalG(U)$,并且
对于$X$的任意开子集$U\subseteq V$,以下图表交换:
$$
  \xymatrix{
     \mcalF(U)  \ar[d]_{\fai_U}
    &\mcalF(V)  \ar[d]_{\fai_V} \ar[l]_{\rho_{UV}}
  \\
     \mcalG(U)
    &\mcalG(V)  \ar[l]_{\rho_{UV}}
  }
$$
\end{definition}

设$\fai:\mcalF\to\mcalG$为$X$上的预层同态,则我们可以定义
$\ker^p\fai$,$\im^p\fai$,$\coker^p\fai$为:对任意开集$U\subseteq X$,
$$(\ker^p\fai)(U):=\ker(\fai_U)$$
$\im^p\fai$与$\coker^p\fai$也完全类似。容易验证它们都是预层,
分别称为预层同态$\fai$的\textbf{核预层}、\textbf{像预层}、\textbf{余核预层}。
这里的上标“$p$”是指“预层”(presheaf)。

\begin{prop}设$\mcalF,\mcalG$为$X$上的层,
$\fai:\mcalF\to\mcalG$为预层同态,则预层$\ker^p\fai$是层。
\end{prop}

\begin{proof}直接验证$\ker^p\fai$满足层的粘合公理和唯一性公理。
设$\Bigset{U_\afa}{\afa\in\mcalI}$为$X$的开子集$U$的一族开覆盖,
注意到$(\ker^p\fai)(U_\afa)\subseteq\mcalF(U_\afa)$,以及$\mcalF$为层(满足粘合公理),
因此易知$\ker^p\fai$也满足粘合公理。$\ker^p\fai$的唯一性公理也是由$\mcalF$的层性质直接得到的。
\end{proof}

从此以后,若$\mcalF$与$\mcalG$都为层,则我们将核预层$\ker^p\fai$简记为$\ker\fai$.

\begin{rem}好吧,刚才的命题几乎显然。但是要注意,即使
$\mcalF$与$\mcalG$都是层,$\im^p\fai$与$\coker^p\fai$未必是层。
它们并没有$\ker^p\fai$的良好性质。
\end{rem}

\begin{example}考虑拓扑空间$X=\bbC\setminus\{0\}$,
令$\mcalF:=\mcalO_X$为$X$上的全纯函数层,$\mcalG:=\mcalO^*_X$定义为:
对于$X$的开集$U$,
$$\mcalO^*_X(U):=\Bigset{f\in\mcalO_X(U)}{f(z)\neq 0\,,\forall z\in U}$$
容易验证$\mcalO^*_X$为(取值于集合的)层。考虑层同态
\begin{eqnarray*}
\exp:\mcalF&\to&\mcalG\\
f\in\mcalF(U)&\mapsto&e^f
\end{eqnarray*}
则$\im^p\exp$\textbf{不是}层。
\label{指数预层同态}
\end{example}

\begin{proof}
只需要考虑函数$z\in\mcalO_X^*(X)$.对任意单连通的开子集$U\subseteq X$,
易知$z\in\mcalO^*_X(U)$满足$z\in(\im^p\exp)(U)$,
但是$z\in\mcalO_X^*(X)$并不位于$(\im^p\exp)(X)$当中,从而
$\im^p\exp$不满足粘合公理。
\end{proof}

\begin{notation}(层的限制)
设$\mcalF$是拓扑空间$X$上的层,$U$为$X$的开子集,
则自然有拓扑空间$U$上的层$\mcalF|_U$如下:对$U$中的开集$V$
(注意$V$也是$X$中的开集),定义
$$\mcalF|_U(V):=\mcalF(V)$$
相应的限制映射也自然给出。容易验证$\mcalF|_U$是拓扑空间$U$上的层,
称为$\mcalF$在$U$上的\textbf{限制}。
\end{notation}

关于层的构造,我们再介绍层的直和:

\begin{example}(层的直和)

设$\mcalF$与$\mcalG$为拓扑空间$X$上的取值于(同一个)Abel范畴的层,
则定义$\mcalF$与$\mcalG$的\textbf{直和层}$\mcalF\oplus\mcalG$如下:
对$X$中的开集$U$,
$(\mcalF\oplus\mcalG)(U):=\mcalF(U)\oplus\mcalG{U}$.
\end{example}
容易验证$\mcalF\oplus\mcalG$也为$X$上的层。类似也可以定义多个层的直和。
特别地,对于层$\mcalF$以及正整数$n$,记
$\mcalF^{\oplus n}:=
\underbrace{\mcalF\oplus\mcalF\oplus\cdots\oplus\mcalF}_{n\text{个}}$


\section{预层的层化}

\begin{definition}(预层的芽)
\index{germ\kong 芽}
\index{stalk\kong 茎条}

设$\mcalF$为$X$上的预层,$x\in X$,则称
$$\mcalF_x:=\rlim_{x\in U}\mcalF(U)$$
为$\mcalF$在$x$处的\textbf{茎条}(stalk),其中$U$取遍$x$的开邻域。
$\mcalF_X$中的元素称为$x$处的\textbf{芽}(germ)。
\end{definition}

我们不再回顾范畴论中的余极限(or 归纳极限、正向极限)的概念。
典型的例子是,若$\mcalO_X$为复流形$X$上的解析函数环层,则对于$x\in X$,
$\mcalO_{X,x}$即为通常的在$x$处的解析函数芽环。

回顾层的粘合公理、唯一性公理,用茎条、芽的语言可以给出上述公理的等价表述:
\begin{prop}设$\mcalF$是拓扑空间$X$上的预层,则

(1)$\mcalF$满足粘合公理$\iff$对任意开集$U$,以及对任意
$s(x)\in\mcalF_x(\forall x\in U)$,如果对任意$x\in U$,存在$x$的开邻域$V\subseteq U$,
以及$s(x)$的代表元$t\in\mcalF(V)$,使得对任意$y\in V$,成立$s(y)=t_y$,那么
存在$S\in\mcalF(U)$,使得对任意$x\in U$成立$S_x=s(x)$。

(2)$\mcalF$满足唯一性公理$\iff$对任意开集$U$,以及对任意
$s\in\mcalF(U)$,如果对任意$x\in U$,$s_x=0$,那么$s=0$.
\end{prop}
\begin{proof}
  由有关定义出发,几乎显然。
\end{proof}

\begin{prop}设$\mcalF$与$\mcalG$为$X$上的预层,$\fai:\mcalF\to\mcalG$
为预层同态,则对任意$x\in X$,$\fai$自然诱导茎条同态
$$\fai_x:\mcalF_x\to\mcalG_x$$
\end{prop}
\begin{proof}
  由余极限$\rlim$的函子性直接得到。
\end{proof}
具体构造是,对任意$F_x\in\mcalF_x$,取$F_x$的代表元$F\in\mcalF(U)$,
其中$U$为$x$的某个开邻域。之后,$\fai_x(F_x)=(\fai_U(F))_x$.

\begin{definition}(预层的层空间)
\index{sheaf space\kong 层空间}

设$\mcalF$为拓扑空间$X$上的预层,则定义拓扑空间
$$\mcalFtil:=\coprod_{x\in X}\mcalF_x$$
其拓扑由拓扑基$\Bigset{\Omg_{F,U}}{U\subseteq X\text{为开子集},F\in\mcalF(U)}$
生成,其中$\Omg_{F,U}=\Bigset{F_x\in\mcalF_x}{x\in U}$.
称拓扑空间$\mcalFtil$为预层$\mcalF$的\textbf{层空间}(sheaf space)。
\end{definition}

具体地,若芽$F_x\in\mcalFtil$,取$F_x$的代表元$F\in\mcalF(U)$,
其中$U$为$x$的一个(充分小的)开邻域,则$\Bigset{F_y}{y\in U}$
为$F_X$在$\mcalFtil$中的一个开邻域。我们由自然的映射
\begin{eqnarray*}
\Pi:\mcalFtil&\to&X\\
s\in\mcalF_x&\mapsto&x
\end{eqnarray*}
则容易验证$\Pi:\mcalFtil\to X$为连续映射,且对于任意$F\in\mcalF(U)$,
$\Pi:\Omg_{F,U}\to U$为拓扑同胚。

\begin{definition}(预层的层化)
\index{sheafification\kong 层化}

设$\mcalF$是$X$上的预层,对$X$的开子集$U$,定义
$$
  \mcalF^+(U)
:=\Bigset{s:U\to\mcalFtil}{s\text{为连续映射,并且}\Pi\circ s=\id_U}
$$
称$\mcalF^+$为预层$\mcalF$的\textbf{层化}(sheafification).
\end{definition}
具体地,对于$s:U\to\mcalFtil$,$s\in\mcalF^+(U)$当且仅当
对任意的$x\in U$,$s(x)\in\mcalF_x$,并且存在$x$的开邻域$V\subseteq U$,
以及存在$F\in\mcalF(V)$,使得$s(y)=F_y$对任意$y\in V$成立。

\begin{prop}设$\mcalF$为$X$上的预层,则$\mcalF^+$为$X$上的层,
并且有典范的预层同态$\theta:\mcalF\to\mcalF^+$如下:对任意开集$U$,
\begin{eqnarray*}
\theta_U:\mcalF(U)&\to&\mcalF^+(U)\\
s&\mapsto&\stil:U\to\mcalFtil\quad(x\mapsto s_x)
\end{eqnarray*}
\end{prop}
\begin{proof}
$\mcalF^+$的粘合公理与唯一性公理几乎显然成立。
\end{proof}

我们更习惯于把有预层同态$\theta:\mcalF\to\mcalF^+$称为$\mcalF$的层化。
容易验证,对任意$x\in X$,由茎条同构$\mcalF_X\cong\mcalF^+_x$;此外也
容易验证,如果$\mcalF$本身是层,那么$\theta$为层同构,即“层的层化同构于其本身”。

\begin{prop}(层化的泛性质)

设$\mcalF$为拓扑空间$X$上的预层,则对于$X$上的任何层$\mcalG$,
以及预层同态$\fai:\mcalF\to\mcalG$,存在唯一的层同态$\psi:\mcalF^+\to\mcalG$,
使得以下图表交换:
$$
  \xymatrix{
     \mcalF \ar[r]^{\forall\fai} \ar[d]_\theta
    &\mcalG
  \\
     \mcalF^+ \ar@{-->}[ur]_{\exists!\,\psi}
    &
  }
$$
\end{prop}

\begin{proof}
对任意$x\in X$,$\fai:\mcalF\to\mcalG$诱导了
$\fai_x:\mcalF_x\to\mcalG_x$,
再注意$\mcalF_X\cong\mcalF^+_x$,
从而自然给出$\psi_x:\mcalF^+_x\to\mcalG_x$.
易验证$\Bigset{\psi_x}{x\in X}$确定了层同态
$\psi:\mcalF^+\to\mcalG$,且$\psi\circ\theta=\fai$.

$\psi$的唯一性是显然的。
\end{proof}

\begin{example}回顾常值预层$\bbC_X$
(见例子\ref{常值预层-def}),则其层化
$\bbC_X^+$为,对任意开集$U$,
$$\bbC_X^+(U)=\Bigset{f:U\to\bbC}{f\text{为局部常值函数}}$$
称之为$X$上的\textbf{局部常值层}。
\end{example}

\begin{example}回顾例子\ref{指数预层同态}中的预层同态
$$\exp:\mcalO_X\to\mcalO_X^*$$
则像预层$\im^p(\exp)$的层化$(\im^p\exp)^+\cong\mcalO_X^*$.
\end{example}

\begin{definition}(像层、余核层与商层)

设$\mcalF$与$\mcalG$为拓扑空间$X$上的层,$\fai:\mcalF\to\mcalG$为层同态。

(1)定义$\im\fai:=(\im^p\fai)^+$,称之为$\fai$的\textbf{像层};

(2)定义$\coker\fai:=(\coker^p\fai)^+$,称之为$\fai$的\textbf{余核层};

(3)若对于任意开集$U$,$\fai_U:\mcalF(U)\to\mcalG(U)$为单同态,则称
$\fai$为\textbf{层单同态},此时也称$\mcalF$为$\mcalG$的\textbf{子层},
并且定义\textbf{商层}$\mcalF/\mcalG:=\coker\fai$.
\end{definition}

无非是将相应的预层加以层化。此外容易验证,层同态$\fai:\mcalF\to\mcalG$为单同态,
当且仅当对任意$x\in X$,$\fai_x:\mcalF_x\to\mcalG_x$为单同态。

\begin{rem}设$\fai:\mcalF\to\mcalG$为层同态,
则像层$\im\fai$自然地视为$\mcalG$的子层:
$$
  \xymatrix{
     \mcalF  \ar[r]^\fai  \ar[d]_{\faitil}
    &\mcalG
  \\
     \im^p\fai  \ar[ur]^{i'}  \ar[r]^\theta
    &\im\fai    \ar@{-->}[u]_i
  }
$$
层同态$i:\im\phi\to\mcalG$由层化的泛性质给出,并且逐茎条看,显然$i$为层单同态。
\end{rem}

\begin{definition}(层满同态)

设$\fai:\mcalF\to\mcalG$为层同态,称$\fai$为\textbf{层满同态},若
$\im\fai:=(\im^p\fai)^+\cong\mcalG$.
\end{definition}

由有关定义可以验证,层同态$\fai:\mcalF\to\mcalG$为层满同态,
当且仅当对任意$x\in X$,$\fai_x:\mcalF_x\to\mcalG_x$为满同态。
由此可推出,$\fai$为层同构,当且仅当对任意$x\in X$,$\fai_x$为茎条同构。

\section{层的顺像与逆像}

\begin{notation}对于拓扑空间$X$,定义$X$上的Abel群层范畴$\Abcat_X$为:

(1)$\Abcat(X)$中的对象为$X$上的取值于Abel群的层;

(2)对象之间的态射为相应的层同态。
\end{notation}
显然这是一个范畴。类似可定义“$X$上的集值层范畴”$\Setcat_X$,
“$X$上的交换环层范畴”$\Ringcat_X$,
以及对于交换环$A$,我们可定义$X$上的$A$-模层范畴$A\modcat_X$等等。

一般地,将$X$上(所有种类的)层之全体记作$\Shcat_X$,
这自然也给出一个范畴,称为$X$上的\textbf{层范畴}。
类似地,$X$上的所有预层也构成范畴,记为$\pShcat_X$.

\begin{rem}对于拓扑空间$X$,以及$X$的开集$U$,则有“取截面”函子
\begin{eqnarray*}
  \Gma(U,-):\Abcat_X&\to&\Abcat\\
            \mcalF&\mapsto&\mcalF(U)
\end{eqnarray*}
其中$\Abcat$为Abel群范畴。容易验证函子$\Gma(U,-)$是\textbf{左正合函子},
即对于$\Abcat_X$中任意的左短正合列$0\to\mcalF\to\mcalG\to\mcalH$,
该函子诱导的Abel群同态序列
$0\to\mcalF(U)\to\mcalG(U)\to\mcalH(U)$
也是正合的。\label{取截面函子左正合-rem}
\end{rem}

函子$\Gma(U,-)$的左正合性是后文将要介绍的层上同调理论的基础。

\begin{definition}(层的顺像)

设$f:X\to Y$为拓扑空间的连续映射,$\mcalF$是$X$上的层,则定义
$\mcalF$的\textbf{推出}(push-forward),也称为\textbf{顺像}(direct image)
$f_*\mcalF$为:
对$Y$的开子集$U$,$(f_*\mcalF)(U):=\mcalF(f^{-1}(U))$.
\end{definition}

显然$f_*\mcalF$为$Y$上的预层。
容易验证,若$\mcalF$是层,则预层$f_*\mcalF$也是层。
事实上,顺像$f_*$具有函子性,具体地说,若$\fai:\mcalF\to\mcalG$为
$X$上的层同态,则$f$诱导了$Y$上的层同态$f_*\fai:f_*\mcalF\to f_*\mcalG$,
并且使得有关图表交换。换句话说,我们有函子$f_*:\Shcat_X\to\Shcat_Y$.

容易验证,$f_*\mcalF$在$y\in Y$处的茎条为
$$
  (f_*\mcalF)_y\cong
\rlim_{y\in V}\mcalF(f^{-1}(V))
$$

\begin{definition}(层的逆像)
\index{inverse image\kong 逆像}
\label{层的逆像-def}

设$f:X\to Y$为拓扑空间之间的连续映射,$\mcalG$为$Y$上的层,
则定义$X$上的层$f^{-1}\mcalG$为:对$X$的任意开集$U$,
$$
  (f^{-1}\mcalG)(U)
:=\rlim_{V\in f(U)}\mcalG(V)
$$
其中$V$取遍$Y$中的包含$f(U)$的开子集。称$f^{-1}\mcalG$为$\mcalG$关于$f$
的\textbf{逆像}(inverse image)
\end{definition}

显然如此定义的$f^{-1}\mcalG$为$X$上的预层。利用余极限的泛性质,
也能验证当$\mcalG$为层时,$f^{-1}\mcalG$也为层。
容易验证对$Y$中的开集$V$,成立
$$(f^{-1}\mcalG)(f^{-1}(V))\cong\mcalG(V)$$
此外对任意$x\in X$,成立
$$(f^{-1}\mcalG)_x\cong\mcalG_{f(x)}\eqno{(*)}$$

容易验证$f^{-1}:\Shcat_Y\to\Shcat_X$为层范畴之间的函子。

\begin{rem}(逆像的层空间)

设$f:X\to Y$为拓扑空间之间的连续映射,$\mcalG$为$Y$上的层,
则有层空间的拓扑同胚
$$\widetilde{f^{-1}\mcalG}\cong X\times_Y\mcalGtil$$
也就是说,存在下述纤维积图表:
$$
  \xymatrix{
     \widetilde{f^{-1}\mcalG}  \ar[r]^\afa  \ar[d]
    &\mcalGtil                              \ar[d]
  \\
     X                         \ar[r]^f
    &Y
  }
$$
\end{rem}
其中映射$\afa$由$(*)$式诱导。由拓扑空间纤维积的具体构造,容易验证以上。

\begin{prop}(伴随对)

设$f:X\to Y$为拓扑空间之间的连续映射,
则$f^{-1}$为$f_*$的左伴随函子。也就是说对于任意$\mcalF\in\Shcat_X$
以及$\mcalG\in\Shcat_Y$,存在(关于$X,Y$)自然的一一对应
$$
  \Hom_{\Shcat_X}
    (f^{-1}\mcalG,\mcalF)
\xle{1\text{-}1}
  \Hom_{\Shcat_Y}
    (\mcalG,f_*\mcalF)
$$
\end{prop}

\begin{proof}[证明大意]
我们只给出此一一对应的构造,其余细节从略(反复使用各种泛性质)。
对于任意的
\begin{eqnarray*}
\psi:&\mcalG\to f_*\mcalF\\
\fai:&f^{-1}\mcalG\to\mcalF
\end{eqnarray*}

首先我们定义$\afa:
  \Hom_{\Shcat_Y}
    (\mcalG,f_*\mcalF)
\to
  \Hom_{\Shcat_X}
    (f^{-1}\mcalG,\mcalF)$如下:对$X$中开集$U$,
$[\afa(\psi)]_U$由以下交换图表给出:
$$
  \xybigcol
  \xymatrix{
     \mcalG(W)
       \ar[r]^-{\psi_W}
       \ar[d]^\rho
    &(f_*\mcalF)(W)=\mcalF(f^{-1}(W))
       \ar[d]^\rho
  \\
     \mcalG(V)
       \ar[r]^-{\psi_V}
       \ar[d]
    &(f_*\mcalF)(V)=\mcalF(f^{-1}(V))
       \ar[d]^{\rho}
  \\
     \rlim\limits_{V\supseteq f(U)}\mcalG(V)=(f^{-1}\mcalG)(U)
       \ar@{-->}[r]^-{[\afa(\psi)]_U}
    &\mcalF(U)
  }
$$
其中$W\supseteq V$为$Y$中的包含$f(U)$的开集。

再定义$\beta:  \Hom_{\Shcat_X}
    (f^{-1}\mcalG,\mcalF)
\to
  \Hom_{\Shcat_Y}
    (\mcalG,f_*\mcalF)$如下:对$Y$中的开集$V$,
$[\beta(\fai)]_V$由以下交换图表给出:
$$
  \xybigcol
  \xymatrix{
     (f^{-1}\mcalG)(f^{-1}(V))
       \ar@{=}[d]
       \ar[r]^-{\fai_{f^{-1}(V)}}
    &\mcalF(f^{-1}(V))
       \ar@{=}[d]
  \\
     \mcalG(V)
       \ar@{-->}[r]^-{[\beta(\fai)]_V}
    &(f_*\mcalF)(V)
  }
$$
其余细节从略。
\end{proof}

\section{局部自由模层与向量丛}

\begin{definition}($\mcalA$-模层)

设$\mcalA$为拓扑空间$X$上的(含幺交换)环层,$\mcalM$为$X$上的Abel群层,
称$\mcalM$为$\mcalA$-模层,如果对$X$的任何开集$V\supseteq U$,
$\mcalM(U)$具有$\mcalA(U)$-模结构$\mcalA(U)\times\mcalM(U)\to\mcalM(U)$,
并且下述图表交换:
$$
  \xymatrix{
     \mcalA(V)\times\mcalM(V)
       \ar[r]
       \ar[d]
    &\mcalM(V)
       \ar[d]
  \\
     \mcalA(U)\times\mcalM(U)
       \ar[r]
    &\mcalM(U)
  }
$$
\end{definition}

例如,考虑复流形$X$上的解析函数环层$\mcalO_X$,则全纯切向量场、全纯微分形式等等,
都可视为$\mcalO_X$-模层。再比如,环层$\mcalA$也有自然的$\mcalA$-模层结构。
一般地,对于拓扑空间$X$上的环层$\mcalA$,我们有
$X$上的$\mcalA$-模层范畴$\mcalA\modcat_X$,
自行定义此范畴中的态射“$\mcalA$-模层同态”。
能够验证,$\mcalA\modcat_X$为Abel范畴。

容易验证,对于$\mcalA$-模层$\mcalM$,则对任意$x\in X$,
茎条$\mcalM_x$有自然的$\mcalA_x$-模结构。

\begin{definition}(局部自由层)
\index{locally free sheaf\kong 局部自由层}

设$\mcalS$为拓扑空间$X$上的$\mcalA$-模层,
称$\mcalS$为\textbf{局部自由$\mcalA$-模层},
简称\textbf{局部自由层}(locally free sheaf),如果对任意$x\in X$,
存在$x$的开邻域$U$,使得有层同构
$$\mcalS|_U\cong(\mcalA|_U)^{\oplus r}$$
其中$r$为正整数,称为局部自由层$\mcalS$的秩。
\end{definition}

特别地,对任意$x\in X$,存在$x$的开邻域$U$,使得
$\mcalS(U)\cong(\mcalA(U))^{\oplus r}$
(但是定义中的“层限制”的语言更强)。
事实上$\mcalS$为局部自由层当且仅当对任意$x\in X$,
存在$x$的开邻域$U$,以及截面$F_{1,x},F_{2,x},...,F_{r,x}\in\mcalS(U)$,
使得对任意$y\in U$,环同态
\begin{eqnarray*}
     \mcalA_y^{\oplus r}&\to&\mcalS_y\\
     (w_1,w_2,...,w_r)
&\mapsto&\sum_{i=1}^{r}w_iF_{i,x}
\end{eqnarray*}
为同构。如此选取的$\Bigset{F_{i,x}\in\mcalA(U)}{1\leq i\leq r}$称为$\mcalS$
的一个\textbf{局部标架}。

\begin{notation}(局部自由层局部标架的转移函数)

设$\mcalS$为拓扑空间$X$上的秩为$r$的局部自由$\mcalA$-模层。
取$X$的一族开覆盖$X=\bigcup\limits_{\afa\in\mcalI}U_\afa$,
以及对于任意$\afa\in\mcalI$,取$\mcalS$在$U_\afa$上的局部标架
$$F_\afa:=\Bigset{F_\afa^i\in\mcalS(U_\afa)}{1\leq i\leq r}$$
则$F_\afa$自然诱导了层同构(仍记作$F_\afa$)
$$F_\afa:\mcalA|_{U_\afa}^{\oplus r}\xra{\sim}\mcalS|_{U_\afa}$$

对于$\afa,\beta\in\mcalI$,若$U_\afa\cap\mcalU_\beta\neq\vkong$,
则考虑如下图表:
$$
  \xymatrix{
     \mcalA|_{U_\afa\cap U_\beta}^{\oplus r}
       \ar[r]^{F_\afa}
    &\mcalS|_{U_\afa\cap U_\beta}
       \ar@{=}[d]
  \\
     \mcalA|_{U_\afa\cap U_\beta}^{\oplus r}
       \ar[r]^{F_\beta}
       \ar@{-->}[u]^{G_{\afa\beta}}
    &\mcalS|_{U_\afa\cap U_\beta}
  }
$$
称层自同构$G_{\afa\beta}:=F_\afa^{-1}\circ F_\beta$为
局部标架$F_\afa$与$F_\beta$之间的转移函数。
\end{notation}

对于$x\in U_\afa\cap U_\beta$,
$$(G_{\afa\beta})_x:\mcalA_x^{\oplus r}\to\mcalA_x^{\oplus r}$$
可以表达为在基$\Bigset{(F_\beta^i)_x}{1\leq i\leq r}$与
$\Bigset{(F_\afa^i)_x}{1\leq i\leq r}$下的矩阵,称此矩阵为\textbf{转移矩阵}。

对于$\afa,\beta,\gma\in\mcalI$,
如果$U_\afa\cap U_\beta\cap U_\gma\neq\vkong$,
则显然有
$
  \left\{
    \begin{array}{l}
      G_{\afa\afa}=\id_{\mcalA|_{U_\afa}^{\oplus r}}
    \\
      G_{\afa\beta}=G_{\beta\afa}^{-1}
    \\
      G_{\afa\beta}\circ G_{\beta\gma}\circ G_{\gma\afa}
      =\id_{\mcalA|_{U_{\afa\beta\gma}}^{\oplus r}}\\
    \end{array}
  \right.
$,其中$U_{\afa\beta\gma}:=U_\afa\cap U_\beta\cap U_{\gma}$.\vs

上述的语言与\textbf{向量丛}十分相似,事实上局部自由层是向量丛概念的推广。

\begin{Example}(拓扑向量丛)

设$X$为拓扑空间,$\mcalC_X$为$X$上的连续函数环层,则有自然的一一对应
$$
  \Big\{
    \text{$X$上的局部自由$\mcalC_X$-模层}
  \Big\}
\xle{\text{$1$-$1$}}
  \Big\{
    \text{$X$上的(拓扑)向量丛}
  \Big\}
$$
\end{Example}

\begin{proof}
若$\mcalE$为$X$上的局部自由$\mcalC_X$-模层,取$X$的一组局部标架覆盖
$X=\bigcup\limits_{\afa\in\mcalI}U_\afa$,以及$U_\afa$上的局部标架
$F_\afa=\Bigset{F_\afa^i}{1\leq i\leq r}$,
则对于任意的$\afa,\beta\in\mcalI$,若$U_\afa\cap U_\beta\neq\vkong$,
则对任意$x\in U_\afa\cap U_\beta$,转移函数$(G_{\afa\beta})_x$
在相应标架上的矩阵(仍记为$(G_{\afa\beta})_x$)给出了映射
\begin{eqnarray*}
     U_\afa\cap U_\beta &\to& \GL(r,\bbC)\\
     x&\mapsto& (G_{\afa\beta})_x
\end{eqnarray*}
易验证该映射连续,并且满足向量丛转移函数的相容条件,
从而这些转移函数可以粘合成一个向量丛。
反之,对于拓扑向量丛$E\to X$,
该向量丛的截面层显然为局部自由$\mcalC_X$-模层。
容易验证上述给出的对应是互逆的,从而得到一一对应。
\end{proof}

\begin{example}(全纯向量丛)

设$X$为复流形,$\mcalO_X$为$X$上的全纯函数环层,
则类似地有一一对应
$$
  \Big\{
    \text{$X$上的局部自由$\mcalO_X$-模层}
  \Big\}
\xle{\text{$1$-$1$}}
  \Big\{
    \text{$X$上的全纯向量丛}
  \Big\}
$$
\end{example}
光滑流形上的光滑向量丛也完全类似。\vs

最后,需要注意局部自由层范畴不是Abel范畴:

\begin{Example}(摩天大厦层)

考虑拓扑空间(复流形)$X=\bbC$,$X$上的局部自由$\mcalO_X$-模层
$\mcalS_1=\mcalS_2:=\mcalO_X$.考虑$\mcalO_X$-模层同态
$\fai:\mcalS_1\to\mcalS_2$为:对任意开集$U\subseteq X$,
\begin{eqnarray*}
\fai_U:\mcalS_1(U)&\to&\mcalS_2(U)\\
f(z)&\mapsto&zf(z)
\end{eqnarray*}
则其余核层$\coker\fai$\textbf{不是}局部自由$\mcalO_X$-模层。
\end{Example}
容易验证,对$X$中的开集$U$,成立
$
  \coker\fai(U)
\cong
  \left\{
    \begin{array}{ll}
      \bbC  &  (0\in U)\\
      0     &  (0\not\in U)
    \end{array}
  \right.
\index{skyscrapter sheaf\kong 摩天大厦层}
$,明显不是局部自由层。此层称为\textbf{摩天大厦层}(skyscraper sheaf)。

\section{凝聚层及其基本性质}

\begin{definition}(局部有限生成$\mcalA$-模层)

设$\mcalM$为拓扑空间$X$上的$\mcalA$-模层,称$\mcalA$是
\textbf{局部有限生成}的,若对任意$x\in X$,存在$x$的邻域$U$,
以及正整数$r$,使得有层同态短正合列
$$\mcalA|_U^{\oplus r}\surj\mcalM|_U\to 0$$
\end{definition}

或者等价地,存在$x$的开邻域$U$,以及截面$F_1,F_2,...,F_r\in\mcalM(U)$,
使得对任意$y\in U$,$\Bigset{(F_i)_y\in\mcalM_y}{1\leq i\leq r}$
是$\mcalM_y$的一组$\mcalA_x$-模生成元。

显然,局部自由层一定是局部有限生成的。

\begin{definition}(关系层)

设$\mcalM$是拓扑空间$X$上的$\mcalA$-模层,
对于$X$的开集$U$,以及$F_1,F_2,...,F_r\in\mcalM(U)$,
称层同态
\begin{eqnarray*}
     \fai:\mcalA|_U^{\oplus r} &\to&  \mcalM|_{U}\\
     (g_1,g_2,...,g_{r})&\mapsto&\sum_{i=1}^{r}g_iF_i
\end{eqnarray*}
的核层$\mcalR(F_1,F_2,...,F_r):=\ker\fai$为截面
$F_1,F_2,...,F_r$的\textbf{关系层}。
\end{definition}
这个定义当中并不要求$\fai$为层满同态,
也就是说$\mcalM$未必为局部有限生成的。
只要给定若干局部截面,就可以定义它们的关系层。

\begin{definition}(凝聚层)
\index{coherent sheaf\kong 凝聚层}

对于拓扑空间$X$上的$\mcalA$-模层$\mcalM$,称$\mcalA$为
\textbf{凝聚层}(coherent sheaf),如果:

(1)$\mcalA$为局部有限生成的;

(2)对$X$的任意开集$U$,以及
任意截面$F_1,F_2,...,F_r\in\mcalM(U)$,
关系层$\mcalR(F_1,F_2,...,F_r)$也是局部有限生成的。
\end{definition}

通过适当缩小$x\in X$的邻域$U$,容易验证$\mcalM$是凝聚层
一定是\textbf{局部有限呈示}的,即
对任意$x\in X$,存在$x$的开邻域$U$,以及正整数$p,q$,
使得存在$U$上的$\mcalA|_U$-模层正合列
$$
  \mcalA|_U^{\oplus p}\to
  \mcalA|_U^{\oplus q}\surj
  \mcalM|_U\to 0
$$
由定义容易知道,\textbf{凝聚层的局部有限生成子层也是凝聚的}。

此外,对于$X$上的交换环层$\mcalA$,
称$\mcalA$为局部有限生成的(切转:凝聚的),如果$\mcalA$作为$\mcalA$-模层
是局部有限生成的(切转:凝聚的)。

凝聚层的下列基本性质是纯线性代数的:

\begin{prop}(凝聚层的基本性质)

设$\mcalA$为拓扑空间$X$上的交换环层,$\mcalF,\mcalG$为
凝聚$\mcalA$-模层,$\fai:\mcalF\to\mcalG$为$\mcalA$-模层同态,则
$\ker\fai,\im\fai,\coker\fai$均为凝聚$\mcalA$-模层。
\end{prop}

\begin{proof}
显然$\im\fai$是局部有限生成的,
从而为凝聚层$\mcalG$的局部有限生成子层,故也为凝聚层。
再看$\ker\fai$作为凝聚层$\mcalF$的子层,
只需要说明$\ker\fai$是局部有限生成的。
对任意$x\in X$,由于$\mcalF$局部有限生成,取$x$的开邻域
$U$,以及截面$F_1,F_2,...,F_q\in\mcalF(U)$为$\mcalF|_U$的生成元,
于是有$\fai(F_1),\fai(F_2),...,\fai(F_q)\in\mcalG(U)$.
由$\mcalG$的凝聚性,取关系层$\mcalR(\fai(F_1),\fai(F_2),...,\fai(F_q))$
的一组生成元$G_1,G_2,...,G_r\in\mcalA(U)^{\oplus q}$,
其中$G_i=(G_i^1,G_i^2,...,G_i^q)$,即有以$\mcalA(U)$为系数的
矩阵$(G_i^j)$,其中$1\leq i\leq r,1\leq j\leq q$.
则容易验证$\Bigset{\sum\limits_{j=1}^qG_i^jF_j}{1\leq i\leq r}$
是$\ker\fai|_U$的一组生成元,因此$\ker\fai$是局部有限生成的,
进而由$\mcalF$的凝聚性知$\ker\fai$也是凝聚的。

再看$\coker\fai$的凝聚性。$\coker\fai$作为局部有限生成层$\mcalG$的商层,
显然也是局部有限生成的。然后对$X$的任意开集$U$,以及任意截面
$G_1,G_2,...,G_q\in\coker\fai(U)$,断言关系层
$\mcalR(G_1,G_2,...,G_q)\subseteq\mcalA|_U^{\oplus q}$是局部有限生成的。
对于任意$x\in U$,取$x$在$U$中的(足够小)邻域$U'$,
$G_i(1\leq i\leq q)$在$U'$上的限制仍记为$G_i$.
取截面$G_i\in\coker\fai(U')$在$\mcalG$中的代表元$\Gtil_i\in\mcalG(U')$,
再取$F_1,F_2,...,F_p\in\mcalF(U')$为$\mcalF|_{U'}$的生成元,
考虑关系层
$$
  \mcalR(F_1,...,F_p;\Gtil_1,...,\Gtil_q)
\subseteq\mcalA|_{U'}^{\oplus(p+q)}
$$
由$\mcalG$的凝聚性(不断缩小$U'$),取其一组生成元
$$
  \Bigset{H_i=(H_i^1,H_i^2,...,H_i^{p+q})\in\mcalA(U')^{\oplus(p+q)}}
  {1\leq i\leq r}
$$
则容易验证(纯线性代数,细节略)
$$
  \Bigset{\Htil_i=(\pi(H_i^{p+1}),...,\pi(H_i^{p+q}))\in\mcalA(U')^{\oplus q}}
  {1\leq i\leq r}
$$
是关系层$\mcalR(G_1,G_2,...,G_q)|_{U'}$的生成元
(其中$\pi:\mcalG\to\coker\fai$为典范投影),从而
关系层$\mcalR(G_1,G_2,...,G_q)$是局部有限生成的,因此$\coker\fai$凝聚。
\end{proof}

\begin{rem}对于拓扑空间$X$,$\mcalA$为$X$上的交换环层,
记$\mcalA\Cohcat_X$为$X$上的凝聚$\mcalA$-模层范畴,
这是$\mcalA\modcat_X$的子范畴。
上述性质表明$\mcalA\Cohcat_X$是Abel范畴。
\end{rem}


\begin{prop}设$\mcalA$为拓扑空间$X$上的交换环层,则
对于$\mcalA$-模层同态短正合列
$$
  0\to\mcalF_1\to\mcalF_2\to\mcalF_3\to 0
$$
此正合列中任何两个为凝聚层均可推出第三个也为凝聚层。
\label{凝聚短正合列-prop}
\end{prop}

\begin{proof}
  只需再证明$\mcalF_1,\mcalF_3$凝聚能推出$\mcalF_2$凝聚。
先断言$\mcalF_2$是局部有限生成的。对任意$x\in X$,
取$x$的(足够小的)开邻域$U$,并且取
$F_1,F_2,...,F_p\in\mcalF_1(U)$为$\mcalF_1|_U$的生成元,
再取$G_1,G_2,...,G_q\in\mcalF_3(U)$为$\mcalF_3|_U$的生成元。
将$G_j$在$F_2(U)$中的代表元记为$\Gtil_j\,(1\leq j\leq q)$,
则容易验证$\{F_1,F_2,...,F_p;\Gtil_1,\Gtil_2,...,\Gtil_q\}$为
$\mcalF_2|_U$的一组生成元。从而$\mcalF_2$是局部有限生成的。

对$X$的任意开集$U$,以及$S_1,S_2,...,S_r\in\mcalF_2(U)$,
断言关系层$\mcalR(S_1,S_2,...,S_r)$是局部有限生成的。
任取$x\in U$,记截面$S_1,...,S_r$在$\mcalF_3(U)$上的投影分别为
$\Sbar_1,...,\Sbar_r$.由$\mcalF_3$的凝聚性,
$\mcalR(\Sbar_1,...,\Sbar_r)$是局部有限生成的,从而取$x$在$U$中的
(足够小且不妨不断缩小的)开邻域$U'$,以及
$\mcalR(\Sbar_1,...,\Sbar_r)|_{U'}$的生成元矩阵
$$
   H
:=
  \begin{pmatrix}
    H_1^1  & \cdots & H_t^1 \\
    \vdots &        & \vdots\\
    H_1^r  & \cdots & H_t^r
  \end{pmatrix}
\in
  \mcalA(U')^{r\times t}
$$
即每个$H_j^i\in\mcalA(U')$,$H$中的列向量$\in\mcalR(\Sbar_1,...,\Sbar_r)(U')$,
矩阵$H$的$r$个列向量构成$\mcalR(\Sbar_1,...,\Sbar_r)|_{U'}$的生成元。再令
$$
  (F_1,F_2,...,F_t)
:=(S_1,S_2,...,S_r)H
$$
则易验证$(F_1,F_2,...,F_t)\in\mcalF_1(U')^{\oplus t}$.
由$\mcalF_1$的凝聚性,取$\mcalR(F_1,...,F_t)|_{U'}$的生成元矩阵
$$
   K
:=
  \begin{pmatrix}
    K_1^1  & \cdots & K_s^1 \\
    \vdots &        & \vdots\\
    K_1^t  & \cdots & K_s^t
  \end{pmatrix}
\in
  \mcalA(U')^{t\times s}
$$
则容易验证$HK$为$\mcalR(S_1,...,S_r)|_{U'}$的生成元矩阵,从而
$\mcalR(S_1,...,S_r)$是局部有限生成的。

综上所述,若$\mcalF_1$与$\mcalF_3$凝聚,则$\mcalF_2$也凝聚。
\end{proof}

\begin{cor}设$\mcalF$是拓扑空间$X$上的凝聚$\mcalA$-模层,则

(1)任意$n\geq 1$,$\mcalF^{\oplus n}$也是凝聚$\mcalA$-模层;

(2)对$X$的任意开集$U$,以及任意$F_1,F_2,...,F_p\in\mcalF(U)$,
则关系层$\mcalR(F_1,F_2,...,F_p)$也是凝聚的($\mcalA|_U$-模层)。
\end{cor}

\begin{proof}
\textbf{(1)}注意短正合列
$0\to\mcalF\to\mcalF\oplus\mcalF^{\oplus(n-1)}\to\mcalF^{\oplus(n-1)}\to 0$,
反复利用性质\ref{凝聚短正合列-prop}作归纳即可。

\textbf{(2)}由(1)知$(\mcalF|_U)^{\oplus p}$是凝聚的,
因此$\mcalR(F_1,F_2,...,F_p)$作为$(\mcalF|_U)^{\oplus p}$
的局部有限生成子层,也是凝聚的。
\end{proof}

\begin{cor}若拓扑空间$X$上的$\mcalA$-模层$\mcalM$是凝聚的,
并且$\mcalM$的子层$\mcalF,\mcalG$也是凝聚的,那么
$\mcalF\cap\mcalG$也为凝聚$\mcalA$-模层。
\end{cor}

\begin{proof}
考虑层同态$\fai:\mcalF\to\mcalM/\mcalG$为如下复合:
$$\mcalF\inj\mcalM\surj\mcalG$$
注意$\mcalF$与$\mcalM/\mcalG$都是凝聚的,再注意
$
  \mcalF\cap\mcalG\cong\ker\fai
$,因此$\mcalF\cap\mcalG$也凝聚。
\end{proof}

\section{Oka凝聚定理}
本节介绍多复变函数论、复几何中的重要结果:
\textbf{对于复流形$X$,解析函数环层$\mcalO_X$是凝聚层}。
这也是凝聚层的重要例子。注意凝聚性是局部性质,于是我们不妨$X=\bbC^n$.
我们只需要证明,对$\bbC^n$的任意开子集$U$,以及任意
$F_1,F_2,...,F_q\in\mcalO_X(U)$,
关系层$\mcalR(F_1,F_2,...,F_q)$是局部有限生成的。

现在,对任意$x\in X$,由于$\mcalO_{X,x}$为诺特环,从而
$\mcalR(F_1,...,F_q)_x\subseteq\mcalO_{X,x}^{\oplus q}$
为有限生成$\mcalO_{X,x}$-模。但这与希望要证的“$\mcalR(F_1,...,F_q)$局部有限生成”
还差些东西。我们暂时只能说明存在$x$的邻域$U'\subseteq U$,
以及有限多个$\mcalO_X^{\oplus q}$在$U'$的截面,使得它们在$x$的芽
生成$\mcalR(F_1,...,F_q)_x$;但我们希望\textbf{对$x$附近的任何点$y$},
这些截面在$y$处的芽也生成$\mcalR(F_1,...,F_q)_y$——这是不显然的。

\begin{lemma}(重要引理)
\label{Oka凝聚-重要引理-lem}

对于$n\geq 2$,记$\bbC^n=\Bigset{(z',z_n)}
{z'=(z_1,...,z_{n-1})\in\bbC^{n-1},z_n\in\bbC}$,
设$F_1,F_2,...,F_q$为定义在$(0,0)\in\bbC^n$附近的解析函数,
则存在$(0,0)$的邻域$\yc:=\yc'\times\yc_n$,
其中$\yc'$与$\yc_n$分别为$\bbC^{n-1}$与$\bbC$中的以原点为中心的多圆柱,
使得对任意$w=(w',w_n)\in\yc$,
$
  \Bigset{
           (K^1,K^2,...,K^q)
           \in\mcalO_{\yc,w}^{\oplus q}
         }
         {K^j\in\mcalK,\,\forall\,1\leq j\leq q}
$是
$\mcalO_{\yc,w}$-模$\mcalR(F_1,F_2,...,F_q)_w$
的一组生成元,其中
$$\mcalK:=\Bigset{f(z',z)\in\mcalO_{\yc',w'}[z_n]}{\deg_{z_n}f\leq\mu}$$
$$
  \mu:=
  \max\Bigset{\Ord_{z_n}(F_{k})_0}{1\leq k\leq q}
$$
\end{lemma}

\begin{proof}
对$F_1,F_2,...,F_q$在原点处使用Weierstrass预备定理,
适当乘以原点附近的可逆解析函数
(不会改变$\mcalR(F_1,...,F_q)$在原点的足够小邻域的限制),
不妨设$F_1,...,F_q\in\mcalO_{\yc',0}[z_n]$为定义在原点附近的
关于$z_n$的Weieretrass多项式。此外,不妨
$$
  \deg_{z_n}F_q=\mu
$$

\textbf{Step 1}
对于$w=(w',w_n)\in\yc$,关于$z_n$的Weierstrass多项式$F_q$
(通过平移)自然也视为关于$(z_n-w_n)$的Weierstrass多项式(次数仍为$\mu$)。
对$F_q$在$w$处使用Weierstrass预备定理,
令$F_q=f'f''$,其中$f'\in\mcalO_{\yc',w'}[z_n]$
为关于$(z_n-w_n)$的Weierstrass多项式,
$f''\in\mcalO_{\yc,w}$在$w$附近可逆。
注意$F_q$与$f'$都为Weierstrass多项式,
从而由引理\ref{Weierstrass多项式整除引理-lemma}可知
$f''\in\mcalO_{\yc',w'}[z_n]$为关于$(z_n-w_n)$的多项式。
分别记$\mu',\mu''$为多项式$f',f''$关于$z_n$的次数,则$\mu=\mu'+\mu''$.\vs

\textbf{Step 2}
我们习惯将$\mcalR(F_1,F_2,...,F_q)_w$中的元素记成列向量。
对于任意的
$
  \begin{pmatrix}
    g^1 \\
    g^2 \\
    \vdots \\
    g^q
  \end{pmatrix}
\in
  \mcalR(F_1,F_2,...,F_q)_w
$,对于$1\leq j\leq q-1$,将$g^j$除以Weierstrass多项式$F_{q,w}$,
由Weierstrass除法定理,得
$$
  g^j=F_{q,w}T^j+R^j
\qquad
  (1\leq j\leq q-1)
$$
其中$T^j\in\mcalO_{\yc ,w}$以及$R^j\in\mcalO_{\yc',w'}[z_n]$,
且$\deg_{z_n}R^j<\mu'$.
而对于$j=q$,令
$$
  R^q
:=
  g^q+\sum_{j=1}^{q-1}F_{j,w}T^j
$$
则容易验证
$$
  \begin{pmatrix}
    g^1 \\
    g^2 \\
    \vdots \\
    g^q
  \end{pmatrix}
=
  \begin{pmatrix}
    F_{q,w} &   &   &   \\
      & F_{q,w} &   &   \\
      &   & \ddots  &   \\
      &   &   & F_{q,w} \\
    -F_{1,w} & -F_{2,w} & \cdots & -F_{q-1,w}
  \end{pmatrix}
  \begin{pmatrix}
    T^1 \\
    T^2 \\
    \vdots \\
    T^{q-1}
  \end{pmatrix}
+
  \begin{pmatrix}
    R^1 \\
    R^2 \\
    \cdots \\
    R^q
  \end{pmatrix}
=
  \begin{pmatrix}
    F_{q,w} &  &  & R^1 \\
      & \ddots &   & \vdots \\
      &   & F_{q,w} & R^{q-1} \\
    -F_{1,w} & \cdots & -F_{q-1,w} & R^q
  \end{pmatrix}
  \begin{pmatrix}
    T^1 \\
    \vdots \\
    T^{q-1} \\
    1
  \end{pmatrix}
$$

\textbf{Step 3}
我们得到了$q$阶方阵
$
  G:=
    \begin{pmatrix}
    F_{q,w} &  &  & R^1 \\
      & \ddots &   & \vdots \\
      &   & F_{q,w} & R^{q-1} \\
    -F_{1,w} & \cdots & -F_{q-1,w} & R^q
  \end{pmatrix}
$.容易验证$G$的每一列都位于$\mcalR(F_1,F_2,...,F_q)_w$之中;
并且除了第$(q,q)$-分量$G^q_q=R_q$,$G$的其余矩阵元都位于$\mcalK$中,
即为次数不超过$\mu$的关于$z_n$的$\mcalO_{\yc',w'}$-系数的多项式。
最后,我们适当调整矩阵$G$的最后一列。

注意到$G$的第$q$列位于$\mcalR(F_1,F_2,...,F_q)_w$之中,以及$F_q=f'f''$,从而
$$
  \sum_{j=1}^{q-1}
    F_{j,w}R^j
  + f'f''R^q
=0
$$
注意$\deg_{z_n}\left(
  \sum\limits_{j=1}^{q-1}
    F_{j,w}R^j
\right)<\mu+\mu'$,因此
$f'f''R^q\in\mcalO_{\yc',w'}[z_n]$并且
$\deg_{z,n}(f'f''R^q)<\mu+\mu'$.
又因为$f'$是关于$z_n$的次数为$\mu'$的Weierstrass多项式,
从而由引理\ref{Weierstrass多项式整除引理-lemma}可知,
$f''R^q\in\mcalO_{\yc',w'}[z_n]$,并且$\deg_{z_n}(f''R^q)<\mu$.
从而考虑
$$
  \begin{pmatrix}
    g^1 \\
    g^2 \\
    \vdots \\
    g^q
  \end{pmatrix}
=
  \begin{pmatrix}
    F_{q,w} &  &  & f''R^1 \\
      & \ddots &   & \vdots \\
      &   & F_{q,w} & f''R^{q-1} \\
    -F_{1,w} & \cdots & -F_{q-1,w} & f''R^q
  \end{pmatrix}
  \begin{pmatrix}
    T^1 \\
    \vdots \\
    T^{q-1} \\
    1/f''
  \end{pmatrix}
$$
易知上式中的矩阵的每个矩阵元都位于$\mcalK$,
并且每一列都位于$\mcalR(F_1,F_2,...,F_q)_w$,
因此
$
  \begin{pmatrix}
    g^1 \\
    \vdots \\
    g^n
  \end{pmatrix}
$由上述矩阵($q$个列向量)$\mcalO_{\yc,w}$-生成。从而证毕。
\end{proof}

\begin{thm}(Oka凝聚定理)

对于复流形$X$,$X$上的解析函数环层$\mcalO_X$是凝聚的。
\end{thm}

\begin{proof}
如之前所述,不妨$X=\bbC^n$,以及对于任意开集$U\subseteq\bbC^n$以及
任意$F_1,...,F_q\in\mcalO_{\bbC^n}(U)$,我们不妨$U$是
以原点为中心的多圆柱区域,不妨$F_1,...,F_q$为关于$z_n$的Weierstrass多项式。

对$X=\bbC^n$的维数$n$归纳。$n=0$时平凡。对于$n\geq 1$,
如果$\mcalO_{\bbC^{n-1}}$是凝聚的,则对于$(0,0)\in\bbC^{n-1}\times\bbC$
的多圆柱邻域$\yc=\yc'\times\yc_n$,以及
$F_1,F_2,...,F_q\in\mcalO_{\yc'}[z_n]$为Weierstrass多项式,
它们关于$z_n$的最高次数记为$\mu$.只需证$\mcalF(F_1,F_2,...,F_q)$局部有限生成。
对于任意的$w\in\yc$,以及
$
  \begin{pmatrix}
    g^1 \\
    \vdots \\
    g^q
  \end{pmatrix}
\in\mcalR(F_1,F_2,...,F_q)_w
$,由重要引理\ref{Oka凝聚-重要引理-lem}
可知,存在$q\times(\mu+1)$矩阵
$U=(U^j_\afa)_{1\leq j\leq q\atop 0\leq\afa\leq\mu}$,使得
$$
  \begin{pmatrix}
    g^1 \\
    \vdots \\
    g^q
  \end{pmatrix}
=
  \begin{pmatrix}
    U^1_0   & \cdots & U^1_\mu \\
    \vdots  &        & \vdots  \\
    U^q_0   & \cdots & U^q_\mu
  \end{pmatrix}
  \begin{pmatrix}
    z_n^0   \\
    \vdots  \\
    z_n^\mu
  \end{pmatrix}
$$
其中$U^j_\afa\in\mcalO_{\yc',w'}$,
视为定义在$w'\in\yc'$附近的解析函数,自然也视为定义在$w\in\yc$附近的
(不显含$z_n$的)解析函数。注意$F_k\in\mcalO_{\yc'}[z_n]$也为关于$z_n$的
(次数不超过$\mu$的)(Weierstrass)多项式,从而
$$
  (F_1,\cdots F_q)
=
  (z_n^0,\cdots z_n^\mu)
  \begin{pmatrix}
    H^0_1   & \cdots & H^0_q  \\
    \vdots  &        & \vdots \\
    H^\mu_1 & \cdots & H^\mu_q
  \end{pmatrix}
$$
即得$(\mu+1)\times q$的矩阵$H$,$H$的每个矩阵元都位于$\mcalO_{\yc'}$之中,
当然也是定义在$w\in\yc$附近的(不显含$z_n$的)解析函数。
注意到
$$
  0
=
  (F_1,\cdots,F_q)
  \begin{pmatrix}
    g^1   \\
    \vdots \\
    g^q
  \end{pmatrix}
=
  (z_n^0,\cdots z_n^\mu)
  HU
  \begin{pmatrix}
    z_n^0   \\
    \vdots  \\
    z_n^\mu
  \end{pmatrix}
=:
  \sum_{k=0}^{2\mu}L_k(U)z_n^k
$$
因此比较$z_n$各次幂的系数,知$L_k(U)=0,\,\forall0\leq k\leq 2\mu$.\vs

我们将矩阵$U$视为层$\mcalO_{\yc'}^{\oplus q(\mu+1)}$
在$w'$附近的截面,对于$0\leq k\leq2\mu$,$L_k$为层同态
$$
  L_k:\mcalO_{\yc'}^{\oplus q(\mu+1)}|_{\Omg'}
\to
  \mcalO_{\yc'}|_{\Omg'}
$$
并且$L_k$只与$F_1,F_2,...,F_q$有关。其中$\Omg'$为$w'$
在$\yc'$中的(足够小、不断缩小的)邻域。\vs

由归纳假设,$\mcalO_{\yc'}$是凝聚的,因此
$\mcalO_{\yc'}^{\oplus q(\mu+1)}$也凝聚,因此对任意$0\leq k\leq 2\mu$,
核层$\ker L_k$也凝聚,从而$\bigcap\limits_{k=0}^{2\mu}\ker L_k$凝聚,
故局部有限生成。因此存在截面
$U_1,U_2,...,U_N\in\mcalO_{\yc'}^{\oplus q(\mu+1)}(\Omg')$,使得
$\{U_1,U_2,...,U_N\}$为$\mcalO_{\yc'}^{\oplus q(\mu+1)}|_{\Omg'}$的
子层$\bigcap\limits_{k=0}^{2\mu}\ker L_k$的生成元。
其中对于$1\leq l\leq N$,$U_l$为$q\times(\mu+1)$矩阵,
其矩阵元取值于$\mcalO_{\yc'}(\Omg')$.

容易验证,以下$N$个$q$维列向量
$$
  \Bigset{U_l
           \begin{pmatrix}
             z_n^0  \\
             \vdots \\
             z_n^\mu
           \end{pmatrix}
         }
         {1\leq l\leq N}
$$
构成关系层$\mcalR(F_1,...,F_q)|_{\Omg'}$的一组生成元。
这就证明了$\mcalO_{\yc}$的凝聚性,证毕。
\end{proof}


\section{层的上同调}
%$X$ a topological space, $\mcalF$- sheaf (of abelian groups).

本节开始,我们讨论拓扑空间$X$上的$\mcalA$-模层,即考虑范畴$\mcalA\modcat_X$.
先简单回顾一些同调代数的记号、结论。对于$X$上的$\mcalA$-模层$\mcalF$,
$\mcalF$的\textbf{消解}(resolution)\index{resolution\kong 消解}
是指形如下述的$\mcalA\modcat_X$中的正合序列:
$$
0\to \mcalF\xra{j}\mcalG_0\xra{d^0}\mcalG_1\xra{d^1}\cdots
$$
%\begin{definition}  (resolution)
%(1)a resolution of $\mcalF$ is an exact sequence
%$$0\to \mcalF\xra{j}\mcalF\xra{d^0}\mcalF\xra{d^1}\to\cdots$$
%\end{definition}

注意范畴$\Abcat$是Abel范畴,
因此我们可以考虑该范畴中的\textbf{内射对象}(injective object),
即“\textbf{内射层}”(injective sheaf)。
\index{injective sheaf\kong 内射层}
具体地,Abel群层$\mcalF$是内射的,若
对任意的层单同态$i:\mcalF\inj\mcalG$,
以及任意的层同态$\fai:\mcalF\to\mcalH$,都存在
层同态$\psi:\mcalG\to\mcalH$,使得$\fai=\psi\circ i$,
如下图:
$$
  \xymatrix{
     \mcalF
       \ar@{^(->}[r]^i
       \ar[rd]_{\fai}
    &\mcalG
       \ar@{-->}[d]^{\exists\,\psi}
  \\
    &\mcalH
  }
$$
我们承认以下事实:

\begin{thm}对于拓扑空间$X$,以及$X$上的交换环层$\mcalA$,
范畴$\mcalA\modcat_X$是\textbf{足够内射}的,
即对于$X$上任意的$\mcalA$-模层$\mcalF$,都存在内射层$\mcalI$,
以及层单同态$\mcalF\inj\mcalI$.
\end{thm}
由同调代数,容易知道$\mcalA\modcat_X$足够内射,
当且仅当对任何$\mcalA$-模层$\mcalF$,
都存在$\mcalF$的\textbf{内射消解}(injective resolution)
\index{injective resolution\kong 内射消解}
$$
  0\to\mcalF\inj\mcalI^0
  \to\mcalI^1\to\mcalI^2\to\cdots
$$
即上述序列正合,并且$\mcalI_k$为内射层($\forall k\geq 0$)。

%\begin{thm}
%the category of sheaves of abelian sheaves have enough
%injective objects, i.e.  any $\mcalF$ can be
%embedded in some injective sheaf.
%\end{thm}

\begin{definition}(层的上同调)

对于拓扑空间$X$上的$\mcalA$-模层$\mcalF$,
任取$\mcalF$的一个内射消解$0\to\mcalF\inj\mcalI\updot$:
$$0\to\mcalF\to\mcalI^0\xra{\td}\mcalI^1\xra{\td}\mcalI^2\to\cdots$$
将函子$\Gma(X,-)$(见注记\ref{取截面函子左正合-rem})作用于其上,
诱导了如下的$\mcalA(X)$-模上链复形
$$0\to\Gamma(X,\mcalF)
\inj\Gamma(X,\mcalI^0)\xra{\td}
\Gamma(X,\mcalI^1)\xra{\td}
\Gamma(X,\mcalI^2)\to\cdots$$
定义$\mcalF$的第$q$阶上同调群
$$H^q(X,\mcalF):=H^q(\Gamma(X,\mcalI\updot))$$
\end{definition}
由函子$\Gma(X,-)$的左正合性可知,
$$
  H^0(X,\mcalF)
=
  \frac{\ker(\td:\Gma(X,\mcalI^0)\to\Gma(X,\mcalI^1))}
       {\im(0\to\Gma(X,\mcalI^0))}
\cong
  \im(\Gma(X,\mcalF)\inj\Gma(X,\mcalI^0))
\cong \Gma(X,\mcalF)=\mcalF(X)
$$
即为$\mcalF$的整体截面。

\begin{rem}
由同调代数的有关知识,上述$H^q(X,\mcalF)$是良定的,
与$\mcalF$的内射消解无关。
\end{rem}

\begin{definition}(松弛层)
\index{flabby\kong 松弛(层)}

称拓扑空间$X$上的$\mcalA$-模层$\mcalS$是\textbf{松弛的}
(flabby 或 flasque),如果对$X$的任意开集$U$,限制同态
$\mcalS(X)\to\mcalS(U)$是满射。
\end{definition}

对于$\mcalA$-模层$\mcalF$,望文生义,
$\mcalF$的\textbf{松弛消解}(flabby resolution)
是指层正合列
$$
  0\to\mcalF\inj\mcalS^0
  \to\mcalS^1\to\mcalS^2\to\cdots
$$
其中每个$\mcalS^k(k\geq 0)$都是松弛层。

我们承认如下事实:

\begin{thm}对于$X$上的$\mcalA$-模层$\mcalF$,
若$0\to\mcalF\inj\mcalS\updot$为$\mcalF$的一个松弛消解,
则成立
$$
  H\updot(X,\mcalF)\cong H^q(\Gma(X,\mcalS\updot))
$$
特别地,若$\mcalF$为松弛层,则$H^q(X,\mcalF)=0$对任意$q\geq 1$成立。
\end{thm}

也就是说,我们可以利用松弛消解来计算层上同调。
然而,松弛消解一定存在吗?答案是肯定的。

\begin{notation}(典范松弛层)

设$\mcalF$为$X$上的$\mcalA$-模层,对于$X$的开集$U$,记
$$
  \God(\mcalF)(U)
:=
  \Bigset
  {f:U\to\coprod_{x\in U}\mcalF_x}
  {f(x)\in\mcalF_x,\,\forall x\in U}
$$
则$\God(\mcalF)$为$X$上的松弛层,并且有典范的层单同态
$$j:\mcalF\inj\God(\mcalF)$$
称$\God(\mcalF)$为关于$\mcalF$的\textbf{典范松弛层},
也称为\textbf{Godement构造}。
\end{notation}

\begin{proof}
$\God(\mcalF)$的松弛性几乎显然。
典范同态$j:\mcalF\inj\God(\mcalF)$如下给出:
对$X$的任意开子集$U$,
\begin{eqnarray*}
  j(U):\mcalF(U)&\to&\God(\mcalF)(U)\\
  s&\mapsto&(x\mapsto s_x)
\end{eqnarray*}
易知如此定义的$j$是层单同态。
\end{proof}

也就是说$\mcalA\modcat_X$中的任何对象都是某个松弛层的子层,即
“足够松弛”。从而由同调代数的标准技术
(与“足够内射$\iff$存在内射消解”完全一样)可知,
$X$上的任何$\mcalA$-模层都存在松弛消解。

%the two definitions of Sheaf Cohomology are isomorphic.
%Godement's construction
%$$God(\mcalF)(U):=
%\{f:U\to\bigcup_{x\in U}\mcalF_x|
%f(y)\in\mcalF_y,\forall y\in U\}
%:=\prod_{x\in U}\mcalF_x$$
%$God(\mcalF)$ is a sheaf, and it is flabby. and there is a canonical
%morphism $\mcalF(U)\to God(F)(U)$ by $x\mapsto(x\mapsto s_x)$ is injective.
%$$\mcalF^0:=God(\mcalF)$$
%$$0\to\mcalF\xra{j}\mcalF^0\surj\coker(j)=\mcalF^0\big/\mcalF$$
%and consider
%$$\mcalF^1:=God(\coker(j))$$
%......then construct by induction... this is a flabby resolution of $\mcalF$.

在一些更特殊的情形下,我们可以去计算某些层的上同调。
回顾:拓扑空间$X$称为\textbf{仿紧}(paracompact)的
\index{paracompact\kong 仿紧},
如果$X$是Hausdorff的,并且$X$的任何开覆盖都存在局部有限开加细。
众所周知,度量空间都是仿紧的,紧空间都是仿紧的,仿紧空间都是正规的。

\begin{definition}(单位分解环层)%(resolution by fine sheaves)
\index{fine sheaf}

设$X$为仿紧空间,$\mcalA$为$X$上的交换环层,
称$\mcalA$为\textbf{fine sheaf},
若对$X$的任何开覆盖$\mcalU=\Bigset{\U_\afa}{\afa\in\mcalI}$,
存在一族截面$f_\afa\in\mcalA(X)$,使得
$$
  \supp(f_\afa)
:=
  \overline{\Bigset{x\in X}{(f_\afa)_x\neq 0}}
\subseteq U_\afa
$$
$$
  \sum_{\afa\in\mcalI}
    f_\afa
\equiv 1\in\mcalA(X)
$$
其中上述求和是局部有限的。
\end{definition}

%$\mcalA$ is a sheaf of ring,
%$X$ is a paracompact topological space, $\mcalA$
%is called a fine sheaf, if for any open covering
%$$X=\bigcup_{\alpha}V_{\alpha}\quad,\mcalV:=\{V_{\alpha}\}$$
%there exists a partition of unit subordinate to $\mcalV$,
%(i.e. $\exists f_{\alpha}\in\mcalA(V_{\alpha}),supp(\alpha)
%:=\overline{\{x\in V_{\alpha}|f_{\alpha,x}\neq 0\}}\subseteq V_{\alpha}$, and
%$\sum_{\alpha}f_{\alpha}=1$(the sum is locally finite) )

Fine sheaf的典型例子是光滑流形上众所周知的单位分解定理:
 
\begin{example}(单位分解定理)

设$X$为光滑流形,$\mcalC^\infty$为$X$上的光滑函数环层,
则$\mcalC^\infty$是fine sheaf。
\end{example}

%$X$ is a differential manifold,
%$\mcalC^{\infty}$ is the sheaf of smooth functions,
%then $\mcalC^{\infty}$ is a fine sheaf.

事实上,若仿紧空间$X$上的环层$\mcalA$是fine sheaf,
则任何$\mcalA$-模层都是上同调平凡的:

\begin{thm}
设仿紧空间$X$上的环层$\mcalA$为fine sheaf,
则对于任意$\mcalA$-模层$\mcalF$,
$$H^q(X,\mcalF)=0\qquad (\forall q\geq 1)$$
\end{thm}
%$\mcalS$ is a sheaf of $\mcalA$-modules,
%$\mcalA$ is a fine sheaf. then for any $q\geq 1$,
%$$H^q(X,\mcalS)=0$$

\begin{proof}
任取$\mcalF$的内射消解$0\to\mcalF\inj\mcalI\updot$,
其中$\td^k:\mcalI^k\to\mcalI^{k+1}\,(\forall k\geq 0)$,
则对任意$q\geq 1$,有
$$
  H^q(X,\mcalF)
\cong
  \frac{\ker(\td^{q}_X:\mcalI^{q}(X)\to\mcalI^{q+1}(X))}
       {\im(\td^{q-1}_X:\mcalI^{q-1}(X)\to\mcalI^q(X))}
$$
而对于任意截面$s\in\ker\td^q_X\subseteq\mcalI^q(X)$,
由$\cdots\to\mcalI^{q-1}\to\mcalI^q\to\mcalI^{q+1}\to\cdots$在$\mcalI^q$
处的正合性可知,存在$X$的开覆盖$\mcalU=\Bigset{U_\afa}{\afa\in\mcalJ}$,
以及$s'_\afa\in\mcalI^{q-1}(U_\afa)$,使得
$\td^{q-1}_{U_\afa}(s'_\afa)=s|_{U_\afa}$.

由于$\mcalA$为fine sheaf,从而取
$f_\afa\in\mcalA(X)$,使得$\supp(f_\afa)\subseteq U_\afa$,
并且$\sum\limits_{\afa\in\mcalJ}f_\afa=1$
为局部有限和。从而有
$$
  s':=
  \sum_{\afa\in\mcalJ}
    f_\afa s_\afa'
\in\mcalI^{q-1}(X)
$$
并且$\td^{q-1}_Xs'=s$.这表明$H^q(X,\mcalF)=0$.
\end{proof}

\begin{cor}设$X$为光滑流形,$E\to X$为$X$上的光滑向量丛,
自然也视为$X$上的$\mcalC^\infty$-模层。则对任意$q\geq 1$,
$$H^q(X,E)=0$$
\end{cor}
看来光滑流形上“常见的”层的上同调都是平凡的。

%Consider a flabby(or injective) resolution
%$$0\to\mcalS\xra{j}\mcalI^0\xra{\td}\mcalI^1\xra{\td}\mcalI^2\cdots$$
%where any $\mcalI^k(k\geq 0)$ is a sheaf of $\mcalA$-modules.by definition,
%$$H^q(X,m\mcalS):=\frac{\ker\td:\Gamma(\mcalI^q)\to\Gamma(\mcalI^{q+1})}
%                       {\Im\td:\Gamma(\mcalI^{q-1})\to\Gamma(\mcalI^{q})}$$
%Let $\alpha\in\ker\{\td:\Gamma(\mcalI^q)\to\Gamma(\mcalI^{q+1})\}$
%by the exactness of resolution, $\exists$ an open covering $\mcalU=(U_{i})_{i}$,
%s.t. $\alpha|_{U_{i}}=\td\beta_{i}$where $\beta_{i}\in\mcalT^{q-1}(U_{i})$.
%Let $(\beta_{i})_{i}$ be the partition of unit w.r.t. $\mcalU$.consider
%$$\beta:=\sum_{i}f_i\beta_i$$(well defined). Then $\td\beta=\alpha$....

%\begin{rem}
%we call $\mcalA$ is a \textbf{soft sheaf},
%if for any closed set $K\subseteq X$,
%the morphism
%$$\mcalA(X)\to \mcalA(K)$$
%is surjective.
%where $\mcalA(K):=\Gamma(K,\mcalA|_K)$
%\end{rem}
%fact: $\mcalA$ is fine if and only if
%$\mcalH om(\mcalA,\mcalA)$ is soft.
%(omit)

\section{\Cech 上同调}

设$X$为拓扑空间,$\mcalA$为$X$上的交换环层,
$\mcalF$为$\mcalA$-模层。对于$X$的开覆盖
$\mcalU=\Bigset{U_\afa}{\afa\in\mcalI}$,我们记
$$
  U_{\afa_0\afa_1\cdots\afa_k}
:=
  \bigcap_{j=0}^k
    U_{\afa_j}
\qquad
(\forall \afa_0,...,\afa_k\in\mcalI)
$$

%$X$- a topological space, $\mcalF$- a sheaf of abelian group.
%$$\mcalU=(U_{\alpha})_{\alpha\in I}$$
%is an open covering.
%notation:$U_{\alpha_1,...,\alpha_q}:=\bigcap_{i=1}^qU_{\alpha_i}$.

\begin{definition}(\Cech 上同调)

记号同上,并且给定开覆盖$\mcalU$的指标集$\mcalI$
上的一个良序$\preceq$,则对任意$q\geq 0$,记
$$
  C^q(\mcalU,\mcalF)
:=
  \prod_{(\afa_0,...,\afa_q)\in\mcalI^{q+1}
         \atop \afa_0\prec\afa_1\prec\cdots\prec\afa_q
        }
    \mcalF(U_{\afa_0\cdots\afa_q})
$$
对于$c\in C^q(\mcalU,\mcalF)$,记$c$
的$\mcalF(U_{\afa_0\cdots\afa_q})$-分量为$c_{\afa_0\cdots\afa_q}$.
再定义$\delta^q:C^q(\mcalU,\mcalF)\to C^{q+1}(\mcalU,\mcalF)$为:
对任意$c\in C^q(\mcalU,\mcalF)$,
$$
  (\delta^q(c))_{\afa_0\afa_1\cdots\afa_{q+1}}
:=
  \sum_{k=0}^{q+1}
    (-1)^k
    c_{\afa_0\cdots\widehat{\afa_k}\cdots\afa_{q+1}}
    |_{U_{\afa_0\cdots\afa_{q+1}}}
$$
则容易验证$\Bigset{\delta^q:C^q(\mcalU,\mcalF)\to C^{q+1}(\mcalU,\mcalF)}
{q\geq 0}$为上链复形,称之为\textbf{\Cech 上链复形},
相应的上同调
$$
  \check{H}\updot(\mcalU,\mcalF)
:=H\updot(C\updot(\mcalU,\mcalF))
$$
称为$\mcalF$关于开覆盖$\mcalU$的\textbf{\Cech 上同调}.
\index{\Cech 上同调}
\end{definition}

容易验证上述定义的$\delta\updot$满足$\delta^2=0$,
从而$(C\updot(\mcalU,\mcalF),\delta\updot)$的确为上链复形。

此外由定义容易看出,若$\mcalU$为有限覆盖,$|\mcalI|=n<+\infty$,
那么$C^n(\mcalU,\mcalF)=0$,并且对任意$q\geq n$有
$\check{H}^q(\mcalU,\mcalF)=0$.

%\u{C}ech $q$-chain w.r.t $\mcalU$:
%$$C^q(\mcalU,\mcalF):=\prod_{(\alpha_1,...,\alpha_q)
%\in\mcalI^{q+1}}\mcalF(U_{\alpha_1,...,\alpha_q})$$
%$$c\in C^q(\mcalU,\mcalF)$$means that we have a family of sections
%$C_{\alpha_1,...,\alpha_q}\in\mcalF(U_{\alpha_1,...,\alpha_q})$
%with the relation$$C_{\alpha_0,...,\alpha_j,...,\alpha_i,...}=-C_{...}$$
%\u(C)ech differential:
%$$\delta^q:C^q(\mcalU,\mcalF)\to C^{q+1}(\mcalU,\mcalF)$$
%$$\delta^q(c)_{\alpha_0,...,\alpha_{q+1}}:=\sum_{0\leq k\leq q+1}(-1)^k
%c_{...\hat{\alpha_k}...}|_{U_{\alpha_0,...,\alpha_{q+1}}}$$
%\begin{prop}$$\delta^q\circ\delta^q=0$$\end{prop}
%so, we have \u{C}ech cohomology
%$$H^q(\mcalU,\mcalF):=\ker\delta^q\big/\im\delta^{q-1}$$

\begin{example}(第零阶\Cech 上同调)

记号同之前,则有$\check{H}^0(\mcalU,\mcalF)=\ker\delta^0$.
而对于$c\in C^0(\mcalU,\mcalF)=\prod\limits_{\afa\in\mcalI}\mcalF(U_\afa)$,
有$(\delta^0 c)_{\afa\beta}=(c_\beta-c_\afa)|_{U_{\afa\beta}}$,
因此有
$$
  \ker\delta^0
=
  \Bigset{c=(c_\afa)_{\afa\in\mcalI}\in\prod_{\afa\in\mcalI}\mcalF(U_\afa)}
         {c_\afa|_{U_{\afa\beta}}=c_\beta|_{U_{\afa\beta}},
         \,\forall\afa,\beta\in\mcalI}
\xle{\text{层的粘合公理}}
  \mcalF(X)
$$
即$\check{H}^0(\mcalU,\mcalF)\cong\mcalF(X)$为$\mcalF$的整体截面之全体。
\end{example}

%example:
%$$C^0(\mcalU,\mcalF):=\prod_{\alpha\in I}\mcalF(U_{\alpha})$$
%$$c=(c_{\alpha})_{\alpha\in I}\in C^0(\mcalU,\mcalF)$$
%$$\delta^0c=0\iff(\delta^0c)_{\alpha_0\alpha_1}
%:=(c_{\alpha_1}-c_{\alpha_0})|_{U_{\alpha_0\alpha_1}}=0$$
%so, $c_{\alpha_0}=c_{\alpha_1}$ on $U_{\alpha_0\alpha_1}$.
%$\rightsquigarrow$ $H^0(\mcalU,\mcalF)=\mcalF(X)$.

\begin{example}
(1) consider $X=\triangle\setminus\{0\}$, where $\triangle=
\{(z_1,z_2)||z_1|<1,|z_2|<1\}$. Consider the covering
$$\mcalU=U_1\cup U_2$$
where
$$U_1:=\{(z_1,z_2)\in\triangle|z_1\neq 0\}=\bbD^*\times\bbD$$
$$U_2:=\{(z_1,z_2)\in\triangle|z_2\neq 0\}=\bbD\times\bbD^*$$

then
$$U_1\cap U_2=\bbD^*\times\bbD^*$$

consider $H^0(X,\mcalO)=\mcalO(X)\cong\mcalO(\triangle)
=\{f:\triangle\to\bbC\text{holomorphic}\}$.

$$H^1(\mcalU,\mcalO)=\ker\delta^1\big/\im\delta^0$$
$$\delta^1:C^1(\mcalU,\mcalO)\to C^2(\mcalU,\mcalO)\subseteq
\prod_{\alpha_0,\alpha_1,\alpha_2}\mcalO(U_{\alpha_0,\alpha_1,\alpha_2})=0$$
$$\ker\delta^1=
C^1(\mcalU,\mcalO)=\{c=c(\alpha_0,\alpha_1)|c_{\alpha_0,\alpha_1}\in\mcalO(U_{\alpha_0\alpha_1})\}
=\{c\in\mcalO(U_1\cap U_2)\}=\{c=\sum_{m,n\in\bbZ}a_{mn}z_1^mz_2^n\text{convergent}\}$$

$$\delta^0:C^0(\mcalU,\mcalO)\to\mcalC^1(\mcalU,\mcalO)$$
$$(\delta^0c)_{12}=(c_2-c_1)|_{U_{12}}$$
where $c_2\in\mcalO(U_2)$ and $c_1\in\mcalO(U_1)$.
note that
$$\mcalO(U_1)=\{c(z_1,z_2)=\sum_{m\in\bbZ,n\geq 0}a_{mn}z_1^mz_2^n\text{convergent}\}$$
$$\mcalO(U_2)=\{c(z_1,z_2)=\sum_{n\in\bbZ,m\geq 0}a_{mn}z_1^mz_2^n\text{convergent}\}$$

So, $H^1(\mcalU,\mcalO)=
\{c(z_1,z_2)=\sum_{m,n<0}a_{mn}z_1^mz_2^n\}$
\end{example}

\begin{example}(complex projective space)

$$\bbC P^n:=(\bbC^{n+1}\setminus\{0\})\big/\sim$$
$$(z_0,...,z_n)\sim\lmd(z_0,...,z_n)$$
for some $\lmd\in\bbC^*$.

$$\bbC P^n=\{[z_0,...,z_n]|\text{not all $z_k=0$},z_i\in\bbC\}
=\bigcup_{0\leq p\leq n}V_k$$
where
$$V_k=\{[z_0,...,z_n]|z_k\neq 0\}
\cong \{(\frac{z_0}{z_k},...,1,...,\frac{z_n}{z_k})|
z_i\in\bbC,i\neq k, z_k\neq 0\}\cong \bbC^n$$
this is a holo chart.

$$\bbC P^1=V_0\cup V_1,\mcalV=\{V_0,\mcalV_1\}$$
HW: compute $H^q(\mcalV,\mcalO)$.

Answer:
$$H^0\cong\bbC,H^1\cong 0$$
\end{example}

%%%%%%%%%%%%%%%%2019.3.26 第五周 周二%%%%%%%%%%%%%%%%%%%%%%%%%%%%%

%\textbf{Correction}:
%$\mcalA$: Sheaf of rings (with unit)
%$X$: paracompact topological space,
%\begin{definition}
%$\mcalA$ is called fine, if for any open covering
%$\mcalU=(V_{\alpha})_{\alpha\in\mcalI}$,there exist
%$s_{\alpha}\in\mcalA(X)$
%such that such that $supp(s_{\alpha})\subseteq V_{\alpha}$,
%$$\sum_{\alpha}s_{\alpha}=1$$(this is a locally finite sum)
%\end{definition}

Recall:

Cech cohomology: $X$ topological space,
$\mcalU=(U_{\alpha})_{\alpha\in\mcalI}$,
$$C^q(\mcalU,\mcalF)=
\prod_{\alpha_0<...<\alpha_q}\mcalF(\U_{\alpha_1,...,\alpha_q})$$
$$\delta^q:C^q(\mcalU,\mcalF)\to C^{q+1}(\mcalU,\mcalF)$$

fact: $H^0(\mcalU,\mcalF)=\Gamma(X,\mcalF)$.

Today:

\begin{definition}
Let $\mcalV=(V_{\beta})_{\beta\in J}$ be another open covering,
then $\mcalV$ is called a refinement of $\mcalU$, if there exists a map
$$\rho:\mcalJ\to\mcalI$$
such that
$$V_{\beta}\subseteq U_{\rho(\beta)}$$
\end{definition}

\begin{prop}
Let $\mcalV$ be a refinement of $\mcalU$, then $\rho$ induces a map
$$\rho^q:C^q(\mcalU,\mcalF)\to C^q(\mcalV,\mcalF)$$
$$(\rho^qC)_{\beta_0,...,\beta_q}\mapsto
C_{\rho(\beta_0),...,\rho(\beta_q)}|_{V_{\beta_0,...,\beta_q}}$$
$\rho$ is a morphism of complexes.
\end{prop}

%%%%%%改用电子版了%%%%%%
so, $\rho$ induces a map
$$H^q(\rho):H^q(\mcalU,\mcalF)\to H^q(\mcalV,\mcalF)$$

Let $\tilde{\rho}:\mcalJ\to\mcalI$ be another refinement of $\mcalU$

(induces $H^q(\tilde{\rho}):H^q(\mcalU,\mcalF)\to H^q(\mcalV,\mcalF)$)
then $\rho,\tilde{\rho}$ are homotopic
%%%homotopy%%%
(chain homotopy$\rightsquigarrow H^q(\rho)=H^q(\tilde(\rho))$)

so, if $\rho:\mcalJ\to \mcalI$ is refinement, then
$$H^q(\rho)$$
is independent of the refinement.

\begin{definition}
$$\check{H}^q(X,\mcalF):=\lim_{\to\atop\mcalU} H^q(\mcalU,\mcalF)$$
i.e. $a\in H^q(\mcalU,\mcalF)\sim\in H^q(\mcalV,\mcalF)$ iff
$\exists$ a refinement $\mcalW$ of $\mcalU$ and $\mcalV$ such that
$a,b$ have the same image in $H^q(\mcalW,\mcalF)$
\end{definition}

\begin{rem}
$$\check{H}^0(X,\mcalF)=\Gamma(X,\mcalF)$$

Exercise: For $q=1$, if $\mcalV$ is a refinement of $\mcalU$,
then
$$H^1(\mcalU,\mcalF)\to H^1(\mcalV,\mcalF)$$
is injective.
\end{rem}

so ,for any open cover $\mcalU$,
$$H^1(\mcalU,\mcalF)\to \check{H}^1(X,\mcalF)$$
is injective.

\textbf{Homological Algebra}
recall:let $(K\updot,\td_k),(L\updot,\td_l)$ and $(M\updot,\td_M)$,
if we have a short exact sequence
$$0\to K\updot\xra{\fai} L\updot\xra{\psi}M\updot\to 0$$
then it induces a long exact sequence :
$$
  \cdots\to H^q(K\updot)\to
  H^q(L\updot)\to
  H^q(M\updot)\to
  H^{q+1}(K\updot)\to\cdots
$$

analogy of Cech cohomology: $X$ is a topological space,
$\mcalU$ is an open covering of $X$.
$\mcalA$ and $\mcalB$ sheaves on $X$, Let
$$\fai:\mcalA\to\mcalB$$
be a morphism, then it induces
$$\fai\updot:C\updot(\mcalU,\mcalA)\to C\updot(\mcalU,\mcalB)$$

Let
$$0\to \mcalA\to\mcalB\to\mcalC\to 0$$
be an exact sequence of sheaves, then we have:
for any open set $\Omg$,
$$0\to \mcalA(\Omg)\to \mcalB(\Omg)\to\mcalC(\Omg)$$
left exact.

Example: consider
$$0\to\bbZ\to\mcalO\xra{exp}\to0$$
is exact on $bbC^{\times}:=\bbC\setminus\{0\}$

but we have :
$$0\to\mcalA(\Omg)\xra{\psi}\mcalB(\Omg)\to\im\psi(\Omg)\to 0$$
is exact.

First we have the following exact sequence
$$C^q(\mcalU,\mcalA)\to C^q(\mcalU,\mcalB)\to C^q_{\mcalB}(\mcalU,\mcalC)\to 0$$
where $C_{\mcalB}^q$ is the image of ...

then we get an exact sequence
$$0\to (C\updot(\mcalU,\mcalA),\delta)\to
(C\updot(\mcalU,\mcalB),\delta)\to
(C\updot_{\mcalB}(\mcalU,\mcalC),\delta)\to 0$$

it induces a long exact sequence
$$\cdots\to
H^q(\mcalU,\mcalA)\to
H^q(\mcalU,\mcalB)\to
H^q_{\mcalB}(\mcalU,\mcalC)\to
H^{q+1}(\mcalU,\mcalA)\to\cdots
$$

\begin{thm}
If $X$ is paracompact,
$$0\to\mcalA\to\mcalB\to\mcalC\to 0$$
is a sheaf exact sequence.
Then there is a long exact sequence
$$
\cdots\to
\check{H}^q(X,\mcalA)\to
\check{H}^q(X,\mcalB)\to
\check{H}^q(X,\mcalC)\to
\check{H}^{q+1}(X,\mcalZ)\to \cdots
$$
\end{thm}
\begin{proof}

Key lemma: need to prove
$$\lim_{\to\atop\mcalU}H^q(\mcalU,\mcalC)=
\lim_{\to\atop\mcalU}H^q_{\mcalB}(\mcalU,\mcalC)
$$
if $X$ is paracompact.

Omit.
\end{proof}


if
$$0\to \mcalA\to\mcalB\to\mcalC\to 0$$
exact,

recall:(cohomology by resolutions)
$$
0\to\mcalA\to\mcalF^0\to\mcalF^1\to\cdots
$$
flabby resolution. then it induces
$$0\to\Gamma(X,\mcalA)\to\Gamma(X,\mcalF^0)\to\Gamma(X,\mcalF^1)\to\cdots$$
then define the sheaf cohomology...

we have a long exact sequence
$$
\cdots\to H^q(X,\mcalA)\to H^q(X,\mcalB)\to H^q(X,\mcalC)\to H^{q+1}(X,\mcalA)\to\cdots
$$
it is homological algebra...

\begin{thm}(Leray's acyclic theorem)
Let $\mcalU=(U_{\alpha})_{\alpha\in\mcalI}$ be an open covering of $X$,
($\mcalF$ is a sheaf on $X$), if satisfying
$$H^k(U_{\alpha_0,...,\alpha_q})=0$$
for any $k \geq 1$ ,then
$$H^q(\mcalU,\mcalF)\cong \check(H)^q(X,\mcalF)$$

and if $X$ is paracompact ,we also have
$$H^q(\mcalU,\mcalF)\cong \check(H)^q(X,\mcalF)\cong H^q(X,\mcalF)$$

\end{thm}
(this $\mcalU$ is called acyclic covering)

\section{de Rham- Weil 定理}
\begin{definition}
$\mcalF$ is a sheaf on $X$, $\Omg$ is an open set of $X$,
then $\mcalF$ is called \textbf{acyclic sheaf} if
$$H^q(\Omg,\mcalF)=0$$
for any $q\geq 1$.
\end{definition}

\begin{thm}
Let
$$0\to\mcalF\to(L\updot,\td)$$
be an acyclic resolution of $\mcalF$
(i.e. $L^q$is acyclic on $X$)
then
$$H^q(X,\mcalF)\cong H^q(\Gamma(X,L\updot),\td)$$
for any $q\geq 0$.
\end{thm}

(先看例子)

\begin{example}
Let $X$ be a differential manifold,
$\mcalE^p$:sheaf of smooth $p$-forms, then we have a resolution
(de Rham complex)
$$0\to\bbR\inj\mcalE^0\xra{\td}\mcalE^1
\xra{\td}\mcalE^2\xra{\td}\mcalE^3\to\cdots$$
where $\td$ differential operators.
(Why it is a resolution? because of Poincare lemma...locally solvable..)

Note that
$$\mcalE^0=\mcalC^{\infty}$$
$\mcalE^p$ is a sheaf of $C^{\infty}$-modules..

then we have
$$H^q(X,\mcalE^p)=0$$
for all $q\geq1$

and then
$$H^q(X,\bbR)\cong
\frac{\ker(\td:\Gamma(X,\mcalE^q)\to\Gamma(X,\mcalE^{q+1}))}
     {\im(\td:\Gamma(X,\mcalE^{q-1})\to\Gamma(X,\mcalE^q))}
=H_{DR}^q(X,\mcalR)
$$
\end{example}

\begin{example}
Let $X$ be a complex manifold,
$\mcalE^{p,q}$ sheaf of smooth $(p,q)$ forms,
$\Omg^p$ is the sheaf of holomorphic $p$-forms
(i.e. $(p,0)$-form $\fai$ with $\pbar\fai=0$).

Then we have resolution
$$0\to \Omg^p\xra{j}\mcalE^{p,0}\xra{\pbar}\mcalE^{p,1}\xra{\pbar}\mcalE^{p,2}\to\cdots$$
(Why it is a resolution?  because of the Dolbeault lemma),remain to Exercise...

$$H^q(X,\Omg^p)\cong H^{p,q}_{\pbar}(X,\bbC)$$
\end{example}

%%%%%%%%%%%%%%%%%2019.3.28第五周周四%%%%%%%%%%%%%%%%%%%%%%%%%

Today: de Rham-Weil Isomorphism Thm

\begin{thm}
Let $X$ be a topological space, $\mcalF$ be a sheaf of abelian groups on $X$,
$$0\to\mcalF\to(\mcalL\updot,\td)$$
be an acyclic resolution, i.e.
$$H^k(X,\mcalL^q)=0$$
for all $k\geq 1$ and $q\geq 0$.
Then,
$$H^q(X,\mcalF)\cong H^q((\Gamma(\mcalL\updot),\td))$$
\end{thm}

\begin{proof}
Since
$$
0\to \mcalF\xra{j}\mcalL^0\xra{\td^0}\mcalL^1\xra{\td^1}\mcalL^2\to\cdots
$$
be an exact sequence, denote
$$\mcalZ^q:=\ker \td^q$$
then we have short exact sequences
$$0\to \mcalZ^q\to\mcalL^q\to\mcalZ^{q+1}\to 0$$
for any $q$. They induce long exact sequence of cohomology groups:
$$\cdots\to H^k(X,\mcalZ^q)\to H^k(X,\mcalL^q)\to H^k(X,\mcalZ^{q+1})\xra{\p}
H^{k+1}(X,\mcalL^q)\to H^{q+1}(X,\mcalL^q)\to\cdots$$
For any $k\geq 1$, since $\mcalL^q$ are acyclic on $X$,
$$H^k(X,\mcalZ^{q+1})\cong H^{k+1}(X,\mcalZ^q)$$
and for $k=0$, we have
$$
0\to H^0(X,\mcalZ^q)\to H^0(X,\mcalL^q)\to H^0(X,\mcalZ^{q+1})\to
H^1(X,\mcalZ^q)\to H^1(X,\mcalL^q)=0\to\cdots
$$
so,
$$H^1(X,\mcalZ^q)\cong H^0(X,\mcalZ^{q+1})\big/\im\td^q
\cong H^{q+1}((\Gamma(\mcalL\updot),\td))$$

$$H^{q+1}(\Gamma(\mcalL\updot))\cong H^1(X,\mcalZ^q)\cong H^2(X,\mcalZ^{q-1})\cong\cdots
H^{q+1}(X,\mcalZ^0)=H^{q+1}(X,\mcalF)$$
\end{proof}

%%%%%%%上次课讲了两个经典例子:De Rham cohomology and Doulbeault cohomology%%%%%%55

$$0\to\bbR\to\mcalE^0\xra{\td}\mcalE^1\xra{\td}\mcalE^2\to\cdots$$
(de Rham resolution) then we have
$$H^k(X,\mcalR)\cong H_{DR}^k(X;\mcalR)$$
(if $X$ is compact ,then by Hodge theory, it also isomorphic to $\ker(\td\td^*+\td^*\td)$)

Another example:$X$ is a complex manifold, then
$$0\to\Omg^p\to\mcalE^{p,0}\xra{\pbar}\mcalE^{p,1}\xra{\pbar}\mcalE^{p,2}\to\cdots$$
then
$$H^{q}(X,\Omg^p)\cong H_{\pbar}^{p,q}(X,\bbC)$$
(RHS$=$ Dolbeault cohomology)


$X$ be a smooth manifold, we define
$$C_q(X,\bbZ):=\text{the free abelian group generated by continuous map}$$
$$\phi:\triangle_q:=\{(t_1,...,t_{q+1})\in[0,1]^{q+1}|\sum_{i=1}^nt_i=1\}$$
and we define (for $\phi\in C_q(X,\bbZ)$)
$$\p\phi:=\sum_{i=1}^{q+1}(-1)^q\phi|_{\triangle_{q,i}}$$
$$\triangle_{q,i}:=\{t\in\triangle_q|t_i=0\}$$
we define
$$(C_{sing}\updot,\p)$$
be the dual complex of $(C^{sing}\downdot),\p$.

(These are all Basic Algebraic Topology)

For any open $U\subseteq X$, we have
$$U\to C_{sing}^q(U,\bbZ)$$
we get a sheaf
$$\mcalC_{sing}^q$$

FACT: $(\mcalC_{sing}\updot,\p)$ is a flabby resolution of $\bbZ$. (check!)So,
$$H^q_{sing}(X,\bbZ)= H^q(\Gamma(\mcalC_{sing}\updot),\p)
\cong H^q(X,\bbZ)$$

%%%%%%%%%%%%%%%以后还要讲谱序列,so scared...好怕怕……%%%%%%%%%%%%%%%%%%%5
%推荐读一读,现在学的东西应该能读懂了。。。
%Tate : rigid analytic spaces
%%%%%%%%%%%%%%%%%%%%%%%%%%%%%%%%%%%


