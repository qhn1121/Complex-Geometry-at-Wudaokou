%%%%%%%%%%%%%%%%%%%%%%%%%%%%%%%%%%%%%%%%%%%%%%%%%
%%%%%%%%%%%%%%%%%%%%%%%%%%%%%%%%%%%%%%%%%%%%%%%%%
%       2019.3.21启用电子稿                     %
%%%%%%%%%%%%%%%%%%%%%%%%%%%%%%%%%%%%%%%%%%%%%%%%%

\chapter{层与层上同调}
\section{层的上同调}
Today:

Sheaf cohomology

$X$ a topological space, $\mcalF$- sheaf (of abelian groups).

\begin{definition}  (resolution)

(1)a resolution of $\mcalF$ is an exact sequence

$$0\to \mcalF\xra{j}\mcalF\xra{d^0}\mcalF\xra{d^1}\to\cdots$$

\end{definition}

\begin{definition}
A sheaf $\mcalA$ is called injective, if
if for any injective morphism $j:\mcalA\to \mcalB$
and for any morphism $\fai:\mcalA\to\mcalS$,
there exists an extension $\psi :\mcalB\to \mcalS$,such that
%%%%%diagram%%%
\end{definition}
\begin{thm}
the category of sheaves of abelian sheaves have enough
injective objects, i.e.  any $\mcalF$ can be
embedded in some injective sheaf.
\end{thm}

\begin{definition}
Consider an injective resolution of $\mcalF$, i.e. an exact sequence
$$0\to\mcalF\to\mcalI^0\xra{\td}\mcalI^1\xra{\td}\mcalI^2\to\cdots$$
where every $\mcalI^k(k\geq 0)$ is injective.


$\rightsquigarrow $induces a sequence
$$0\to\Gamma(X,\mcalF)
\to\Gamma(X,\mcalI^0)\xra{\td}
\Gamma(X,\mcalI^1)\xra{\td}
\Gamma(X,\mcalI^2)\to\cdots$$

Then
$$H^q(X,\mcalF):=H^q(\Gamma(X,\mcalI\updot))$$

\end{definition}

then, $H^0(X,\mcalF)=\Gamma(X,\mcalF)$.

\begin{definition}
A sheaf $\mcalS$ is called a flabby (flasque ,in France) ,if
for any open set $\Omg\subseteq X$, the morphism
$$\mcalS(X)\to\mcalS(\Omg)$$
is surjective.
\end{definition}

\begin{definition}
$$0\to\mcalF\xra{j}\mcalF^0\xra{d^0}\to\mcalF^1$$
is an exact sequence is called a flabby resolution, if
any $\mcalF^k$ is flabby.
\end{definition}

\begin{definition}
$$H^q(X,\mcalF):=...\text{by flabby resolution...}$$
\end{definition}

\begin{proof}
Homological Algebra...omit.
\end{proof}

the two definitions of Sheaf Cohomology are isomorphic.


Godement's construction

$$God(\mcalF)(U):=
\{f:U\to\bigcup_{x\in U}\mcalF_x|
f(y)\in\mcalF_y,\forall y\in U\}
:=\prod_{x\in U}\mcalF_x$$

$God(\mcalF)$ is a sheaf, and it is flabby. and there is a canonical
morphism $\mcalF(U)\to God(F)(U)$ by $x\mapsto(x\mapsto s_x)$ is injective.

$$\mcalF^0:=God(\mcalF)$$
$$0\to\mcalF\xra{j}\mcalF^0\surj\coker(j)=\mcalF^0\big/\mcalF$$
and consider
$$\mcalF^1:=God(\coker(j))$$
......then construct by induction... this is a flabby resolution of $\mcalF$.

\begin{definition}(resolution by fine sheaves)

$\mcalA$ is a sheaf of ring,
$X$ is a paracompact topological space, $\mcalA$
is called a fine sheaf, if for any open covering
$$X=\bigcup_{\alpha}V_{\alpha}\quad,\mcalV:=\{V_{\alpha}\}$$
there exists a partition of unit subordinate to $\mcalV$,
(i.e. $\exists f_{\alpha}\in\mcalA(V_{\alpha}),supp(\alpha)
:=\overline{\{x\in V_{\alpha}|f_{\alpha,x}\neq 0\}}\subseteq V_{\alpha}$, and
$\sum_{\alpha}f_{\alpha}=1$(the sum is locally finite)
 )
\end{definition}

\begin{example}
$X$ is a differential manifold,
$\mcalC^{\infty}$ is the sheaf of smooth functions,
then $\mcalC^{\infty}$ is a fine sheaf.
\end{example}

\begin{thm}
$\mcalS$ is a sheaf of $\mcalA$-modules,
$\mcalA$ is a fine sheaf. then for any $q\geq 1$,
$$H^q(X,\mcalS)=0$$
\end{thm}
\begin{proof}
Consider a flabby(or injective) resolution
$$0\to\mcalS\xra{j}\mcalI^0\xra{\td}\mcalI^1\xra{\td}\mcalI^2\cdots$$
where any $\mcalI^k(k\geq 0)$ is a sheaf of $\mcalA$-modules.

by definition,
$$H^q(X,m\mcalS):=\frac{\ker\td:\Gamma(\mcalI^q)\to\Gamma(\mcalI^{q+1})}
                       {\Im\td:\Gamma(\mcalI^{q-1})\to\Gamma(\mcalI^{q})}$$

Let $\alpha\in\ker\{\td:\Gamma(\mcalI^q)\to\Gamma(\mcalI^{q+1})\}$
by the exactness of resolution, $\exists$ an open covering $\mcalU=(U_{i})_{i}$,
s.t. $\alpha|_{U_{i}}=\td\beta_{i}$
where $\beta_{i}\in\mcalT^{q-1}(U_{i})$.
Let $(\beta_{i})_{i}$ be the partition of unit w.r.t. $\mcalU$.
consider
$$\beta:=\sum_{i}f_i\beta_i$$
(well defined). Then $\td\beta=\alpha$....
\end{proof}

\section{\u{C}ech上同调}
\textbf{\u{C}ech cohomology}

$X$- a topological space, $\mcalF$- a sheaf of abelian group.
$$\mcalU=(U_{\alpha})_{\alpha\in I}$$
is an open covering.

notation:$U_{\alpha_1,...,\alpha_q}:=\bigcap_{i=1}^qU_{\alpha_i}$.

\u{C}ech $q$-chain w.r.t $\mcalU$:

$$C^q(\mcalU,\mcalF):=\prod_{(\alpha_1,...,\alpha_q)
\in\mcalI^{q+1}}\mcalF(U_{\alpha_1,...,\alpha_q})$$

$$c\in C^q(\mcalU,\mcalF)$$
means that we have a family of sections
$C_{\alpha_1,...,\alpha_q}\in\mcalF(U_{\alpha_1,...,\alpha_q})$
with the relation
$$C_{\alpha_0,...,\alpha_j,...,\alpha_i,...}=-C_{...}$$

\u(C)ech differential:
$$\delta^q:C^q(\mcalU,\mcalF)\to C^{q+1}(\mcalU,\mcalF)$$
$$\delta^q(c)_{\alpha_0,...,\alpha_{q+1}}
:=\sum_{0\leq k\leq q+1}(-1)^k
c_{...\hat{\alpha_k}...}|_{U_{\alpha_0,...,\alpha_{q+1}}}$$

\begin{prop}
$$\delta^q\circ\delta^q=0$$
\end{prop}

so, we have \u{C}ech cohomology
$$H^q(\mcalU,\mcalF):=\ker\delta^q\big/\im\delta^{q-1}$$

example:
$$C^0(\mcalU,\mcalF):=\prod_{\alpha\in I}\mcalF(U_{\alpha})$$
$$c=(c_{\alpha})_{\alpha\in I}\in C^0(\mcalU,\mcalF)$$
$$\delta^0c=0\iff(\delta^0c)_{\alpha_0\alpha_1}
:=(c_{\alpha_1}-c_{\alpha_0})|_{U_{\alpha_0\alpha_1}}=0$$
so, $c_{\alpha_0}=c_{\alpha_1}$ on $U_{\alpha_0\alpha_1}$.

$\rightsquigarrow$ $H^0(\mcalU,\mcalF)=\mcalF(X)$.

\begin{example}
(1) consider $X=\triangle\setminus\{0\}$, where $\triangle=
\{(z_1,z_2)||z_1|<1,|z_2|<1\}$. Consider the covering
$$\mcalU=U_1\cup U_2$$
where
$$U_1:=\{(z_1,z_2)\in\triangle|z_1\neq 0\}=\bbD^*\times\bbD$$
$$U_2:=\{(z_1,z_2)\in\triangle|z_2\neq 0\}=\bbD\times\bbD^*$$

then
$$U_1\cap U_2=\bbD^*\times\bbD^*$$

consider $H^0(X,\mcalO)=\mcalO(X)\cong\mcalO(\triangle)
=\{f:\triangle\to\bbC\text{holomorphic}\}$.

$$H^1(\mcalU,\mcalO)=\ker\delta^1\big/\im\delta^0$$
$$\delta^1:C^1(\mcalU,\mcalO)\to C^2(\mcalU,\mcalO)\subseteq
\prod_{\alpha_0,\alpha_1,\alpha_2}\mcalO(U_{\alpha_0,\alpha_1,\alpha_2})=0$$
$$\ker\delta^1=
C^1(\mcalU,\mcalO)=\{c=c(\alpha_0,\alpha_1)|c_{\alpha_0,\alpha_1}\in\mcalO(U_{\alpha_0\alpha_1})\}
=\{c\in\mcalO(U_1\cap U_2)\}=\{c=\sum_{m,n\in\bbZ}a_{mn}z_1^mz_2^n\text{convergent}\}$$

$$\delta^0:C^0(\mcalU,\mcalO)\to\mcalC^1(\mcalU,\mcalO)$$
$$(\delta^0c)_{12}=(c_2-c_1)|_{U_{12}}$$
where $c_2\in\mcalO(U_2)$ and $c_1\in\mcalO(U_1)$.
note that
$$\mcalO(U_1)=\{c(z_1,z_2)=\sum_{m\in\bbZ,n\geq 0}a_{mn}z_1^mz_2^n\text{convergent}\}$$
$$\mcalO(U_2)=\{c(z_1,z_2)=\sum_{n\in\bbZ,m\geq 0}a_{mn}z_1^mz_2^n\text{convergent}\}$$

So, $H^1(\mcalU,\mcalO)=
\{c(z_1,z_2)=\sum_{m,n<0}a_{mn}z_1^mz_2^n\}$
\end{example}

\begin{example}(complex projective space)

$$\bbC P^n:=(\bbC^{n+1}\setminus\{0\})\big/\sim$$
$$(z_0,...,z_n)\sim\lmd(z_0,...,z_n)$$
for some $\lmd\in\bbC^*$.

$$\bbC P^n=\{[z_0,...,z_n]|\text{not all $z_k=0$},z_i\in\bbC\}
=\bigcup_{0\leq p\leq n}V_k$$
where
$$V_k=\{[z_0,...,z_n]|z_k\neq 0\}
\cong \{(\frac{z_0}{z_k},...,1,...,\frac{z_n}{z_k})|
z_i\in\bbC,i\neq k, z_k\neq 0\}\cong \bbC^n$$
this is a holo chart.

$$\bbC P^1=V_0\cup V_1,\mcalV=\{V_0,\mcalV_1\}$$
HW: compute $H^q(\mcalV,\mcalO)$.

Answer:
$$H^0\cong\bbC,H^1\cong 0$$
\end{example}

%%%%%%%%%%%%%%%%2019.3.26 第五周 周二%%%%%%%%%%%%%%%%%%%%%%%%%%%%%

\textbf{Correction}:

$\mcalA$: Sheaf of rings (with unit)

$X$: paracompact topological space,

\begin{definition}
$\mcalA$ is called fine, if for any open covering
$\mcalU=(V_{\alpha})_{\alpha\in\mcalI}$,there exist
$s_{\alpha}\in\mcalA(X)$
such that such that $supp(s_{\alpha})\subseteq V_{\alpha}$,
$$\sum_{\alpha}s_{\alpha}=1$$
(this is a locally finite sum)
\end{definition}

\begin{rem}
we call $\mcalA$ is a \textbf{soft sheaf},
if for any closed set $K\subseteq X$,
the morphism
$$\mcalA(X)\to \mcalA(K)$$
is surjective.
where $\mcalA(K):=\Gamma(K,\mcalA|_K)$
\end{rem}

fact: $\mcalA$ is fine if and only if
$\mcalH om(\mcalA,\mcalA)$ is soft.
(omit)

Recall:

Cech cohomology: $X$ topological space,
$\mcalU=(U_{\alpha})_{\alpha\in\mcalI}$,
$$C^q(\mcalU,\mcalF)=
\prod_{\alpha_0<...<\alpha_q}\mcalF(\U_{\alpha_1,...,\alpha_q})$$
$$\delta^q:C^q(\mcalU,\mcalF)\to C^{q+1}(\mcalU,\mcalF)$$

fact: $H^0(\mcalU,\mcalF)=\Gamma(X,\mcalF)$.

Today:

\begin{definition}
Let $\mcalV=(V_{\beta})_{\beta\in J}$ be another open covering,
then $\mcalV$ is called a refinement of $\mcalU$, if there exists a map
$$\rho:\mcalJ\to\mcalI$$
such that
$$V_{\beta}\subseteq U_{\rho(\beta)}$$
\end{definition}

\begin{prop}
Let $\mcalV$ be a refinement of $\mcalU$, then $\rho$ induces a map
$$\rho^q:C^q(\mcalU,\mcalF)\to C^q(\mcalV,\mcalF)$$
$$(\rho^qC)_{\beta_0,...,\beta_q}\mapsto
C_{\rho(\beta_0),...,\rho(\beta_q)}|_{V_{\beta_0,...,\beta_q}}$$
$\rho$ is a morphism of complexes.
\end{prop}

%%%%%%改用电子版了%%%%%%
so, $\rho$ induces a map
$$H^q(\rho):H^q(\mcalU,\mcalF)\to H^q(\mcalV,\mcalF)$$

Let $\tilde{\rho}:\mcalJ\to\mcalI$ be another refinement of $\mcalU$

(induces $H^q(\tilde{\rho}):H^q(\mcalU,\mcalF)\to H^q(\mcalV,\mcalF)$)
then $\rho,\tilde{\rho}$ are homotopic
%%%homotopy%%%
(chain homotopy$\rightsquigarrow H^q(\rho)=H^q(\tilde(\rho))$)

so, if $\rho:\mcalJ\to \mcalI$ is refinement, then
$$H^q(\rho)$$
is independent of the refinement.

\begin{definition}
$$\check{H}^q(X,\mcalF):=\lim_{\to\atop\mcalU} H^q(\mcalU,\mcalF)$$
i.e. $a\in H^q(\mcalU,\mcalF)\sim\in H^q(\mcalV,\mcalF)$ iff
$\exists$ a refinement $\mcalW$ of $\mcalU$ and $\mcalV$ such that
$a,b$ have the same image in $H^q(\mcalW,\mcalF)$
\end{definition}

\begin{rem}
$$\check{H}^0(X,\mcalF)=\Gamma(X,\mcalF)$$

Exercise: For $q=1$, if $\mcalV$ is a refinement of $\mcalU$,
then
$$H^1(\mcalU,\mcalF)\to H^1(\mcalV,\mcalF)$$
is injective.
\end{rem}

so ,for any open cover $\mcalU$,
$$H^1(\mcalU,\mcalF)\to \check{H}^1(X,\mcalF)$$
is injective.

\textbf{Homological Algebra}
recall:let $(K\updot,\td_k),(L\updot,\td_l)$ and $(M\updot,\td_M)$,
if we have a short exact sequence
$$0\to K\updot\xra{\fai} L\updot\xra{\psi}M\updot\to 0$$
then it induces a long exact sequence :
$$
  \cdots\to H^q(K\updot)\to
  H^q(L\updot)\to
  H^q(M\updot)\to
  H^{q+1}(K\updot)\to\cdots
$$

analogy of Cech cohomology: $X$ is a topological space,
$\mcalU$ is an open covering of $X$.
$\mcalA$ and $\mcalB$ sheaves on $X$, Let
$$\fai:\mcalA\to\mcalB$$
be a morphism, then it induces
$$\fai\updot:C\updot(\mcalU,\mcalA)\to C\updot(\mcalU,\mcalB)$$

Let
$$0\to \mcalA\to\mcalB\to\mcalC\to 0$$
be an exact sequence of sheaves, then we have:
for any open set $\Omg$,
$$0\to \mcalA(\Omg)\to \mcalB(\Omg)\to\mcalC(\Omg)$$
left exact.

Example: consider
$$0\to\bbZ\to\mcalO\xra{exp}\to0$$
is exact on $bbC^{\times}:=\bbC\setminus\{0\}$

but we have :
$$0\to\mcalA(\Omg)\xra{\psi}\mcalB(\Omg)\to\im\psi(\Omg)\to 0$$
is exact.

First we have the following exact sequence
$$C^q(\mcalU,\mcalA)\to C^q(\mcalU,\mcalB)\to C^q_{\mcalB}(\mcalU,\mcalC)\to 0$$
where $C_{\mcalB}^q$ is the image of ...

then we get an exact sequence
$$0\to (C\updot(\mcalU,\mcalA),\delta)\to
(C\updot(\mcalU,\mcalB),\delta)\to
(C\updot_{\mcalB}(\mcalU,\mcalC),\delta)\to 0$$

it induces a long exact sequence
$$\cdots\to
H^q(\mcalU,\mcalA)\to
H^q(\mcalU,\mcalB)\to
H^q_{\mcalB}(\mcalU,\mcalC)\to
H^{q+1}(\mcalU,\mcalA)\to\cdots
$$

\begin{thm}
If $X$ is paracompact,
$$0\to\mcalA\to\mcalB\to\mcalC\to 0$$
is a sheaf exact sequence.
Then there is a long exact sequence
$$
\cdots\to
\check{H}^q(X,\mcalA)\to
\check{H}^q(X,\mcalB)\to
\check{H}^q(X,\mcalC)\to
\check{H}^{q+1}(X,\mcalZ)\to \cdots
$$
\end{thm}
\begin{proof}

Key lemma: need to prove
$$\lim_{\to\atop\mcalU}H^q(\mcalU,\mcalC)=
\lim_{\to\atop\mcalU}H^q_{\mcalB}(\mcalU,\mcalC)
$$
if $X$ is paracompact.

Omit.
\end{proof}


if
$$0\to \mcalA\to\mcalB\to\mcalC\to 0$$
exact,

recall:(cohomology by resolutions)
$$
0\to\mcalA\to\mcalF^0\to\mcalF^1\to\cdots
$$
flabby resolution. then it induces
$$0\to\Gamma(X,\mcalA)\to\Gamma(X,\mcalF^0)\to\Gamma(X,\mcalF^1)\to\cdots$$
then define the sheaf cohomology...

we have a long exact sequence
$$
\cdots\to H^q(X,\mcalA)\to H^q(X,\mcalB)\to H^q(X,\mcalC)\to H^{q+1}(X,\mcalA)\to\cdots
$$
it is homological algebra...

\begin{thm}(Leray's acyclic theorem)
Let $\mcalU=(U_{\alpha})_{\alpha\in\mcalI}$ be an open covering of $X$,
($\mcalF$ is a sheaf on $X$), if satisfying
$$H^k(U_{\alpha_0,...,\alpha_q})=0$$
for any $k \geq 1$ ,then
$$H^q(\mcalU,\mcalF)\cong \check(H)^q(X,\mcalF)$$

and if $X$ is paracompact ,we also have
$$H^q(\mcalU,\mcalF)\cong \check(H)^q(X,\mcalF)\cong H^q(X,\mcalF)$$

\end{thm}
(this $\mcalU$ is called acyclic covering)

\textbf{de Rham- Weil theorem}
\begin{definition}
$\mcalF$ is a sheaf on $X$, $\Omg$ is an open set of $X$,
then $\mcalF$ is called \textbf{acyclic sheaf} if
$$H^q(\Omg,\mcalF)=0$$
for any $q\geq 1$.
\end{definition}

\begin{thm}
Let
$$0\to\mcalF\to(L\updot,\td)$$
be an acyclic resolution of $\mcalF$
(i.e. $L^q$is acyclic on $X$)
then
$$H^q(X,\mcalF)\cong H^q(\Gamma(X,L\updot),\td)$$
for any $q\geq 0$.
\end{thm}

(先看例子)

\begin{example}
Let $X$ be a differential manifold,
$\mcalE^p$:sheaf of smooth $p$-forms, then we have a resolution
(de Rham complex)
$$0\to\bbR\inj\mcalE^0\xra{\td}\mcalE^1
\xra{\td}\mcalE^2\xra{\td}\mcalE^3\to\cdots$$
where $\td$ differential operators.
(Why it is a resolution? because of Poincare lemma...locally solvable..)

Note that
$$\mcalE^0=\mcalC^{\infty}$$
$\mcalE^p$ is a sheaf of $C^{\infty}$-modules..

then we have
$$H^q(X,\mcalE^p)=0$$
for all $q\geq1$

and then
$$H^q(X,\bbR)\cong
\frac{\ker(\td:\Gamma(X,\mcalE^q)\to\Gamma(X,\mcalE^{q+1}))}
     {\im(\td:\Gamma(X,\mcalE^{q-1})\to\Gamma(X,\mcalE^q))}
=H_{DR}^q(X,\mcalR)
$$
\end{example}

\begin{example}
Let $X$ be a complex manifold,
$\mcalE^{p,q}$ sheaf of smooth $(p,q)$ forms,
$\Omg^p$ is the sheaf of holomorphic $p$-forms
(i.e. $(p,0)$-form $\fai$ with $\pbar\fai=0$).

Then we have resolution
$$0\to \Omg^p\xra{j}\mcalE^{p,0}\xra{\pbar}\mcalE^{p,1}\xra{\pbar}\mcalE^{p,2}\to\cdots$$
(Why it is a resolution?  because of the Dolbeault lemma),remain to Exercise...

$$H^q(X,\Omg^p)\cong H^{p,q}_{\pbar}(X,\bbC)$$
\end{example}

%%%%%%%%%%%%%%%%%2019.3.28第五周周四%%%%%%%%%%%%%%%%%%%%%%%%%

Today: de Rham-Weil Isomorphism Thm

\begin{thm}
Let $X$ be a topological space, $\mcalF$ be a sheaf of abelian groups on $X$,
$$0\to\mcalF\to(\mcalL\updot,\td)$$
be an acyclic resolution, i.e.
$$H^k(X,\mcalL^q)=0$$
for all $k\geq 1$ and $q\geq 0$.
Then,
$$H^q(X,\mcalF)\cong H^q((\Gamma(\mcalL\updot),\td))$$
\end{thm}

\begin{proof}
Since
$$
0\to \mcalF\xra{j}\mcalL^0\xra{\td^0}\mcalL^1\xra{\td^1}\mcalL^2\to\cdots
$$
be an exact sequence, denote
$$\mcalZ^q:=\ker \td^q$$
then we have short exact sequences
$$0\to \mcalZ^q\to\mcalL^q\to\mcalZ^{q+1}\to 0$$
for any $q$. They induce long exact sequence of cohomology groups:
$$\cdots\to H^k(X,\mcalZ^q)\to H^k(X,\mcalL^q)\to H^k(X,\mcalZ^{q+1})\xra{\p}
H^{k+1}(X,\mcalL^q)\to H^{q+1}(X,\mcalL^q)\to\cdots$$
For any $k\geq 1$, since $\mcalL^q$ are acyclic on $X$,
$$H^k(X,\mcalZ^{q+1})\cong H^{k+1}(X,\mcalZ^q)$$
and for $k=0$, we have
$$
0\to H^0(X,\mcalZ^q)\to H^0(X,\mcalL^q)\to H^0(X,\mcalZ^{q+1})\to
H^1(X,\mcalZ^q)\to H^1(X,\mcalL^q)=0\to\cdots
$$
so,
$$H^1(X,\mcalZ^q)\cong H^0(X,\mcalZ^{q+1})\big/\im\td^q
\cong H^{q+1}((\Gamma(\mcalL\updot),\td))$$

$$H^{q+1}(\Gamma(\mcalL\updot))\cong H^1(X,\mcalZ^q)\cong H^2(X,\mcalZ^{q-1})\cong\cdots
H^{q+1}(X,\mcalZ^0)=H^{q+1}(X,\mcalF)$$
\end{proof}

%%%%%%%上次课讲了两个经典例子:De Rham cohomology and Doulbeault cohomology%%%%%%55

$$0\to\bbR\to\mcalE^0\xra{\td}\mcalE^1\xra{\td}\mcalE^2\to\cdots$$
(de Rham resolution) then we have
$$H^k(X,\mcalR)\cong H_{DR}^k(X;\mcalR)$$
(if $X$ is compact ,then by Hodge theory, it also isomorphic to $\ker(\td\td^*+\td^*\td)$)

Another example:$X$ is a complex manifold, then
$$0\to\Omg^p\to\mcalE^{p,0}\xra{\pbar}\mcalE^{p,1}\xra{\pbar}\mcalE^{p,2}\to\cdots$$
then
$$H^{q}(X,\Omg^p)\cong H_{\pbar}^{p,q}(X,\bbC)$$
(RHS$=$ Dolbeault cohomology)


$X$ be a smooth manifold, we define
$$C_q(X,\bbZ):=\text{the free abelian group generated by continuous map}$$
$$\phi:\triangle_q:=\{(t_1,...,t_{q+1})\in[0,1]^{q+1}|\sum_{i=1}^nt_i=1\}$$
and we define (for $\phi\in C_q(X,\bbZ)$)
$$\p\phi:=\sum_{i=1}^{q+1}(-1)^q\phi|_{\triangle_{q,i}}$$
$$\triangle_{q,i}:=\{t\in\triangle_q|t_i=0\}$$
we define
$$(C_{sing}\updot,\p)$$
be the dual complex of $(C^{sing}\downdot),\p$.

(These are all Basic Algebraic Topology)

For any open $U\subseteq X$, we have
$$U\to C_{sing}^q(U,\bbZ)$$
we get a sheaf
$$\mcalC_{sing}^q$$

FACT: $(\mcalC_{sing}\updot,\p)$ is a flabby resolution of $\bbZ$. (check!)So,
$$H^q_{sing}(X,\bbZ)= H^q(\Gamma(\mcalC_{sing}\updot),\p)
\cong H^q(X,\bbZ)$$

%%%%%%%%%%%%%%%以后还要讲谱序列,so scared...好怕怕……%%%%%%%%%%%%%%%%%%%5
%推荐读一读,现在学的东西应该能读懂了。。。
%Tate : rigid analytic spaces
%%%%%%%%%%%%%%%%%%%%%%%%%%%%%%%%%%%


