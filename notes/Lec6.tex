\chapter{正性与消灭定理}

\section{Bochner-Kodaira-Nakaino 恒等式}
positivity and vanishing theorem

$X$-Kahler manifold, i.e. $\exists$ Hermitian metric $\omg$ s.t. $\td\omg=0$,
$\td=\td'+\td''$, $\td'=\p,\td''=\pbar$.
$$\yc_\td=[\td,\td^*]=\td\td^*+\td^*\td$$
$$\yc_{\td'}=[\td',\td'^*]$$
$$\yc_{\td''}=[\td'',\td''^*]$$
$\td\curvearrowright C^{\infty}(X,\wedgeform{p,q})$.

Fact: $\omg$ is Kahler $\iff\yc_{\td'}=\yc_{\td''}=\frac{1}{2}\yc_\td$.

Let $\underline{\bbC}:=X\times\bbC$ be the trivial line bundle,
$\td$ can be regraded as the Chern connection on $\underline{\bbC}$.

$(E,h)$-Hermitian holomorphic vector bundle over $(X,\omg)$,
with Chern connection $D_E=D_E'+D_E''$. $(D_E''=\pbar)$.

$$C^\infty(X,\wedgeform{p,q}\ten E)$$
has an inner product induced by $\omg,h$.
$\rightsquigarrow$ adjoint operators $D_E^*=D_E'^*+D_E''^*$.

$\rightsquigarrow\yc_E=[D_E,D_E^*]=D_ED_E^*+D_E^*D_E$, and
$\yc_E'\,,\,\yc_E''$. (self adjoint, elliptic operators)

Question: relation between $\yc_E'$ and $\yc_E''$?

\begin{thm}(Bochner-Kodaira-Nakaino identity)
$$
  \yc_E''-\yc_E'
=
  \left[
    \sqrt{-1}
    \Theta_E
  ,
    \Lmd
  \right]
$$
where $\Theta_E$ is the Chern curvature of $D_E$.
\end{thm}

Recall: $\Theta_E=D_E^2$, when $D_E$ is Chern connectoin, we have
$$D_E'^2=0\qquad D_E''^2=0$$
i.e. $\Theta_E=[D_E',D_E'']$.

Remark: $E$ is flat(i.e. $D_E^2=0$)$\iff \yc_E'=\yc_E''$.

\begin{proof}
based on following identities:
$$[D_E''^*,L]=\sqrt{-1}D_E'$$
$$[D_E'^*,L]=-\sqrt{-1}D_E''$$
$$[\Lmd,D_E']=-\sqrt{-1}D_E'^*$$
$$[\Lmd,D_E'']=\sqrt{-1}D_E''^*$$

then (by super Jacobi identity):
\begin{eqnarray*}
  \yc_E''=[D_E'',D_E''^*]
&=&
  -\sqrt{-1}
  \left[
    D_E''
  ,
    [\Lmd,D_E']
  \right]
=
  -\sqrt{-1}
  \left(
    [\Lmd,[D_E',D_E'']]
   +[D_E',[D_E'',\Lmd]]
  \right)\\
&=&
  -\sqrt{-1}
  \left(
    [\Lmd,\Theta_E]
   +[D_E',\sqrt{-1}D_E'^*]
  \right)
\end{eqnarray*}
so,
$$\yc_E''-\yc_E'=[\sqrt{-1}\Theta_E,\Lmd]$$
\end{proof}

\begin{lemma}(normal frame)

Let $X$ be a complex manifold, then for any $x_0\in X$,
and any holomorphic chart $(z_1,...,z_n)$ centered at $x_0$,
there exists a holomorphic frame $\{e_\lmd\}_{\lmd=1}^{r:=rank E}$
of $E$ near $x_0$ such that
$$
  \left\langle
    e_{\lmd}(z),e_{\mu}(z)
  \right\rangle
=
  \delta_{\lmd,\mu}-
  \sum_{1\leq j,k \leq n}
  C_{jk\lmd\mu}
  z_j\zbar_k
+
 O(|z|^3)
$$
where $(C_{jk\lmd\mu})$ are the coefficients of the Chern curvature
$$\Theta_E(x_0)=
  \sum_{1\leq j,k\leq n\atop 1\leq\lmd,\mu\leq r}
    C_{jk\lmd\mu}
    \td z_j\wedge\td \zbar_{k}\ten e_\lmd^*\ten e_\mu
$$
\end{lemma}


need to verify: $\forall s\in C^\infty(X,\wedgeform{p,q}\ten E),x_0\in X$,
$$[D_E''^*,L]s(x_0)=\sqrt{-1}D_E's(x_0)$$
w.r.t the normal frame $(e_\lmd)_{\lmd=1}^r$ near $x_0$, assume
$$s=\sum_{\lmd=1}^{n}\sgm_\lmd\ten e_\lmd$$
then
$$
  D_Es(z)=\sum_{\lmd=1}^{n}
    \td\sgm_\lmd\ten e_\lmd +O(|z|)
$$
$$
  D_E^*s(z)=\sum_{\lmd=1}^{n}
    \td^*\sgm_\lmd\ten e_\lmd +O(|z|)
$$

$$D_E''^*=\sum_{\lmd=1}^{r} \td''^*\sgm_\lmd\ten e_\lmd+O(|z|)$$
$$
  \Rightarrow
  [D_E''^*,L]s=
  D_E''^*(\sum\omg\wedge\sgm_\lmd\ten e_\lmd)
 -\omg\wedge
 \left(
   \sum_{\lmd=1}^{r}
   \td''^*\sgm_\lmd\ten e_\lmd+O(|z|)
 \right)
=
  \sum_{\lmd=1}^{r}
    [\td''^*,L]\sgm_\lmd\ten e_\lmd+O(|z|)
$$
Similarly,
$$
  D_E's
=
  \sum_{\lmd=1}^{r}
    \td'\sgm_\lmd\ten e_\lmd+O(|z|)
$$
we have:
$$
  [d''^*,L]=\sqrt{-1}\td'
$$
(because $\omg$ is Kahler)

...

$(E,h)$ hermitian holomorphic vector bundle over Kahler manifold $(X,\omg)$.
we have BKN identity
$$
  \yc_E''-\yc_E'=[\sqrt{-1}\Theta_E,\Lmd]
$$

Recall: $L^2$-Hodge theory. $X$ compact manifold,then
$$
  H^{p,q}(X,E):=
  \frac{\ker D_E''}{\im D_E''}
\cong
  \ker\yc_E''
$$
(harmonic form)

Take $u\in C^\infty(X,\wedgeform(p,q)\ten E)$,applying BKN identity to $u$,
$$\yc_E''u-\yc_E'u=[\sqrt{-1}\Theta_E,\Lmd]u$$

note that
$$\ppair{\yc_E'u}{u}=\norm{D_E'u}^2+\norm{D_E'^*u}^2\geq 0$$
$$
  \Rightarrow
  \norm{D_E''u}^2+\norm{D_E''^*u}^2
\geq
  \ppair{[\sqrt{-1}\Theta_E,\Lmd]}{u}
$$
i.e.
$$
  \norm{D_E''u}^2+\norm{D_E''^*u}^2
\geq
  \int_X\pair{[\sqrt{-1}\Theta_E,\Lmd]}{u}
  \td Vol
$$

Observation: if $u\in\ker\yc_E''$, and $[\sqrt{-1}\Theta_E,\Lmd]$
has "positivity",
then $LHS=0$. So, $H^{p,q}(X,E)=0$.

\begin{definition}(Positivity)

We call $[\sqrt{-1}\Theta_E,\Lmd]$ is positive at $x_0\in X$, if
for any $0\neq v\in\left(\wedgeform{p,q}\ten E\right)_{x_0}$, we have
$$\pair{[\sqrt{-1}\Theta_E,\Lmd]v}{v}>0$$

....positive on $X$, if ... at each point
\end{definition}

\begin{thm}
If $[\sqrt{-1}\Theta_E,\Lmd]$ is positive on $X$, then
$$H^{p,q}(X,E)=0$$
\end{thm}

Special case: $E$ is a holomorphic line bundle, with Hermitian metric $h$,
$$\Theta_E=-\td'\td''\log h$$
$\Rightarrow\sqrt{-1}\Theta_E$ is a real $\td$-closed $(1,1)$-form on $X$.

locally,
$$
     \afa=
\sqrt{-1}\sum_{1\leq i,j\leq n}
  a_{ij}\td z_i\wedge\td\zbar_j
$$

$\afa$ is real $\iff$ $\afa=\overline{\afa}$,
(i.e. locally $(a_{ij})$ is an hermitian matrix)

\begin{definition}
a real $(1,1)$-form $\afa$ is called positive, if
$(a_{ij})_{ij}$ is positive definite.
\end{definition}

\begin{lemma}
If $\sqrt{-1}\Theta_E$ is positive, then $\omg:=\sqrt{-1}\Theta_E$ gives a
Kahler metric on $X$.
\end{lemma}

\begin{lemma}
If $\omg=\sqrt{-1}\Theta_E>0$, and $\Lmd$ is the adjoint of $L=\omg\wedge$,
then
$$[\sqrt{-1}\Theta_E,\Lmd]$$
is positive on $\wedgeform{p,q}\ten E$
whenever $p+q\geq n+1$.
\end{lemma}

\begin{lemma}
Let $\afa$ be a real $(1,1)$-form, $\omg$ a Kahler metric, assume
the eigenvalue of $\afa$ at $x_0$ is $\afa_1\leq\afa_2\leq\cdots\leq\afa_n$,
then (in the coordinate chart $(z_1,z_2,...,z_n)$, and
$u=\sum\limits_{|I|=p\atop|J|=q}u_{IJ}\td z_I\wedge\td\zbar_J$)
$$
  [\afa,\Lmd]u
=
  \sum_{I,J}
    \left(
      \sum_{i\in I}\afa_i
    + \sum_{j\in J}\afa_j
    - \sum_{k=1}^n\afa_k
    \right)
    u_{IJ}\td z_I\wedge\td\zbar_J
$$
\end{lemma}

\begin{cor}
$
  \afa=\omg
$, then
$$[\omg,\Lmd]u=(p+q-n)u$$
\end{cor}

\begin{cor}
Take an orthonormal frame $e$ of $E$, then
for any $u=\sum \limits_{|I|=p\atop|J|=q}u_{IJ}\td z_I\wedge\td\zbar_J\ten e$, we have
$$
  \pair{
  [\sqrt{-1}\Theta_E,\Lmd
  ]u
  }{u}
=(p+q-n)|u|^2
$$
\end{cor}

\begin{thm}
If $[\sqrt{-1}\Theta_E,\Lmd]$ is positive on $X$,
then
$$H^{p,q}(X,E)=0$$
\end{thm}

\begin{thm}If $E$ is a holomorphic line bundle
with a smooth hermitian metric $h$ s.t. $\sqrt{-1}\Theta_{(E,h)}\geq 0$,
then$H^{p,q}(X,E)=0$ whenever $p+q\geq n+1$.
\end{thm}
de Rham-Weil...$\cong H^q(X,\Omg^p\ten E)$.

\begin{definition}(canonical bundle)
$$K_X=\det T^*X$$
determinate bundle of cotangent bundle,
is called canonical bundle.
($\mcalO(K_X)=\Omg_X^n$)
\end{definition}


\begin{definition}
$X$ is called Fano, if $K_X^*=\det(TX)$ has a matric with positive curvature.

$X$ is called Calabi-Yau,if $K_X$ has a metric with vanishing curvature.

$X$ is of general type, if $K_X$ has a metric with positive curvature.
\end{definition}


\begin{cor}(Kodaira vanishing theorem)
$E$ is a positive line bundle, then
$$H^q(X,K_X\ten E)=0$$
for any $q\geq 1$.
\end{cor}

So, if $X$ is Fano, $(\iff K_X^*)$ positive,
$K_X\ten K_X^*=\underline{\bbC}$,
$\Rightarrow H^1(X,\mcalO)=0,\Rightarrow H^1(X,\bbR)=0$,

%%%%%%%%%%%%%%%%2019.5.14%%%%%%%%%%%%%%%%%%%%%%%%%5

Recall: BKN-inequality.

holomorphic Hermitian vector bundle $(E,h)\to (X,\omg)$, $\omg$ is Kahler.
For any $u\in C^\infty(X,\wedgeform{p,q}\ten E)$, we have
$$
  \norm{D''u}^2+\norm{D''^*u}^2
\geq
  \int_X
    \pair{[\sqrt{-1}\Theta_E,\Lmd_\omg]u}{u}
    \td Vol
$$

Recall: If $[\sqrt{-1}\Theta_E,\Lmd_\omg]$ is positive on
$C^\infty(X,\wedgeform{p,q}\ten E)$, then
$H^{p,q}(X,E)=0$.

\begin{thm}(Kodaira-Nakano vanishing theorem)

If $E$ is a holomorphic line bundle with a smooth metric $h$ s.t.
$\sqrt{-1}\Theta_{(E,h)}>0$,then
$[\sqrt{-1}\Theta_E,\Lmd_\omg]$ is positive on
$C^\infty(X,\wedgeform{p,q}\ten E)$ whenever $p+q\geq n+1$.

$\Rightarrow H^{p,q}(X,E)=0$ when $p+q\geq n+1$.
\end{thm}
(Last time)

Today:
\begin{thm}(Girbau vanishing theorem, 1976)

$E$ is a holomorphic line bundle over compact Kahler manifold,
with smooth metric $h$ s.t. $\sqrt{-1}\Theta_{(E,h)}\geq 0$,
and has at least $n-s+1$ positive eigenvalues at every points of $X$, then
$$H^{p,q}(X,E)=0$$
if $p+q\geq n+s$.
\end{thm}

$\afa$: a \textbf{real} $(1,1)$-form on $X$,
locally $\afa=\sqrt{-1}\sum\afa_{ij}\td z_i\wedge\td\zbar_j$.
then we have a matrix $M(\afa)=(\afa_{ij})_{n\times n}$,
($\afa$ is real $\Rightarrow$)a hermite matrix.

we call $\afa$ has at least $k$ positive eigenvalues at $x$,
if $M(\afa)(x)$ has $k$ positive eigenvalues.
(Remark: It is well defined)

\begin{proof}
Claim: there exists some Kahler metric $\omg$
s.t. $[\sqrt{-1}\Theta,\Lmd]$ is positive.

Fix a Kahler metric $\omg$,
for $p\in X$, choose a holomorphic chart $(z_1,...,z_n)$,
s.t. $\omg(p)=\sqrt{-1}\sum\td z_j\wedge\td\zbar_j$ and
$\sqrt{-1}\Theta_E(p)=\sqrt{-1}\sum\limits_{j=1}^{n}\gma_j\td z_j\wedge\td\zbar_j$.
WLOG, $0\leq\gma_1\leq\gma_2\leq\cdots\leq\gma_n$,
and for any $j\geq s$, $\gma_j>0$.

Consider
$$\omg_\veps:=\veps\omg+\sqrt{-1}\Theta_E$$
for $\veps>0$, then $\omg_\veps$ is a Kahler metric.
$\omg_\veps(p)=\sqrt{-1}\sum\limits_{j}(\veps+\gma_j)\td z_j\wedge\td\zbar_j$.

$\Rightarrow$ the eigenvalue of $\sqrt{-1}\Theta$ with respective to
$\omg_\veps(p)$ is given by
$$\gma_{j,\veps}=\frac{\gma_j}{\veps+\gma_j}=
\frac{1}{1+\frac{\veps}{\gma_j}}$$

Claim: $[\sqrt{-1}\Theta,\Lmd_{\omg_\veps}]$ is positive on
$\wedgeform{p,q}\ten E$ when $p+q\geq n+s$, $0<\veps<<1$.

Take $u=\sum\limits u_{IJ}\td w_T\wedge\td\wbar_J\ten e$,then
$$
  \pair{[\sqrt{-1}\Theta_E,\Lmd_{\omg_\veps}]}{u}
=
  \sum_{|I|=p\atop|J|=q}
    \left(
      \sum_{i\in I}
        \gma_{i,\veps}
     +\sum_{j\in J}
       \gma_{j,\veps}
     +\sum_{k=1}^{n}
       \gma_{k,\veps}
    \right)
    |u_{IJ}|^2
\geq
  (\gma_{1,\veps}+...+\gma_{p,\veps}-\gma_{q+1,\veps}-...-\gma_{n,\veps})
  |u|^2
$$
note that $\gma_{j,\veps}\geq 1-\frac{\veps}{\gma_s}$
if $j\geq s$, $\gma_{j,\veps}\in[0,1)$ for all $j$. it
$$
\geq
  \left(
    (q+s-1)(1-\frac{\veps}{\gma_s})
   -(n-p)
  \right)
  |u|^2
>0
$$
if $p+q\geq n+s$ and $0<\veps<<1$.
\end{proof}

\begin{rem}(Kawamata-Viewheg vanishing theorem)

$E\to (X,\omg)$ is a holomorphic line bundle over a compact Kahler manifold.

Definition: $E$ is called positive, ...(positive="ample" in AG).
numerically effective(nef) if for any $\veps>0$, there is a smooth metric $h_\veps$
s.t. $\sqrt{-1}\Theta_{h_\veps}\geq -\veps\omg$.

Theorem: If $E$ is nef, and $\int_X c_1(E)^n>0$, then
$H^{q}(X,K_X\ten E)=0$ for $q\geq 1$.
\end{rem}

\textbf{Positivity concept of vector bundles(rank $>1$)}

$(E,h)\to (X,\omg)$ Hermitian vector bundle of rank $r$,
over a complex manifold(may not Kahler).

Denote $(e_1,...,e_r)$ a local orthonormal frame of $E$,
$(z_1,...,z_n)$ local holomorphic chart,
Chern curvature of $(E,h)$:
$$
  \Theta_{(E,h)}
= \sum_{1\leq j,k\leq n\atop 1\lmd,\mu\leq r}
    c_{ik\lmd\mu}\td z_j\wedge\td\zbar_k\ten e_\lmd^*\ten e_\mu
$$

Fact: $\sqrt{-1}\Theta_E$ induces a Hermitian operator $\theta_E$
on $TX\ten E$.

Let $u,v$ be local sections of $TX\ten E$,
$$u=\sum_{1\leq j\leq n\atop1\leq\lmd\leq r}
u_{k\mu}\pp{z_k}\ten e_\mu
$$
$$
  \theta_{E}(u,v)
:=
  \sum_{1\leq j,k\leq n\atop 1\leq\lmd,\mu\leq r}
    c_{jk\lmd\mu} u_{j\lmd}\overline{v_{k\mu}}
$$

\begin{definition}
We call $E$ Nakano positive, if $\theta_E$ is positive.
(i.e for any non-zero local section $u\in TX\ten E$, $\theta_E(u,u)>0$)

We call $E$ Griffith positive, if for any
$0\neq \xi\in T_xX$, $s\in E_x,s\neq 0$,
$$\theta_E(\xi\ten s,\xi\ten s)>0$$
\end{definition}

\begin{rem}
By definition, Nakano positivity $\Rightarrow$ Griffith positivity.

If $E$ is line bundle,
Nakano positivity $\iff$ Griffith positivity.
(and $\iff$ positivity of lines bundles)
\end{rem}

\begin{thm}(Demailly-Skota, 1979)

$E$ is Griffith positive $\Rightarrow$ $E\ten \det E$ is Nakano positive.
\end{thm}
\begin{proof}
  Omit. Non-trivial.
\end{proof}

Notation: $E>_{Nak}0$ ($E$ is Nakano positive).
Similarly, $E>_{Giff}0$...

\begin{prop}

(1)$E$ is Griffith positive if and only if $E^*$ is Griffith negative.

(2) Consider an exact sequence of holomorphic vector bundles:
$$0\to S\to E\to Q\to 0$$
then if $E$ is Griffith positive, then $Q$ is Griffith positive.
If $E$ is Griffith negative, then $S$ is Griffith negative.
If $E$ is Nakano negative, then $S$ is Nakano negative.
\end{prop}

\begin{proof}
Omit. Compute curvature...
\end{proof}

Remark: In general, $E$ is Nakano positive, $\not\Rightarrow$
$Q$ is Nakano positive.

\begin{thm}(Nakano vanishing theorem)

$(X,\omg)$ is compact Kahler of dimension $n$,
$(E,h)$ is a Nakano positive holomorphic Hermitian vector bundle, then
$$H^{n,q}(X,E)=0\qquad \forall q\geq 1$$
\end{thm}

\begin{proof}
$E$ is Nakano positive, check:
$$[\sqrt{-1}\Theta_E,\Lmd_\omg]$$
is positive on $\wedgeform{n,q}\ten E$ for $(q\geq 1)$
\end{proof}

\section{Ampleness}
%\textbf{Ampleness}

$E\to X$, $E$: holomorphic line bundle of rank $r$,
$X$:complex manifold.

\begin{definition}(Jet vector bundle)
$$J^kE=\bigcup_{x\in X}(J^kE)_x$$
where
$$(J^kE)_x=\mcalO_x(E)\Big/\mfkm_x^{k+1}\mcalO_x(E)$$
$\mfkm_x\subseteq\mcalO_x$ be the maximal ideal of $\mcalO_x$.
\end{definition}

In local coordinate,
$$
  (J^kE)_x
=
  \Big\{
    \sum_{1\leq\lmd\leq r \atop |\afa|\leq k}
      C_{\lmd\afa}(z-x)^{\afa} e_{\lmd}(z)
  \Big\}
$$

\begin{prop}
$J^kE$ is a holomorphic vector bundle of rank $=r{n+k\choose n}$.
\end{prop}

\begin{proof}
  Exercise.
\end{proof}

\begin{definition}
$E$ is called very ample, if the following maps:
$$H^0(X,E)\to (J^1E)_x$$
$$H^0(X,E)\to E_x\oplus E_y$$
are surjective, for all $x,y\in X$, $x\neq y$.

$E$ is called ample, if $S^mE:=\Sym^m E$ is very ample for some $m\in\bbN$.
\end{definition}

(ample: "足够多的全纯截面")

\begin{thm}(Kodaira)

$L$-holomorphic line bundle, $X$ is a compact complex manifold.
Then $L$ is positive if and only if $L$ is ample.
\end{thm}

%%%%%%%%%%%%%%%%%2019.5.16%%%%%%%%%%%%%%%%%%%%%%%%%%
%%%%%%%%%%%%%%%%%第12周了吗?%%%%%%%%%%%%%%%%%%%%%%%

\section{除子}

We will prove:

\begin{thm}
$L\to X$ holomorphic line bundle over a compact complex manifold,
then $L$ is positive $\iff L$ is ample.
\end{thm}

\textbf{We need:}

(1)Kodiara vanishing theorem.

(2)Blow-up of complex manifold

(3)Relation between divisor and line bundles.

\textbf{analytic cycles, divisors and meromorphic functions}

\begin{definition}
$X$ be a analytic set in some complex manifold, then the set
$X_{reg}$ is a dense subset of $X$.
Denote the connected component of $X_{reg}$ by $X_\afa$,
$\overline{X_\afa}$ is the closure of $X_\afa$ in $X$,
then $\overline{X_\afa}$ is called a global irreducible component of $X$.

In particular, $X$ is the union of global irreducible components.
\end{definition}

\begin{example}(Global irreducibility is different from local irreducibility)

$V=\Bigset{(x,y)\in\bbC^2}{y^2=x^2(1+x)}$ is an analytic set in $\bbC^2$,
$V_{reg}=V\setminus\{0\}$ is connected. So,
$V=\overline{V_{reg}}$ is globally irreducible.

On the other hand, $(V,0)$ is a reducible as an analytic germ.
\end{example}

\begin{definition}(analytic cycles)

$X$ is a complex manifold,  a $q$-cycle (with integer coefficient)
is a formal linear combination $\sum\limits\lmd_j V_j$, $\lmd_j\in\bbZ$,
and $V_j$ is a global analytic sets of $X$ of dimension $q$.
\end{definition}

So, we get a group $C_{cyl}^q(X)$.

an element of $Cycl^{n-1}(X)$ is called a divisor.
(Weil divisor)
($Div(X)$)

If $D$ is an irreducible analytic set of dimension $n-1$
then the divisor given by $D$ is called a prime divisor.

\begin{rem}
For any open set $U\subseteq X$, $U\to Cycl^q(U)$ induces a sheaf
$Cycl^q$ of $X$ with the germ $Cycl_x^q$ given by $q$-dimension analytic germs at $X$.
\end{rem}

\begin{thm}
$X$ is a connected complex manifold, $f\in\mcalO(X)$,
then we have $f^{-1}(0)$ is emply of $\dim_{\bbC}$ isempty of $n-1$.
\end{thm}


\begin{definition}(Cartier-divisor)

A divisor $D=\sum\lmd_j D_j$ locally given by a $\bbC$ linear combination of $div(f)$.
$f$ is locally holomorphic functions.
\end{definition}

\begin{definition}
$X$ is a compact , $\beta\in\mcalO(X)$,
$D_j$ is a global irreponent of $f^{(-1)0}$,
$$m_j:=Ord_z(f)$$
for all $z\in D_j{\text{reg}}\setminus \bigcup_{k\neq j} D_k $
$m_j$ be the vanishing order along $D_j$.
\end{definition}

\begin{thm}
$(A,x)$ an analytic germ of $\dim_{\bbC}$=n-1.
$(A,x)=(g)$for sone $g\in \mcalO X$,and $g$ is a product of
$(J_{A_j,x})=(g_j)$.

(2) Let $f\in\theta_x$ with $(f^{-1}(0),x)\subseteq(A,x)$,then
$f=u\coprod_j g_jm^{m_j}$, where $m_j=ord_z (f)$
\end{thm}

\begin{prop}If $X$ is a complex manifold, then any Weil divisor is
also a Cartier divisor.
\end{prop}

Remark: NOT true for singular points.

Meromorphic function: $X$ complex manifold, $\mcalO_X$
sheaf of functions on $X$.
$$\mfkm_x:=\Bigset{\frac{g_x}{h_x}}
{g_x,h_x\in\mcalO_x\text{and $h_x$ is not zeor in $\mcalO_x$}}$$
$$
  \mcalM:=\bigcup_{x\in X}\mfkm_x
$$
  with the topology given by the basis
$$
  \Bigset{\frac{G_x}{H_x}}
  {x\in V, G,H\in\mcalO(V)}
$$

\begin{example}
$f(z_1,z_2)=\frac{z_1}{z_2}$
\end{example}

\begin{definition}
Let $F\in\mfkm(X)$,denote
$P(X):=\not\in \Bigset{x\in X}{f_x\not\in\mcalO_x}$.
Pole set pf $f$, and
$Z(f):=P(\frac{1}{z})$ zero set of $f$.
\end{definition}

\begin{thm}
$f\in\mfkm(X)$, if $P(d)$ (or$Z(f)$) is not empty,
then $P(f)$ is analytic set of $\dim=\bbH$.
\end{thm}

\begin{definition}
$P(f)\cup Z(f)$ is called the indeterminiary of set of $f$,
(in particular, codimension $P(M)\cap Z(f)\geq 2$)
\end{definition}

\begin{prop}
Given $f\in\mcalM(X)$, we get a divisor:
$$div(f)=\sum a_jA_j-\sum b_jB_j$$
where $a_j=$ the vanishing order of $f$ along $A_j$,
$A_j$ a globally irreducible component of $Z(f)$,
$b_j=$...along of $\frac{1}{f}$ along $B_j$,
$B_j$:...component of $P(f)$.
\end{prop}

\begin{example}
$f=\frac{z_1}{z_2}\in\mcalM(\bbC^2)$, then
$P(f)=\{z_2=0\}$ and $Z(f)=\{z_1=0\}$, and
$$div(f)=[z_1=0]-[z_2=0]$$
\end{example}

Consider: $X$ - complex manifold, $\mcalO^*$:
sheaf of invertible holomorphic functions,

$\mcalM^*$: Sheaf of non-zero meromorphic functions

$\mcalD iv$: Sheaf of $(n-1)$-cycles.

\begin{prop}
We have an exact sequences:
$$0\to\mcalO^*\to\mcalM^*\to\mcalD iv\to 0$$

In particular, $\mcalD iv=\mcalM^*/\mcalO^*$.

long exact sequence:

$$0\to H^0(X,\mcalO^*)\to H^0(X,\mcalM^*)\to H^0(X,\mcalD iv)
\to H^1(X,\mcalO^*)\to H^1(X,\mcalM^*)\to\cdots$$
\end{prop}
where, note that :
$$H^0(X,\mcalD iv)=Div(X)\qquad H^1(X,\mcalO^*)=Pic(X)$$


Consider $Div(X)=H^0(X,\mcalM^*/\mcalO^*)\to Pic(X)$,
$f\in H^0(X,\mcalM^*/\mcalO^*)\iff$ we have an open covering
$X=\bigcup_i U_i$ and $f_i\in\mcalM^*(U_i)$ with
$\frac{f_i}{f_j}\in\mcalO^*(U_i\cap U_j)$.

$f\in H^0(X,\mcalM^*/\mcalO^*)\xra{\fai}
(U_i\cap U_j,g_{ij}\in\mcalO^*(U_i\cap U_j))
\in \check{H}^1(\mcalU,\mcalO^*)\inj H^1(X,\mcalO^*)$.

\begin{definition}
A divisor $D$ is called principal divisor, if $D= div(h)$ for some
$h\in\mcalM^*(X)$.
\end{definition}

\begin{prop}
$\ker\fai=\{\text{principal divisors}\}$,
i.e. $\mcalO(D)$ is trivial $\iff$ $D= div(f)$ for some global meromorphic functions.
\end{prop}

\begin{prop}
$$\mcalO(D_1+D_2)=\mcalO(D_1)\ten\mcalO(D_2)$$
$$\mcalO(-D)=\mcalO(D)^*$$
\end{prop}

\begin{definition}
$D_1,D_2\in Div(X)$ is called linear equivalent, if $D_1-D_2$ is principal,
denoted by $D_1\sim D_2$. We have an injection:
$$Div(X)/\sim\inj Pic(X)$$
\end{definition}

Remark: in general, $D\to\mcalO(D)$ is not surjective.

If $X\inj\bbP^n$, then $Div(X)/\sim\cong Pic(X)$.

\begin{prop}
$L\to X$ holomorphic line bundle over a complex manifold,
we have a canonical map:
$$H^0(X,L)\setminus\{0\}\to Div(X)$$
$$s\to Z(s)$$
\end{prop}

\begin{proof}
$s\in H^0(X,L)\iff$ the data $(U_i,f_i\in\mcalO(U_i))$,
$L$ is determined by $g_{ij}\in\mcalO^*(\mcalU_i\cap\mcalU_j)$.

$Z(s)$ locally given by $div(f_i)$.
($div(f_i)=div(f_j)$ on $U_i\cap U_j$)
\end{proof}

\begin{prop}
$s_i\in H^0(X,L_i)\setminus\{0\}, i=1,2$ ,we have
$Z(s_1\ten s_2)=Z(s_1)+Z(s_2)$.
\end{prop}

\begin{prop}
Let $s\in H^0(X,L)\setminus\{0\}$, then $\mcalO(Z(s))\cong L$.
\end{prop}
\begin{proof}
Assume $X=\bigcup U_i$ with $L$ determined by $g_{ij}\in\mcalO^*(U_i\cap U_j)$,
$s\in H^0(X,L)$ determined by $(U_i,f_i\in\mcalO(U_i))$.

so,$\mcalO(Z(s))$ is the line bundle given by $\frac{f_i}{f_j}\in\mcalO^*(U_i\cap U_j)$.

note that $f_i=g_{ij}f_j$.
\end{proof}

\begin{cor}
Let $s_i\in H^0(X,L_i)\setminus\{0\},i=1,2$, then
$$Z(s_1)\sim Z(s_2)\iff L_1\cong L_2$$
\end{cor}
use the fact: $\mcalO(Z(s_i))=L_i$ and $\mcalO(\text{principal divisor})\cong\mcalO_X$
trivial line bundle.

\begin{prop}
Consider the map
$$Div(X)\to Pic(X)$$
$$D\to\mcalO(D)$$
then the image is generated by line bundles with non-zero holomorphic sections.
\end{prop}

%%%%%%%%%%%%2019.5.21%%%%%%%%%%%%%%%%%%%
%%%%%%%%%%%快期末了,抓紧复习%%%%%%%%%%%%%%

\section{Blow-up}

Local picture: $U\subseteq\bbC^n$ open subset, $Y\subseteq U$
linear subspace, $codim_UY=k$, e.g. assume $Y=\Bigset{z\in U}{z_1=...=z_k=0}$.

Consider the space
$$U_Y:=\Bigset{([w],z)\in\bbP^{k-1}\times U}{w_iz_j=w_jz_i,\,1\leq i,j\leq k}
\subseteq\bbP^{k-1}\times U\xra{\pi_2}U$$

\begin{definition}
$U_Y$ is called the blow-up of $U$ along $Y$.
\end{definition}

\begin{prop}
$U_Y$ is a smooth complex submanifold of $\bbP^{k-1}\times U$, and
$\dim_{\bbC}U_Y=\dim_{\bbC}U=n$. And
$\tau:U_Y\to U$ is a holomorphic map with
$$
  \tau|_{U_Y\setminus\tau^{-1}(Y)}:
  U_Y\setminus\tau^{-1}(Y)\cong U\setminus Y
$$

And for any $y\in Y$, $\tau^{-1}(y)=\bbP^{k-1}\times\{y\}$
is complex projective space.

\end{prop}

Locally, on then chart $w_1\neq 0$, denote $\hat{w}_i=\frac{w_i}{w_1}$
for all $2\leq i\leq k$. Then $z_i=\what_iz_1$.Then
$(z_1,\what_2,...,\what_k,z_{k+1},...,z_n)$ gives a holomorphic chart
of $U_Y$.

Denote $(z_1,...,z_n)=(z_1,\what_2,...,\what_k,z_{k+1},...,z_n)$,
then $z_1=\xi_1$, $z_2=\xi_1\xi_2$,...,$z_k=\xi_1\xi_k$, and
$z_{k+l}=\xi_{k+l}$for $k\geq l$.

In this coordinate system, $\tau^{-1}(Y)=
\Bigset{\xi\in U_Y}{\xi_1=0}$.

$\Rightarrow\tau^{-1}(Y)$ is a (smooth) hypersurface in $U_Y$.And,
$\tau^{-1}(Y)\cong \bbP(N_{Y/U})$,
where $N_{Y/U}$ is the normal bundle of $Y$ in $U$.
$$(0\to T_Y\to T_U|_Y\to N_{Y/U}\to 0)$$

If $codim_UY=1$ hypersurface, then $U_Y\cong U$.

\textbf{Global construction}

$Y$ is a complex submanifold of $X$, $\dim_{\bbC}=n$,$\dim_{\bbC}Y=k\leq n$.

\begin{lemma}
  If $f_1,...,f_k$ and $g_1,...,g_k$ are two (local) definition of $Y$,
defining equations of $Y$, $Y=\Bigset{f_z(z)=...=f_k(z)=0}{,}$,
then $\td f_1,...,\td f_k$ are linely independent along $Y$.
And $\exists$ a matrix $(m_{ij})$ of holomorphic functions,
s.t. $g_i=\sum_{j=1}^{k}M_{n,j}f_j$ for any $1\leq i\leq k$.

The matrix $(M_i^j)$ is invertible along $Y$,and determined uniquely
by $(f_1,...,f_k)$ and $g_1,...,g_k$.
\end{lemma}

\begin{proof}
Assume $f_i=z_i$ for $1\leq i\leq k$ is a local coordinate system
$\equiv 0$.For ever $g_i$, $g_i|_{z_1,...,z_k=0}$

Consider the Taylor expansion of $\\g_i$, we set
$$g_i=\sum_{j=1}^kM_i^j(z)z_j$$

$\td g_i=\sum_{j=1}^{k}\td M_{i}^jz_j+
\sum_{j=1}^{k} M_i^j\td z_j$.

$(\td g_1,...,\td g_k)|_{Y}$
and $(\td z_1,...,\td z_k)|_Y$ are $L.I$,
so $M_i^j|_{Y}$ is invertible.

Assume $Y\cap U=\{f_1^U=...=f_k^{U}=0\}$,
$Y\cap V=\{f_1^V=f_2^V=...=f_k^V=0\}$ and
$(M_{i,UV}^j)_{1\leq i,j\leq k}$ is the
\end{proof}

$0\to T_Y\to T_X|_Y\to N_{Y/D}$, the dual
$$N^*_{Y/ X}\to T_X^*|_Y\to T^*_Y$$

$(M_i^j,UV)$ gives the translation matrix middle of $N_{Y/X}^*$

\begin{lemma} $\exists$ isomorphism $\phi_{UV}:
\tau_U^{-1}(U\cap V)\cong \tau_V^{-1}(U\cap V)$.
\end{lemma}

\begin{proof}
  Assume $f_i^U=\sum_{j=1}^{k}=\sum_{j=1}^{k}M_{i,UV}^jf_j^V$.
Define $\phi_{UV}([w],z)=([M^{-t}w],z)$, then
$\phi_{UV}$ satisfies the two properties.
\end{proof}

\begin{definition}(The blow-up of $X$ along $Y$)(Global blow up)

$\Bl_YX$:the blow-up of $X$ along $Y$
is defined as the complex manifold by gluing the $U_Y$
and $\Omg:=X\setminus S_Y$,
where $S_Y$ is some neighborhood of $Y$.
\end{definition}

we have a holomorphic map: $\tau:\Bl_YX\to X$.

\begin{prop}
$\tau:\Bl_YX\to X$ satisfies :

(1)$\tau^{-1}(Y)$ is a smooth complex submanifold of $\Bl_YX$,
with $\dim_{\bbC}=n-1$,(It is called the excepted divisor of $\tau$)

(2)$\tau:\Bl_YX\setminus \tau^{-1}(Y)\to X\setminus Y$ is an isomorphism.

(2)$\tau$ is a proper map(any pre-image of compact set is compact).
\end{prop}

\begin{proof}
Check.
\end{proof}

\textbf{projective bundle} $E\to X$ is a holomorphic vector bundle(of rank $r$)
over a complex manifold(of complex dimension $n$),
then we can define projective bundle $\bbP(E)$,
$$\bbP(E):=\Bigset{(x,[\xi])}{x\in X,\,\xi\in E_x\setminus\{0\}}$$

$\bbP(E)$ is a complex manifold of dimension $n+r-1$
(if $X=\{pt\}$, then $\bbP(E)$ is just the projective space)

We have a tautological line bundle on $\bbP(E)$:
$$\mcalO_E(-1)_{(x,[\xi])}=\bbC\xi$$
$\mcalO_E(-1)$ is a holomorphic line bundle on $\bbP(E)$.

\textbf{Exercise:}Assume $(E,h)$ is an hermitian vector bundle with
metric $h$,then $h$ induces a metric on $\htil$ on $\mcalO_E(-1)$,
then the Chern curvature $\Theta$ of $\htil$ satisfies:
for any $x\in X$, $\sqrt{-1}\Theta|_{\bbP(E_x)}<0$.

\begin{thm}
  $\tau:\Bl_YX\to X$ blow-up along $Y$, $E:=\tau^{-1}(Y)$
exceptional divisor, $\mcalO(E)$:
the holomorphic line bundle associated to $E$, then

(1) $\tau: E\to Y$ is just the map $\bbP(N_{Y/X})\to Y$

(2) $\mcalO(E)|_E\cong \mcalO_{P(N_{Y/X})}(-1)
\cong N_{E/\Bl_YX}$ the normal bundle of $E$ in $\Bl_YX$.
\end{thm}

\begin{proof}
  Exercise.
\end{proof}

\begin{cor}
If $X$ is a (compact) Kahler manifold, $Y$ is a compact submnifold
of $X$, then the blow-up $\Bl_YX$ is also a (compact) Kahler manifold.
\end{cor}

\begin{proof}
$\tau:\Bl_YX\to X$, let $\omg$ be a Kahler matric on $X$,
then $\tau^*\omg$ is a semi-positive $(1,1)$-form on $\Bl_YX$,
positive on $\Bl_YX\setminus E$, and the kernel of $\tau^{-1}\omg$
along $E$ is given by the tangent space of the fiber $E\to Y$.

Define the metric $h$ on $\mcalO(E)$ as follows:
on $E$,$h$ is induced by the metric on $N_{Y/X}$ induced by the metric on $N_{Y/X}$,
and we extend $h$ to a neighborhood of $E$;
outside a neighborhood of $E$,($\mcalO(E)|_{\Bl_YX\setminus E}$ is trivial),
$h$ is given by the trivial metric.

Then ,we glue these two metrics to get a matric on $\mcalO(E)$.
Denote the curvature $\theta:=\sqrt{-1}\Theta(\mcalO(-E),h)$/

Claim: $C\tau^*\omg+\theta>0$ for $C\gg 1$
\end{proof}

%%%%%%%%%%%2019.5.23  第13周%%%%%%%%%%%%%%%%%%%%%


\section{Kodaira Embedding Theorem}

Recall: $L\to X$ holomorphic line bundle with a smooth metric $h$
over compact complex manifold.

$L$ is called positive if the curvature $\sqrt{-1}\Theta_{(L,h)}$
is a positive $(1,1)$-form.

$L$ is called ample, if $L^{\ten m}:= mL$ is very ample for $m\gg 1$.

Recall: a holomorphic vector bundle $E$ is called very ample,
if the following maps
$$H^0(X,E)\to E_x\oplus E_y\qquad \forall x\neq y\in X$$
$$H^0(X,E)\to (J^1E)_x\qquad \forall x\in X$$
are surjective.

\begin{prop}
$X$ is a complex manifold of dimension $n$, $Y\subseteq X$
is a complex submanifold of codimension $k$. $\tau:\Xhat\to X$
blow-up along $Y$. $E:=\tau^{-1}(Y)$ exceptional divisor. Then
$$K_{\Xhat}=\tau^{*}K_X\ten\mcalO((k-1)E)$$
\end{prop}

(Recall: $K_X=\det T^*X=\wedgeform{n}T^*X$,
locally free sheaf of holomorphic $n$-terms $\Omg_X^n$).

\begin{proof}
locally, $\tau$ can be written as
$$\tau:(w_1,...,w_n)\to (z_1,...,z_n)$$
$$z_1=w_1,\,z_2=w_2,...,z_k=w_kw_1,...,z_{k+l}=w_{k+l}$$

$$
  \Rightarrow
  \tau^*(\td z_1\wedge\td z_2\wedge\cdots\wedge\td z_n)
= w_1^{k-1}\td w_1\wedge\td w_2\wedge\cdots\wedge\td w_n
$$
(local holomorphic frame of $K_X$ and $K_{\Xhat}$...
$w_1^{k-1}$-local section of $\mcalO(E)$)


Recall: $L$-line bundle, $\{g_{ij}\}$ transition function,
a local section is the following data
$f_i=g_{ij}f_j$. If $e_i$ the local frame on $U_i$, then
$f_ie_i=f_je_j$ on $U_i\cap U_j$.

之后check两个线丛的转移函数相同.
\end{proof}



\begin{lemma}Let $\Xhat$ be the blow up of $X$ along
$\{x_1,...,x_N\}\subseteq X$, ($N$ distinct points),
denote $E$ the exceptional divisor, then
$$
  H^1(\Xhat,\mcalO(-mE)\ten\tau^*(kL))=0
$$
for $m\geq 1$, $k\geq Cm$ for $C\gg 1$
\end{lemma}

\begin{proof}
$$
  H^1(\Xhat,\mcalO(-mE)\ten\tau^*(kL))
=
  H^1(\Xhat,K_{\Xhat}\ten K^{-1}_{\Xhat}\ten\mcalO(-mE)\ten\tau^*(kL))
=
  H^{n,1}(\Xhat,F)
$$
where $F:= K^{-1}_{\Xhat}\ten\mcalO(-mE)\ten\tau^*(kL)$.

By Kodaira-Nakano vanishing, if $F$ is positive,
then $H^{n,1}(\Xhat,F)=0$.

Note that
$$F=\mcalO(-mE)\ten\tau^*K_X^{-1}\ten\mcalO((1-n)E)\ten\tau^*(kL)$$
$$=\tau^*K_X^{-1}\ten\mcalO(-(m+n-1)E)\ten\tau^*(kL)$$

We know, $\exists C_0\gg 1$ s.t. $C_0L\ten K_X^{-1}$ is positive,
and $\exists C\gg 1$,s.t. $C\tau^*L\ten\mcalO(-E)$ is positive.

So, For $k\geq Cm\,\,(C\gg 1)$, $F$ is positive.

Let $v_j\in H^0(\Omg_j,kL)$ be a local section of $kL$,
s.t. $v_j$ generates the $m$-jet at $x_j$. Let
$\psi_j\in C^\infty(X,\bbR)$ s.t. $supp\psi_j\subset\subset\Omg_j$,
$0\leq\psi_j\leq 1$, $\psi_j\equiv 1$ around $x_j$. Denote
$$
  v:=\sum_{j=1}^{n}
    \psi_jv_j
$$
a smooth section of $kL$.
$$
  \td''v=\sum_{j}\td''\psi_jv_j\in C_{(0,1)}^{\infty}(X,kL)
$$
satisfies $\td''v=0$ near $x_j$ for $1\leq j\leq N$.

Lemma:(Exercise)
$$H^0(X,M)\to H^0(\Xhat,\tau^*M)$$
$$s\mapsto\tau^*s$$
is an isomorphism for any line bundle $M$.

Lemma:(Exercise)
a section of $\tau^*M$ with vanishing order$=k$ along $E$ is the
pull-back of a section of $M$ with vanishing order $=k$ at $x_j$.
\vs

Denote $S_E\in H^0(\Xhat,\mcalO(E))$ the canonical section of $E$,
$$
  w=S_E^{-(m+1)}\ten\tau^*(\td''v)\in
  C_{(0,1)}^{\infty}
  (\Xhat,\mcalO(-(m+1)E)\ten\tau^*(kL))
$$
and $\td''w=0$. Vanishing of $H^0(\Xhat,\mcalO(-(m+1))\ten\tau^*(kL))$
implies $w=\td'' u$ for some $u\in C^\infty(\Xhat,\mcalO(-(m+1)E)\ten\tau^{-1}kL)$.
$$S_E^{-(m+1)}\tau^*(\td''v)=\td''u$$
$$\Rightarrow\td''(\tau^*v-S_E^{(m+1)}u)=0$$
so, $\tau^*v-s_E^{(m+1)}u$ is a holomorphic section of $\tau^*(kL)$.
Using $s_E^{(m+1)}u=\tau^*f$ for some $f\in H^0(X,kL)$ with vanishing order
$= m+1$ along $x_j$.

Claim: denote $g:= v-f$ is the holomorphic sections generating
the $m$-jets at $x_j$. $\td''(\tau^*g)=0\Rightarrow\tau^*g$ is holomorphic,
$\Ord_{x_j}(f)=m+1$. So, $J^m(g)_{x_j}=J^m(v)_{x_j}$.

\end{proof}

\begin{thm} $L\to X$ positive line bundle, $x_1,...,x_N\in X$
are $N$ distinct points on $X$, then there exists $C>0$, s.t.
$$
  H^0(X,kL)\surj\bigoplus_{j=1}^N
  \left(
    J^m(kL)
  \right)_{x_j}
$$
is surjective for all $m\geq 0$ and $k\geq Cm$
\end{thm}

\begin{proof}
\end{proof}


\begin{thm}(Kodaira)

Line bundle $L$ is positive $\iff$ it is ample.
\end{thm}

(微分几何的正性与代数几何的正性是等价的)

\begin{proof}(有一边是显然的,留作习题)

proof of "$L$ ample $\Rightarrow$ $L$ positive".

Exercise: If $A$ is a very ample line bundle on $X$,
$H^0(X,A)$ has a basis $\{s_0,...,s_N\}$, then the map
$$\Phi:X\to\bbP(H^0(X,A))$$
$$s\mapsto[s_0(x);s_1(x);...;s_N()]$$
(Kodaira map) is a holomorphic embedding.

(Hint: $H^0(X,A)\surj A_x\oplus A_y$ means that $\Phi$ is injective;
$H^0(X,A)\surj(J^1(A))_x$ means that $\Phi_*$ is injective.)

Exercise: denote the tautological line bundle on $\bbP(H^0(X,A))$
by $\mcalO(1)$, then $A=\Phi^*\mcalO(1)$.

Cor: $A$ is very ample $\Rightarrow A$ is positive.

Given any inner product on $H^0(X,A)$, we get a metric $h$ on $\mcalO(1)$,
the curvature $\Theta(\mcalO(n))$ of $h$ is positive.
$$\Rightarrow\Theta(A)=\Phi^*\Theta(\mcalO(1))$$
$\Phi$ is embedding $\Rightarrow$ $\Theta(A)$ is positive.
\end{proof}

$L$ positive $\Rightarrow L$ ample, i.e. $mL$ is very ample,
$$\Rightarrow\Phi_{H^0(X,mL)}:X\inj\bbP(H^0(X,mL))$$
holomorphic embedding($\Rightarrow X$ is an analytic submanifold of $\bbP(H^0(X,mL))$)

$\xra{\text{Chow theorem}}X$ is an algebraic set of $\bbP(H^0(X,mL))$
(i.e. $X=\bigcup\limits_{j=1}^t\{P_j=0\}$, $P_j$-homogenous polynomial)

a compact complex manifold $X$ admitting a positive line bundle $L$
if and only if $X$ is an algebraic manifold.

$$
  0\to\bbZ\to\mcalO\xra{e^{2\pi\sqrt{-1}}}\mcalO^*\to 0
$$
$$\rightsquigarrow
H^1(X,\mcalO^*)\xra{C_1}H^2(X,\bbZ)\to H^2(X,\mcalO)\to...
$$
and $H^2(X,\bbZ)\to H^2(X,\bbC)\cong H^{2,0}\oplus H^{1,1}\oplus H^{0,2}$,
and $H^2(X,\mcalO)\cong H^{0,2}(X,\bbC)$.

$\Rightarrow\forall\afa\in H^2(X,\bbZ)\cup H^{1,1}(X,\bbC)$,
we have a holomorphic line bundle $L$ s.t. $\afa=c_1(L)$.

$L$ admitting a positive line bundle $\iff$ $X$ admitting a class
$\afa\in H^{2}(X,\bbZ)\cup H^{1,1}$ with a positive representative.

%%%%%%%%%%%%%20190528%%%%%%%%%%%%%%%%%%%%%%%%
%%%%%%%%%%%%%第十四周%%%%%%%%%%%%%%%%%%%%%%%%%
Recall:
\begin{thm}
$L\to X$ positive line bundle over compact complex manifold, then
$\forall x_1,...,x_N\in X$,$\exists C>0$(depends on $X$), s.t.
$$
  H^0(X,L^k)
\surj
  \bigoplus_{i=1}^N
  (J^mL^k)_{x_i}
\eqno{(*)}
$$
whenever $m\geq 0$ and $k\geq C(m+1)$
\end{thm}


For fixed $(x_1,...,x_N)$, we proved $\exists C(x_1,...,x_N)>0$,
s.t. $(*)$ holds.

Observation:$(*)$ is an open condition with respect to $(x_1,...,x_N)$.

$\Rightarrow\exists$ open set $U(x_i)$
s.t. $\forall(y_1,...,y_N)\in\prod_{i=1}^{N}U(x_i)$,
$(*)$ holds for $C=C(x_1,...,x_N)$.

$m=0,N=1$,$H^0(X,L^k)\surj (L^k)_x\iff\exists$ section $s$
s.t. $s(x)\neq 0$ (for $y$ near $x$, $s(y)\neq 0$)

$\pi:Y\to X$ blow-up along $x_1,...,x_N$, with exception divisor $E$,

\textbf{FACT:}$\exists C\gg 1$, s.t. $C\pi^*L+\mcalO(-E)$ is positive.

(这些已证明)

(more generally, if $\omg$ is a Kahler metric on $X$,denote
$\{\omg\}\in H^{1,1}(X,\bbR)$
the Kahler associated to $\omg$,
then $\exists C\gg 1$ s.t. $C\pi^*\omg+c_1(-E)$
is a Kahler class)

\begin{prop}
Define the Seshadri constant
$$
    \mcalE(x_1,...,x_N;\omg)
:=  \sup\Bigset{t\geq 0}
               {\pi^*\omg+t\cdot c_1(-E)\text{is a Kahler class}}
$$
Then $\mcalE(x_1,...,x_N;\omg)$ is a lower-semi-continuous
function w.r.t $x_1,...,x_N$ .

So,
$$
  \inf\Bigset
  {\mcalE(x_1,...,x_N;\omg)}
  {(x_1,...,x_N)\in \underbrace{X\times\cdots\times X}_{N}}>0
$$
\end{prop}
\begin{proof}
  Too difficult. omit.
\end{proof}

\begin{rem}(如果感兴趣)

Nagata conjecture

Biran-Nagata conjecture

Symplectic packing/embedding of bundles
\end{rem}

\begin{thm} $L$ is a positive line bundle, for $k\gg 1$,
$$
  \Phi_{H^0(X,L^k)}:
  X\inj\bbP(H^0(X,L^k))
$$
$$
  x\mapsto[s_0(x):...:s_N(x)]
$$
is a holomorphic embedding.(Where $\{s_j\}_{j=0}^N$
is a basis of $H^0(X,L^k)$)

So, (Chow theorem), $X$ is an algebraic manifold.
\end{thm}

Chow theorem 1949:

\begin{thm}(Chow theorem ,1949)

Let $A$ be an analytic set of $\bbP^n$,
then $A$ is an algebraic set, i.e.
$$
  A=\bigcap_{j=1}^N
  \{P_j(z_0,...,z_n)=0\}
$$
where $P_j$ is a homogeneous polynomial.
\end{thm}

Using the Remmert-Stein theorem:

$X$- a complex manifold, $A\subseteq X$ an analytic set,
$Z\subseteq X\setminus A$ is an analytic subset (of $X\setminus A$).
If $\dim(Z,x)>\dim A$ for all $x\in Z$,
then the closure $\overline{Z}$ in $X$ is also an analytic set of $X$.

Consider the natural map $\pi:\bbC^{n+1}\setminus\{0\}\to\bbP^n$,
then $Z:=\pi^{-1}(A)$ is an analytic set of $\bbC^{n+1}\setminus\{0\}$.
we have $\dim Z\geq 1>\dim\{0\}$,
Using Remmart-Stein, $\overline{Z}$ is an analytic set of $\bbC^{n+1}$.
So, for a small disk $\yc$ around $0\in\bbC^{n+1}$,
$$\overline{Z}\cap\yc
=\bigcap_{j=1}^N
\{f_j(z_1,...,z_n)=0\}
$$
where $f_j\in\mcalO(\yc)$.

Let $f_j=\sum\limits_{k=0}^\infty P_{j,k}$ be the Taylor expansion
of $f_j$, where $P_{j,k}$ is a homogenous polynomials of degree $k$.

Claim: $\overline{Z}\cap\yc=
\left(
  \bigcap_{j,k}\{P_{j,k}=0\}
\right)\cap\yc$.
Denote $W:=\bigcap_{j,k}\{P_{j,k}=0\}$,

$W\cap\yc\subseteq\overline{Z}\cap$ is obvious.

By the definition of $\pi$, $Z$ is invariant by homotheties,
so, for any $z\in\overline{Z}\cap\yc$, $|t|\ll 1$,
we have $f_j(t,z)=0$.Write
$$
  f_j(tz)
=
  \sum_{k=0}^{\infty}
    P_{j,k}(z)t^k
=0\quad
\Rightarrow
\quad
P_{j,k}(z)=0
$$
so, $\overline{Z}\cap\yc\subseteq W\cap\yc$.

$\Rightarrow\overline{Z}=W$ by the $\bbC^*$-invariance of $\overline{Z}$ and $W$.
By the noetherian property of $\bbC[z_0,...,z_n]$, $\exists$
finite polynomials $P_j,\quad 1\leq j\leq k$, s.t.
$$
  W=\bigcap_{j=1}^k
  \{P_j=0\}
$$

\begin{cor}
Any analytic subset of an algebraic variety is also algebraic.
\end{cor}

\textbf{Lefschetz's $(1-1)$-theorem}

Exercise: $X$ is a compact complex manifold,
$L,A$ be two holomorphic line bundles over $X$, $A$ is positive($\iff$ ample).
Then for $k\gg 1$, $H^0(X,L\ten A^k)\neq\{0\}$.
(与之前证明几乎完全一样)

Recall:$0\to\mcalO\to\mcalM^*\to Div\to 0$ induces
$$
  Div(X):=H^0(X,Div)\to H^1(X,\mcalO^*)=:Pic(X)
$$

\begin{thm}
If $X$ is an algebraic manifold, then for all $L\in Pic(X)$,
$\exists$ divisor $D$ s.t. $L=\mcalO(D)$.
\end{thm}

\begin{proof}
Take non-zero sections $S\in H^0(X,L\ten A^{k})$, $t\in H^0(X,A^k)$,
then $\frac{s}{t}$ is a meromorphic section of $L$.
Let $D$ be the divisor associated to $\frac{s}{t}$, then
$$L\cong\mcalO(D)$$
\end{proof}

\begin{thm}(Lelong-Poincare equation)

Let $s\in H^0(X,L)\setminus\{0\}$, then
$$
  \frac{\sqrt{-1}}{\pi}
  \p\pbar\log|s|_{h}
=
  [s^{-1}(0)]-\frac{\sqrt{-1}}{2\pi}
  \Theta_{(L,h)}
\eqno{(*)}
$$
where $[s^{-1}(0)]$ is defined as follows:
$$
  \pair{[s^{-1}(0)]}{\psi}
=
  \int_{s^{-1}(0)}
    \psi
$$
where $\psi$ is an $(n-1,n-1)$-form on $X$.
(假设..有度量;在分布意义下求导)
\end{thm}

(Current of integration)

\begin{proof}
  (以后再证)
\end{proof}

$(*)\Rightarrow$
$$
  c_1(L)
=
  \{\frac{\sqrt{-1}}{2\pi}\Theta_{(L,h)}\}
=
  \{[s^{-1}(0)]\}
$$

\begin{rem}
$(*)$ also holds for moromorphic sections.
\end{rem}

\begin{cor}
$X$ be an algebraic manifold, then $\forall \afa\in H^{1,1}(X,\bbQ)$,
we have a divisor $D$ with $\bbQ$-coefficients s.t.
$$
  [\afa]=\{[D]\}
$$
\end{cor}
(Hodge conjecture for $(1,1)$-classes)

\textbf{Fact:}$X$ is a compact complex manifold,
$V\subseteq X$ is an analytic set of pure
$\dim_{\bbC}V=p$. Then the current $[V]$ associated
to $V_{\text{reg}}$:
$$
  \pair{[V]}{\psi}
:=
  \int_{V_{\text{reg}}}
    \psi|_{V_{\text{reg}}}
$$
where $\psi\in C^\infty(X,\wedgeform{p,p})$,
defines a class
$\{[V]\}\in H^{n-p,n-p}(X,\bbZ)$.

\textbf{Hodge conjecture:} $X$ is a complex algebraic manifold,
then for all $\afa\in H^{n-p,n-p}(X,\bbQ)$,
$\exists$ analytic sets $V_k$ of pure dimension $p$ and
rational numbers $r_k$, s.t.
$$
  \afa\in
  \{\sum_{k=1}^Nr_k[V_k]\}
$$
(这个猜想作为练习,说不定就做出来了………………)

Known case: $p=n-1$, it is Lef. $(1,1)$-theorem.

Exercise: also true for $p=1$ (Using Hard Lef)
And, $p=0,p=n$...\vspp

下学期课程:《Analytic methods in algebraic geometry》
敬请期待233333333333

Pluripotential theroy/Positive currents


%%%%%%%%%%%%%%20190530第十四周%%%%%%%%%%%%%%%%%%%%%%

\textbf{Positive currents and Lelong numbers}

Currents on $C^\infty$ manifolds. Let $X$ is a $C^\infty$,
oriented manifold, with $\dim_{\bbR}X=n$.
Denote $C^s(X,\wedgeform{p}T^*X)$ the space
of $s$-differentiable $p$-forms on $X$.

Let $\Omg\subseteq X$ be a coordinate open set,
$u\in C^s(X,\wedgeform{p}T^*X)$,write
$$u=\sum_{|I|=p} u_I\td x_I$$
where $u_I$ are $s$-differentiable functions.

Assume $K\subset\!\subset\Omg$, define
$$
  P^s_K(u)
:=
  \sup_{x\in K}
    \max_{|I|=p\atop|\afa|\leq s}
      |D^\afa u_I(x)|
$$
,where $D^\afa=\frac{\p^\afa}{\p x_1^{\afa_1}\cdots\p x_n^{\afa_n}}$.

$\mcalE^p(X)$- the space $C^\infty(X,\wedgeform{p}T^*X)$
equipped with the topology defined by all the
$P^s_k(\cdot)$, when $s,K\subset\!\subset\Omg$ vary/

$\,^s\mcalE^p(X)$- the space $C^s(X,\wedgeform{p}T^*X)$
equipped with the topology defined by all the
$P^s_K(\cdot)$, when $K\subset\!\subset\Omg$ vary.

Let $K\subseteq X$ be a compact subset,
$\mcalD^p(K)$- the space of $u\in\mcalE^p(X)$with induced topology
and $\supp(u)\subseteq K$.
$$\mcalD^p(X):=\bigcup_{K\text{ is compact subset of $X$}}
\mcalD^p(K)$$
similarly we can define $\,^s\mcalD^p(K)$ and $\,^s\mcalD^p(K)$.

\begin{rem}
(1) $\mcalE^p(X)$(resp. $\,^s\mcalE^p(X)$)
can be defined by countably many $P^s_k$.
In particular, $\mcalE^p$, $\,^s\mcalE^p$ are
Fr\'{e}chet spaces.

(2)$\mcalD^p(X)$ is dense in $\mcalE^p(X)$.
\end{rem}

\begin{definition}(Current,流,[de Rham 1955])

The spaces of currents of $\dim=p$ (or $\deg=n-p$)
denoted by $\mcalD'_p(X):=$the topological dual of $\mcalD^p(X)$.
i.e. $\mcalD'_p(X)$ is the set of linear forms $T$ on $\mcalD^p(X)$
s.t. $T|_{\mcalD^p(K)}$ is continuous.
\end{definition}

\begin{rem}
A current can be considered as a form with
distribution coefficients.
More precisely,
$$
  T=\sum_{|I|=n-p}
    T_I\td x_I
$$
is a current on $\Omg$, if and only if $T_I$
is a distribution on $\Omg$.
\end{rem}

Let $u=f\td x_{I^c}
\in\mcalD^p(\Omg)$, we have
$$
  \pair{T}{f\td x_I}
:=T_I(f)
$$
On the other hand, if $T$ is a current of dimension $p$,
then we define
$$
  T_I(f):=\pair{T}{f\td x_{I^c}}
$$
we can verify $T=\sum\limits_{|I|=p}T_I\td x_I$.

\begin{notation}
$$
  \pair{T}{U}
:=
  \int_XT\wedge U
$$
\end{notation}

\begin{example}(Current of integration)

$Z\subseteq X$ is an oriented closed submanifold with
dimension $p$, with boundary $\p Z$. We can define the current
$[Z]$ as followes:
$$\pair{[Z]}{u}:=\int_Zu$$
for all $u\in\mcalD^p(X)$.
\end{example}

\begin{example}
Let $f$ be a form of degree $n-p$, with
coefficients in $L^1_{loc}(X)$. We define the following current $T_f$:
$$
  \pair{T_f}{u}
:=
  \int_Xf\wedge u
$$
\end{example}

(一般地,current 与current 无法作外积,
好比测度与测度无法相乘)

Exterior derivative on Currents:

\begin{definition}
Let $T$ be a current of dimension $p$(or degree $n-p$),
$u\in\mcalD^{p-1}(X)$. We define $\td T$ is the
current is defined as followes:
$$
  \pair{\td T}{u}
:=
  (-1)^{n-p+1}
  \pair{T}{\td U}
$$
\end{definition}

(why $(-1)^{n-p+1}$?)

\begin{rem}
if $X$ has no boundary,
$T\in T_f$, $f\in\mcalE^{n-p}(X)\cap L^1_{loc}(X)$,then
$$
  0=\int_X\td(f\wedge u)
=\int_X\td f\wedge u+(-1)^{n-p}f\wedge\td u
$$
$$
  \pair{T_{\td f}}{u}
=(-1)^{n-p+1}\pair{T_f}{\td u}
$$
\end{rem}

\begin{example}
$Z\subseteq X$ is a $p$-dimensional submanifold with
boundary $\p Z$, then by the stokes formula:
for $u\in\mcalD^{p-1}(X)$,
$$
  \pair{\td[Z]}{u}
=
  (-1)^{n-p+1}
  \pair{[Z]}{\td u}
=
  (-1)^{n-p+1}\int_Z\td u
= (-1)^{n-p+1}\int_{\p Z}u
= (-1)^{n-p+1}\pair{[\p Z]}{u}
$$
so we have
$$\td[Z]=(-1)^{n-p+1}[\p Z]$$
\end{example}

\begin{definition}
A current $T\in\mcalD'_p(X)$ is called closed
if $\td T=0$.
\end{definition}

Example: if $T=T_f$ for some $f\in\mcalE^p(X)\cap L^1_{loc}(X)$,
then $T_f$ is closed if and only if $f$ is closed(differential form).

Example: $Z\subseteq X$ a submanifold of $X$, then
$\td[Z]=0\iff Z$ has no boundary.

\textbf{Now, Let $X$ be a complex manifold...}
of $\dim_{\bbC}=n$, similar to the smooth case,
we can define:
$$
  \mcalE^{p,q}(X)\quad
  \,^s\mcalE^{p,q}(X)\quad
  \mcalD^{p,q}(X)\quad
  \,^s\mcalD^{p,q}(X)
$$

\begin{definition}
the space of currents of bidimenison $(p,q)$
(or bidegree $(n-p,n-q)$) denoted by
$$
  \mcalD'_{p,q}(X)
$$
is the topological dual of $\mcalD^{p,q}(X)$
\end{definition}

Sinilar to the operator $\td$, we can define
$\td'=\p$ and $\td''=\pbar$ on $\mcalD'_{p,q}(X)$
by duality.

\begin{example}
$V\subseteq X$ complex submanifold of $\dim_{\bbC}=p$,
then the current of integration $[V]$ defined by
$$
  \pair{[V]}{u}
=
  \int_V u|_V
$$
we have $[v]\in\mcalD'_{p,p}(X)$.(Why?!)
\end{example}

\begin{example}(Non-trivial example)

Let $V\subset X$ is an irreducible analytic set of
pure dimension $\dim_\bbC=p$. define the current
$[V]$ as follows:
$$
  \pair{[V]}{u}
:=
  \int_{V_{\reg}}u
$$
(in general, $V_{\reg}$ is an open submanifold of $\dim_{\bbC}=p$.)
\end{example}

FACT: $[V]$ also defines a closed currents.

\begin{example}(trivial example)
$$
  \mcalE^{p,q}(X)\cap L^1_{\loc}(X)
  \subseteq(\mcalD')^{p,q}(X)
$$
\end{example}


\section{微分形式的正性}
\textbf{positivity of forms}

\begin{definition}
A $(p,p)$-form $\fai\in\wedgeform{p,p}(\bbC^n)$
is called strongly positive if $\fai$
is a convex combination as follows:
$$
  \fai
=
  \sum_{s=1}^k
    c_s\sqrt{-1}
    \afa_{s,1}\wedge\overline{\afa_{s,1}}
    \wedge\cdots\wedge
    \sqrt{-1}\afa_{s,p}\wedge\overline{\afa_{s,p}}
$$
where $c_s\geq 0$, $\afa_{s,j}\in\wedgeform{1,0}(\bbC^n)$
\end{definition}
(in particular, $\fai\equiv0$ is strongly positive)

\begin{definition}
A $(p,p)$-form $\psi\in\wedgeform{p,p}(\bbC^n)$ is called positive,
if $\forall\fai\in\wedgeform{n-p,n-p}(\bbC^n)$ strongly positive,
we have $\psi\wedge\fai\geq 0$.
\end{definition}

\begin{rem}(1)
Let $\fai=\sqrt{-1}\sum\limits_{i,j}\fai_{ij}\td z_i\wedge\td\zbar_j$
be a real $(1,1)$-form, then $\fai$ is positive /strongly positive
$\iff[\fai_{ij}]_{i,j}\geq 0$.

(2) denote $\mcalC_{sp}^k:=$ the strongly positive $(k,k)$-forms,
($\mcalC_{sp}^k$ is a convex cone in $\wedgeform{k,k}$)
$\mcalC_p^k:=$ the set of positive $(k,k)$-forms.
(also a convex cone)
\end{rem}

Check: $\mcalC_{sp}^k\subseteq\mcalC_p^k$.

Consider a $(p,p)$-form as a linear function on $\wedgeform{n-p,n-p}$,
then $C_p^k=(C_{sp}^{n-k})^*$

(General setting: $\mcalC\subseteq V$, $V$
is a real vector space of finite dimension,
$\mcalC$ is a convex cone, $V^*$ is the dual of $V$, then we can define
$\mcalC^*:=\Bigset{l\in V^*}{l|_{\mcalC}\geq 0}$)

Take double dual:
$$\overline{\mcalC_{sp}^{n-k}}=(\mcalC_p^k)^*$$
(general fact: $(\mcalC^*)^*=\overline{\mcalC}$)

\begin{cor}
A $(p,p)$-form $\fai$ is strongly positive if and only if
for all $\psi\in\wedgeform{n-p,n-p}$ positive,
$\fai\wedge\psi\geq 0$. In particular,
"strongly positivity" is dual to "positivity".
\end{cor}

%%%%%%%%%%2019.6.4  特别的日子%%%%%%%%%%%%%%%%%%%
%%%%%%%%%%%%%马上期末了!!%%%%%%%%%%%%%%%%%%%%%%%

Recall: positivity of forms.

a $(p,p)$-form $\fai\in\wedgeform{p,p}(\bbC^n)$
is strongly positive if
$$
  \fai=\sum_{c_s\geq 0}
    c_s\ii\afa_{s,1}\wedge\afabar_{s,1}
    \wedge\cdots\wedge
    \ii\afa_{s,p}\wedge\afabar_{s,p}
$$
where $\afa_{s,j}\in\wedgeform{1,0}$

A $(p,p)$-form $\psi\in\wedgeform{p,p}$
is positive if $\psi\wedge\fai\geq 0$ for all $\fai\in\wedgeform{n-p,n-p}$
strongly positive.

Rem: $\{\text{positive $(p,p)$-forms}\}$ and
$\{\text{strongly positive $(n-p,n-p)$-forms}\}$
are dual to each other.

Rem: an $(1,1)$-form $\fai=\ii\sum\limits_{i,j}\fai_{ij}\td z_i\wedge\td\zbar_j$
is strongly positive $\iff$ positive $\iff$
the matrix $[\fai_{ij}]$ is a semi-positive Hermitian matrix.

Rem: for $p=1,n-1,0,n$, strongly positive $\iff$ positive.
In particular, an $(n-1,n-1)$-form
$\psi=c\sum\limits_{ij}\psi_{ij}\widehat{\td z_i\wedge\td\zbar_j}$
(where $\widehat{\td z_i\wedge\td\zbar_j}$ is the $(n-1,n-1)$-form
s.t. $\td z_i\wedge\td \zbar_j\wedge\widehat{\td z_i\wedge\td\zbar_j}
=\td z_1\wedge\td\zbar_1\wedge\cdots\wedge\td z_n\wedge\td\zbar_n$.)
is strongly positive $\iff$ $[\psi_{ij}]\geq 0$.

\begin{prop}
Let $(z_1,z_2,...,z_n)$ be a coordinate system of $\bbC^n$,
then $\wedgeform{p,p}$ admits a basis consisting of strongly
positive form
$$
  \beta_s
=
  \ii\beta_{s,1}\wedge\betabar_{s,1}
  \wedge\cdots\wedge
  \ii\beta_{s,p}\wedge\betabar_{s,p}
$$
with $1\leq s\leq {n\choose p}^2$, every $\beta_{s,j}\in\wedgeform{1,0}$
is of the type $\td z_j\pm\td z_k, \td z_j\pm\ii\td z_k$,
$1\leq j,k\leq n$.
\end{prop}

\begin{proof}
note the identity
$$
  4\td z_j\wedge\td \zbar_k
=
  (\td z_j+\td z_k)\wedge
  \overline{(\td z_j+\td z_k)}
-
  (\td z_j-\td z_k)\wedge
  \overline{(\td z_j-\td z_k)}
+
  \ii(\td z_j+\ii\td z_k)
  \wedge
  \overline{
    \ii(\td z_j+\ii\td z_k)
  }
-
  \ii(\td z_j-\ii\td z_k)
  \wedge
  \overline{
    \ii(\td z_j-\ii\td z_k)
  }
$$
and
$$
  \td z_{i_1}\wedge\cdots\wedge\td z_{i_p}
\wedge
  \td\zbar_{j_1}\wedge\cdots\wedge\td\zbar_{j_p}
=
  \pm\bigwedge_{1\leq s\leq p}
  (\td z_{is}\wedge\td\zbar_{js})
$$

a set of generators contain a basis.
\end{proof}

\begin{cor}
All positive $(p,p)$-forms are real, i.e.
if $u\in\wedgeform{p,p}_{\bbC}$ is positive,and
$$u=(\ii)^{p^2}
\sum_{|I|=|J|=p}
u_{IJ}
\td z_I\wedge\td z_{J}
$$
then $u=\ubar$, i.e. $u_{IJ}=\overline{u_{JI}}$
\end{cor}

\begin{rem}
For all $2\leq p\leq n-2$,
$$\{\text{strongly positive $(p,p)$-forms}\}
\subsetneqq
\{\text{positive $(p,p)$-forms}\}$$

The wedge product of positive forms may NOT be positive
\end{rem}

\begin{rem}Ref:

\verb"Blocki-Pli\'{s}:Square or positive $(p,p)$-forms 2013"

For $p=2$, if $\afa$ is a positive $(2,2)$-form,
then $\afa^2$ is a positive $(4,4)$-form.
\end{rem}
(困难的线性代数题)

Positivity of currents

\begin{definition}
A current $T\in\mcalD'_{p,p}(X)$ is called positive
if $\pair{T}{\fai}\geq 0$ for all $\fai\in\mcalD^{p,p}(X)$
strongly positive.

$T$ is called strongly positive, if
$\pair{T}{\fai}\geq 0$ for all $\fai\in\mcalD^{p,p}(X)$ positive.
\end{definition}

\begin{example}
Let $V\subseteq X$ be a complex submanifold of $\dim=p$,
then the current of integration $[V]$ is a
strongly positive $(p,p)$-current $\in\mcalD'_{p,p}(X)$.
\end{example}

\begin{proof}
Exercise.
\end{proof}

\begin{rem}
In general, $\{\text{strongly positive $(p,p)$-currents}\}
\subseteq\{\text{positive $(p,p)$-current}\}$
with equality if $p=0,1,n-1,n$.
\end{rem}

\begin{rem}
Every positive $(p,p)$-current
$T=_{loc}(-1)^p\sum\limits_{|I|=|J|=p}T_{IJ}\td z_I\wedge\td\zbar_J$
is real ,i.e. $T_{IJ}=\overline{T_{JI}}$ (as distribution) and
$T_{I,I}\geq 0$. and
$$
  \lmd_I\lmd_J|T_{IJ}|\leq 2^p\sum\lmd_M^2T_{MM}
$$
if $I\cap J\subseteq M\subseteq I\cup J$,
where $\lmd_I=\prod_{k\in I}\lmd_K$, $\lmd_k\geq 0$.
\end{rem}

\begin{proof}
Omit.
\end{proof}

\textbf{Weak topology of currents}

\begin{definition}
The weak topology on $\mcalD'_p(X)$ is the
topology defined by the collection of semi-norms
$$
  T\mapsto |\pair{T}{f}|
  \qquad
  f\in\mcalD^p(X)
$$
\end{definition}

\begin{definition}
A set $B\subseteq\mcalD_p'(X)$ is called
weakly bounded if
$|\pair{T}{f}|$ if bounded for ever $T\in B$ and $f\in\mcalD^p(X)$.
\end{definition}

\begin{rem}
By Banach-Alaoglu Theorem(泛函大招),
every weakly bounded closed subset $B\subseteq \mcalD'_p(X)$
is weakly compact.
\end{rem}

\begin{thm}
$(X,\omg)$-Hermitian manifold, then
the set
$$\Bigset{T\in\mcalD'_{p,p}(X)\text{positive}}
         {\sup_{\theta\in C^\infty(X)\text{with compact supp}}
           \pair{T}{\theta\omg^p}\leq 1}
$$
is weakly compact in $\mcalD'_{p,p}$
\end{thm}

\textbf{$\td$-closed positive $(1,1)$-currents and plurisubharmonic (PSH) functions }

$T\in\mcalD^{1,1}(X),T\geq 0,\td T=0$...

\begin{example}
$D\in\Divs(X)$ effective, then $[D]$ is a $\td$-closed
positive $(1,1)$-current.
\end{example}

PSH functions

\begin{definition}
  $\Omg\subseteq X$ open subset, $u:\Omg\to[-\infty,+\infty)$
is called \textbf{PSH}(多重次调和), if

(1)$u$ is upper-semi-continuous (usc)
(rmk: $u$ has a upper bound on any compact subset)

(2) for any complex line $L\subseteq\bbC^n$,
$u|_{\Omg\cap L}$ is subharmonic.
(i.e. $\forall a\in\Omg,\xi\in\bbC^n$ with $|\xi|<d(a,\p\Omg)$,
we have the mean value inequality
$$
  u(a)\leq\frac{1}{2\pi}
  \int_{0}^{2\pi}
    u(a+e^{i\theta}\xi)\td \theta
$$
Denote $PSH(\Omg):=$ the set of PSH functions on $\Omg$
)
\end{definition}

\begin{prop}
(1) for all $u\in\PSH(\Omg)$, $u$ is subharmonic, i.e.
for any $a\in\Omg,r<d(a,\p\Omg)$ we have
$$
  u(a)\leq\frac{1}{|B(a,r)|}
  \int_{B(a,r)}u(z)\td\lmd(z)
$$

(2)Let $\{u_k\}_{k\geq 1}\subseteq\PSH(\Omg)$,
assume $u_k\downarrow u$, then $u\in\PSH(\Omg)$.

(cor: $u\in\PSH(\Omg)$, then either $u\equiv-\infty$ or $u\in L^1_{\loc}(\Omg)$)

(3)Let $u\in\PSH(\Omg)\cap L^1_{\loc}(\Omg)$,
take $(\rho_\veps)_{\veps>0}$ a family of smoothing kernels,
then
$$u*\rho_\veps\in C^{\infty}(\Omg_\veps)\cap\PSH(\Omg_\veps)$$
where $\Omg_\veps:=\Bigset{x\in\Omg}{d(x,\p\Omg)>\omg}$,
and $u*\rho_\veps\downarrow u$ as $\veps\to 0^+$.

(4) Let $u_1,...,u_p\in\PSH(\Omg)$,
$x:\bbR^p\to \bbR$ be a convex function and increasing w.r.t. every variable,
then $x(u_1,...,u_p)\in\PSH(\Omg)$.

In particular, $\max\{u_1,...,u_p\}$,$\log(e^{u_1}+\cdots+e^{u_p})$
are also in $\PSH(\Omg)$.

(5)If $u\in C^2(\Omg)$,then $u\in\PSH(\Omg)\iff\ii\p\pbar u$
is a (strongly) positive $(1,1)$-form, i.e. Hessian
$$\left(\pmfrac{u}{z_i}{\zbar_j}\right)\geq 0$$
\end{prop}

Use the formula
$$
  \frac{1}{2\pi}
  \int_{0}^{2\pi}
    u(a+e^{i\theta}\xi)\td\theta-u(a)
=
  \frac{2}{\pi}
  \int_{0}^{1}\frac{\td t}{t}
  \int_{|\xi|<t}
    Hu(a+\zeta\xi)(\xi)\td\lmd(\zeta)
$$
where $Hu(z)(\xi):=\sum\limits_{1\leq i,j\leq n}
\pmfrac{u}{z_i}{\zbar_j}(z)\xi_i\overline{\xi}_j$.

\begin{proof}
Exercise. Analysis.
\end{proof}

\begin{thm}
(1) If $u\in\PSH(\Omg)\cap L_{\loc}^1(\Omg)$,
then the current $\ii\p\pbar u\geq 0$.

(2) if $f$ is a distribution on $\Omg$ and $\ii\p\pbar f\geq 0$
in the sense of currents, then $\exists !u\in\PSH(\Omg)\cap L^1_{\loc}(\Omg)$
s.t. $f$ is the distribution associative to $u$.
\end{thm}

\begin{proof}
(1) $u\in\PSH(\Omg)\cap L_{\loc}^1(\Omg)
\Rightarrow\ii\p\pbar u=\lim\limits_{\veps\to 0}\ii\p\pbar(u*\rho_\veps)$.

(2)Assume $\ii\p\pbar f\geq 0$ as currents,then
$$
  \ii\p\pbar(f*\rho_\veps)
=
  (\ii\p\pbar f)*\rho_\veps
$$
so $f*\rho_\veps\in C^\infty(\Omg_\veps)\cap\PSH(\Omg_\veps)$.
\end{proof}

define:
$$u:=\lim_{\veps\to 0}
f*\rho_\veps\in\PSH(\Omg)
$$

\begin{cor}
$$\PSH(\Omg)\cap L_{\loc}^1(\Omg)$$
is a closed convex cone in $L^1_{\loc}(\Omg)$.

Every bounded set (in $L^1_{\loc}(\Omg)$) in $\PSH(\Omg)$
is (relatively) compact.
\end{cor}

%%%%%%%%%%%2019.6.6倒数第二节课%%%%%%%%%%%%%%%%%%%
%%%%%%%%%%%%%%%%%%第15周 周四%%%%%%%%%%%%%%%%%%%%%

Recall: $\Omg\in\bbC^n$ open set,
$u\in\PSH(\Omg)$ if $u$ is u.s.c. and $u|_L$
is subharmonic $\iff\ii\td'\td'' u$ is a positive $(1,1)$-current on $\Omg$.

Remark(check!): any $\td$-closed $(1,1)$ current $T$
on $\Omg$ can be written as $T=\ii\td'\td''\fai$ locally.
$\Rightarrow$ if $T$ is a $\td$-closed positive $(1,1)$-current on $\Omg$,
then locally $T=\ii\td'\td''u$ for some PSH-function $u$.
This $u$ is called \textbf{local potential} of $T$.

Now, $X$ is a complex manifold,
observation: $F:\Omg_1\to\Omg_2$ holomorphic map,
then $F^*(\ii\td'\td''\fai)=\ii\td'\td''(F^*\fai)$.

\begin{definition}
$u\in\PSH(X)$, if:

(1) $u$ is u.s.c.

(2) $u|_\Omg$ is PSH on every coordinate open set.
\end{definition}

\begin{prop}
$f:X\to Y$ is holomorphic map between two complex manifolds,
$u\in\PSH(Y)$, then $f^*u\in\PSH(X)$.
\end{prop}

\begin{proof}
Easy.
\end{proof}

\begin{example}
(1)Let $f\in H^0(X,\mcalO_X)$, then
$\log |f|\in\PSH(X)$.
\end{example}

\begin{proof}
consider $f:X\to \bbC$, and
$\log|f|=f^*(\log|z|)$, note that
$\log|z|\in\PSH(\bbC)$
\end{proof}
note that $\frac{\ii}{2\pi}\td'\td''\log|z|=\delta_0$ is Dirac distribution.

\begin{example}For any $f_1,f_2,...,f_k\in H^0(X,\mcalO_X)$, then
$$\log(|f_1|^{\afa_1}+\cdots+|f_k|^{\afa_k})\in\PSH(X)$$
for any $\afa_j\geq 0,1\leq j\leq k$.
\end{example}

\textbf{Skoda-El Mir extension theorem}

Let $X$ be a complex manifold

\begin{definition}
$E\subseteq X$ is called pluripolar(多重极点) if
$X$ has an open covering $(\Omg)$, we have
$u\in\PSH(\Omg)\cap L^1_{\loc}(\Omg)$ s.t.
$$E\cap\Omg=\Bigset{z\in\Omg}{u(z)=-\infty}$$
\end{definition}

if "$=$" holds, then $E$ is called \textbf{complete} pluripolar set.

\begin{example}
If $A$ is an analytic set, then $A$ is complete pluripolar set.
\end{example}

Locally ,$A\cap\Omg=\{f_1=\cdots f_k=0\}$ where $f_j\in\mcalO(\Omg)$,
it also $= U^{-1}(-\infty)$, where
$$u=\log(|f_1|^2+\cdots+|f_k|^2)$$

\begin{thm}(Skoda 1982; El mir 1984; Sibory 1985)

$X$ be a complex manifold, $E\subseteq X$ be (closed) pluripolar set in $X$,
$T$ is a closed positive $(p,p)$-current on $X\setminus E$,
s.t. $T$ has locally finite mass(局部有限质量) near $E$.
(i.e. $\forall$ compact neighborhood $\Omg$ we have
$$\int_{\Omg\setminus E}T\wedge\omg^{n-p}<+\infty$$
)

Then, $1_{X\setminus E}T$ is also a closed positive $(p,p)$-current on $X$.
\end{thm}
(this current is called:current obtained by extending $T$ by $0$ on $E$.)

\begin{proof}
(1)Notice that : $\td$-closeness is a local property(check!)

(2)we need to verify $1_{\Omg\setminus E}T$ is $\td-$closed on
every open set $\Omg$ s.t. $\Omg\cap 1=
\Bigset{z\in\Omg}{v(z)=-\infty}$ where $v\in\PSH(\Omg)\cap L^1_{\loc}(\Omg)$,

(3)We can assume $v\leq 0$ on $\Omg$ after shrinking $\Omg$.

Let $g:\bbR\to\bbR$ smooth, convex, increasing. $g(t)\equiv 0$ if $t\leq 1$,
and $g(0)=1$. %笔记上有示意图%

define $v_k=g(k^{-1}v*\rho_{\veps_k})$ where $\veps_k\to 0$ fast.

(After shrinking $\Omg$ a little bit, $v_k\in\PSH(\Omg)\cap C^\infty(\Omg)$)

Then, $0\leq v_k\leq 1$, $v_k=0$ in a neighborhood of $E\cap\Omg$.
And $\lim\limits_{k\to\infty}v_k(x)=1$ for any $x\in\Omg\setminus E$.

Let $\theta\in C^\infty[0,1]$ s.t.
$\theta =0$ on $[0,\frac{1}{3}]$, $\theta=1$ on $[\frac{1}{3},1]$,
 and $\theta$ is smooth, increasing.

So, $\theta\circ v_k= 0$ near $E\cap \Omg $, and $\to 1$ on $\Omg\setminus E$.
Moreover, $\theta\circ v_k\to 1_{\Omg\setminus E}$.

In particular, $\widetilde{T}=1_{\Omg\setminus E}T
=\lim\limits(k\to\infty)(\theta\circ v_k)T$
in the sense of currents w.r.t. weak topology.
Using the continuity of $\td$ on currents,
$$
  \td\widetilde{T}
=\lim_{k\to\infty}\td(\theta\circ v_k)\wedge T
$$
Need to verify $\lim\limits_{k\to\infty}\td(\theta\circ v_k)\wedge T=0
\iff\lim\limits_{k\to\infty}T\wedge\td'(\theta\circ v_k)=0$.

First assume $T$ is a current of bi-degree $(n-1,n-1)$:
need to verity $\forall \afa\in\mcalD^{1,0}(\Omg)$ we have
$$
  \pair{T\wedge\td'(\theta\circ v_k)}{\afabar}
=
  \pair{T}{\theta'(v_k)\td'v_k\wedge\afabar}\to 0
$$

Prop: (Linear algebra) For $\gma\in\mcalD^{1,0}(\Omg\setminus E)$,
$$\gma\mapsto\pair{T}{\ii\gma\wedge\gmabar}$$
is a non-negative Hermitian form, so we have Cauchy-Schwarz inequality:
$$
  |\pair{T}{\ii\beta\wedge\gmabar}|^2
\leq
  \pair{T}{\ii\beta\wedge\betabar}
  +\pair{T}{\ii\gma\wedge\gmabar}
$$
so we have
$$
  |\pair{T}{\theta'(v_k)\td'v_k\wedge\afabar}|^2
\leq
  \underbrace{
  \pair{T}{\theta'(v_k)^2\afa\wedge\afabar}
  }_{(1)}
  \underbrace{
  \pair{T}{\psi\td'v_k\wedge\td''v_k}
  }_{(2)}
$$
where $0\leq\psi\leq 1$ s.t. $\supp\psi\supseteq\supp\afa$, $\psi\equiv 1$
near $\supp\afa$.

$(1):T$ has locally finite mass near $E$, $\Rightarrow$
$$
  \int_{\Omg\setminus E}T\wedge\ii\afa\wedge\afabar
<+\infty
$$
note that $\theta'(v_k)\to 0$ on $\Omg$,
use Lebesgue dominated convergent theorem, $(1)\to 0$ as $k\to\infty$.

(2): use the formula:
$$
  \ii\td'\td'' v_k^2
=
  2v_k\ii\td'\td'' v_k+2\ii\td' v_k\wedge\td'' v_k
\geq
  2\ii\td'v_k\wedge\td'' v_k
$$
so,
\begin{eqnarray*}
  (2)\leq \pair{T}{\psi\ii\td'\td'' v_k^2}
&=&
  \pair{T}{v_k^2\ii\td'\td''\psi}
=
  \int_{\Omg\setminus E}
    v_k^2T\wedge\ii\td'\td''\psi\\
&\leq&
  \int_{\Omg\setminus E}
    T\wedge\ii\td'\td''\psi
\leq
  C(\psi)\int_{\Omg\setminus E}T\wedge\omg
<+\infty
\end{eqnarray*}
Conclusion:$\td\Ttil=0$.

General bidegree $(p,p)$: Take $\Gma=\ii\gma_1
\wedge\gmabar_1\wedge\cdots\wedge\ii\gma_{n-p-1}\wedge\gmabar_{n-p-1}$
where $\gma_j\in\mcalD^{1,0}(\Omg)$ with constant coefficients,
then $T\wedge T$ is a current of bidegree $(n-1,n-1)$
with locally finite mass near $E$.
$$0=
  \td(1_{\Omg\setminus E}(T\wedge\Gma))
= \td(\Ttil\wedge\Gma)=\td\Ttil\wedge\Gma
$$
by first case, $\Rightarrow$
$\td\Ttil\wedge\Gma=0$ for any such $\Gma$.

$\Rightarrow\forall \Gma\in\mcalD^{n-p-1,n-p-1}(\Omg)$,
$\td\Ttil\wedge\Gma=0$, so $\td\Ttil=0$.
\end{proof}

\begin{cor}
$T$ is a $\td$-closed positive current on $X$,
$E\subseteq X$ (closed) complete pluripolar set, Then
$1_{X\setminus E}T$ is a $\td$-closed positive current.
\end{cor}
(Since $1_{X\setminus E}$ is the trivial extension of $T|_{X\setminus E}$)
($1_{E}T$ is a $\td$-closed positive current)
($1_ET=T-1_{X\setminus E}T$)

\begin{cor}(Exercise)

$A\subseteq X$ an analytic subset of $X$
$A$ is of pure dimension $\dim_{\bbC}=p$, then the current
$[A]$ is a \textbf{$\td$-closed} positive current of bidimension $(p,p)$
\end{cor}

\begin{proof}
(1)$[A_{\reg}]$ is a closed positive current on $X\setminus A_{\sing}$.

(2)$[A_{\reg}]$ has locally finite mass near $A_{\sing}$.
(要用解析集局部参数化定理,可怕)
(Using the local parameterization theorem of analytic set)

(3)$[A]$ is the trivial extension of $[A_{\reg}]$,
i.e. $[A]=1_{X\setminus A_{\sing}}[A_{\reg}]$
\end{proof}

Cor: if $X$ is compact (over $\bbC$), then (cohomology of current)
$$\{[A]\}\in H^{p,p}(X,\bbR)$$

%%%%%%%%2019-6-11%%%%%%%%%%%%%%%%%%%%
%%%%%%%%%%%最后一课%%%%%%%%%%%%%%%%%%%%%%%%

介绍一些结果,不怎么证明。

\textbf{Support theorems for normal currents}

$X$- $C^\infty$-mfd, $\mcalD^p(X)\subseteq\,^s\mcalD^p(X)$,dual,
$\mcalD'^p(X)\supseteq\,^s\mcalD^p(X)$.

\begin{definition}
$T\in\mcalD'^p(X)$ is called a normal current if
$T$ and $\td T$ are currents of order $0$.
\end{definition}

\begin{example}
$X$ be complex manifold, then any positive closed current is normal.
\end{example}

\begin{proof}
  Omit.
\end{proof}

\begin{definition}
$T\in\mcalD'(X)$, support of $T$, $\supp T$ is defined by the
smallest closed set $A$ s.t. $T|_{\mcalD^p(X/\setminus A)}$ is zero.
\end{definition}

\begin{thm}$X$ be a complex manifold,
let $T\in\mcalD_{p,p}'(X)$ be a normal current,
if $\supp T\subseteq A$, where $A$ is an analytic set with
$\dim_{\bbC}A<p$, then $T\equiv 0$.
\end{thm}

\begin{proof}
  Omit.
\end{proof}

\begin{thm}
Assume $A$ is an analytic set of $\dim_{\bbC}=p$
with global irreducible components $A_j$ of pure dimension $p$.
$T$ is a $\td$-closed current of order zero,
with $\supp T\subseteq A$. Then
$$
  T=\sum_{\lmd_j\in\bbC}
  \lmd_j[A_j]
$$
moreover, if $T$ is positive, then all $\lmd_j\geq 0$.
\end{thm}

\begin{proof}
  Omit.
\end{proof}

\textbf{Application}: Lelong-{\phjwll} equation.

Baby case: consider $f(z)=z$ on $\bbC$, then
$$\frac{\ii}{\pi}\td'\td''\log|f|=\delta_0$$
i.e. $\fai\in C^\infty(\bbC)$ with compact supp, then
$$\pair{\frac{\ii}{\pi}\td'\td''\log|f|}{\fai}=\fai(0)$$

In other words, $\frac{\ii}{\pi}\td'\td''\log|z|=[f^{-1}(0)]$.

\begin{thm}$X$ is a complex manifold, $f\in H^0(X,\mcalO_X)\setminus\{0\}$.
Let $Z_f=\sum_{m_j\in\bbZ_{>0}} m_jz_j$ be the zero-divisor of $f$,
then
$$
  \frac{\ii}{\pi}\td'\td''\log|f|
=
  [Z_f]
=
  \sum_{m_j\in\bbZ_{>0}}m_j[z_j]
$$
\end{thm}

\begin{proof}
Observation: if $f(x)\neq 0$, then $\td'\td''\log|f|(x)=0$.
The support of $\frac{\ii}{\pi}\td'\td''\log|f|$
is contained in $Z=\bigcup\limits_j Z_j$.
\end{proof}

\begin{proof}

Take a point $a\in Z_j\cap Z_{\reg}$
(note that $Z_{\sing}=\bigcup\limits_{j} Z_{j,\sing}
\cup\bigcup\limits_{j\neq k}(Z_j\cap Z_k)$),
$\dim Z_{\sing}\leq n-2$.

Take a coordinate $(z_1,...,z_n)$ around $a$ s.t.
$$Z_j=\{z_1=0\}$$
near $a$. then $f(z)=u(z)z_1^{m_j}$ near $a$,
where $u(a)\neq 0$ is an unit.So,
$$
  \frac{\ii}{\pi}\td'\td''\log|f|
=\frac{\ii}{\pi}\td'\td''\log|u(z)|+ m_j
\frac{\ii}{\pi}\td'\td''\log|z_1|
$$
Claim: $\frac{\ii}{\pi}\td'\td''\log|z_1|=[z_1=0]^{m_j}$.
Consider the projection $\pi:\bbC^n\to\bbC$,
$(z_1,...,z_n)\to z_1$, then
$\frac{\ii}{\pi}\td'\td''\log|z_1|=\pi^*\delta_0=[Z_1=0]$,
$\Rightarrow$
$$
  \frac{\ii}{\pi}\td'\td''\log|f|
=[Z_f]
$$
holes on $Z_{\reg}$.so, $\frac{\ii}{\pi}\td'\td''\log|f|-[Z_f]$
is a normal current (bidim= $(n-1,n-1)$)
with support $\subseteq Z_{\sing}$ and
$\dim Z_{\sing}<n-1$,so then current
$$\frac{\ii}{\pi}\td'\td''=[Z_k]$$

\end{proof}

\begin{cor}Let $L$ be a holomorphic line bundle on $X$,and
$s\in H^{0,0}(X,L)\setminus\{0\}$.(with a metric $h$).
Then
$$
  \frac{\-i}{\pi}\td'\td''\log|h|
=[Z_s]
$$
\end{cor}

Recall, Line bundle $h$, chern curvature
$$\Theta=1\p\pbar\log|s|=[Z_s]-
\frac{\ii}{2\pi}\td'\td''\log h$$
$$[Z_s]-\frac{\ii}{2\pi}\Theta_{...}$$


\begin{cor}
$(c_1(l))=\{\frac{\ii}{2\pi}\Theta_{L,H}\}=
\{Z_LS\}$where $s\in H^{0,0}H^0(X,L)\setminus{o}$

Morgeneally, if $s$ is nontri=vial section of $L$,yhen
$$
  \frac{\ii}{\pi}\log|s|_{\bbR}=[\mcalD]-\frac{\ii}{2\pi}\Theta_{(L,H)}.
 $$ where $\mcalD'_X$ is the dividor associated of %$\pi,\pi$

So if $.$ is algebraic, then for any Holomorphic line bundle $L$,
$c_{i,i}*L=\{[D]\}.$
\end{cor}

%再来介绍一个,可视为代数几何中的超越版本

\textbf{Lolong number of a (positive) current(1957)}

For $\Omg\subseteq\bbC^n$ open set, $T\in\mcalD'_{p,p}(\Omg)$,
is closed positive.define
$$\nu(T,x,r):=
\frac{1}{r^{2p}}
\int_{B(x,r)}T\wedge
\left(\frac{\ii}{\pi}\td'td''|z|^2\right)^p
=
  \frac{\sgm_T(B(X,B))\updot}
       {\pi^pr^{2p}/p!}
$$
where $|z|^2=\sum_{k=1}^|z_k|^2$,
$$\bullet:=
\int_{B(x,r)}
\sgm_T
$$
where $\sgm_T=T\wedge\frac{1}{p!}\left(
\ii\td'\td''|z|^2\right)^p$
is the \textbf{trace measure} in $\bbC^p$ with residues $r$.

\begin{thm}
$$\nu(T,x)=\lim_{r\to 0}\downarrow \nu(T,x,r)$$
is called the Lelong number of $T$ at $x$.
\end{thm}

\begin{thm}
if $u\in\PSH(\Omg)$, then
$$
  \nu(\frac{\ii}{\pi}\td'\td'' u,x)
=
  \sup\Bigset{t\geq 0}
             {u(z)\leq t\log|z-x|+O(1)\text{near $x$}}
=
  \liminf_{z\to x}\frac{u(z)}{\log|z-x|}
$$
in particular, if $u=\log |f|$, $f\in\mcalO(\Omg)$, then
$$
  \nu(\frac{\ii}{\pi}\td'\td''\log|f|,x)
=
  \Ord_x(f)
=
  \max\Bigset{k}{\tD^\afa f(x)=0,\forall |\afa|<k}
$$
\end{thm}

\begin{proof}
  Omit.
\end{proof}

\begin{thm}(Thie 1967)

If $T=[A]$ is the current of integration given by an
analytic set; then for all $x\in A$, we have
$$\nu(T,x)=\text{mult}_xA$$
the multiplicity of $A$ at $x$.
\end{thm}

\begin{proof}
  Omit.
\end{proof}

\begin{thm}(Siu, 1974)

$X$ -complex manifold, $T\in\mcalD'_{p,p}(X)$ positive closed,
then the set $E_c(T):=\Bigset{x\in X}{\nu(T,x)\geq c}$
(where $c>0$ is a positive constant)
is an analytic set of $\dim_{\bbC}\leq p$.
\end{thm}

\begin{proof}
  Omit.
\end{proof}

Rem: special case $p=1$, Siu's thm...(when $T\in\mcalD'_{1,1}(X)$)

\begin{thm}(Demailly 1992)
$\Omg\subseteq\bbC^n$ bounded psudoconvex
(i.e. a set with PSH exhaustion function) open set,
$\fai\in\PSH(\Omg)\cap L^1_{\loc}(\Omg)$, For $m\in\bbN$,
denote
$$
  \mcalH_{\Omg}(m,\fai)=
  \Bigset{f\in\mcalO(\Omg)}
  {\int_\Omg|f|^2e^{-2m\fai}<+\infty}
$$
(FACT: it is a Hilbert space)
Take $(\sgm_l)_l$ an orthonormal basis of $\mcalH_\Omg(m\fai)$,
define $\fai_m=\frac{1}{2m}\log\sum\limits_l|\sgm_l|^2\in\PSH(\Omg)$,
then $\exists c_1,c_2>0$(independent of $m$) s.t.

(1)$\fai(z)-\frac{c_1}{m}\leq\fai_m(z)
  \leq\sup_{|\xi-z|<r}\fai(\xi)+\frac{1}{m}\log\frac{c_2}{r^n}$

(2)$\nu(\fai,z)-\frac{n}{m}\leq\nu(\fai_m,z)\leq\nu(\fai,z)$
\end{thm}

\begin{proof}
  Omit.
\end{proof}

\begin{cor}
Consider $E_c(\fai):=\Bigset{z\in\Omg}{\nu(\fai,z)\geq c}$,then
$E_c(\fai)=\bigcap\limits_{m\geq m_0\gg 1}E_{c-\frac{n}{m}}(\fai_m)$.
\end{cor}

\begin{proof}
  Omit.
\end{proof}

%课程结束,后天(本周四考试)
%6+2题(两道加分题,6道简单题?2道40分送分题?),90分钟
%有一道题,今天讲课说口头说了答案?。。。

%关键词: Hodge theory, HL, LD, HRR.,
% Kahler, Hodge decomposition , Hodge isomorphism... 还要会用
%Vanishing theorem (要知道结果)
%其它比较基础。。。 Kahler manifold 的定义?
%有两个Exercise?
%所以已经送了80分了?

