\chapter{正性与消灭定理}
positivity and vanishing theorem

$X$-Kahler manifold, i.e. $\exists$ Hermitian metric $\omg$ s.t. $\td\omg=0$,
$\td=\td'+\td''$, $\td'=\p,\td''=\pbar$.
$$\yc_\td=[\td,\td^*]=\td\td^*+\td^*\td$$
$$\yc_{\td'}=[\td',\td'^*]$$
$$\yc_{\td''}=[\td'',\td''^*]$$
$\td\curvearrowright C^{\infty}(X,\wedgeform{p,q})$.

Fact: $\omg$ is Kahler $\iff\yc_{\td'}=\yc_{\td''}=\frac{1}{2}\yc_\td$.

Let $\underline{\bbC}:=X\times\bbC$ be the trivial line bundle,
$\td$ can be regraded as the Chern connection on $\underline{\bbC}$.

$(E,h)$-Hermitian holomorphic vector bundle over $(X,\omg)$,
with Chern connection $D_E=D_E'+D_E''$. $(D_E''=\pbar)$.

$$C^\infty(X,\wedgeform{p,q}\ten E)$$
has an inner product induced by $\omg,h$.
$\rightsquigarrow$ adjoint operators $D_E^*=D_E'^*+D_E''^*$.

$\rightsquigarrow\yc_E=[D_E,D_E^*]=D_ED_E^*+D_E^*D_E$, and
$\yc_E'\,,\,\yc_E''$. (self adjoint, elliptic operators)

Question: relation between $\yc_E'$ and $\yc_E''$?

\begin{thm}(Bochner-Kodaira-Nakaino identity)
$$
  \yc_E''-\yc_E'
=
  \left[
    \sqrt{-1}
    \Theta_E
  ,
    \Lmd
  \right]
$$
where $\Theta_E$ is the Chern curvature of $D_E$.
\end{thm}

Recall: $\Theta_E=D_E^2$, when $D_E$ is Chern connectoin, we have
$$D_E'^2=0\qquad D_E''^2=0$$
i.e. $\Theta_E=[D_E',D_E'']$.

Remark: $E$ is flat(i.e. $D_E^2=0$)$\iff \yc_E'=\yc_E''$.

\begin{proof}
based on following identities:
$$[D_E''^*,L]=\sqrt{-1}D_E'$$
$$[D_E'^*,L]=-\sqrt{-1}D_E''$$
$$[\Lmd,D_E']=-\sqrt{-1}D_E'^*$$
$$[\Lmd,D_E'']=\sqrt{-1}D_E''^*$$

then (by super Jacobi identity):
\begin{eqnarray*}
  \yc_E''=[D_E'',D_E''^*]
&=&
  -\sqrt{-1}
  \left[
    D_E''
  ,
    [\Lmd,D_E']
  \right]
=
  -\sqrt{-1}
  \left(
    [\Lmd,[D_E',D_E'']]
   +[D_E',[D_E'',\Lmd]]
  \right)\\
&=&
  -\sqrt{-1}
  \left(
    [\Lmd,\Theta_E]
   +[D_E',\sqrt{-1}D_E'^*]
  \right)
\end{eqnarray*}
so,
$$\yc_E''-\yc_E'=[\sqrt{-1}\Theta_E,\Lmd]$$
\end{proof}

\begin{lemma}(normal frame)

Let $X$ be a complex manifold, then for any $x_0\in X$,
and any holomorphic chart $(z_1,...,z_n)$ centered at $x_0$,
there exists a holomorphic frame $\{e_\lmd\}_{\lmd=1}^{r:=rank E}$
of $E$ near $x_0$ such that
$$
  \left\langle
    e_{\lmd}(z),e_{\mu}(z)
  \right\rangle
=
  \delta_{\lmd,\mu}-
  \sum_{1\leq j,k \leq n}
  C_{jk\lmd\mu}
  z_j\zbar_k
+
 O(|z|^3)
$$
where $(C_{jk\lmd\mu})$ are the coefficients of the Chern curvature
$$\Theta_E(x_0)=
  \sum_{1\leq j,k\leq n\atop 1\leq\lmd,\mu\leq r}
    C_{jk\lmd\mu}
    \td z_j\wedge\td \zbar_{k}\ten e_\lmd^*\ten e_\mu
$$
\end{lemma}


need to verify: $\forall s\in C^\infty(X,\wedgeform{p,q}\ten E),x_0\in X$,
$$[D_E''^*,L]s(x_0)=\sqrt{-1}D_E's(x_0)$$
w.r.t the normal frame $(e_\lmd)_{\lmd=1}^r$ near $x_0$, assume
$$s=\sum_{\lmd=1}^{n}\sgm_\lmd\ten e_\lmd$$
then
$$
  D_Es(z)=\sum_{\lmd=1}^{n}
    \td\sgm_\lmd\ten e_\lmd +O(|z|)
$$
$$
  D_E^*s(z)=\sum_{\lmd=1}^{n}
    \td^*\sgm_\lmd\ten e_\lmd +O(|z|)
$$

$$D_E''^*=\sum_{\lmd=1}^{r} \td''^*\sgm_\lmd\ten e_\lmd+O(|z|)$$
$$
  \Rightarrow
  [D_E''^*,L]s=
  D_E''^*(\sum\omg\wedge\sgm_\lmd\ten e_\lmd)
 -\omg\wedge
 \left(
   \sum_{\lmd=1}^{r}
   \td''^*\sgm_\lmd\ten e_\lmd+O(|z|)
 \right)
=
  \sum_{\lmd=1}^{r}
    [\td''^*,L]\sgm_\lmd\ten e_\lmd+O(|z|)
$$
Similarly,
$$
  D_E's
=
  \sum_{\lmd=1}^{r}
    \td'\sgm_\lmd\ten e_\lmd+O(|z|)
$$
we have:
$$
  [d''^*,L]=\sqrt{-1}\td'
$$
(because $\omg$ is Kahler)

...

$(E,h)$ hermitian holomorphic vector bundle over Kahler manifold $(X,\omg)$.
we have BKN identity
$$
  \yc_E''-\yc_E'=[\sqrt{-1}\Theta_E,\Lmd]
$$

Recall: $L^2$-Hodge theory. $X$ compact manifold,then
$$
  H^{p,q}(X,E):=
  \frac{\ker D_E''}{\im D_E''}
\cong
  \ker\yc_E''
$$
(harmonic form)

Take $u\in C^\infty(X,\wedgeform(p,q)\ten E)$,applying BKN identity to $u$,
$$\yc_E''u-\yc_E'u=[\sqrt{-1}\Theta_E,\Lmd]u$$

note that
$$\ppair{\yc_E'u}{u}=\norm{D_E'u}^2+\norm{D_E'^*u}^2\geq 0$$
$$
  \Rightarrow
  \norm{D_E''u}^2+\norm{D_E''^*u}^2
\geq
  \ppair{[\sqrt{-1}\Theta_E,\Lmd]}{u}
$$
i.e.
$$
  \norm{D_E''u}^2+\norm{D_E''^*u}^2
\geq
  \int_X\pair{[\sqrt{-1}\Theta_E,\Lmd]}{u}
  \td Vol
$$

Observation: if $u\in\ker\yc_E''$, and $[\sqrt{-1}\Theta_E,\Lmd]$
has "positivity",
then $LHS=0$. So, $H^{p,q}(X,E)=0$.

\begin{definition}(Positivity)

We call $[\sqrt{-1}\Theta_E,\Lmd]$ is positive at $x_0\in X$, if
for any $0\neq v\in\left(\wedgeform{p,q}\ten E\right)_{x_0}$, we have
$$\pair{[\sqrt{-1}\Theta_E,\Lmd]v}{v}>0$$

....positive on $X$, if ... at each point
\end{definition}

\begin{thm}
If $[\sqrt{-1}\Theta_E,\Lmd]$ is positive on $X$, then
$$H^{p,q}(X,E)=0$$
\end{thm}

Special case: $E$ is a holomorphic line bundle, with Hermitian metric $h$,
$$\Theta_E=-\td'\td''\log h$$
$\Rightarrow\sqrt{-1}\Theta_E$ is a real $\td$-closed $(1,1)$-form on $X$.

locally,
$$
     \afa=
\sqrt{-1}\sum_{1\leq i,j\leq n}
  a_{ij}\td z_i\wedge\td\zbar_j
$$

$\afa$ is real $\iff$ $\afa=\overline{\afa}$,
(i.e. locally $(a_{ij})$ is an hermitian matrix)

\begin{definition}
a real $(1,1)$-form $\afa$ is called positive, if
$(a_{ij})_{ij}$ is positive definite.
\end{definition}

\begin{lemma}
If $\sqrt{-1}\Theta_E$ is positive, then $\omg:=\sqrt{-1}\Theta_E$ gives a
Kahler metric on $X$.
\end{lemma}

\begin{lemma}
If $\omg=\sqrt{-1}\Theta_E>0$, and $\Lmd$ is the adjoint of $L=\omg\wedge$,
then
$$[\sqrt{-1}\Theta_E,\Lmd]$$
is positive on $\wedgeform{p,q}\ten E$
whenever $p+q\geq n+1$.
\end{lemma}

\begin{lemma}
Let $\afa$ be a real $(1,1)$-form, $\omg$ a Kahler metric, assume
the eigenvalue of $\afa$ at $x_0$ is $\afa_1\leq\afa_2\leq\cdots\leq\afa_n$,
then (in the coordinate chart $(z_1,z_2,...,z_n)$, and
$u=\sum\limits_{|I|=p\atop|J|=q}u_{IJ}\td z_I\wedge\td\zbar_J$)
$$
  [\afa,L]
=
  \sum_{I,J}
    \left(
      \sum_{i\in I}\afa_i
    + \sum_{j\in J}\afa_j
    - \sum_{k=1}^n\afa_k
    \right)
    u_{IJ}\td z_I\wedge\td\zbar_J
$$
\end{lemma}

\begin{cor}
$
  \afa=\omg
$, then
$$[\omg,\Lmd]u=(p+q-n)u$$
\end{cor}

\begin{cor}
Take an orthonormal frame $e$ of $E$, then
for any $u=\sum \limits_{|I|=p\atop|J|=q}u_{IJ}\td z_I\wedge\td\zbar_J\ten e$, we have
$$
  \pair{
  [\sqrt{-1}\Theta_E,\Lmd
  ]u
  }{u}
=(p+q-n)|u|^2
$$
\end{cor}

\begin{thm}
If $[\sqrt{-1}\Theta_E,\Lmd]$ is positive on $X$,
then
$$H^{p,q}(X,E)=0$$
\end{thm}

\begin{thm}If $E$ is a holomorphic line bundle
with a smooth hermitian metric $h$ s.t. $\sqrt{-1}\Theta_{(E,h)}\geq 0$,
then$H^{p,q}(X,E)=0$ whenever $p+q\geq n+1$.
\end{thm}
de Rham-Weil...$\cong H^q(X,\Omg^p\ten E)$.

\begin{definition}(canonical bundle)
$$K_X=\det T^*X$$
determinate bundle of cotangent bundle,
is called canonical bundle.
($\mcalO(K_X)=\Omg_X^n$)
\end{definition}


\begin{definition}
$X$ is called Fano, if $K_X^*=\det(TX)$ has a matric with positive curvature.

$X$ is called Calabi-Yau,if $K_X$ has a metric with vanishing curvature.

$X$ is of general type, if $K_X$ has a metric with positive curvature.
\end{definition}


\begin{cor}(Kodaira vanishing theorem)
$E$ is a positive line bundle, then
$$H^q(X,K_X\ten E)=0$$
for any $q\geq 1$.
\end{cor}

So, if $X$ is Fano, $(\iff K_X^*)$ positive,
$K_X\ten K_X^*=\underline{\bbC}$,
$\Rightarrow H^1(X,\mcalO)=0,\Rightarrow H^1(X,\bbR)=0$,

%%%%%%%%%%%%%%%%2019.5.14%%%%%%%%%%%%%%%%%%%%%%%%%5

Recall: BKN-inequality.

holomorphic Hermitian vector bundle $(E,h)\to (X,\omg)$, $\omg$ is Kahler.
For any $u\in C^\infty(X,\wedgeform{p,q}\ten E)$, we have
$$
  \norm{D''u}^2+\norm{D''^*u}^2
\geq
  \int_X
    \pair{[\sqrt{-1}\Theta_E,\Lmd_\omg]u}{u}
    \td Vol
$$

Recall: If $[\sqrt{-1}\Theta_E,\Lmd_\omg]$ is positive on
$C^\infty(X,\wedgeform{p,q}\ten E)$, then
$H^{p,q}(X,E)=0$.

\begin{thm}(Kodaira-Nakano vanishing theorem)

If $E$ is a holomorphic line bundle with a smooth metric $h$ s.t.
$\sqrt{-1}\Theta_{(E,h)}>0$,then
$[\sqrt{-1}\Theta_E,\Lmd_\omg]$ is positive on
$C^\infty(X,\wedgeform{p,q}\ten E)$ whenever $p+q\geq n+1$.

$\Rightarrow H^{p,q}(X,E)=0$ when $p+q\geq n+1$.
\end{thm}
(Last time)

Today:
\begin{thm}(Girbau vanishing theorem, 1976)

$E$ is a holomorphic line bundle over compact Kahler manifold,
with smooth metric $h$ s.t. $\sqrt{-1}\Theta_{(E,h)}\geq 0$,
and has at least $n-s+1$ positive eigenvalues at every points of $X$, then
$$H^{p,q}(X,E)=0$$
if $p+q\geq n+s$.
\end{thm}

$\afa$: a \textbf{real} $(1,1)$-form on $X$,
locally $\afa=\sqrt{-1}\sum\afa_{ij}\td z_i\wedge\td\zbar_j$.
then we have a matrix $M(\afa)=(\afa_{ij})_{n\times n}$,
($\afa$ is real $\Rightarrow$)a hermite matrix.

we call $\afa$ has at least $k$ positive eigenvalues at $x$,
if $M(\afa)(x)$ has $k$ positive eigenvalues.
(Remark: It is well defined)

\begin{proof}
Claim: there exists some Kahler metric $\omg$
s.t. $[\sqrt{-1}\Theta,\Lmd]$ is positive.

Fix a Kahler metric $\omg$,
for $p\in X$, choose a holomorphic chart $(z_1,...,z_n)$,
s.t. $\omg(p)=\sqrt{-1}\sum\td z_j\wedge\td\zbar_j$ and
$\sqrt{-1}\Theta_E(p)=\sqrt{-1}\sum\limits_{j=1}^{n}\gma_j\td z_j\wedge\td\zbar_j$.
WLOG, $0\leq\gma_1\leq\gma_2\leq\cdots\leq\gma_n$,
and for any $j\geq s$, $\gma_j>0$.

Consider
$$\omg_\veps:=\veps\omg+\sqrt{-1}\Theta_E$$
for $\veps>0$, then $\omg_\veps$ is a Kahler metric.
$\omg_\veps(p)=\sqrt{-1}\sum\limits_{j}(\veps+\gma_j)\td z_j\wedge\td\zbar_j$.

$\Rightarrow$ the eigenvalue of $\sqrt{-1}\Theta$ with respective to
$\omg_\veps(p)$ is given by
$$\gma_{j,\veps}=\frac{\gma_j}{\veps+\gma_j}=
\frac{1}{1+\frac{\veps}{\gma_j}}$$

Claim: $[\sqrt{-1}\Theta,\Lmd_{\omg_\veps}]$ is positive on
$\wedgeform{p,q}\ten E$ when $p+q\geq n+s$, $0<\veps<<1$.

Take $u=\sum\limits u_{IJ}\td w_T\wedge\td\wbar_J\ten e$,then
$$
  \pair{[\sqrt{-1}\Theta_E,\Lmd_{\omg_\veps}]}{u}
=
  \sum_{|I|=p\atop|J|=q}
    \left(
      \sum_{i\in I}
        \gma_{i,\veps}
     +\sum_{j\in J}
       \gma_{j,\veps}
     +\sum_{k=1}^{n}
       \gma_{k,\veps}
    \right)
    |u_{IJ}|^2
\geq
  (\gma_{1,\veps}+...+\gma_{p,\veps}-\gma_{q+1,\veps}-...-\gma_{n,\veps})
  |u|^2
$$
note that $\gma_{j,\veps}\geq 1-\frac{\veps}{\gma_s}$
if $j\geq s$, $\gma_{j,\veps}\in[0,1)$ for all $j$. it
$$
\geq
  \left(
    (q+s-1)(1-\frac{\veps}{\gma_s})
   -(n-p)
  \right)
  |u|^2
>0
$$
if $p+q\geq n+s$ and $0<\veps<<1$.
\end{proof}

\begin{rem}(Kawamata-Viewheg vanishing theorem)

$E\to (X,\omg)$ is a holomorphic line bundle over a compact Kahler manifold.

Definition: $E$ is called positive, ...(positive="ample" in AG).
numerically effective(nef) if for any $\veps>0$, there is a smooth metric $h_\veps$
s.t. $\sqrt{-1}\Theta_{h_\veps}\geq -\veps\omg$.

Theorem: If $E$ is nef, and $\int_X c_1(E)^n>0$, then
$H^{q}(X,K_X\ten E)=0$ for $q\geq 1$.
\end{rem}

\textbf{Positivity concept of vector bundles(rank $>1$)}

$(E,h)\to (X,\omg)$ Hermitian vector bundle of rank $r$,
over a complex manifold(may not Kahler).

Denote $(e_1,...,e_r)$ a local orthonormal frame of $E$,
$(z_1,...,z_n)$ local holomorphic chart,
Chern curvature of $(E,h)$:
$$
  \Theta_{(E,h)}
= \sum_{1\leq j,k\leq n\atop 1\lmd,\mu\leq r}
    c_{ik\lmd\mu}\td z_j\wedge\td\zbar_k\ten e_\lmd^*\ten e_\mu
$$

Fact: $\sqrt{-1}\Theta_E$ induces a Hermitian operator $\theta_E$
on $TX\ten E$.

Let $u,v$ be local sections of $TX\ten E$,
$$u=\sum_{1\leq j\leq n\atop1\leq\lmd\leq r}
u_{k\mu}\pp{z_k}\ten e_\mu
$$
$$
  \theta_{E}(u,v)
:=
  \sum_{1\leq j,k\leq n\atop 1\leq\lmd,\mu\leq r}
    c_{jk\lmd\mu} u_{j\lmd}\overline{v_{k\mu}}
$$

\begin{definition}
We call $E$ Nakano positive, if $\theta_E$ is positive.
(i.e for any non-zero local section $u\in TX\ten E$, $\theta_E(u,u)>0$)

We call $E$ Griffith positive, if for any
$0\neq \xi\in T_xX$, $s\in E_x,s\neq 0$,
$$\theta_E(\xi\ten s,\xi\ten s)>0$$
\end{definition}

\begin{rem}
By definition, Nakano positivity $\Rightarrow$ Griffith positivity.

If $E$ is line bundle,
Nakano positivity $\iff$ Griffith positivity.
(and $\iff$ positivity of lines bundles)
\end{rem}

\begin{thm}(Demailly-Skota, 1979)

$E$ is Griffith positive $\Rightarrow$ $E\ten \det E$ is Nakano positive.
\end{thm}
\begin{proof}
  Omit. Non-trivial.
\end{proof}

Notation: $E>_{Nak}0$ ($E$ is Nakano positive).
Similarly, $E>_{Giff}0$...

\begin{prop}

(1)$E$ is Griffith positive if and only if $E^*$ is Griffith negative.

(2) Consider an exact sequence of holomorphic vector bundles:
$$0\to S\to E\to Q\to 0$$
then if $E$ is Griffith positive, then $Q$ is Griffith positive.
If $E$ is Griffith negative, then $S$ is Griffith negative.
If $E$ is Nakano negative, then $S$ is Nakano negative.
\end{prop}

\begin{proof}
Omit. Compute curvature...
\end{proof}

Remark: In general, $E$ is Nakano positive, $\not\Rightarrow$
$Q$ is Nakano positive.

\begin{thm}(Nakano vanishing theorem)

$(X,\omg)$ is compact Kahler of dimension $n$,
$(E,h)$ is a Nakano positive holomorphic Hermitian vector bundle, then
$$H^{n,q}(X,E)=0\qquad \forall q\geq 1$$
\end{thm}

\begin{proof}
$E$ is Nakano positive, check:
$$[\sqrt{-1}\Theta_E,\Lmd_\omg]$$
is positive on $\wedgeform{n,q}\ten E$ for $(q\geq 1)$
\end{proof}

\textbf{Ampleness}

$E\to X$, $E$: holomorphic line bundle of rank $r$,
$X$:complex manifold.

\begin{definition}(Jet vector bundle)
$$J^kE=\bigcup_{x\in X}(J^kE)_x$$
where
$$(J^kE)_x=\mcalO_x(E)\Big/\mfkm_x^{k+1}\mcalO_x(E)$$
$\mfkm_x\subseteq\mcalO_x$ be the maximal ideal of $\mcalO_x$.
\end{definition}

In local coordinate,
$$
  (J^kE)_x
=
  \Big\{
    \sum_{1\leq\lmd\leq r \atop |\afa|\leq k}
      C_{\lmd\afa}(z-x)^{\afa} e_{\lmd}(z)
  \Big\}
$$

\begin{prop}
$J^kE$ is a holomorphic vector bundle of rank $=r{n+k\choose n}$.
\end{prop}

\begin{proof}
  Exercise.
\end{proof}

\begin{definition}
$E$ is called very ample, if the following maps:
$$H^0(X,E)\to (J^1E)_x$$
$$H^0(X,E)\to E_x\oplus E_y$$
are surjective, for all $x,y\in X$, $x\neq y$.

$E$ is called ample, if $S^mE:=\Sym^m E$ is very ample for some $m\in\bbN$.
\end{definition}

(ample: "足够多的全纯截面")

\begin{thm}(Kodaira)

$L$-holomorphic line bundle, $X$ is a compact complex manifold.
Then $L$ is positive if and only if $L$ is ample.
\end{thm}

%%%%%%%%%%%%%%%%%2019.5.16%%%%%%%%%%%%%%%%%%%%%%%%%%
%%%%%%%%%%%%%%%%%第12周了吗?%%%%%%%%%%%%%%%%%%%%%%%

We will prove:

\begin{thm}
$L\to X$ holomorphic line bundle over a compact complex manifold,
then $L$ is positive $\iff L$ is ample.
\end{thm}

\textbf{We need:}

(1)Kodiara vanishing theorem.

(2)Blow-up of complex manifold

(3)Relation between divisor and line bundles.

\textbf{analytic cycles, divisors and meromorphic functions}

\begin{definition}
$X$ be a analytic set in some complex manifold, then the set
$X_{reg}$ is a dense subset of $X$.
Denote the connected component of $X_{reg}$ by $X_\afa$,
$\overline{X_\afa}$ is the closure of $X_\afa$ in $X$,
then $\overline{X_\afa}$ is called a global irreducible component of $X$.

In particular, $X$ is the union of global irreducible components.
\end{definition}

\begin{example}(Global irreducibility is different from local irreducibility)

$V=\Bigset{(x,y)\in\bbC^2}{y^2=x^2(1+x)}$ is an analytic set in $\bbC^2$,
$V_{reg}=V\setminus\{0\}$ is connected. So,
$V=\overline{V_{reg}}$ is globally irreducible.

On the other hand, $(V,0)$ is a reducible as an analytic germ.
\end{example}

\begin{definition}(analytic cycles)

$X$ is a complex manifold,  a $q$-cycle (with integer coefficient)
is a formal linear combination $\sum\limits\lmd_j V_j$, $\lmd_j\in\bbZ$,
and $V_j$ is a global analytic sets of $X$ of dimension $q$.
\end{definition}

So, we get a group $C_{cyl}^q(X)$.

an element of $Cycl^{n-1}(X)$ is called a divisor.
(Weil divisor)
($Div(X)$)

If $D$ is an irreducible analytic set of dimension $n-1$
then the divisor given by $D$ is called a prime divisor.

\begin{rem}
For any open set $U\subseteq X$, $U\to Cycl^q(U)$ induces a sheaf
$Cycl^q$ of $X$ with the germ $Cycl_x^q$ given by $q$-dimension analytic germs at $X$.
\end{rem}

\begin{thm}
$X$ is a connected complex manifold, $f\in\mcalO(X)$,
then we have $f^{-1}(0)$ is emply of $\dim_{\bbC}$ isempty of $n-1$.
\end{thm}


\begin{definition}(Cartier-dividiot)

A divisor $D=\sum\lmd_j D_j$ locally giveb by a $\bbX$ linear combination of $div(f)$.
$f$ is locally holomorphic functions.
\end{definition}

\begin{definition}
$X$ is a compact , $\beta\in\mcalO(X)$,
$D_j$ is a global irreponent of $f^{(-1)0}$,
$$m_j:=Ord_z(f)$$
for all $z\in D_j{\text{reg}}\setminus \bigcup_{k\neq j} D_k $
$m_j$ be the vanishing order along $D_j$.
\end{definition}

\begin{thm}
$(A,x)$ an analytic germ of $\dim_{\bbC}$=n-1.
$(A,x)=(g)$for sone $g\in \mcalO X$,and $g$ is a product of
$(J_{A_j,x})=(g_j)$.

(2) Let $f\in\theta_x$ with $(f^{-1}(0),x)\subseteq(A,x)$,then
$f=u\coprod_j g_jm^{m_j}$, where $m_j=ord_z (f)$
\end{thm}

\begin{prop}If $X$ is a complex manifold, then any Weil divisor is
also a Cantier divisor.
\end{prop}

Remark: NOT true for singular points.

Meromorphic function: $X$ complex manifold, $\mcalO_X$
sheaf of functions on $X$.
$$\mfkm_x:=\Bigset{\frac{g_x}{h_x}}
{g_x,h_x\in\mcalO_x\text{and $h_x$ is not zeor in $\mcalO_x$}}$$
$$
  \mcalM:=\bigcup_{x\in X}\mfkm_x
$$
  with the topology given by the basis
$$
  \Bigset{\frac{G_x}{H_x}}
  {x\in V, G,H\in\mcalO(V)}
$$

\begin{example}
$f(z_1,z_2)=\frac{z_1}{z_2}$
\end{example}

\begin{definition}
Let $F\in\mfkm(X)$,denote
$P(X):=\not\in \Bigset{x\in X}{f_x\not\in\mcalO_x}$.
Pole set pf $f$, and
$Z(f):=P(\frac{1}{z})$ zero set of $f$.
\end{definition}

\begin{thm}
$f\in\mfkm(X)$, if $P(d)$ (or$Z(f)$) is not empty,
then $P(f)$ is analytic set of $\dim=\bbH$.
\end{thm}

\begin{definition}
$P(f)\cup Z(f)$ is called the indeterminiary of set of $f$,
(in particular, codimension $P(M)\cap Z(f)\geq 2$)
\end{definition}

\begin{prop}
Given $f\in\mcalM(X)$, we get a divisor:
$$div(f)=\sum a_jA_j-\sum b_jB_j$$
where $a_j=$ the vanishing order of $f$ along $A_j$,
$A_j$ a globally irreducible component of $Z(f)$,
$b_j=$...along of $\frac{1}{f}$ along $B_j$,
$B_j$:...component of $P(f)$.
\end{prop}

\begin{example}
$f=\frac{z_1}{z_2}\in\mcalM(\bbC^2)$, then
$P(f)=\{z_2=0\}$ and $Z(f)=\{z_1=0\}$, and
$$div(f)=[z_1=0]-[z_2=0]$$
\end{example}

Consider: $X$ - complex manifold, $\mcalO^*$:
sheaf of invertible holomorphic functions,

$\mcalM^*$: Sheaf of non-zero meromorphic functions

$\mcalD iv$: Sheaf of $(n-1)$-cycles.

\begin{prop}
We have an exact sequences:
$$0\to\mcalO^*\to\mcalM^*\to\mcalD iv\to 0$$

In particular, $\mcalD iv=\mcalM^*/\mcalO^*$.

long exact sequence:

$$0\to H^0(X,\mcalO^*)\to H^0(X,\mcalM^*)\to H^0(X,\mcalD iv)
\to H^1(X,\mcalO^*)\to H^1(X,\mcalM^*)\to\cdots$$
\end{prop}
where, note that :
$$H^0(X,\mcalD iv)=Div(X)\qquad H^1(X,\mcalO^*)=Pic(X)$$


Consider $Div(X)=H^0(X,\mcalM^*/\mcalO^*)\to Pic(X)$,
$f\in H^0(X,\mcalM^*/\mcalO^*)\iff$ we have an open covering
$X=\bigcup_i U_i$ and $f_i\in\mcalM^*(U_i)$ with
$\frac{f_i}{f_j}\in\mcalO^*(U_i\cap U_j)$.

$f\in H^0(X,\mcalM^*/\mcalO^*)\xra{\fai}
(U_i\cap U_j,g_{ij}\in\mcalO^*(U_i\cap U_j))
\in \check{H}^1(\mcalU,\mcalO^*)\inj H^1(X,\mcalO^*)$.

\begin{definition}
A divisor $D$ is called principal divisor, if $D= div(h)$ for some
$h\in\mcalM^*(X)$.
\end{definition}

\begin{prop}
$\ker\fai=\{\text{principal divisors}\}$,
i.e. $\mcalO(D)$ is trivial $\iff$ $D= div(f)$ for some global meromorphic functions.
\end{prop}

\begin{prop}
$$\mcalO(D_1+D_2)=\mcalO(D_1)\ten\mcalO(D_2)$$
$$\mcalO(-D)=\mcalO(D)^*$$
\end{prop}

\begin{definition}
$D_1,D_2\in Div(X)$ is called linear equivalent, if $D_1-D_2$ is principal,
denoted by $D_1\sim D_2$. We have an injection:
$$Div(X)/\sim\inj Pic(X)$$
\end{definition}

Remark: in general, $D\to\mcalO(D)$ is not surjective.

If $X\inj\bbP^n$, then $Div(X)/\sim\cong Pic(X)$.

\begin{prop}
$L\to X$ holomorphic line bundle over a complex manifold,
we have a canonical map:
$$H^0(X,L)\setminus\{0\}\to Div(X)$$
$$s\to Z(s)$$
\end{prop}

\begin{proof}
$s\in H^0(X,L)\iff$ the data $(U_i,f_i\in\mcalO(U_i))$,
$L$ is determined by $g_{ij}\in\mcalO^*(\mcalU_i\cap\mcalU_j)$.

$Z(s)$ locally given by $div(f_i)$.
($div(f_i)=div(f_j)$ on $U_i\cap U_j$)
\end{proof}

\begin{prop}
$s_i\in H^0(X,L_i)\setminus\{0\}, i=1,2$ ,we have
$Z(s_1\ten s_2)=Z(s_1)+Z(s_2)$.
\end{prop}

\begin{prop}
Let $s\in H^0(X,L)\setminus\{0\}$, then $\mcalO(Z(s))\cong L$.
\end{prop}
\begin{proof}
Assume $X=\bigcup U_i$ with $L$ determined by $g_{ij}\in\mcalO^*(U_i\cap U_j)$,
$s\in H^0(X,L)$ determined by $(U_i,f_i\in\mcalO(U_i))$.

so,$\mcalO(Z(s))$ is the line bundle given by $\frac{f_i}{f_j}\in\mcalO^*(U_i\cap U_j)$.

note that $f_i=g_{ij}f_j$.
\end{proof}

\begin{cor}
Let $s_i\in H^0(X,L_i)\setminus\{0\},i=1,2$, then
$$Z(s_1)\sim Z(s_2)\iff L_1\cong L_2$$
\end{cor}
use the fact: $\mcalO(Z(s_i))=L_i$ and $\mcalO(\text{principal divisor})\cong\mcalO_X$
trivial line bundle.

\begin{prop}
Consider the map
$$Div(X)\to Pic(X)$$
$$D\to\mcalO(D)$$
then the image is generated by line bundles with non-zero holomorphic sections.
\end{prop}

%%%%%%%%%%%%2019.5.21%%%%%%%%%%%%%%%%%%%
%%%%%%%%%%%快期末了,抓紧复习%%%%%%%%%%%%%%

\section{Blow-up}

Local picture: $U\subseteq\bbC^n$ open subset, $Y\subseteq U$
linear subspace, $codim_UY=k$, e.g. assume $Y=\Bigset{z\in U}{z_1=...=z_k=0}$.

Consider the space
$$U_Y:=\Bigset{([w],z)\in\bbP^{k-1}\times U}{w_iz_j=w_jz_i,\,1\leq i,j\leq k}
\subseteq\bbP^{k-1}\times U\xra{\pi_2}U$$

\begin{definition}
$U_Y$ is called the blow-up of $U$ along $Y$.
\end{definition}

\begin{prop}
$U_Y$ is a smooth complex submanifold of $\bbP^{k-1}\times U$, and
$\dim_{\bbC}U_Y=\dim_{\bbC}U=n$. And
$\tau:U_Y\to U$ is a holomorphic map with
$$
  \tau|_{U_Y\setminus\tau^{-1}(Y)}:
  U_Y\setminus\tau^{-1}(Y)\cong U\setminus Y
$$

And for any $y\in Y$, $\tau^{-1}(y)=\bbP^{k-1}\times\{y\}$
is complex projective space.

\end{prop}

Locally, on then chart $w_1\neq 0$, denote $\hat{w}_i=\frac{w_i}{w_1}$
for all $2\leq i\leq k$. Then $z_i=\what_iz_1$.Then
$(z_1,\what_2,...,\what_k,z_{k+1},...,z_n)$ gives a holomorphic chart
of $U_Y$.

Denote $(z_1,...,z_n)=(z_1,\what_2,...,\what_k,z_{k+1},...,z_n)$,
then $z_1=\xi_1$, $z_2=\xi_1\xi_2$,...,$z_k=\xi_1\xi_k$, and
$z_{k+l}=\xi_{k+l}$for $k\geq l$.

In this coordinate system, $\tau^{-1}(Y)=
\Bigset{\xi\in U_Y}{\xi_1=0}$.

$\Rightarrow\tau^{-1}(Y)$ is a (smooth) hypersurface in $U_Y$.And,
$\tau^{-1}(Y)\cong \bbP(N_{Y/U})$,
where $N_{Y/U}$ is the normal bundle of $Y$ in $U$.
$$(0\to T_Y\to T_U|_Y\to N_{Y/U}\to 0)$$

If $codim_UY=1$ hypersurface, then $U_Y\cong U$.

\textbf{Global construction}

$Y$ is a complex submanifold of $X$, $\dim_{\bbC}=n$,$\dim_{\bbC}Y=k\leq n$.

\begin{lemma}
  If $f_1,...,f_k$ and $g_1,...,g_k$ are two (local) definition of $Y$,
defining equations of $Y$, $Y=\Bigset{f_z(z)=...=f_k(z)=0}{,}$,
then $\td f_1,...,\td f_k$ are linely independent along $Y$.
And $\exists$ a matrix $(m_{ij})$ of holomorphic functions,
s.t. $g_i=\sum_{j=1}^{k}M_{n,j}f_j$ for any $1\leq i\leq k$.

The matrix $(M_i^j)$ is invertible along $Y$,and determined uniquely
by $(f_1,...,f_k)$ and $g_1,...,g_k$.
\end{lemma}

\begin{proof}
Assume $f_i=z_i$ for $1\leq i\leq k$ is a local coordinate system
$\equiv 0$.For ever $g_i$, $g_i|_{z_1,...,z_k=0}$

Consider the Taylor expansion of $\\g_i$, we set
$$g_i=\sum_{j=1}^kM_i^j(z)z_j$$

$\td g_i=\sum_{j=1}^{k}\td M_{i}^jz_j+
\sum_{j=1}^{k} M_i^j\td z_j$.

$(\td g_1,...,\td g_k)|_{Y}$
and $(\td z_1,...,\td z_k)|_Y$ are $L.I$,
so $M_i^j|_{Y}$ is invertible.

Assume $Y\cap U=\{f_1^U=...=f_k^{U}=0\}$,
$Y\cap V=\{f_1^V=f_2^V=...=f_k^V=0\}$ and
$(M_{i,UV}^j)_{1\leq i,j\leq k}$ is the
\end{proof}

$0\to T_Y\to T_X|_Y\to N_{Y/D}$, the dual
$$N^*_{Y/ X}\to T_X^*|_Y\to T^*_Y$$

$(M_i^j,UV)$ gives the translation matrix middle of $N_{Y/X}^*$

\begin{lemma} $\exists$ isomorphism $\phi_{UV}:
\tau_U^{-1}(U\cap V)\cong \tau_V^{-1}(U\cap V)$.
\end{lemma}

\begin{proof}
  Assume $f_i^U=\sum_{j=1}^{k}=\sum_{j=1}^{k}M_{i,UV}^jf_j^V$.
Define $\phi_{UV}([w],z)=([M^{-t}w],z)$, then
$\phi_{UV}$ satisfies the two properties.
\end{proof}

\begin{definition}(The blow-up of $X$ along $Y$)(Global blow up)

$\Bl_YX$:the blow-up of $X$ along $Y$
is defined as the complex manifold by gluing the $U_Y$
and $\Omg:=X\setminus S_Y$,
where $S_Y$ is some neighborhood of $Y$.
\end{definition}

we have a holomorphic map: $\tau:\Bl_YX\to X$.

\begin{prop}
$\tau:\Bl_YX\to X$ satisfies :

(1)$\tau^{-1}(Y)$ is a smooth complex submanifold of $\Bl_YX$,
with $\dim_{\bbC}=n-1$,(It is called the excepted divisor of $\tau$)

(2)$\tau:\Bl_YX\setminus \tau^{-1}(Y)\to X\setminus Y$ is an isomorphism.

(2)$\tau$ is a proper map(any pre-image of compact set is compact).
\end{prop}

\begin{proof}
Check.
\end{proof}

\textbf{projective bundle} $E\to X$ is a holomorphic vector bundle(of rank $r$)
over a complex manifold(of complex dimension $n$),
then we can define projective bundle $\bbP(E)$,
$$\bbP(E):=\Bigset{(x,[\xi])}{x\in X,\,\xi\in E_x\setminus\{0\}}$$

$\bbP(E)$ is a complex manifold of dimension $n+r-1$
(if $X=\{pt\}$, then $\bbP(E)$ is just the projective space)

We have a tautological line bundle on $\bbP(E)$:
$$\mcalO_E(-1)_{(x,[\xi])}=\bbC\xi$$
$\mcalO_E(-1)$ is a holomorphic line bundle on $\bbP(E)$.

\textbf{Exercise:}Assume $(E,h)$ is an hermitian vector bundle with
metric $h$,then $h$ induces a metric on $\htil$ on $\mcalO_E(-1)$,
then the Chern curvature $\Theta$ of $\htil$ satisfies:
for any $x\in X$, $\sqrt{-1}\Theta|_{\bbP(E_x)}<0$.

\begin{thm}
  $\tau:\Bl_YX\to X$ blow-up along $Y$, $E:=\tau^{-1}(Y)$
exceptional divisor, $\mcalO(E)$:
the holomorphic line bundle associated to $E$, then

(1) $\tau: E\to Y$ is just the map $\bbP(N_{Y/X})\to Y$

(2) $\mcalO(E)|_E\cong \mcalO_{P(N_{Y/X})}(-1)
\cong N_{E/\Bl_YX}$ the normal bundle of $E$ in $\Bl_YX$.
\end{thm}

\begin{proof}
  Exercise.
\end{proof}

\begin{cor}
If $X$ is a (compact) Kahler manifold, $Y$ is a compact submnifold
of $X$, then the blow-up $\Bl_YX$ is also a (compact) Kahler manifold.
\end{cor}

\begin{proof}
$\tau:\Bl_YX\to X$, let $\omg$ be a Kahler matric on $X$,
then $\tau^*\omg$ is a semi-positive $(1,1)$-form on $\Bl_YX$,
positive on $\Bl_YX\setminus E$, and the kernel of $\tau^{-1}\omg$
along $E$ is given by the tangent space of the fiber $E\to Y$.

Define the metric $h$ on $\mcalO(E)$ as follows:
on $E$,$h$ is induced by the metric on $N_{Y/X}$ induced by the metric on $N_{Y/X}$,
and we extend $h$ to a neighborhood of $E$;
outside a neighborhood of $E$,($\mcalO(E)|_{\Bl_YX\setminus E}$ is trivial),
$h$ is given by the trivial metric.

Then ,we glue these two metrics to get a matric on $\mcalO(E)$.
Denote the curvature $\theta:=\sqrt{-1}\Theta(\mcalO(-E),h)$/

Claim: $C\tau^*\omg+\theta>0$ for $C\gg 1$
\end{proof}

%%%%%%%%%%%2019.5.23  第13周%%%%%%%%%%%%%%%%%%%%%


\section{Kodaira Embedding Theorem}

Recall: $L\to X$ holomorphic line bundle with a smooth metric $h$
over compact complex manifold.

$L$ is called positive if the curvature $\sqrt{-1}\Theta_{(L,h)}$
is a positive $(1,1)$-form.

$L$ is called ample, if $L^{\ten m}:= mL$ is very ample for $m\gg 1$.

Recall: a holomorphic vector bundle $E$ is called very ample,
if the following maps
$$H^0(X,E)\to E_x\oplus E_y\qquad \forall x\neq y\in X$$
$$H^0(X,E)\to (J^1E)_x\qquad \forall x\in X$$
are surjective.

\begin{prop}
$X$ is a complex manifold of dimension $n$, $Y\subseteq X$
is a complex submanifold of codimension $k$. $\tau:\Xhat\to X$
blow-up along $Y$. $E:=\tau^{-1}(Y)$ exceptional divisor. Then
$$K_{\Xhat}=\tau^{*}K_X\ten\mcalO((k-1)E)$$
\end{prop}

(Recall: $K_X=\det T^*X=\wedgeform{n}T^*X$, 
locally free sheaf of holomorphic $n$-terms $\Omg_X^n$).

\begin{proof}
locally, $\tau$ can be written as
$$\tau:(w_1,...,w_n)\to (z_1,...,z_n)$$
$$z_1=w_1,\,z_2=w_2,...,z_k=w_kw_1,...,z_{k+l}=w_{k+l}$$

$$
  \Rightarrow
  \tau^*(\td z_1\wedge\td z_2\wedge\cdots\wedge\td z_n)
= w_1^{k-1}\td w_1\wedge\td w_2\wedge\cdots\wedge\td w_n
$$
(local holomorphic frame of $K_X$ and $K_{\Xhat}$...
$w_1^{k-1}$-local section of $\mcalO(E)$)


Recall: $L$-line bundle, $\{g_{ij}\}$ transition function,
a local section is the following data
$f_i=g_{ij}f_j$. If $e_i$ the local frame on $U_i$, then
$f_ie_i=f_je_j$ on $U_i\cap U_j$.

之后check两个线丛的转移函数相同.
\end{proof}



\begin{lemma}Let $\Xhat$ be the blow up of $X$ along
$\{x_1,...,x_N\}\subseteq X$, ($N$ distinct points),
denote $E$ the exceptional divisor, then
$$
  H^1(\Xhat,\mcalO(-mE)\ten\tau^*(kL))=0
$$
for $m\geq 1$, $k\geq Cm$ for $C\gg 1$
\end{lemma}

\begin{proof}
$$
  H^1(\Xhat,\mcalO(-mE)\ten\tau^*(kL))
=
  H^1(\Xhat,K_{\Xhat}\ten K^{-1}_{\Xhat}\ten\mcalO(-mE)\ten\tau^*(kL))
=
  H^{n,1}(\Xhat,F)
$$
where $F:= K^{-1}_{\Xhat}\ten\mcalO(-mE)\ten\tau^*(kL)$.

By Kodaira-Nakano vanishing, if $F$ is positive,
then $H^{n,1}(\Xhat,F)=0$.

Note that 
$$F=\mcalO(-mE)\ten\tau^*K_X^{-1}\ten\mcalO((1-n)E)\ten\tau^*(kL)$$
$$=\tau^*K_X^{-1}\ten\mcalO(-(m+n-1)E)\ten\tau^*(kL)$$

We know, $\exists C_0\gg 1$ s.t. $C_0L\ten K_X^{-1}$ is positive,
and $\exists C\gg 1$,s.t. $C\tau^*L\ten\mcalO(-E)$ is positive.

So, For $k\geq Cm\,\,(C\gg 1)$, $F$ is positive.

Let $v_j\in H^0(\Omg_j,kL)$ be a local section of $kL$,
s.t. $v_j$ generates the $m$-jet at $x_j$. Let
$\psi_j\in C^\infty(X,\bbR)$ s.t. $supp\psi_j\subset\subset\Omg_j$,
$0\leq\psi_j\leq 1$, $\psi_j\equiv 1$ around $x_j$. Denote
$$
  v:=\sum_{j=1}^{n}
    \psi_jv_j
$$
a smooth section of $kL$.
$$
  \td''v=\sum_{j}\td''\psi_jv_j\in C_{(0,1)}^{\infty}(X,kL)
$$
satisfies $\td''v=0$ near $x_j$ for $1\leq j\leq N$.

Lemma:(Exercise)
$$H^0(X,M)\to H^0(\Xhat,\tau^*M)$$
$$s\mapsto\tau^*s$$
is an isomorphism for any line bundle $M$.

Lemma:(Exercise)
a section of $\tau^*M$ with vanishing order$=k$ along $E$ is the
pull-back of a section of $M$ with vanishing order $=k$ at $x_j$.
\vs 

Denote $S_E\in H^0(\Xhat,\mcalO(E))$ the canonical section of $E$,
$$
  w=S_E^{-(m+1)}\ten\tau^*(\td''v)\in
  C_{(0,1)}^{\infty}
  (\Xhat,\mcalO(-(m+1)E)\ten\tau^*(kL))
$$
and $\td''w=0$. Vanishing of $H^0(\Xhat,\mcalO(-(m+1))\ten\tau^*(kL))$
implies $w=\td'' u$ for some $u\in C^\infty(\Xhat,\mcalO(-(m+1)E)\ten\tau^{-1}kL)$.
$$S_E^{-(m+1)}\tau^*(\td''v)=\td''u$$
$$\Rightarrow\td''(\tau^*v-S_E^{(m+1)}u)=0$$
so, $\tau^*v-s_E^{(m+1)}u$ is a holomorphic section of $\tau^*(kL)$.
Using $s_E^{(m+1)}u=\tau^*f$ for some $f\in H^0(X,kL)$ with vanishing order
$= m+1$ along $x_j$.

Claim: denote $g:= v-f$ is the holomorphic sections generating 
the $m$-jets at $x_j$. $\td''(\tau^*g)=0\Rightarrow\tau^*g$ is holomorphic,
$\Ord_{x_j}(f)=m+1$. So, $J^m(g)_{x_j}=J^m(v)_{x_j}$. 

\end{proof}

\begin{thm} $L\to X$ positive line bundle, $x_1,...,x_N\in X$
are $N$ distinct points on $X$, then there exists $C>0$, s.t.
$$
  H^0(X,kL)\surj\bigoplus_{j=1}^N
  \left(
    J^m(kL)
  \right)_{x_j}
$$ 
is surjective for all $m\geq 0$ and $k\geq Cm$
\end{thm}

\begin{proof}
\end{proof}


\begin{thm}(Kodaira)

Line bundle $L$ is positive $\iff$ it is ample.
\end{thm}

(微分几何的正性与代数几何的正性是等价的)

\begin{proof}(有一边是显然的,留作习题)

proof of "$L$ ample $\Rightarrow$ $L$ positive".

Exercise: If $A$ is a very ample line bundle on $X$,
$H^0(X,A)$ has a basis $\{s_0,...,s_N\}$, then the map
$$\Phi:X\to\bbP(H^0(X,A))$$
$$s\mapsto[s_0(x);s_1(x);...;s_N()]$$
(Kodaira map) is a holomorphic embedding.

(Hint: $H^0(X,A)\surj A_x\oplus A_y$ means that $\Phi$ is injective;
$H^0(X,A)\surj(J^1(A))_x$ means that $\Phi_*$ is injective.)

Exercise: denote the tautological line bundle on $\bbP(H^0(X,A))$
by $\mcalO(1)$, then $A=\Phi^*\mcalO(1)$.

Cor: $A$ is very ample $\Rightarrow A$ is positive.

Given any inner product on $H^0(X,A)$, we get a metric $h$ on $\mcalO(1)$,
the curvature $\Theta(\mcalO(n))$ of $h$ is positive.
$$\Rightarrow\Theta(A)=\Phi^*\Theta(\mcalO(1))$$
$\Phi$ is embedding $\Rightarrow$ $\Theta(A)$ is positive.
\end{proof}

$L$ positive $\Rightarrow L$ ample, i.e. $mL$ is very ample,
$$\Rightarrow\Phi_{H^0(X,mL)}:X\inj\bbP(H^0(X,mL))$$
holomorphic embedding($\Rightarrow X$ is an analytic submanifold of $\bbP(H^0(X,mL))$)

$\xra{\text{Chow theorem}}X$ is an algebraic set of $\bbP(H^0(X,mL))$
(i.e. $X=\bigcup\limits_{j=1}^t\{P_j=0\}$, $P_j$-homogenous polynomial)

a compact complex manifold $X$ admitting a positive line bundle $L$
if and only if $X$ is an algebraic manifold.

$$
  0\to\bbZ\to\mcalO\xra{e^{2\pi\sqrt{-1}}}\mcalO^*\to 0
$$
$$\rightsquigarrow
H^1(X,\mcalO^*)\xra{C_1}H^2(X,\bbZ)\to H^2(X,\mcalO)\to...
$$
and $H^2(X,\bbZ)\to H^2(X,\bbC)\cong H^{2,0}\oplus H^{1,1}\oplus H^{0,2}$,
and $H^2(X,\mcalO)\cong H^{0,2}(X,\bbC)$.

$\Rightarrow\forall\afa\in H^2(X,\bbZ)\cup H^{1,1}(X,\bbC)$,
we have a holomorphic line bundle $L$ s.t. $\afa=c_1(L)$.

$L$ admitting a positive line bundle $\iff$ $X$ admitting a class
$\afa\in H^{2}(X,\bbZ)\cup H^{1,1}$ with a positive representative.





 


