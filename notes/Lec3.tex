\chapter{Hermite向量丛}
\section{向量丛的联络与曲率}

先回顾一下光滑向量丛的联络(活动标架版本),这是黎曼几何的标准内容。
我们考察光滑的实向量丛或者复向量丛。为表述方便,令$\bbK:=\bbR$或$\bbC$.

\begin{notation}(向量值微分形式)

设$X$为光滑流形,$E\to X$为$X$上的光滑$\bbK$-向量丛
,记$$\Omg^p(X,E):=\Gma(X,(\wedgeform{p}T^*M)\ten E)$$
为$X$上的取值于$E$的光滑$p$-形式空间。
\end{notation}
%For $X$ be a smooth manifold, $E$ is a vector bundle(real or complex), denote
%is the space of $k$-differential $p$-forms with values in $E$.

我们将$\Omg\updot(X,E):=\bigoplus\limits_{p\geq 0}\Omg^p(X,E)$
自然视为分次线性空间,使得该空间的$p$次齐次子空间为$\Omg^p(X,E)$.
注意到$X$上的微分$p$-形式空间$\Omg^p(X)\cong\Omg^p(X,\bbK)$,
此处的$\bbK$为$X$上的平凡线丛。
注意$\Omg\updot(X)$上的外积结构$\wedge$,它自然诱导了
$$
  \wedge:
  \Omg^p(X)\times\Omg^q(X,E)
  \to\Omg^{p+q}(X,E)
$$
事实上这给出了$\Omg\updot(X,E)$的一个分次$\Omg\updot(X)$-模结构。

局部地,取向量丛$E$的一个局部平凡化
$$\theta_{\alpha}:E|_{U_{\alpha}}\cong U_{\alpha}\times \bbK^{r}$$
记$\{e_1,...,e_r\}$为$U_\afa$上的一组局部标架,则
$\Omg^p(X,E)$中的元素$s$在此局部标架下形如
$$s=\sum_{\lmd=1}^r\fai^{\lmd}\ten e_{\lmd}
=:\fai^\lmd\ten e_\lmd$$
其中每个$\fai_\lmd$均为光滑$p$-形式,并采用Einstein求和约定。

%Locally, consider a trivialization of $E$,($\rightsquigarrow$ frame $(e_1,...e_r)$)
%$$s\in \sum\fai_{\lmd}(x)\ten e_{\lmd}(x)$$where $\fai_{\lmd}$ is a $p$-form.

\begin{definition}(向量丛上的联络)
\index{connection\kong 联络}

设$E\to X$为光滑流形$X$上的光滑向量丛,
丛$E$上的\textbf{联络}(connection)
是指作用在分次向量空间$\Omg\updot(X,E)$上的
次数为$1$的齐次线性映射$\tD:\Omg\updot(X,E)
\to\Omg^{\bullet+1}(X,E)$,
并且满足:
$$
  \tD(\fai\wedge s)
=
  \td\fai\wedge s
 +(-1)^p\fai\wedge\tD s
$$
对任意$\fai\in\Omg^p(X)$以及$s\in\Omg^q(X,E)$成立。
\end{definition}

%a (linear) connection on $E$ is a linear differential operator
%of order $1$ acting on$C^{\infty}\downdot(X,E)$:
%$$D:C_{p}^{\infty}(X,E)\to C_{p+1}^{\infty}(X,E)$$
%$$D(f\wedge x):= \td f\wedge s+(-1)^p f\wedge Ds$$
%where $f\in C^{\infty}(X,\wedgeform{p}T^*M)$, $s\in C^{\infty}(X,E)$.

注意定义中并没要求$\tD^2=0$,一般地$(\Omg\updot(X,E),\tD)$
并不是上链复形。易知向量丛$E\to X$,
$E$上的联络之全体,构成$\bbK$-线性空间.
联络的典型例子是,考虑$X$上的平凡线丛$\bbK$,则
$\Omg\updot(X,\bbK)\cong\Omg\updot(X)$上的外微分$\td$
即为丛$\bbK$上的联络。\vs

取$E\to X$的局部平凡化坐标卡$U$及其局部标架$\{e_1,e_2,...,e_r\}$,
则对任意$t\in\Omg^p(X,E)$,其中$r:=\rank E$.若在该局部标架下
$t=\fai^\lmd\ten e_\lmd$,则有
$$
  \tD t=
    \td\fai^\lmd\ten e_\lmd
   +(-1)^p\fai^\lmd\wedge\tD e_\lmd
$$
其中$\tD e_\lmd\in\Omg^1(U,E)$.
可见只要确定了$\tD$在$e_\lmd\in\Omg^0(X,E)$上的作用,
则联络$\tD$被唯一确定。
我们令$\tD e_\lmd=a_\lmd^\mu\ten e_\mu$,其中$a_\lmd^\mu\in\Omg^1(U)$,
称为$\tD$关于局部标架$\{e_1,e_2,...,e_r\}$的\textbf{联络$1$-形式},
称$r\times r$矩阵$A:=(a^\mu_\lmd)$为
$\tD$关于局部标架$\{e_1,e_2,...,e_r\}$的\textbf{系数矩阵}。
若记$e:=(e_1,e_2,...,e_r)$为标架排成的行向量,
则有紧凑的表达式$\tD e=eA$.

在局部标架$\{e_1,e_2,...,e_r\}$下,$\Omg^p(X,E)$中的元素$t$
可以用以$p$-形式为分量的列向量$\fai:=(\fai^\lmd)_{1\leq\lmd\leq r}$来表示,
即$t=\fai^\lmd\ten e_\lmd$.在此意义下,有
\begin{eqnarray*}
     \tD t
&=&
     \td\fai^\lmd\ten e_\lmd
    +(-1)^p\fai^\lmd\wedge\tD e_\lmd
 =
     \td\fai^\lmd\ten e_\lmd
    +(-1)^p\fai^\lmd\wedge a_\lmd^\mu\ten e_\mu\\
&=&
     \left(
       \td\fai^\mu+a_\lmd^\mu\wedge\fai^\lmd
     \right)\ten e_\mu
 =
     (\td\fai+A\wedge\fai)^\mu\ten e_\mu
\end{eqnarray*}
或者简记为
$$\tD\fai=\td\fai+A\wedge\fai$$

%Locally, consider a local trivialization
%$$\theta:E|_{\Omg}\xra{\sim}\Omg\times\bbK^r$$
%with a frame $\{e_1,...,e_r\}$. any section
%$t\in C^{\infty}_p(\Omg,E)$ can be written as
%$$t=\sum_{1\leq\lmd\leq r}\sgm_{\lmd}\ten e_{\lmd}$$
%$$Ds=\sum_{\lmd=1}^r\td\sgm_{\lmd}\wedge e_{\lmd}+(-1)^p
%\sgm_{\lmd}\wedge De_{\lmd}$$ where$$De_{\lmd}\in C_1^{\infty}(\Omg, E)$$
%can be written as$$De_{\lmd}=\sum_{\mu=1}^r
%a_{\mu\lmd}\ten e_{\mu}$$where "$a_{\mu \lmd}$" is called the coefficients of $D$
%with respect to frame $\{e_1,...,e_r\}$ .so,$$D(t)=\sum_{\lmd,\mu}\td\sgm_{\lmd}\wedge
%e_{\lmd}+(-1)^p\sgm_{\lmd}\wedge a_{\mu\lmd}\wedge e_{\mu}
%=\sum_{\mu}\sum_{\lmd}\left(\td\sgm_{\mu}+ a_{\mu\lmd}\wedge\sgm_{\lmd}
%\right)$$%平凡化下,用矩阵再写一下,自己脑补%$$Dt=\td\sgm+A\wedge\sgm$$where $A=(a_{\mu\lmd})$.
%RMK: connection always exists!

%%%%%%%2019.4.2第六周周二%%%%%%%%%%%%%%

%Recall: for any (connected) smooth manifold,
%$E\to X$ is a smooth vector bundle,Connection:
%$$D:C^\infty_p(X,E)\to C_{p+1}^\infty(X,E)$$
%where $C^\infty_p(X,E):=C^\infty(X,\wedge^pT^*M\ten E)$
%$$D(f\wedge s)=\td f\wedge s+(-1)^{\deg f}f\wedge Ds$$
%Essentially,$$D:C^\infty(X,E)\to C_1^\infty(X,E)$$
%Locally, consider a trivialization
%$\theta:E|_\Omg\xra{\sim}\Omg\times \bbK^r$, and a local frame
%$(e_1,...,e_r)$ where $e_k(x)=\theta^{-1}(x,\begin{pmatrix}
%0\\\vdots\\1 (k^{th})\\\vdots\\0\end{pmatrix})$.
%Let $s\in C^\infty(\Omg,E)$, i.e.$$s=\sum_{i=1}^r\sgm_ie_i$$
%where $\sgm_i$ are smooth functions.$$Ds=\td\sgm+A\wedge\sgm$$where
%$$\sgm=\begin{pmatrix}\sgm_1\\\vdots\\\sgm_r\end{pmatrix}
%\quad A={a_{ij}}$$

\begin{prop}(联络矩阵的变换)

设$E\to X$为光滑向量丛,$\tD$为$E$上的一个联络。设
$(U,e)$与$(\Util,\etil)$为丛$E$的两个局部平凡化,
其中$e=(e_1,e_2...,e_r),\,\etil=(\etil_1,\etil_2,...,\etil_r)$
为相应的局部标架。记它们之间的转移函数为
$\etil_\lmd=g_\lmd^\mu e_\mu$,$G:=(g_\lmd^\mu)$.
若$A,\Atil$分别为联络$\tD$关于
标架$e,\etil$的系数矩阵,则有变换关系
$$
  \Atil
=
  G^{-1}AG+G^{-1}\td G
$$
\end{prop}

\begin{proof}
将$\etil_\lmd=g_\lmd^\mu e_\mu$写为紧凑的矩阵形式,
有$\etil=eG$.从而我们有
\begin{eqnarray*}
\tD\etil&=&\tD(eG)=(\tD e)G+e\td G=eAG+e\td G\\
\tD\etil&=&\etil\Atil=eG\Atil
\end{eqnarray*}
因此整理得$\Atil=G^{-1}AG+G^{-1}\td G$.
\end{proof}

%consider another trivialization
%$$\tilde\theta:E|_\Omg\xra{\sim}\Omg\times\bbK^r$$
%$\rightsquigarrow$ a local frame $(\tilde{e_1},...,\tilde{e_r})$.
%Then there exists a invertible linear transform s.t.
%$$\tilde{e_k}=g_k^me_m$$ assume
%$$De_k=a_k^le_l\qquadD\tilde{e_k}=\tilde{a}_k^l\tilde{e}_l$$
%we have$$\td g_k^ne_n+g_k^ma_m^ne_n=\tilde{a}_k^lg_l^ne_n$$
%$$\rightsquigarrow\quad\tilde{a}_k^lg_l^n(g^{-1})^p_n
%=\td g_k^n(g^{-1})^p_n+g_k^ma_m^n(g^{-1})^p_n$$
%$$\rightsquigarrow\quad\tilde{a}_l^p=\td g_k^n(g^{-1})^p_n)
%+g_k^ma^n_m(g^{-1})^p_n$$$$\rightsquigarrow\quad
%\tilde{A}=\td g\cdot g^{-1}+g\cdot A\cdot g^{-1}$$
%\textbf{Curvature}$$H_D:=D^2$$%H要加一个圈。。。

\begin{definition}(曲率)
\index{curvature\kong 曲率}

设$E\to X$为光滑流形$X$上的光滑$\bbK$-向量丛,
$\tD$为$E$上的一个联络,则记
$$\Theta:=\tD\circ\tD$$
为联络$\tD$的\textbf{曲率}(curvature).
\end{definition}

在局部标架$e=(e_1,e_2,...,e_k)$下,对于$t\in\Omg^p(X,E)$,
若$t=e\fai$,其中$\fai$为分量为$p$-形式的列向量,
利用局部标架下的联络公式$\tD\fai=\td\fai+A\wedge\fai$,
其中$A$为联络$\tD$在关于此标架的系数矩阵,可得
\begin{eqnarray*}
     \Theta\fai
&=&
     \tD(\td\fai+A\wedge\fai)\\
&=&
     \td(\td\fai+A\wedge\fai)
    +A\wedge(\td\fai+A\wedge\fai)\\
&=&
     \td^2\fai+\td A\wedge\fai
    -A\wedge\td\fai+A\wedge\td\fai+A\wedge A\wedge\fai\\
&=&
     (\td A+A\wedge A)\wedge\fai
\end{eqnarray*}
即在局部标架$e$下,曲率算子$\Theta$的矩阵$\Omg=\td A+A\wedge A$,
此矩阵的矩阵元为光滑$2$-形式,称为\textbf{曲率形式}。

%locally,$$D^2s=D(\td\sgm+A\wedge\sgm)=
%\td(\td\sgm+A\wedge\sgm)+A\wedge(\td\sgm+A\wedge\sgm)$$
%$$=\td A\wedge\sgm-A\wedge\td\sgm+A\wedge\td\sgm+A\wedge A\wedge\sgm
%=(\td A+A\wedge A)\wedge\sgm$$so we have$$H=\td A+A\wedge A$$

\begin{prop}(曲率形式在不同局部平凡化下的变化)

设$E\to X$为光滑向量丛,$\tD$为$E$上的联络,
$\Theta$为联络$\tD$的曲率。
设$e=(e_1,e_2,...,e_r)$与$\etil=(\etil_1,\etil_2,...,\etil_r)$
为$E$的两组局部标架,转移矩阵$G$满足$\etil=eG$.
则曲率$\Theta$在标架$e,\etil$下的矩阵$\Omg,\Omgtil$满足
$$
  \Omgtil=G^{-1}\Omg G
$$
\end{prop}

\begin{proof}
我们已有$\Atil=G^{-1}AG+G^{-1}\td G$,从而
$G\Atil=AG+\td G$,两边外微分得到
$$
  \td G\wedge\Atil+G\td\Atil
= \td A\cdot G-A\wedge\td G
$$
因此有
\begin{eqnarray*}
     \td\Atil
&=&
     G^{-1}\td A\cdot G
    -G^{-1}\td G\wedge\Atil
    -G^{-1}A\wedge\td G\\
&=&
     G^{-1}\td A\cdot G
    -G^{-1}\td G\wedge(G^{-1}AG+G^{-1}\td G)
    -G^{-1}A\wedge\td G\\
&=&
     G^{-1}\td A\cdot G
    -G^{-1}\td G\cdot G^{-1}AG
    -G^{-1}\td G\cdot G^{-1}\td G
    -G^{-1}A\wedge\td G
\end{eqnarray*}
\begin{eqnarray*}
     \Atil\wedge\Atil
&=&
     (G^{-1}AG+G^{-1}\td G)
     \wedge
     (G^{-1}AG+G^{-1}\td G)\\
&=&
     G^{-1}A\wedge AG
    +G^{-1}A\wedge\td G
    +G^{-1}\td G\cdot G^{-1}AG
    +G^{-1}\td G\cdot G^{-1}\td G
\end{eqnarray*}
从而得到
$$
  \Omgtil=\td\Atil+\Atil\wedge\Atil
= G^{-1}(\td A+A\wedge A)G=G^{-1}\Omg G
$$
\end{proof}

\begin{rem}
此定理表明,曲率$\Theta$是丛$E$上的$2$-形式值$(1,1)$型张量,
故称为\textbf{曲率张量}。具体地,
$$\Theta\in\Omg^2(X,\End(E))\cong\Omg^2(X,E^*\ten E)$$
\end{rem}

%Similarly to $\tilde{A},A$ we haveExercise:
%$$\tilde{H}=gHg^{-1}$$曲率在不同平凡化下的表达式。where
%$$\tilde{e}=ge$$$\rightsquigarrow H$ can be
%considered as a section of $C^\infty_2(X,\Hom(E,E))$.
%because$$\tilde{H}\tilde{e}=gHg^{-1}\tilde{e}=gHe$$
%independent of the choice of local frames.

\begin{Example}(对偶丛的联络与曲率)

设$E\to X$为光滑向量丛,$E^*$为$E$的对偶丛。
注意到有如下自然的配对:
\begin{eqnarray*}
     \pair{}{}:
     \Omg^p(X,E^*)\times\Omg^q(X,E)
     &\to&\Omg^{p+q}(X)
\\
     \pair{\fai_\lmd\ten e^\lmd}{\psi^\mu\ten e_\mu}
&:=&
     \pair{e^{\lmd}}{e_\mu}
     \fai_\lmd\wedge\psi^\mu
\end{eqnarray*}
若$\tD_E$为$E$上的联络,则$\tD_E$诱导了对偶丛$E^*$上的联络$\tD_{E^*}$,
使得对任意$s\in\Omg^p(X,E^*)$以及$t\in\Omg^q(X,E)$都成立
$$
  \td\pair{s}{t}
=
  \pair{\tD_{E^*}s}{t}
 +(-1)^p\pair{s}{\tD_E t}
$$
\end{Example}

取$E$的局部标架$e=(e_1,e_2,...,e_r)$,记$e^*:=(e_1^*,e_2^*,...,e_r^*)$
为其对偶标架(也排成行向量),记对偶丛联络$\tD^*:=\tD_{E^*}$
的曲率为$\Theta^*$,它们在对偶标架$e^*$上的矩阵记作$A^*,\Omg^*$,则成立:
$
  \left\{
    \begin{array}{l}
      A^*=-A^T\\
      \Omg^*=-\Omg^T
    \end{array}
  \right.
$.这是因为,由于$\pair{e^{*T}}{e}=I$,从而
$$
  0=\td\pair{e^{*T}}{e}
=\pair{\tD^* e^{*T}}{e}+\pair{e^{*T}}{\tD e}
=\pair{(e^*A^*)^T}{e}+\pair{e^{*T}}{eA}
=A^{*T}+A
$$
因此对偶联络的系数矩阵$A^*=-A^T$.从而对偶联络的曲率矩阵
$$
  \Omg^*=\td A^*+A^*\wedge A^*
        =-\td A^T+A^T\wedge A^T
        =-(\td A+A\wedge A)^T
        =-\Omg^T
$$

这里要特别注意微分形式矩阵的运算,注意$A$的矩阵元为微分$1$-形式,
从而易验证有$(A\wedge A)^T=-A^T\wedge A^T$.
此外,若把对偶标架记为上指标$e^*=(e^1,e^2,...,e^r)^T$,则有
$\tD e^\lmd=-A^\lmd_\mu e^\mu$.

%fact: let $D_E$ be a connection on $E$,then it induces a connection $D_{E^*}$.
%Let $u$ be a local section of $E^*$, $s$ local section of $E$,
%then we define$$\td\langle u,s\rangle=\langle D_{E^*}u,s\rangle
%+\langle u,D_{E}s\rangle$$Exercise:$$H(D_{E^*})=-H(D_E)^T$$

\begin{Example}(直和丛的联络与曲率)

设$E,F\to X$均为$X$上的光滑向量丛,$\tD_E,\tD_F$
分别为$E,F$上的联络,则直和丛$E\oplus F$上自然有联络
$\tD_{E\oplus F}$,使得对任意$u\in \Omg^p(X,E)$
以及$v\in\Omg^p(X,F)$,成立
$$
  \tD_{E\oplus F}(u\oplus v)
=
  \tD_E u\oplus\tD_Fv
$$
\end{Example}
取定$E$的局部标架$e=(e_1,e_2,...,e_r)$
以及$F$的局部标架$f=(f_1,f_2,...,f_s)$,则$E\oplus F$
有局部标架$e\oplus f=(e_1,...,e_r;f_1,...,f_s)$.
容易验证相应的联络、曲率矩阵满足
$$
  A_{E\oplus F}
=
  \begin{pmatrix}
    A_E &  \\
        & A_F
  \end{pmatrix},
\qquad
  \Omg_{E\oplus F}
=
  \begin{pmatrix}
    \Omg_E &  \\
           & \Omg_F
  \end{pmatrix}
$$

%and for two vector bundles $E,F$, connections $D_E,D_F$, then
%$$D_{E\oplus F}:=D_E\oplus D_F$$$$H(E\oplus F)=H_E\oplus H_F$$

\begin{Example}(张量丛的联络与曲率)

设$E,F\to X$均为$X$上的光滑向量丛,$\tD_E,\tD_F$
分别为$E,F$上的联络,则张量丛$E\ten F$上自然有联络
$\tD_{E\ten F}$,使得对任意$E$的截面$u$以及$F$的截面$v$,成立
$$
  \tD_{E\ten F}(u\ten v)
=
  \tD_Eu\ten v+u\ten\tD_Fv
$$
\end{Example}

取定$E$的局部标架$e=(e_1,e_2,...,e_r)$
以及$F$的局部标架$f=(f_1,f_2,...,f_s)$,则$E\ten F$
有局部标架$e\ten f=\Bigset{e_\afa\ten f_\beta}
{1\leq\afa\leq r,\,1\leq\beta\leq s }$.
同意验证有关的联络、曲率矩阵满足
\begin{eqnarray*}
A_{E\ten F} &=& A_E\ten I_F+I_E\ten A_F\\
\Omg_{E\ten F}&=&  \Omg_E\ten I_F+I_E\ten\Omg_F
\end{eqnarray*}
在计算曲率时要当心微分$1$-形式系数的矩阵的外积运算结果的符号。

\begin{Example}(行列式丛的联络与曲率)

设$E\to X$为$X$上的秩为$r$的光滑向量丛,
$\tD_E$为$E$上的联络,则$\tD_E$自然诱导了\textbf{行列式从}
$\det E:=\wedgeform{r}E$上的联络$\tD_{\det E}$,使得
对任意局部标架$e=(e_1,e_2,...,e_r)$,
$$
  \tD_{\det E}
  e_1\wedge e_2\wedge\cdots\wedge e_r
=
  \sum_{k=1}^{r}
  e_1\wedge\cdots\wedge\tD_Ee_k\wedge\cdots\wedge e_r
$$
\end{Example}

$E$的局部标架$e=(e_1,e_2,...,e_r)$诱导了线丛$\det E$的局部标架
$e_1\wedge e_2\wedge\cdots\wedge e_r$,容易验证相应的联络、曲率
矩阵满足
$
  \left\{
    \begin{array}{lcl}
      A_{\det E} &=& \tr A_E\\
      \Omg_{\det E}&=& \tr\Omg_E
     \end{array}
  \right.
$.只需注意到$A_{\det E}$与$\Omg_{\det_E}$都是一阶矩阵,无非是普通的微分形式;
验证曲率时注意$\tr(A_E\wedge A_E)=0$.

%as for tensor product,we define $D_{E\ten F}$ as follows:
%$$D_{E\ten F}(s\ten t)=D_Es\ten t+s\ten D_Ft$$
%check the curvature $$H_{E\ten F}=H_E\ten id_F+id_E\ten H_F$$
%\begin{rem}we can also consider wedge product of vector bundles.
%Consider vector bundles $E_1,...,E_k$,with connections $D_{E_1},...,D_{E_k}$,
%let $s_i\in C_{p_i}^\infty(X,E^i)$ then
%$$D_{E_1\wedge,...,\wedge E_k}(s_1\wedge...\wedge s_k)
%=\sum_{i=1}^k(-1)^{p_1+...+p_{i-1}}s_1
%\wedge...\wedge D_{E_i}s_i\wedge...\wedge s_k$$\end{rem}
%Let $E$ be a vector bundle of rank $r$,then $\wedgeform{r}E$ is a line bundle,
%with transition matrix by $\det(g_{\alpha\beta})$.
%this bundle is denoted by $\det E$.(Det-bundle)
%Let $s_1,...,s_r$ be local sections of $E$,then we have
%$$D_{\det E}(s_1\wedge\cdots\wedge s_r)= tr(H_E)s_1\wedge\cdots\wedge s_r$$

\begin{prop}($\Hom$丛的联络与曲率)
\label{Hom丛的联络与曲率-prop}
设$E,F\to X$为$X$上的光滑向量丛,$\tD_E,\tD_F$分别为$E,F$上的联络,
则$\Hom$丛$\Hom(E,F)\cong E^*\ten F$上自然有
(由张量丛、对偶丛诱导的)联络$\tD_{\Hom(E,F)}$,
相应的曲率记为$\Theta_{\Hom(E,F)}\in\Omg^2(X,\End(\Hom(E,F)))$.
则对任意$f\in\Omg^p(X,\Hom(E,F))$以及$u\in\Omg^q(X,E)$,成立
\begin{eqnarray*}
  \tD_F\pair{f}{u}
&=&
  \pair{\tD_{\Hom(E,F)}f}{u}
 +(-1)^p\pair{f}{\tD_Eu}
\\
  \Theta_F\pair{f}{u}
&=&
  \pair{\Theta_{\Hom(E,F)}f}{u}
 +\pair{f}{\Theta_Eu}
\end{eqnarray*}
\end{prop}

\begin{proof}
  不妨$f=u^*\ten v$,其中$u^*\in\Omg^p(X,E^*),\,v\in\Omg^0(X,F)$.
为方便书写,不妨省略$\tD,\Theta$的下标,这不会产生歧义。
则由对偶丛与张量丛的联络运算法则,易知
\begin{eqnarray*}
     \tD\pair{f}{u}
&=&
     \tD(\pair{u^*}{u}\ten v)\\
&=&
        \left(
          \pair{\tD u^*}{u}
         +(-1)^p\pair{u^*}{\tD u}
        \right)\ten v
    +(-1)^{p+q}\pair{u^*}{u}\ten\tD v\\
&=&
     \pair{\tD u^*\ten v}{u}
    +(-1)^p\pair{u^*\ten v}{\tD u}
    +(-1)^{p+q}(-1)^q\pair{u^*\ten\tD v}{u}\\
&=&
     \pair{\tD(u^*\ten v)}{u}
    +(-1)^p\pair{u^*\ten v}{\tD u}\\
&=&
     \pair{\tD f}{u}
    +(-1)^p\pair{f}{\tD u}
\end{eqnarray*}
利用上式,容易得到$\tD_{\Hom(E,F)}$的曲率的表达式:
\begin{eqnarray*}
      \Theta\pair{f}{u}
&=&
     \tD(\tD\pair{f}{u})
 =
     \tD(\pair{\tD f}{u}+(-1)^p\pair{f}{\tD u})\\
&=&
     \pair{\tD^2 f}{u}
    +(-1)^{p+1}\pair{\tD f}{\tD u}
    +(-1)^{p  }\pair{\tD f}{\tD u}
    +(-1)^{p+p}\pair{f}{\tD^2 u}\\
&=&
     \pair{\Theta f}{u}+\pair{f}{\Theta u}
\end{eqnarray*}
\end{proof}

\begin{prop}(Bianchi恒等式)
\label{Bianchi恒等式-prop}

设$E\to X$为光滑向量丛,$\tD_E$为$E$上的一个联络,其曲率为
$\Theta\in\Omg^2(X,\End(E))$.考虑$\End(E)\cong E^*\ten E$
上的由$\tD_E$诱导的联络$\tD_{\End{E}}$,则有
$$\tD_{\End(E)}\Theta=0\in\Omg^3(X,\End(E))$$
\end{prop}

\begin{proof}对任意$u\in\Omg^p(X,E)$,
由性质\ref{Hom丛的联络与曲率-prop}可知,
$$
  \tD\pair{\Theta}{u}
= \pair{\tD\Theta}{u}
  +(-1)^2\pair{\Theta}{\tD u}
$$
因此有
\begin{eqnarray*}
     \pair{\tD\theta}{u}
&=&
     \tD\pair{\Theta}{u}-\pair{\Theta}{\tD u}\\
&=&
     \tD(\tD^2u)-\tD^2(\tD u)
 =
     0
\end{eqnarray*}
从而由$u$的任意性,得证。
\end{proof}

\section{陈省身示性类}
%chern classes (defined by curvature).
%Let $E\to X$ be a smooth complex vector bundle of rank $r$,
%where $X$ be a complex manifold.(Chern-Weil theory)
%$V$ be a complex vector space, $f:\underbrace{V\times\cdots\times V}_k\to \bbC$
%be a symmetric multi-linear form of degree $k$.
%$\rightsquigarrow f(v):=f(v,v,...,v)$ is a homogeneous polynomial of degree $k$.
%\begin{definition}assume $G$ is a group (left) acting on $V$, s.t.
%$$f(g(v_1),...,g(v_k))=f(v_1,...,v_k)$$
%for any $g\in G,v_i\in V$, then we say $f$ is $G$-invariant.
%\end{definition}Special case: $G=GL(r,\bbC)$ and $V=Lie
%G=\mfkgl{r,\bbC}$ be the Lie algebra of $G$.the action is
%$$(g,M)\mapsto gMg^{-1}$$

对于$r$阶方阵$M$,考虑$r$次多项式
$$
  f(t):=\det(I+tM)
=
  I+f_1(M)t+f_2(M)t^2+\cdots+f_r(M)t^r
$$
由线性代数不难知道,函数$f_k\,(1\leq k\leq r)$在矩阵相似变换下不变,
即对任意的$r$阶可逆矩阵$P$,有
$f_k(P^{-1}MP)=f_k(M)$.此外众所周知,$f_1(M)=\tr M$以及
$f_r(M)=\det M$.

此外,由代数学,$f_k$唯一决定了一个$k$重对称线性泛函
(仍记为$f_k$)$f_k:\Sym^k\mfkgl(r,\bbK)\to\bbK$,
使得对任意矩阵$M\in\mfkgl(r,\bbK)$,成立
$f_k(M)=f_k(\underbrace{M,M,...,M}_{k})$,
并且每个$f_k\in\Sym^k\mfkgl(r,\bbK)$都是$\GL(r,\bbK)$-不变的,
即对任意$r$阶方阵$M_1,M_2,...,M_k$以及$r$阶可逆矩阵$P$,成立
$$
  f_k(P^{-1}M_1P,P^{-1}M_2P,...,P^{-1}M_kP)
= f_k(M_1,M_2,...,M_k)
\eqno{(*)}
$$
对于任意$r$阶矩阵$M\in\mfkgl(r,\bbK)$,
若令$P=e^{tM}$代入$(*)$式,并求$t=0$处的导数,容易得到
$$
  \sum_{j=1}^{k}
    f_k(M_1,...,[M,M_j],...,M_k)
=0\eqno{(**)}
$$

\begin{lemma}\label{Hom丛联络-矩阵表示-lem}
设$E\to X$为光滑向量丛,$\tD_E$为$E$上的一个联络,
取$e=(e_1,e_2,...,e_r)$为$E$的一个局部标架,$\tD_E$在此标架下的系数矩阵为$A$,
则对于任意$\fai\in C^k(X,\End(E))$,$\fai$在此局部标架下
可表示为系数为$k$-形式的$r$阶矩阵。
则在此意义下,$\tD_E$诱导的$\End(E)$上的联络$\tD_{\End(E)}$满足
$$
  \tD\fai=\td\fai+[A,\fai]
$$
其中$[A,\fai]=A\wedge\fai-(-1)^p\fai\wedge A$为分次对易子。
\end{lemma}

\begin{proof}局部标架下直接计算即可。注意$\fai=\fai_i^je^i\ten e_j$,
其中矩阵元$\fai^j_i$为$p$-形式。则有
\begin{eqnarray*}
     \tD(\fai^j_i e^i\ten e_j)
&=&
     \td\fai_i^j e^i\ten e_j
    +(-1)^p
     \left(
       \fai^j_i\wedge(-A_k^i)e^k\ten e_j
      +\fai_i^j e^i\ten A_j^k e_k
     \right)\\
&=&
     \left(
       \td\fai_i^j
      +A_k^j\wedge\fai_i^k
      -(-1)^k\fai_k^j\wedge A_i^k
     \right)e^i\ten e_j\\
&=&
     \left(
       \td\fai+[A,\fai]
     \right)_i^j
     e^i\ten e_j
\end{eqnarray*}
\end{proof}

\begin{prop}(陈省身示性类)
\index{Chern class\kong 陈类}
\label{陈类的定义-prop}

设$E\to X$为秩为$r$的光滑向量丛,$\tD$为$E$的一个联络,
$\Theta$为$\tD$的曲率。设$e=(e_1,e_2,...,e_n)$为$E$的一个局部标架,
$\Theta$在该标架下的矩阵为$\Omg$.记
$$
  c_k(E;\tD)
:=
  f_k(\frac{\sqrt{-1}}{2\pi}\Omg),
\quad
  (1\leq k\leq r)
$$
则事实上$c_k(E;\tD)$为$X$上整体定义的$2k$-形式,并且$\td c_k(E;\tD)=0$,
从而$c_k(E;\tD)\in H^{2k}_{\DR}(X)$.
称$c_k(E;\tD)$为向量丛$E$关于联络$\tD$的第$k$阶\textbf{陈省身示性类},
简称\textbf{陈类}(Chern class).
\end{prop}

\begin{proof}
若$\etil=(\etil_1,\etil_2,...,\etil_r)$为$E$的另一组局部标架,
且有转移矩阵$\etil=eG$,则我们已证$\Omgtil=G^{-1}\Omg G$,
从而$f_k(\frac{\ii}{2\pi}\Omgtil)=f_k(\frac{\ii}{2\pi}\Omg)$,
从而$c_k(E,\tD)\in\Omg^{2k}(X)$整体定义。

再证明所有的$c_k(E,\tD)$都是闭形式。
注意到在某局部标架下,
$$
  c_k(E,\tD)=f_k(\frac{\ii}{2\pi}\Omg)
= \left(\frac{\ii}{2\pi}\right)^k
  f_k(\Omg,\Omg,...,\Omg)
$$
其中$\Omg=\td A+A\wedge A$为曲率$\Theta$在局部标架下的矩阵。
直接外微分,有
\begin{eqnarray*}
     \td c_k(E,\tD)
&=&
     \left(\frac{\ii}{2\pi}\right)^k
     \td f_k(\Omg,\Omg,...,\Omg)
 =
     \left(\frac{\ii}{2\pi}\right)^k
     \sum_{j=1}^{k}
       f_k(\Omg,...,\underbrace{\td\Omg}_{\text{第$j$个}},...,\Omg)\\
&=&
     \left(\frac{\ii}{2\pi}\right)^k
     \sum_{j=1}^{k}
       f_k(\Omg,...,\underbrace{(\tD\Omg-[A,\Omg])}
       _{\text{第$j$个}},...,\Omg)\\
&=&
     -\left(\frac{\ii}{2\pi}\right)^k
     \sum_{j=1}^{k}
       f_k(\Omg,...,\underbrace{[A,\Omg]}
       _{\text{第$j$个}},...,\Omg)
 =0
\end{eqnarray*}
其中注意到Bianch恒等式$\tD\Omg=0$(见性质\ref{Bianchi恒等式-prop})、
引理\ref{Hom丛联络-矩阵表示-lem}以及该引理上方的$(**)$式。
\end{proof}

%Consider$$\det(I+\frac{i}{2\pi}tm)=I+tf_1(M)+t^2f_2(M)+\cdots t^rf_r(M)$$
%$\rightsquigarrow\forall 1\leq k\leq r$, $f_k$ is $G$-invariant.
%Let $E\to X$ complex vector bundle on a complex manifold,
%let $D_E$ be a connection,curvature $H_E\in C_2^\infty(X,\Hom(E,E))$.
% Let $f\in GL(r,\bbC)$- invariant "$k$-form",then
%(1)Let $H_{\alpha},H_{\beta}$ be the curvature forms of $E$ in different trivialization,
%then $f(H_\alpha)=f(H_{\beta})$ , so we get a globally defined $2k$-form.
%assume $H_\alpha=gH_\beta g^{-1}$, then
%$$f(H_\alpha)=f(g H_\beta g^{-1})=f(H_\beta)$$(2) we also have$$\td f(H)=0$$
%locally , $H=H_\alpha=\td a_\alpha+A_\alpha\wedge A_\alpha$,then
%$$\td f(H)=\td f(H_\alpha,H_\alpha,...,H_\alpha)
%=\sum_{i=1}^kf(H_\alpha,...,\underbrace{\td H_\alpha}_{i},...,\H_\alpha)$$
%$$=\sum_{i=1}^kf(H_\alpha,...,\td A_\alpha\wedge A_\alpha-A_\alpha\wedge\td
%A_\alpha,...,H_\alpha)$$Fact:(in Riemannian geometry) For any $x\in X$,
%we always can find a local frame s.t.$A_\alpha(x)=0$.so, choose this frame,
%$$\td f(H)=0$$So, $[f(H)]\in H^{2k}(X,\bbC)$

\begin{lemma}(联络的形变)
\label{联络的形变-lemma}

设$E\to X$为光滑向量丛,$\tD$为$E$的一个联络,
则对于任意$\afa\in\Omg^1(X,\End(E))$,$\tD+\afa$
也是$E$的一个联络,并且其曲率为
$$
  \Theta_{\tD+\afa}
=
  \Theta_{\tD}+\tD\afa+\afa\wedge\afa
$$
其中“$\tD\afa$”当中的“$\tD$”为$\End(E)$上的联络。
\end{lemma}

\begin{proof}显然$\tD+\afa$也是联络,并且在局部标架下的系数矩阵为$A+\afa$,
其中$A$为$\tD$的系数矩阵,$\afa$也用来表示相应的系数矩阵。
我们来验证曲率,有
\begin{eqnarray*}
     \Omg_{\tD+\afa}
&=&
     \td(A+\afa)+(A+\afa)\wedge(A+\afa)\\
&=&
     \td A+\td\afa+A\wedge A+A\wedge\afa+\afa\wedge A+\afa\wedge\afa\\
&=&
     \Omg_{\tD}+\td\afa+[A,\afa]+\afa\wedge\afa
\end{eqnarray*}
注意到引理\ref{Hom丛联络-矩阵表示-lem},从而可知
$
  \Theta_{\tD+\afa}
=
  \Theta_{\tD}+\tD\afa+\afa\wedge\afa
$,证毕。
\end{proof}

\begin{thm}(陈类与向量丛的联络选取无关)

设$E\to X$为秩为$r$的光滑向量丛,则对任意$1\leq k\leq r$,
上同调类$[c_k(E,\tD)\in H^{2k}(X)]$与$E$的联络$\tD$选取无关。
\end{thm}

\begin{proof}
设$\tD_0$与$\tD_1$为丛$E$上的两个联络,对于$t\in[0,1]$,记
$$\tD_t:=(1-t)\tD_0+t\tD_1$$
则$\tD_t$显然也是联络。令$\afa:=\tD_1-\tD_0$,则容易验证
$\afa\in\Omg^1(X,\End(E))$为整体定义的张量。
再记$\afa_t:=\tD_t-\tD_0\in\Omg^1(X,\End(E))$,则$\afa_t=t\afa$.
记联络$\tD_t$的曲率为$\Theta_t$,则由引理\ref{联络的形变-lemma}可知
\begin{eqnarray*}
     \Theta_t
&=&
     \Theta_0+t\tD_0\afa+t^2\afa\wedge\afa
\\
     \tdd{t}\Theta_t
&=&
     \tD_0\afa+2t\afa\wedge\afa\\
&=&
     \tD_t\afa-(\tD_t-\tD_0)\afa+2t\afa\wedge\afa\\
&=&
     \tD_t\afa-t[\afa,\afa]+2t\afa\wedge\afa\\
&=&
     \tD_t\afa
\end{eqnarray*}
因此有
\begin{eqnarray*}
& &
     c_k(E,\tD_1)-c_k(E,\tD_0)
 =
     \int_{0}^{1}
       \tdd{t}c_k(E,\tD_t)\td t
 =
     \left(
       \frac{\ii}{2\pi}
     \right)^k
     \int_{0}^{1}
       \tdd{t}f_k(\Theta_t,\Theta_t,...,\Theta_t)
       \td t\\
&=&
     k\left(
       \frac{\ii}{2\pi}
     \right)^k
     \int_{0}^{1}
       f_k(\tdfrac{\Theta_t}{t},\Theta_t,...,\Theta_t)
       \td t
 =
     k\left(
       \frac{\ii}{2\pi}
     \right)^k
     \int_{0}^{1}
       f_k(\tD_t\afa,\Theta_t,...,\Theta_t)
       \td t\\
&=&
     k\left(
       \frac{\ii}{2\pi}
     \right)^k
     \td\left(
       \int_{0}^{1}
         f_k(\afa,\Theta_t,...,\Theta_t)
         \td t
     \right)
\end{eqnarray*}
最后一步利用了Bianchi恒等式$\tD_t\Theta_t=0$以及
$f_k$的$\mfkgl(r,\bbK)$-不变性(与性质\ref{陈类的定义-prop}的证明过程类似)。
因此$[c_k(E,\tD_1)]=[c_k(E,\tD_0)]$,
即位于同一个上同调类。
\end{proof}

%(3) Claim : the class $[f(H)]$ is independent of the choice of the connections $D_E$.
%Let $D_0,D_1$ be two connections, consider$$D_t=(1-t)D_0+tD_1$$
%$t\in[0,1]$, curvature $H_t$%%%%%%%%将以上所有的H都换成\Theta%%%%%%%%%
%Fact: $\alpha:=A_1-A_0$ is globally defined, and in $C_1^\infty(X,\Hom(E,E))$.
%Fact: $$\frac{\td}{\td t}f(H_t)=k\td f(\alpha,H_t,H_t,...,H_t)$$
%So,$$f(H_1)-f(H_0)=\int_0^1\frac{\td}{\td t}f(H_t)\td t
%=\td\int_0^1f(\alpha,H_t,H_t,...,H_t)\td t$$So,$$[f(H_1)]-[f(H_0)]$$
%\begin{definition}the $k$-th Chern class of $E$
%$$c_k(E):=[f_k(\Theta_E)]\in H^{2k}(X,\bbC)$$\end{definition}
%%%%%%%%%%%2019.4.04第六周星期四;清明节前最后一节课%%%%%%%%%%%%%%%
%Recall: Chern Class$X$ complex manifold, $E\to X$
%is a smooth complex vector bundle of rank $r$.
%$D$ is a connection, curvature $\Theta(D)\in C_2^\infty(X,\Hom(E,E))$.
%linear algebra:$$\det(I+\frac{i}{2\pi}tM)=I+tf_1(M)+t^2f_2(M)+\cdots+t^rf_r(M)$$
%Chern class $\{f_k(\Theta)\}\in H^{2k}_{DR}(X,\bbC)$
%is independent of choice of connection.Today:

\begin{rem}
对于向量丛$E\to X$,既然陈类$[c_k(E,\tD)]$与联络选取无关,
我们可记$[c_k(E)]:=[c_k(E,\tD)]$,其中$\tD$为$E$上的任何一个联络。
\end{rem}
在黎曼几何中,我们知道黎曼度量总存在,并且黎曼度量诱导的Levi-Civita联络非平凡(即不恒为$0$).
类似地,后文将会在向量丛上引入某种度量,该度量诱导了某个非平凡的联络。
也就是我,向量丛上的(不恒为零的)联络总存在。

\begin{example}(线丛的第一陈类)

设$E\to X$为光滑线丛,即$\rank_{\bbK}E=1$.设$\tD$为$E$上的联络,
$A$为$\tD$在某局部标架下的系数矩阵(无非是局部的$1$-形式),此时曲率形式
$\Omg=\td A+A\wedge A=\td A$.
容易验证第一陈类为
$$c_1(E,\tD)=\frac{\ii}{2\pi}\Theta$$
在此特殊情况下,若$\tD_0$与$\tD_1$为线丛$E$的两个联络,则容易验证
$$
  c_1(E,\tD_1)-c_1(E,\tD_0)
=
  \frac{\ii}{2\pi}\td(\tD_1-\tD_0)
$$
其中$\tD_1-\tD_0\in\Omg^1(X)$为整体定义的$1$形式。特别地,
$[c_1(E,\tD_1)]=[c_1(E,\tD_0)]\in H^2(X)$
\end{example}

%Special case: $E$ is a complex line bundle.
%Let $D_0$ be a connection on $E$, locally $D_0e=A_0e$, $A_0$ is $1$-form.
%curvature$$\Theta(D_0)=D_0^2=\td A_0+A_0\wedge A_0=\td A_0$$
%so, curvature is $\td$-exact, so $\td\Theta(D_0)=0$.
%$$\det(I+\frac{i}{2\pi}tM)=I+\frac{i}{2\pi}tM$$
%so, the first Chern class of line bundle is
%$$c_1(E)=\{\frac{i}{2\pi}\Theta(D_0)\}$$
%Let $D_1$ be another connection, locally $D_1e=A_1e$, so
%$\Theta(D_1)=\td A_1$.so,$$\Theta(D_1)-\Theta(D_0)=\td(A_1-A_0)$$
%where$$A_1-A_0\in C_1^\infty(X,\Hom(E,E))$$
%(when $E$ is line bundle ,$\Hom(E,E)\cong E^*\ten E$ is trivial bundle)
%so, $A_1-A_0$ is a globally defined smooth function on $X$. So,
%$$\{\Theta(D_1)\}=\{\Theta(D_0)\}\in H^2(X,\bbC)$$
%independent of the choice of connection.

\section{Hermite向量丛}
本节开始考虑复向量丛。

\begin{definition}(Hermite 向量丛)
\index{Hermite 向量丛}

设$E\to X$为光滑流形$X$上的复向量丛,
称$E$为Hermite向量丛,若任意$x\in X$,
纤维$E_x$上配以Hermite内积结构$h\in\Omg^0(X,\Herm(E,E))$:
\begin{eqnarray*}
  h_x:E_x\times E_x&\to&\bbC\\
  (s_x,t_x)&\mapsto&\bair{s_x}{t_x}
\end{eqnarray*}
并且对于任意局部标架$e(x)=(e_1(x),e_2(x),..,e_r(x))$,
$h_{ij}(x):=\bair{e_i(x)}{e_j(x)}$是光滑的。
此时$h$称为$E$的\textbf{Hermite度量}。
\end{definition}

%\begin{definition}a complex vector bundle
%$E\to X$ of rank $r$ is called a Hermitian vector bundle, if
%we have an inner product on $E$, i.e. locally, consider a local frame
%$\{e_1,...,e_r\}$, we have$$\{e_i(x),e_j(x)\}=h_{ij}(x)$$
%s.t. $(h_{ij}(x))$ is a positive definite Hermitian matrix depending smoothly on $x$.
%\end{definition}\begin{rem}For any complex vector bundle,
%Hermitian structure always exists.\end{rem}证明与黎曼几何类似。(黎曼度量的存在性)

由定义知,在局部标架$e=(e_1,e_2,...,e_r)$下,
矩阵$h=(h_{ij})$为正定Hermite矩阵。我们假定$\bair{}{}$关于第一个位置共轭线性。
我们用符号“$\dag$”来表示矩阵的共轭转置,则$h^\dag=h$.
显然此定义是良定的,若$\etil$为另一组局部标架,并有转移矩阵
$G$使得$\etil=eG$,则有
$$
  \htil_{ij}=\bair{\etil_i}{\etil_j}
= \bair{G_i^ke_k}{G_j^le_l}
= \Gbar_i^kG_j^l\bair{e_k}{e_l}
= \Gbar_i^kh_{kl}G^l_j
= (G^\dag hG)_{ij}
$$
即$\htil=G^\dag hG$.又由于转移矩阵$G$是光滑的,
从而$h$光滑当且仅当$\htil$光滑。

\textbf{任何复向量丛都存在Hermite度量},
原因与黎曼几何当中的“黎曼度量存在性”完全类似。
此外,Hermite度量$h$可自然地线性延拓为
\begin{eqnarray*}
  h:\Omg^p(X,E)\times\Omg^q(X,E)
  &\to&\Omg^{p+q}(X)
\end{eqnarray*}
满足共轭超对称性:对任意$u\in\Omg^p(X,E)$以及$v\in\Omg^q(X,E)$,
$\bair{u}{v}=(-1)^{pq}\overline{\bair{v}{u}}$等等性质。

\begin{definition}(Hermite联络)
\label{Hermite联络-def}

设复向量丛$(E,h)\to X$为Hermite向量丛,$\tD$为$E$上的联络。
称$\tD$与Hermite度量$h$相容,或称$\tD$为\textbf{Hermite 联络},
若对任意$u\in\Omg^p(X,E)$以及$v\in\Omg^q(X,E)$,都有
$$
  \td\bair{u}{v}
=
  \bair{\tD u}{v}+(-1)^p\bair{u}{\tD v}
$$
\end{definition}
特别地,在局部标架$e=(e_1,e_2,...,e_r)$下,成立
$$
  \td h_{ij}=\bair{\tD e_i}{e_j}+\bair{e_i}{\tD e_j}
            =\Abar^k_ih_{kj}+A^k_jh_{ik}
            =(A^\dag h+hA)_{ij}
$$
其中$A$为$\tD$在该标架下的系数。从而有
$$\td h=A^\dag h+hA\eqno(*)$$

%\begin{definition}(Hermitian connection)
%connection compatible with Hermitian metric
%A connection $D$ on $E$ is called Hermitian, if
%$$\td\{e_i,e_j\}=\{De_i,e_j\}+\{e_i,De_j\}$$\end{definition}
%More generally, let $t\in C_p^{\infty}(X,E)$, $s\in C_q^\infty(X,Y)$,
%$$\td\{s,t\}=\{\td t,s\}+(-1)^p\{t,Ds\}$$

\begin{prop}(Hermite联络的曲率)
\label{Hermite 联络的曲率-prop}

设$E\to X$为复向量丛,$h$为$E$的一个Hermite结构,
$\tD$为$E$上的与$h$相容的联络,
$e=(e_1,e_2,...,e_r)$为$E$的一组局部\textbf{幺正标架}
(即$\bair{e_i}{e_j}=\delta_{ij}$),
则$\tD$的曲率$\Theta$在标架$e$下的矩阵$\Omg$满足
$$\Omg^\dag=-\Omg$$
\end{prop}

\begin{proof}只需注意到
\begin{eqnarray*}
     0
&=&
     \td^2\bair{e_i}{e_j}
 =
     \td\left(
       \bair{\tD e_i}{e_j}
      +\bair{e_i}{\tD e_j}
     \right)\\
&=&
     \bair{\tD^2e_i}{e_j}
    -\bair{\tD e_i}{\tD e_j}
    +\bair{\tD e_i}{\tD e_j}
    +\bair{e_i}{\tD^2e_j}
 =
     \Omgbar^j_i+\Omg^i_j
\end{eqnarray*}
因此$\Omg^\dag=-\Omg$.
\end{proof}

容易验证,在一般的标架(未必幺正)下,Hermite联络的曲率的系数矩阵满足
$$\Omg^\dag h+h\Omg=0$$

%\begin{prop}$D$ is a Hermitian connection ,then the curvature
%$$\Theta(D)^*=-\Theta(D)$$(where $(-)^*$ is conjugate transpose of matrix)
%\end{prop}it means that, $i\Theta(D)\in C_2^\infty(X,\text{Herm}(E,E))$
%\begin{proof}$$0=\td^2\{e_i,e_j\}=\td\{De_i,e_j\}+\td\{e_i,De_j\}$$
%$$=\{D^2e_i,e_j\}-\{De_i,De_j\}+\{De_i,De_j\}+\{e_i,D^2e_j\}=
%\{(\Theta+\Theta^*)e_i,e_j\}$$\end{proof}\begin{rem}
%$E$ is a Hermitian line bundle, $D$ is a Hermitian connection,
%then $i\Theta(D)$ is a real $2$-form ,$c_1(E)\in H^2(X,\bbR)$.\end{rem}

\begin{rem}
此性质表明,Hermite联络$\tD$的曲率$\Theta$满足:
$\ii\Theta\in\Omg^2(X,\Herm(E,E))$.特别地,$\ii\Theta$的系数矩阵$\ii\Omg$
的对角元都是实的$2$-形式,从而第一陈类
$$[c_1(E)]=\frac{1}{2\pi}[\tr(\ii\Theta)]\in H^2(X,\bbR)$$
\end{rem}

\section{复流形上的联络}
现在,我们考虑\textbf{复流形}$X$上的光滑向量丛。
对于复流形,注意到微分形式空间的分解
$$\Omg^{m}(X)=\bigoplus_{p+q=m}\Omg^{p,q}(X)$$
现在,若$E\to X$为复流形$E$上的光滑向量丛,我们类似定义
$$\Omg^{p,q}(X,E)=\Gma(X,\wedgeform{p,q}(T^*M)\ten E)$$
为取值于$E$上的$(p,q)$形式空间。在复流形上,外微分算子$\td=\td'+\td''$,
其中$\td':=\p,\td'':=\pbar$.

\begin{definition}($(1,0)$型联络)

对于复流形$X$上的光滑向量丛$E$,称算子
$\tD':\Omg^{\bullet,\bullet}(X,E)\to\Omg^{\bullet+1,\bullet}(X,E)$
为$(1,0)$型联络,如果对任意$u\in\Omg^{p,q}(X)$
以及$v\in\Omg^{p',q'}(X,E)$,成立
$$
  \tD'(u\wedge v)
= \td'u\wedge v+(-1)^{p+q}u\wedge\tD' v
$$
\end{definition}

注意“$(1,0)$型联络”虽然字面上有“联络”,
但它一般不是向量丛的联络(从定义能看出)。
类似地,我们也可以定义$(0,1)$型联络
$\tD'':\Omg^{\bullet,\bullet}(X,E)\to\Omg^{\bullet,\bullet+1}(X,E)$,
使得对任意$u\in\Omg^{p,q}(X)$
以及$v\in\Omg^{p',q'}(X,E)$,成立
$$
  \tD''(u\wedge v)
= \td''u\wedge v+(-1)^{p+q}u\wedge\tD'' v
$$
%\begin{definition}Let $X$ be a complex manifold.
%$D'$ is called a connection of type $(1,0)$ on $E$,
%if for any section $s\in C_{p,q}^\infty(X,E)$, we have
%$D's\in C_{p+1,q}^\infty(X,E)$.A connection $D''$ is called
%a connection of type $(0,1)$, if ...$D''s\in
%C_{p,q+1}^\infty(X,E)$.\end{definition}

\begin{rem}对于复流形上的光滑向量丛$E\to X$,
若$\tD'$与$\tD''$分别为$E$上的$(1,0),(0,1)$型联络,
则易验证$\tD:=\tD'+\tD''$必为$E$上的联络。
反之,$E$上的任何联络$\tD$都可自然地分解为
$(1,0)$部分与$(0,1)$部分之和:$\tD=\tD'+\tD''$.
\end{rem}

设$\tD'$为$(1,0)$型联络,则在$E$的局部(光滑)标架下,
类似定义$\tD'$的系数矩阵$A'$,注意$A'$的每个矩阵元都是$(1,0)$形式;
而$(0,1)$型联络的系数矩阵也类似。类似地,也有局部标架下的表达式$\tD'=\td'+A'$
以及$\tD''=\td''+A''$.

%\begin{rem}Let $E\to X$ be avector bundle.Let $D$ be a
%connection on $E$, locally$$Ds\xra{\sim}\td\sgm+
%A\wedge\sgm$$$$\td\sgm=\p\sgm+\pbar\sgm$$
%so,let $A'$ be the $(1,0)$-part of $A$,...,$$Ds
%=\p\sgm+A'\wedge\sgm+(\pbar\sgm+A''\wedge\sgm)=:D's+D''s$$\end{rem}

\begin{prop}
设$(E,h)\to X$为复流形$X$上的光滑Herimite向量丛,
$\tD$为$E$上的与$h$相容的Hermite联络,则对于
$E$的光滑\textbf{幺正标架}$e=(e_1,e_2,...,e_n)$,
联络$\tD$的系数矩阵$A$的$(1,0),(0,1)$部分$A',A''$满足
$$A''=-(A')^\dag$$
\end{prop}

\begin{proof}直接验证即可,只需注意
$$
  0=\tD\bair{e_i}{e_j}
   =\tD'\bair{e_i}{e_j}+\tD''\bair{e_i}{e_j}
   =\left((\Abar')^j_i+(A'')^i_j\right)
   +\left((A')^i_j+(\Abar'')^j_i\right)
$$
注意其$(1,0)$分量与$(0,1)$分量,从而$A'=-(A'')^\dag$.
\end{proof}

%\begin{prop}
%$E$:Hermitian vector bundle, $D$ is a Hermitian connection, locally,
%take a $C^\infty$-frame $e_1,...,e_r$ which is orthonomal
%(i.e. $\{e_i(x),e_j(x)\}=\delta_{ij}$), then the  connection coefficient
%$A=A'+A''$ satisfies$$(A')^*=-A''$$($\iff \overline(iA)=iA$ )\end{prop}
%\begin{proof}because$$0=\td{e_i,e_j}=\{De_i,e_j\}+\{e_i,De_j\}
%=\{a_i^ke_k,e_j\}+\{e_i,a_j^le_l\}=a^j_i+\overline{a^i_j}$$
%so, $A^*=-A$.\end{proof}

事实上,联络$\tD$在幺正标架下的系数矩阵满足$(A'')^\dag=-A'$
当且仅当$\tD$为Hermite联络。由此不难得出:

\begin{cor}\label{Hermite 联络的分解-cor}
设$E\to X$为复流形$X$上的光滑Hermite向量丛,则对于$E$上的任何$(0,1)$型
联络$\tD''$,存在唯一的Hermite联络$\tD$,使得$\tD$的$(0,1)$部分恰为$\tD''$.
\end{cor}

%\begin{cor}$E\to X$ is a Hermitian vector bundle,
%$D_0''$ is a connection of type $(0,1)$ on $E$.
%Then exists a unique Hermitian connection $D$
%such that $D''=D_0''$.\end{cor}\begin{proof}Let $A''=A_0''$ and $A'
%=-(A_0'')^*\rightsquigarrow A=A'+A''$, and$D$ is given by $A$.\end{proof}

\begin{proof}
  任取$E$的光滑幺正标架$e=(e_1,e_2,...,e_r)$,令
$\tD''$在标架$e$下的系数矩阵为$A''$.令$A:=-(A'')^\dag+A''$,
则矩阵$A$确定了一个Hermite联络$\tD$,使得$\tD$在该标架下的矩阵为$A$.
易验证$\tD$的良定性。
\end{proof}

现在,假设$E\to X$为全纯向量丛。

\begin{Example}(全纯向量丛上的典范$(0,1)$型联络)

设$E\to X$为全纯向量丛,
则微分算子$\td''=\pbar$自然给出了$E$上的一个
$(0,1)$型联络,称为\textbf{典范$(0,1)$}型联络。
\end{Example}

任取$E$的\textbf{全纯标架}$e=(e_1,e_2...,e_r)$,
则对于任意的$u\in\Omg^{p,q}(X,E)$,局部上有
$u=u^\afa\ten e_\afa$,其中$u^\afa\in\Omg^{p,q}(X)$,则令
$$\pbar u:=(\pbar u^\afa)\ten e_\afa$$
特别地,对于全纯标架场$e_\afa$,有$\pbar e_\afa=0$.
容易验证上述定义的良定性,从而$\pbar$诱导了$E$
上的一个$(0,1)$型联络。

%Let $E\to X$ is a holomorphic Hermitian vector bundle,
%observe that $\pbar$ defines a connection of type $(0,1)$ on $E$(check!)
%assume $E$ is a holomorphic line bundle, take a section
%$s\in C_p^\infty(X,E)$, i.e. we have a family of $p$-forms $(s_\afa)$
%such that $s_\afa=g_{\afa\beta}s_\beta$
%where $g_{\afa,\beta}$ is the holomorphic transition matrix.
%$$\pbar s\xra{\sim}\pbar s_\beta$$then
%$$\pbar s_\afa=g_{\afa,\beta}\pbar s_\beta$$
%(so, $\pbar$ is a connection of $(0,1)$)
%this connection is called the canonical connection of type $(0,1)$.

\begin{definition}(全纯Hermite向量丛的陈联络)
\index{Chern connection\kong 陈联络}

设$E\to X$为\textbf{全纯Hermite向量丛}
(即它为全纯向量丛且配以Hermite结构$h$,),
则存在唯一的Hermite联络$\tD$,使得其$(0,1)$分量$\tD''=\pbar$.
称此联络为全纯Hermite向量丛$(E,h)\to X$的\textbf{陈联络}(Chern connection)。
\end{definition}

显然,陈联络的存在唯一性是推论\ref{Hermite 联络的分解-cor}的推论。
全纯Hermite丛的陈联络就像Riemann流形上的Levi-Civita联络一样典范。
另外,注意全纯Hermite向量丛的Hermite结构$h$
在全纯标架下的矩阵\textbf{未必}是全纯的,仅仅光滑即可,
并不存在Hermite结构与全纯结构的“相容性”。

%\begin{definition}Let $E\to X$ holomorphic Hermitian vector bundle,
%the connection $D$ on $E$ is called Chern connection if$$D''=\pbar$$\end{definition}
%\textbf{Curvature of Chern connection}

\begin{lemma}
设$(E,h)\to X$为全纯Hermite向量丛,$\tD=\tD'+\tD''$为其陈联络,
任取$E$的全纯标架$e=(e_1,e_2,...,e_r)$,
将$h$在该标架下的矩阵也记为$h$,
则陈联络$\tD$在该标架下的系数矩阵$A$及其$(1,0),(0,1)$分量$A',A''$满足
$$
  A=A'=h^{-1}\p h
\qquad
  A''=0
$$
\end{lemma}

\begin{proof}
由于$\tD''=\pbar$,从而系数矩阵$A''=0$是显然的,也有$A'=A$.
注意$\tD$首先是Hermite联络,
从而定义\ref{Hermite联络-def}下方的$(*)$式成立,即
$$
  hA+A^\dag h=\td h=\p h+\pbar h
$$
比较$(1,0)$分量,得到$hA=\p h$,因此$A=h^{-1}\p h$.
\end{proof}

{\color{blue}
注意在我这里,Hermite内积$\bair{}{}$\textbf{关于第一个位置}是共轭线性的,
即$\bair{\lmd u}{v}=\overline{\lmd}\bair{u}{v}$.
而在其它版本中,若假定$\bair{}{}$关于第二个位置共轭线性,
则陈联络在全纯标架下的系数矩阵为
$A=\hbarr^{-1}\p\hbarr$.
}

\begin{lemma}
设$(E,h)\to X$为全纯Hermite向量丛,$\tD=\tD'+\tD''$为其陈联络,
则有$(\tD')^2=(\tD'')^2=0$,特别地,$\tD$的曲率
$$\Theta=\tD^2=(\tD'+\tD'')^2=\tD'\tD''+\tD''\tD'$$
\end{lemma}

\begin{proof}
注意$\tD''=\pbar$,从而显然$(\tD'')^2=\pbar^2=0$.
至于$(\tD')^2=0$,我们在全纯局部标架下验证。
对于全纯标架$e=(e_1,e_2...,e_r)$,设$\fai\in\Omg^{0,0}(X,E)$
为$E$的光滑截面,$\fai:=\fai^\afa e_\afa$,
我们将$\fai$在标架$e$下的坐标函数仍记为$\fai$
(这里的$\fai$是$r$维列向量,每个分量都为光滑函数),则有
$$\tD'\fai=\p\fai+A'\fai=\p\fai+h^{-1}\p h\cdot \fai$$
注意到$\p^2=0$,以及$\p(h^{-1})=-h^{-1}\p h\cdot h^{-1}$,因此有
\begin{eqnarray*}
     (\tD')^2\fai
&=&
     \tD'(\p\fai+h^{-1}\p h\cdot\fai)\\
&=&
     \p(\p\fai+h^{-1}\p h\cdot\fai)
    +h^{-1}\p h\cdot(\p\fai+h^{-1}\p h\cdot\fai)\\
&=&
     \p(h^{-1}\p h\cdot \fai)
    +h^{-1}\p h\cdot\p\fai
    +h^{-1}\p h\cdot h^{-1}\p h\cdot\fai\\
&=&
    -h^{-1}\p h\cdot h^{-1}\p h\cdot\fai
    -h^{-1}\p h\cdot\p\fai
    +h^{-1}\p h\cdot\p\fai
    +h^{-1}\p h\cdot h^{-1}\p h\cdot\fai\\
&=&
     0
\end{eqnarray*}
\end{proof}

\begin{prop}(陈联络的曲率)

设$(E,h)\to X$为全纯Hermite向量丛,$\tD$为其陈联络,则
$\tD$的曲率$\Theta$在全纯标架$e=(e_1,e_2,...,e_r)$下的系数矩阵为
$$
  \Omg=\pbar(h^{-1}\p h)
$$
\end{prop}

\begin{proof}
记号同之前,则
\begin{eqnarray*}
     \Omg&=&\td A+A\wedge A
 =
     (\p+\pbar)(h^{-1}\p h)+h^{-1}\p h\cdot h^{-1}\p h
 =
     \pbar(h^{-1}\p h)
\end{eqnarray*}
\end{proof}

%$E\to X$ is holomorphic Hermite vector bundle , $D$ is the Chern connection,
%Locally let $\{e_1,...,e_r\}$ be a holomorphic frame, and two local sections
%$$s,t\in C^\infty(\Omg, E)$$where$$s=\sum_{i=1}^r\sgm_ie_i$$
%$$t=\sum_{i=1}^r t_ie_i$$Since $D$ is Hermitian ,$$\td \{s,t\}=\td
%((\sgm_1,...,\sgm_r)H\begin{pmatrix}t_1\\\vdots\\t_r\end{pmatrix})
%=(\td\sgm)^THt+\sgm^T(\td H)t+\sgm^TH\td(t)$$so, we have
%$$\{Ds,t\}+\{s,Dt\}=(\td\sgm+\overline{H}^{-1}\p\overline{H}
%\wedge\sgm)^T\wedge H\overline{t}+\sgm^T\wedge H\overline{
%(\td t+\overline{H}^{-1}\p\overline{H}\wedge t)}$$
%so ,$$Ds=\td\sgm+\overline{H}^{-1}\p\overline{H}\wedge\sgm$$
%$$D's=\p\sgm+\overline{H}^{-1}\p\overline{H}\wedge\sgm
%=\overline{H}^{-1}\p(\overline{H}\sgm)$$$$D''s=\pbar\sgm$$so,
%$$(D')^2s=\overline{H}^{-1}\p(\overline{H}(\overline{H}^{-1}\p(\overline{H}\sgm)))
%=\cdots=0$$$$(D'')^2s=\cdots=0$$So we have
%$$\Theta(D)=(D'+D'')^2=D'D''+D''D'$$Locally ,
%$$\Theta s=D'D''s+D''D's=\overline{H}^{-1}\p(\overline{H}\pbar\sgm)
%+\pbar(\overline{H}^{-1}\pbar(\overline{H}\sgm))
%=\cdots=\overline{H}^{-1}\p\overline{H}\wedge\pbar\sgm
%+\pbar(\overline{H}^{-1})\sgm$$$$= \pbar(\overline{H}^{-1}\p\overline{H})\sgm
%$$So, Chern curvature$$\Theta_D=\pbar(\overline{H}^{-1}\p\overline{H})$$
%Nexttime: Hodge theory...%%%%%%%%%%%%%%%%%%%%%%%%%%%%%%%%%%%%%%%%%
%%%%%%%%%%%%%2019.4.9周二第七周%%%%%%%%%%%%%%%%%
%Last time: $E\to X$ is a holomorphic vector bundlewith a Hermitian metric $H$.
%Then there is a unique connection $D_E$s.t. ... called Chern connection.
%Curvature of Chern Connection:$$\Theta(D_E)=\pbar(\overline{H}^{-1}\p\overline{H})$$

\begin{rem}注意陈联络首先是Hermite联络,
回顾性质\ref{Hermite 联络的曲率-prop},陈联络的曲率满足
$\ii\Theta\in\Omg^{2}(X,\Herm(E,E))$.
而由$\Omg=\pbar(h^{-1}\p h)$不难看出,对于陈联络的曲率$\Theta$,
$$\ii\Theta\in\Omg^{1,1}(X,\Herm(E,E))$$
\end{rem}

%so,$$i\Theta(D_E)\in C_{1,1}^\infty(X,\Hom(E,E))$$

\begin{example}(全纯线丛的情形)

特别地,若$(E,h)\to X$为全纯Hermite线丛,$\tD$为其陈联络,
则在全纯标架$e$下,曲率形式$\Omg$满足
$$\Omg=\pbar(h^{-1}\p h)=\pbar\p\log h=-\p\pbar\log h$$
\end{example}
若在该标架下,$h=e^{-2\fai}$,则曲率形式
$$
  \Omg=-\p\pbar\log h=2\p\pbar\fai
= 2\pmfrac{\fai}{z^i}{\zbar^j}\td z^i\wedge\td\zbar^j
$$

%\begin{example}(Special case: $E$ is  a holomorphic line bundle)
%locally, let $e$ be ha holomorphic frame,
%$\langle e,e\rangle=h$ is the metric.
%then,$$\Theta=\pbar(h^{-1}\p h)=\pbar\p\log h$$
%so,$$i\Theta(E)=-i\p\pbar\log h$$\end{example}
%if $h=e^{-2\fai}$ where $\fai$ is a smooth function, then
%$$i\Theta(E)=2i\p\pbar\fai=2\sqrt{-1}\sum_{k,l}
%\pmfrac{\fai}{z_k}{\overline{z_l}}\td z_k\wedge\td\overline{z_l}$$
%\textbf{Question}: let $s$ be a local holomorphic section of $E$,
%$$-i\p\pbar\log|s|_h^2=?$$(Hint:$\frac{i}{\pi}\p\pbar\log z=?$单复变,
%按分布意义下求导 .等于狄拉克测度2333333){\color{red}可能是期末题目?}

\begin{prop}
\label{用全纯截面计算Hermite线丛的陈曲率-prop}
设$(E,h)\to X$为全纯Hermite线丛,
设$s$为$E$的任何一个(局部)非零的\textbf{全纯截面},则成立
$$-\ii\p\pbar\log\norm{s}^2_h=\sqrt{-1}\Theta$$
其中$\Theta$为$(E,h)$陈联络的曲率形式。
\end{prop}

\begin{proof}任取$E$的全纯标架$e$,在全纯标架下验证。
把$s$在该标架下的坐标函数仍记为$s$.则注意$s$为全纯标架,
从而坐标函数$s$满足$\pbar s=\p\sbar=0$,因此有
\begin{eqnarray*}
  -\ii\p\pbar\log\norm{s}^2_h
&=&
  -\ii\p\pbar\log(s\sbar h)
=
  -\ii\p\pbar[\log(s\sbar)+\log h]\\
&=&
  -\ii\p\pbar\log h
=
  \ii\Theta
\end{eqnarray*}
\end{proof}

\begin{cor}($\p\pbar$-引理)

设$(E,h)\to X$为复流形$X$上的全纯线丛,
记$\Theta$为$h$的陈联络的曲率形式。则对任意实值光滑函数
$f\in\Omg^0(X,\bbR)$,必存在$E$上的Hermite度量$\t=\htil$,
使得其陈联络的曲率形式$\Thetatil$满足
$$\Thetatil=\Theta+2\pi\p\pbar f$$
\end{cor}
\begin{proof}
取$\htil=e^{-2\pi f}h$即可。
\end{proof}

%Exercise: $E$ is a holomorphic line bundle, denote
%$\theta:=\frac{i}{2\pi}\Theta(D_E)$ real $(1,1)-form$,
%where $D_E$ is Chern connection
%with a metric $h$.Prove: for any smooth function $f\in C^\infty(X,\bbR)$,
%there exists a Hermitian metric $h_f$ s.t.
%$$\frac{i}{2\pi}\Theta_{E,h_f}=\theta+i\p\pbar f$$

\begin{thm}(第一陈类的代数几何解释)

设$X$为复流形,考虑$X$上的层正合列
$0\to\bbZ\inj\mcalO\xra{e^{2\pi i*}}\mcalO^*\to 0$
诱导的上同调长正合列
$$
  \cdots\to
  H^1(X,\mcalO)\to H^1(X,\mcalO^*)
  \xra{\delta}H^2(X,\bbZ)\to H^2(X,\mcalO)\to\cdots
$$
则对任意全纯线丛$E\in H^1(X,\mcalO^*)$,
$\delta(E)\in H^2(X,\bbZ)\inj H^2(X,\bbR)\cong H^2_{\DR}(X,\bbR)$满足
$$\delta(E)=[c_1(E)]$$
\end{thm}

\begin{proof}
{\color{red}见Demailly page.265(待补)}
\end{proof}

%Consider the sequence$$0\to\bbZ\to\mcalO\xra{e^{2\pi i *}}
%\mcalO^*\to 0$$it induces a long exact sequence
%prove: Consider $E$ as an element of $H^1(X,\mcalO^*)$, then the
%image of $\delta(E)$ in $H^2(X,\bbR)\cong H^2_{DR}(X,\bbR)$ is $c_1(E)$.

\section{例子:复射影空间上的典范线丛$\mcalO(-1)$}

设$V$为复线性空间,$\dim_{\bbC}V=n+1$,则我们已经熟悉
复流形$\bbP(V)\cong\bbC\bbP^n$以及其上的典范线丛
$\mcalO(-1):=\Bigset{([\xi],\eta)\in\bbP(V)\times V}{\eta\in[\xi]}$.
回顾$\bbP(V)$具有典范的坐标覆盖$U_i:=\Bigset{[z_0;z_1;...;z_n]}{z_i\neq 0}$,
这也是线丛$\mcalO(-1)$的局部平凡化覆盖。

\begin{Example}
设$V$为Hermite内积空间(酉空间),则线丛$\mcalO(-1)\to h$
上自然配以Hermite内积结构$h$,
该Hermite内积由$V$上的Hermite内积自然诱导。
取线性空间$V$的一组幺正基$(e_0,e_1,...,e_n)$,
该基下的坐标函数记为$(z_0,z_1,...,z_n)^T$.
则在$\bbP(V)$的局部坐标卡
$U_0:=\Bigset{[z_0;z_1;...;z_n]}{z_0\neq 0}\cong\bbC^n$下,
$$\mcalE_0:=e_0+\sum_{k=1}^{n}z_ke_k$$
为$\mcalO(-1)$在$U_0$下的一个非零全纯截面,即全纯标架。
\end{Example}

关于截面$\mcalE_0$,有一点轻微的记号混用。
线丛的全纯截面在局部平凡化坐标卡下,应该是$\mcalU_0\cong\bbC^n$
上的全纯函数。在此意义下,$\mcalE_0(z_1,z_2,...,z_n)\equiv 1$,
是“最简单”的非零全纯函数。

注意Hermite度量,容易知道在局部坐标
$U_0=(z_1,z_2,...,z_n)$以及全纯标架$\mcalE_0$下有
$$
  h=\norm{\mcalE_0}^2_h=1+|z_1|^2+|z_2|^2+\cdots+|z_n|^2
$$
因此陈联络的曲率形式$\Theta(\mcalO(-1))$为
$$
  \Theta(\mcalO(-1))=-\p\pbar\log h
=
  -\p\pbar\log(1+|z_1|^2+|z_2|^2+\cdots+|z_n|^2)
$$

%Assume $V$ has a Hermitian inner product, then $\mcalO(-1)$ has an
%Hermitian structure induced from $V$.
%Let $e_0,...,e_n$ be an orthonormal basis of $V$, then
%$\mcalO(-1)|_{\Omg_0}$ has a non-vanishing holomorphic section:
%$$\mcalE_0(z_1,...,z_n)=e_0+z_1e_1+...+z_ne_n$$
%where$$\Omg_0=\{[1;z_1;...;z_n]|z_j\in\bbC\}\cong\bbC^n$$
%then,$$|\mcalE_0|^2_h=1+|z_1|^2+...+|z_n|^2$$
%so the Chern curvature of $\mcalO(-1)$ on $\Omg_0$ is given by
%$$\Theta=\pbar\p\log(1+|z_1|^2+\cdots+|z_n|^2)$$

\begin{example}(超平面丛$\mcalO(1)$)
\label{复射影空间上的超平面丛-example}

记号同上,考虑$\mcalO(-1)$的对偶丛$\mcalO(1):=\mcalO(-1)^*$,
则$\mcalO(1)$的曲率形式在坐标卡$U_0$下的表达式为
$$
  \Theta(\mcalO(1))=-\Theta(\mcalO(-1))
=
  \p\pbar\log(1+|z_1|^2+|z_2|^2+\cdots+|z_n|^2)
$$
\end{example}

{\color{blue}
记$f(z_1,z_2,...,z_n):=\log(1+|z_1|^2+|z_2|^2+\cdots+|z_n|^2)$,
则矩阵$\left(\pmfrac{f}{z_k}{\zbar_l}\right)_{1\leq k,l\leq n}$
是\textbf{正定}Hermite矩阵(为什么?)。
}

%Denote $\mcalO(1):=\mcalO(-1)^*$, then
%$$\Theta(\mcalO(1))=-\pbar\p\log(1+|z_1|^2+...+|z_n|^2)$$
%on $\Omg_0$.$$i\Theta(\mcalO(1))=i\p\pbar\log(1+|z_0|^2+...+|z_n|^2)
%=\sqrt{-1}\sum_{1\leq k,l\leq n}c_{k,l}\td z_k\wedge\td\overline{z_l}$$
%Exercise: $(c_{kl})$ is a positive definite Hermitian matrix.
%"Fubini-Study metric" on $\bbP(V)$.$\mcalO(1)$ is
%"hyperplane line bundle of $\bbP(V)$".

\begin{thm}
对于$n+1$维复向量空间$V$,则$\bbP(V)$上的线丛$\mcalO(1)$
的第一陈形式$c_1(\mcalO(1))\in\Omg^{1,1}(\bbP(V))$满足
$$
  \int_{\bbP(V)}
    [c_1(\mcalO(1))]^{\wedge n}
=1
$$
\end{thm}

%Exercise: calculate$$\int_{\bbP(V)}\left(
%\frac{i}{2\pi}\Theta(\mcalO(1))\right)^{\wedge n}=?$$
%(Hint: $\bbP(V)\setminus\Omg_0$ is a zero-measure set)
%$E\to X$ : holomorphic line bundle, $D_E$ is a Chern connection.
%$$c_1(E)=\{\frac{i}{2\pi}\Theta(D_E)\}\in H_{DR}^2(X,\bbR)$$

注意$c_1(\mcalO(1))=\frac{\ii}{2\pi}\Theta$,
其中$\Theta$为上文当中的陈联络的曲率形式。
事实上,陈类的定义当中的系数“$\frac{\ii}{2\pi}$”
存在的意义正是要让此定理成立(归一化)。

\begin{proof}
考虑$\bbP(V)$的坐标卡$U_0$,注意$\bbP(V)\setminus U_0$
是零测的,从而只需计算它在$U_0$上的积分。

{\color{red}(待补,似乎要暴力计算?)}
\end{proof}




