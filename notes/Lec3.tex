\chapter{Hermite向量丛}
\section{向量丛的联络与曲率}

先回顾一下光滑向量丛的联络(活动标架版本),这是黎曼几何的标准内容。
我们考察光滑的实向量丛或者复向量丛。为表述方便,令$\bbK:=\bbR$或$\bbC$.

\begin{notation}(向量值微分形式)

设$X$为光滑流形,$E\to X$为$X$上的光滑$\bbK$-向量丛
,记$$\Omg^p(X,E):=\Gma(X,(\wedgeform{p}T^*M)\ten E)$$
为$X$上的取值于$E$的光滑$p$-形式空间。
\end{notation}
%For $X$ be a smooth manifold, $E$ is a vector bundle(real or complex), denote
%is the space of $k$-differential $p$-forms with values in $E$.

我们将$\Omg\updot(X,E):=\bigoplus\limits_{p\geq 0}\Omg^p(X,E)$
自然视为分次线性空间,使得该空间的$p$次齐次子空间为$\Omg^p(X,E)$.
注意到$X$上的微分$p$-形式空间$\Omg^p(X)\cong\Omg^p(X,\bbK)$,
此处的$\bbK$为$X$上的平凡线丛。
注意$\Omg\updot(X)$上的外积结构$\wedge$,它自然诱导了
$$
  \wedge:
  \Omg^p(X)\times\Omg^q(X,E)
  \to\Omg^{p+q}(X,E)
$$
事实上这给出了$\Omg\updot(X,E)$的一个分次$\Omg\updot(X)$-模结构。

局部地,取向量丛$E$的一个局部平凡化
$$\theta_{\alpha}:E|_{U_{\alpha}}\cong U_{\alpha}\times \bbK^{r}$$
记$\{e_1,...,e_r\}$为$U_\afa$上的一组局部标架,则
$\Omg^p(X,E)$中的元素$s$在此局部标架下形如
$$s=\sum_{\lmd=1}^r\fai^{\lmd}\ten e_{\lmd}
=:\fai^\lmd\ten e_\lmd$$
其中每个$\fai_\lmd$均为光滑$p$-形式,并采用Einstein求和约定。

%Locally, consider a trivialization of $E$,($\rightsquigarrow$ frame $(e_1,...e_r)$)
%$$s\in \sum\fai_{\lmd}(x)\ten e_{\lmd}(x)$$where $\fai_{\lmd}$ is a $p$-form.

\begin{definition}(向量丛上的联络)
\index{connection\kong 联络}

设$E\to X$为光滑流形$X$上的光滑向量丛,
丛$E$上的\textbf{联络}(connection)
是指作用在分次向量空间$\Omg\updot(X,E)$上的
次数为$1$的齐次线性映射$\tD:\Omg\updot(X,E)
\to\Omg^{\bullet+1}(X,E)$,
并且满足:
$$
  \tD(\fai\wedge s)
=
  \td\fai\wedge s
 +(-1)^p\fai\wedge\tD s
$$
对任意$\fai\in\Omg^p(X)$以及$s\in\Omg^q(X,E)$成立。
\end{definition}

%a (linear) connection on $E$ is a linear differential operator 
%of order $1$ acting on$C^{\infty}\downdot(X,E)$:
%$$D:C_{p}^{\infty}(X,E)\to C_{p+1}^{\infty}(X,E)$$
%$$D(f\wedge x):= \td f\wedge s+(-1)^p f\wedge Ds$$
%where $f\in C^{\infty}(X,\wedgeform{p}T^*M)$, $s\in C^{\infty}(X,E)$.

注意定义中并没要求$\tD^2=0$,一般地$(\Omg\updot(X,E),\tD)$
并不是上链复形。易知向量丛$E\to X$,
$E$上的联络之全体,构成$\bbK$-线性空间.
联络的典型例子是,考虑$X$上的平凡线丛$\bbK$,则
$\Omg\updot(X,\bbK)\cong\Omg\updot(X)$上的外微分$\td$
即为丛$\bbK$上的联络。\vs 

取$E\to X$的局部平凡化坐标卡$U$及其局部标架$\{e_1,e_2,...,e_r\}$,
则对任意$t\in\Omg^p(X,E)$,其中$r:=\rank E$.若在该局部标架下
$t=\fai^\lmd\ten e_\lmd$,则有
$$
  \tD t=
    \td\fai^\lmd\ten e_\lmd
   +(-1)^p\fai^\lmd\wedge\tD e_\lmd
$$
其中$\tD e_\lmd\in\Omg^1(U,E)$.
可见只要确定了$\tD$在$e_\lmd\in\Omg^0(X,E)$上的作用,
则联络$\tD$被唯一确定。
我们令$\tD e_\lmd=a_\lmd^\mu\ten e_\mu$,其中$a_\lmd^\mu\in\Omg^1(U)$,
称为$\tD$关于局部标架$\{e_1,e_2,...,e_r\}$的\textbf{联络$1$-形式},
称$r\times r$矩阵$A:=(a^\mu_\lmd)$为
$\tD$关于局部标架$\{e_1,e_2,...,e_r\}$的\textbf{系数矩阵}。
若记$e:=(e_1,e_2,...,e_r)$为标架排成的行向量,
则有紧凑的表达式$\tD e=eA$.

在局部标架$\{e_1,e_2,...,e_r\}$下,$\Omg^p(X,E)$中的元素$t$
可以用以$p$-形式为分量的列向量$\fai:=(\fai^\lmd)_{1\leq\lmd\leq r}$来表示,
即$t=\fai^\lmd\ten e_\lmd$.在此意义下,有
\begin{eqnarray*}
     \tD t
&=&
     \td\fai^\lmd\ten e_\lmd
    +(-1)^p\fai^\lmd\wedge\tD e_\lmd
 =
     \td\fai^\lmd\ten e_\lmd
    +(-1)^p\fai^\lmd\wedge a_\lmd^\mu\ten e_\mu\\
&=&
     \left(
       \td\fai^\mu+a_\lmd^\mu\wedge\fai^\lmd
     \right)\ten e_\mu
 =
     (\td\fai+A\wedge\fai)^\mu\ten e_\mu
\end{eqnarray*}
或者简记为
$$\tD\fai=\td\fai+A\wedge\fai$$

%Locally, consider a local trivialization
%$$\theta:E|_{\Omg}\xra{\sim}\Omg\times\bbK^r$$
%with a frame $\{e_1,...,e_r\}$. any section
%$t\in C^{\infty}_p(\Omg,E)$ can be written as
%$$t=\sum_{1\leq\lmd\leq r}\sgm_{\lmd}\ten e_{\lmd}$$
%$$Ds=\sum_{\lmd=1}^r\td\sgm_{\lmd}\wedge e_{\lmd}+(-1)^p
%\sgm_{\lmd}\wedge De_{\lmd}$$ where$$De_{\lmd}\in C_1^{\infty}(\Omg, E)$$
%can be written as$$De_{\lmd}=\sum_{\mu=1}^r
%a_{\mu\lmd}\ten e_{\mu}$$where "$a_{\mu \lmd}$" is called the coefficients of $D$
%with respect to frame $\{e_1,...,e_r\}$ .so,$$D(t)=\sum_{\lmd,\mu}\td\sgm_{\lmd}\wedge 
%e_{\lmd}+(-1)^p\sgm_{\lmd}\wedge a_{\mu\lmd}\wedge e_{\mu}
%=\sum_{\mu}\sum_{\lmd}\left(\td\sgm_{\mu}+ a_{\mu\lmd}\wedge\sgm_{\lmd}
%\right)$$%平凡化下,用矩阵再写一下,自己脑补%$$Dt=\td\sgm+A\wedge\sgm$$where $A=(a_{\mu\lmd})$.
%RMK: connection always exists!

%%%%%%%2019.4.2第六周周二%%%%%%%%%%%%%%

%Recall: for any (connected) smooth manifold,
%$E\to X$ is a smooth vector bundle,Connection:
%$$D:C^\infty_p(X,E)\to C_{p+1}^\infty(X,E)$$
%where $C^\infty_p(X,E):=C^\infty(X,\wedge^pT^*M\ten E)$
%$$D(f\wedge s)=\td f\wedge s+(-1)^{\deg f}f\wedge Ds$$
%Essentially,$$D:C^\infty(X,E)\to C_1^\infty(X,E)$$
%Locally, consider a trivialization
%$\theta:E|_\Omg\xra{\sim}\Omg\times \bbK^r$, and a local frame
%$(e_1,...,e_r)$ where $e_k(x)=\theta^{-1}(x,\begin{pmatrix}
%0\\\vdots\\1 (k^{th})\\\vdots\\0\end{pmatrix})$.
%Let $s\in C^\infty(\Omg,E)$, i.e.$$s=\sum_{i=1}^r\sgm_ie_i$$
%where $\sgm_i$ are smooth functions.$$Ds=\td\sgm+A\wedge\sgm$$where
%$$\sgm=\begin{pmatrix}\sgm_1\\\vdots\\\sgm_r\end{pmatrix}
%\quad A={a_{ij}}$$

\begin{prop}(联络矩阵的变换)

设$E\to X$为光滑向量丛,$\tD$为$E$上的一个联络。设
$(U,e)$与$(\Util,\etil)$为丛$E$的两个局部平凡化,
其中$e=(e_1,e_2...,e_r),\,\etil=(\etil_1,\etil_2,...,\etil_r)$
为相应的局部标架。记它们之间的转移函数为
$\etil_\lmd=g_\lmd^\mu e_\mu$,$G:=(g_\lmd^\mu)$.
若$A,\Atil$分别为联络$\tD$关于
标架$e,\etil$的系数矩阵,则有变换关系
$$
  \Atil
=
  G^{-1}AG+G^{-1}\td G
$$
\end{prop}

\begin{proof}
将$\etil_\lmd=g_\lmd^\mu e_\mu$写为紧凑的矩阵形式,
有$\etil=eG$.从而我们有
\begin{eqnarray*}
\tD\etil&=&\tD(eG)=(\tD e)G+e\td G=eAG+e\td G\\
\tD\etil&=&\etil\Atil=eG\Atil
\end{eqnarray*}
因此整理得$\Atil=G^{-1}AG+D^{-1}\td G$.
\end{proof}

%consider another trivialization
%$$\tilde\theta:E|_\Omg\xra{\sim}\Omg\times\bbK^r$$
%$\rightsquigarrow$ a local frame $(\tilde{e_1},...,\tilde{e_r})$.
%Then there exists a invertible linear transform s.t.
%$$\tilde{e_k}=g_k^me_m$$ assume
%$$De_k=a_k^le_l\qquadD\tilde{e_k}=\tilde{a}_k^l\tilde{e}_l$$
%we have$$\td g_k^ne_n+g_k^ma_m^ne_n=\tilde{a}_k^lg_l^ne_n$$
%$$\rightsquigarrow\quad\tilde{a}_k^lg_l^n(g^{-1})^p_n
%=\td g_k^n(g^{-1})^p_n+g_k^ma_m^n(g^{-1})^p_n$$
%$$\rightsquigarrow\quad\tilde{a}_l^p=\td g_k^n(g^{-1})^p_n)
%+g_k^ma^n_m(g^{-1})^p_n$$$$\rightsquigarrow\quad
%\tilde{A}=\td g\cdot g^{-1}+g\cdot A\cdot g^{-1}$$
%\textbf{Curvature}$$H_D:=D^2$$%H要加一个圈。。。

\begin{definition}(曲率)
\index{curvature\kong 曲率}

设$E\to X$为光滑流形$X$上的光滑$\bbK$-向量丛,
$\tD$为$E$上的一个联络,则记
$$\Theta:=\tD\circ\tD$$
为联络$\tD$的\textbf{曲率}(curvature).
\end{definition}

在局部标架$e=(e_1,e_2,...,e_k)$下,对于$t\in\Omg^p(X,E)$,
若$t=e\fai$,其中$\fai$为分量为$p$-形式的列向量,
利用局部标架下的联络公式$\tD\fai=\td\fai+A\wedge\fai$,
其中$A$为联络$\tD$在关于此标架的系数矩阵,可得
\begin{eqnarray*}
     \Theta\fai
&=&
     \tD(\td\fai+A\wedge\fai)\\
&=&
     \td(\td\fai+A\wedge\fai)
    +A\wedge(\td\fai+A\wedge\fai)\\
&=&
     \td^2\fai+\td A\wedge\fai
    -A\wedge\td\fai+A\wedge\td\fai+A\wedge A\wedge\fai\\
&=&
     (\td A+A\wedge A)\wedge\fai
\end{eqnarray*}
即在局部标架$e$下,曲率算子$\Theta$的矩阵$\Omg=\td A+A\wedge A$,
此矩阵的矩阵元为光滑$2$-形式,称为\textbf{曲率形式}。

%locally,$$D^2s=D(\td\sgm+A\wedge\sgm)=
%\td(\td\sgm+A\wedge\sgm)+A\wedge(\td\sgm+A\wedge\sgm)$$
%$$=\td A\wedge\sgm-A\wedge\td\sgm+A\wedge\td\sgm+A\wedge A\wedge\sgm
%=(\td A+A\wedge A)\wedge\sgm$$so we have$$H=\td A+A\wedge A$$

\begin{prop}(曲率形式在不同局部平凡化下的变化)

设$E\to X$为光滑向量丛,$\tD$为$E$上的联络,
$\Theta$为联络$\tD$的曲率。
设$e=(e_1,e_2,...,e_r)$与$\etil=(\etil_1,\etil_2,...,\etil_r)$
为$E$的两组局部标架,转移矩阵$G$满足$\etil=eG$.
则曲率$\Theta$在标架$e,\etil$下的矩阵$\Omg,\Omgtil$满足
$$
  \Omgtil=G^{-1}\Omg G
$$
\end{prop}

\begin{proof}
我们已有$\Atil=G^{-1}AG+G^{-1}\td G$,从而
$G\Atil=AG+\td G$,两边外微分得到
$$
  \td G\wedge\Atil+G\td\Atil
= \td A\cdot G-A\wedge\td G
$$
因此有
\begin{eqnarray*}
     \td\Atil
&=&
     G^{-1}\td A\cdot G
    -G^{-1}\td G\wedge\Atil
    -G^{-1}A\wedge\td G\\
&=&
     G^{-1}\td A\cdot G
    -G^{-1}\td G\wedge(G^{-1}AG+G^{-1}\td G)
    -G^{-1}A\wedge\td G\\
&=&
     G^{-1}\td A\cdot G
    -G^{-1}\td G\cdot G^{-1}AG
    -G^{-1}\td G\cdot G^{-1}\td G
    -G^{-1}A\wedge\td G
\end{eqnarray*}
\begin{eqnarray*}
     \Atil\wedge\Atil
&=&
     (G^{-1}AG+G^{-1}\td G)
     \wedge
     (G^{-1}AG+G^{-1}\td G)\\
&=&
     G^{-1}A\wedge AG
    +G^{-1}A\wedge\td G
    +G^{-1}\td G\cdot G^{-1}AG
    +G^{-1}\td G\cdot G^{-1}\td G
\end{eqnarray*}
从而得到
$$
  \Omgtil=\td\Atil+\Atil\wedge\Atil
= G^{-1}(\td A+A\wedge A)G=G^{-1}\Omg G
$$
\end{proof}

\begin{rem}
此定理表明,曲率$\Theta$是丛$E$上的$2$-形式值$(1,1)$型张量,
故称为\textbf{曲率张量}。具体地,
$$\Theta\in\Omg^2(X,\Hom(E,E))$$
\end{rem}

%Similarly to $\tilde{A},A$ we haveExercise:
%$$\tilde{H}=gHg^{-1}$$曲率在不同平凡化下的表达式。where
%$$\tilde{e}=ge$$$\rightsquigarrow H$ can be 
%considered as a section of $C^\infty_2(X,\Hom(E,E))$.
%because$$\tilde{H}\tilde{e}=gHg^{-1}\tilde{e}=gHe$$
%independent of the choice of local frames.

\begin{Example}(对偶丛的联络与曲率)

设$E\to X$为光滑向量丛,$E^*$为$E$的对偶丛。
注意到有如下自然的配对:
\begin{eqnarray*}
     \pair{}{}:
     \Omg^p(X,E^*)\times\Omg^q(X,E)
     &\to&\Omg^{p+q}(X)
\\
     \pair{\fai_\lmd\ten e^\lmd}{\psi^\mu\ten e_\mu}
&:=&
     \pair{e^{\lmd}}{e_\mu}
     \fai_\lmd\wedge\psi^\mu     
\end{eqnarray*}
若$\tD_E$为$E$上的联络,则$\tD_E$诱导了对偶丛$E^*$上的联络$\tD_{E^*}$,
使得对任意$s\in\Omg^p(X,E^*)$以及$t\in\Omg^q(X,E)$都成立
$$
  \td\pair{s}{t}
=
  \pair{\tD_{E^*}s}{t}
 +(-1)^p\pair{s}{\tD_E t}
$$
\end{Example}

取$E$的局部标架$e=(e_1,e_2,...,e_r)$,记$e^*:=(e_1^*,e_2^*,...,e_r^*)$
为其对偶标架(也排成行向量),记对偶丛联络$\tD^*:=\tD_{E^*}$
的曲率为$\Theta^*$,它们在对偶标架$e^*$上的矩阵记作$A^*,\Omg^*$,则成立:
$
  \left\{
    \begin{array}{l}
      A^*=-A^T\\
      \Omg^*=-\Omg^T
    \end{array}
  \right.
$.这是因为,由于$\pair{e^{*T}}{e}=I$,从而
$$
  0=\td\pair{e^{*T}}{e}
=\pair{\tD^* e^{*T}}{e}+\pair{e^{*T}}{\tD e}
=\pair{(e^*A^*)^T}{e}+\pair{e^{*T}}{eA}
=A^{*T}+A
$$
因此对偶联络的系数矩阵$A^*=-A^T$.从而对偶联络的曲率矩阵
$$
  \Omg^*=\td A^*+A^*\wedge A^*
        =-\td A^T+A^T\wedge A^T
        =-(\td A+A\wedge A)^T
        =-\Omg^T
$$

这里要特别注意微分形式矩阵的运算,注意$A$的矩阵元为微分$1$-形式,
从而易验证有$(A\wedge A)^T=-A^T\wedge A^T$.

%fact: let $D_E$ be a connection on $E$,then it induces a connection $D_{E^*}$.
%Let $u$ be a local section of $E^*$, $s$ local section of $E$,
%then we define$$\td\langle u,s\rangle=\langle D_{E^*}u,s\rangle
%+\langle u,D_{E}s\rangle$$Exercise:$$H(D_{E^*})=-H(D_E)^T$$

\begin{Example}(直和丛的联络与曲率)

设$E,F\to X$均为$X$上的光滑向量丛,$\tD_E,\tD_F$
分别为$E,F$上的联络,则直和丛$E\oplus F$上自然有联络
$\tD_{E\oplus F}$,使得对任意$u\in \Omg^p(X,E)$
以及$v\in\Omg^p(X,F)$,成立
$$
  \tD_{E\oplus F}(u\oplus v)
=
  \tD_E u\oplus\tD_Fv
$$
\end{Example}
取定$E$的局部标架$e=(e_1,e_2,...,e_r)$
以及$F$的局部标架$f=(f_1,f_2,...,f_s)$,则$E\oplus F$
有局部标架$e\oplus f=(e_1,...,e_r;f_1,...,f_s)$.
容易验证相应的联络、曲率矩阵满足
$$
  A_{E\oplus F}
=
  \begin{pmatrix}
    A_E &  \\
        & A_F
  \end{pmatrix},
\qquad
  \Omg_{E\oplus F}
=
  \begin{pmatrix}
    \Omg_E &  \\
           & \Omg_F
  \end{pmatrix}
$$

%and for two vector bundles $E,F$, connections $D_E,D_F$, then
%$$D_{E\oplus F}:=D_E\oplus D_F$$$$H(E\oplus F)=H_E\oplus H_F$$

\begin{Example}(张量丛的联络与曲率)

设$E,F\to X$均为$X$上的光滑向量丛,$\tD_E,\tD_F$
分别为$E,F$上的联络,则张量丛$E\ten F$上自然有联络
$\tD_{E\ten F}$,使得对任意$E$的截面$u$以及$F$的截面$v$,成立
$$
  \tD_{E\ten F}(u\ten v)
=
  \tD_Eu\ten v+u\ten\tD_Fv
$$
\end{Example}

取定$E$的局部标架$e=(e_1,e_2,...,e_r)$
以及$F$的局部标架$f=(f_1,f_2,...,f_s)$,则$E\ten F$
有局部标架$e\ten f=\Bigset{e_\afa\ten f_\beta}
{1\leq\afa\leq r,\,1\leq\beta\leq s }$.
同意验证有关的联络、曲率矩阵满足
\begin{eqnarray*}
A_{E\ten F} &=& A_E\ten I_F+I_E\ten A_F\\
\Omg_{E\ten F}&=&  \Omg_E\ten I_F+I_E\ten\Omg_F
\end{eqnarray*}
在计算曲率时要当心微分$1$-形式系数的矩阵的外积运算结果的符号。

\begin{Example}(行列式丛的联络与曲率)

设$E\to X$为$X$上的秩为$r$的光滑向量丛,
$\tD_E$为$E$上的联络,则$\tD_E$自然诱导了\textbf{行列式从}
$\det E:=\wedgeform{r}E$上的联络$\tD_{\det E}$,使得
对任意局部标架$e=(e_1,e_2,...,e_r)$,
$$
  \tD_{\det E}
  e_1\wedge e_2\wedge\cdots\wedge e_r
=
  \sum_{k=1}^{r}
  e_1\wedge\cdots\wedge\tD_Ee_k\wedge\cdots\wedge e_r
$$
\end{Example}

$E$的局部标架$e=(e_1,e_2,...,e_r)$诱导了线丛$\det E$的局部标架
$e_1\wedge e_2\wedge\cdots\wedge e_r$,容易验证相应的联络、曲率
矩阵满足
$
  \left\{
    \begin{array}{lcl}
      A_{\det E} &=& \tr A_E\\
      \Omg_{\det E}&=& \tr\Omg_E
     \end{array}
  \right.
$.只需注意到$A_{\det E}$与$\Omg_{\det_E}$都是一阶矩阵,无非是普通的微分形式;
验证曲率时注意$\tr(A_E\wedge A_E)=0$.

%as for tensor product,we define $D_{E\ten F}$ as follows:
%$$D_{E\ten F}(s\ten t)=D_Es\ten t+s\ten D_Ft$$
%check the curvature $$H_{E\ten F}=H_E\ten id_F+id_E\ten H_F$$
%\begin{rem}we can also consider wedge product of vector bundles.
%Consider vector bundles $E_1,...,E_k$,with connections $D_{E_1},...,D_{E_k}$,
%let $s_i\in C_{p_i}^\infty(X,E^i)$ then
%$$D_{E_1\wedge,...,\wedge E_k}(s_1\wedge...\wedge s_k)
%=\sum_{i=1}^k(-1)^{p_1+...+p_{i-1}}s_1
%\wedge...\wedge D_{E_i}s_i\wedge...\wedge s_k$$\end{rem}
%Let $E$ be a vector bundle of rank $r$,then $\wedgeform{r}E$ is a line bundle,
%with transition matrix by $\det(g_{\alpha\beta})$.
%this bundle is denoted by $\det E$.(Det-bundle)
%Let $s_1,...,s_r$ be local sections of $E$,then we have
%$$D_{\det E}(s_1\wedge\cdots\wedge s_r)= tr(H_E)s_1\wedge\cdots\wedge s_r$$

\section{陈省身示性类}
chern classes (defined by curvature).

Let $E\to X$ be a smooth complex vector bundle of rank $r$,
where $X$ be a complex manifold.

(Chern-Weil theory)

$V$ be a complex vector space, $f:\underbrace{V\times\cdots\times V}_k\to \bbC$
be a symmetric multi-linear form of degree $k$.

$\rightsquigarrow f(v):=f(v,v,...,v)$ is a homogeneous polynomial of degree $k$.
\begin{definition}
assume $G$ is a group (left) acting on $V$, s.t.
$$f(g(v_1),...,g(v_k))=f(v_1,...,v_k)$$
for any $g\in G,v_i\in V$, then we say $f$ is $G$-invariant.
\end{definition}

Special case: $G=GL(r,\bbC)$ and $V=Lie G=\mfkgl{r,\bbC}$ be the Lie algebra of $G$.
the action is
$$(g,M)\mapsto gMg^{-1}$$

Consider
$$\det(I+\frac{i}{2\pi}tm)=I+tf_1(M)+t^2f_2(M)+\cdots t^rf_r(M)$$

$\rightsquigarrow\forall 1\leq k\leq r$, $f_k$ is $G$-invariant.

Let $E\to X$ complex vector bundle on a complex manifold,
let $D_E$ be a connection,
curvature $H_E\in C_2^\infty(X,\Hom(E,E))$.
 Let $f\in GL(r,\bbC)$- invariant "$k$-form",then

(1)Let $H_{\alpha},H_{\beta}$ be the curvature forms of $E$ in different trivialization,
then $f(H_\alpha)=f(H_{\beta})$ , so we get a globally defined $2k$-form.

assume $H_\alpha=gH_\beta g^{-1}$, then
$$f(H_\alpha)=f(g H_\beta g^{-1})=f(H_\beta)$$

(2) we also have
$$\td f(H)=0$$
locally , $H=H_\alpha=\td a_\alpha+A_\alpha\wedge A_\alpha$,then
$$\td f(H)=\td f(H_\alpha,H_\alpha,...,H_\alpha)
=\sum_{i=1}^kf(H_\alpha,...,\underbrace{\td H_\alpha}_{i},...,\H_\alpha)$$
$$
=\sum_{i=1}^kf(H_\alpha,...,\td A_\alpha\wedge A_\alpha-A_\alpha\wedge\td A_\alpha,...,H_\alpha)
$$

Fact:(in Riemannian geometry) For any $x\in X$,
we always can find a local frame s.t.
$A_\alpha(x)=0$.

so, choose this frame,
$$\td f(H)=0$$

So, $[f(H)]\in H^{2k}(X,\bbC)$

(3) Claim : the class $[f(H)]$ is independent of the choice of the connections $D_E$.

Let $D_0,D_1$ be two connections, consider
$$D_t=(1-t)D_0+tD_1$$
$t\in[0,1]$, curvature $H_t$
%%%%%%%%将以上所有的H都换成\Theta%%%%%%%%%

Fact: $\alpha:=A_1-A_0$ is globally defined, and in $C_1^\infty(X,\Hom(E,E))$.

Fact: $$\frac{\td}{\td t}f(H_t)=k\td f(\alpha,H_t,H_t,...,H_t)$$

So,
$$f(H_1)-f(H_0)=
\int_0^1\frac{\td}{\td t}f(H_t)\td t
=\td\int_0^1f(\alpha,H_t,H_t,...,H_t)\td t$$
So,
$$[f(H_1)]-[f(H_0)]$$

\begin{definition}
the $k$-th Chern class of $E$
$$c_k(E):=[f_k(\Theta_E)]\in H^{2k}(X,\bbC)$$
\end{definition}

%%%%%%%%%%2019.4.04第六周星期四;清明节前最后一节课%%%%%%%%%%%%%%%
Recall: Chern Class

$X$ complex manifold, $E\to X$ is a smooth complex vector bundle of rank $r$.
$D$ is a connection, curvature $\Theta(D)\in C_2^\infty(X,\Hom(E,E))$.

linear algebra:
$$\det(I+\frac{i}{2\pi}tM)=I+tf_1(M)+t^2f_2(M)+\cdots+t^rf_r(M)$$

Chern class $\{f_k(\Theta)\}\in H^{2k}_{DR}(X,\bbC)$ is independent of choice of connection.

Today:

Special case: $E$ is a complex line bundle.
Let $D_0$ be a connection on $E$, locally $D_0e=A_0e$, $A_0$ is $1$-form.
curvature
$$\Theta(D_0)=D_0^2=\td A_0+A_0\wedge A_0=\td A_0$$
so, curvature is $\td$-exact, so $\td\Theta(D_0)=0$.
$$\det(I+\frac{i}{2\pi}tM)=I+\frac{i}{2\pi}tM$$
so, the first Chern class of line bundle is
$$c_1(E)=\{\frac{i}{2\pi}\Theta(D_0)\}$$

Let $D_1$ be another connection, locally $D_1e=A_1e$, so
$\Theta(D_1)=\td A_1$.so,
$$\Theta(D_1)-\Theta(D_0)=\td(A_1-A_0)$$
where
$$A_1-A_0\in C_1^\infty(X,\Hom(E,E))$$
(when $E$ is line bundle ,$\Hom(E,E)\cong E^*\ten E$ is trivial bundle)

so, $A_1-A_0$ is a globally defined smooth function on $X$. So,
$$\{\Theta(D_1)\}=\{\Theta(D_0)\}\in H^2(X,\bbC)$$
independent of the choice of connection.

\section{Hermite向量丛}

\begin{definition}
a complex vector bundle $E\to X$ of rank $r$ is called a Hermitian vector bundle, if
we have an inner product on $E$, i.e. locally, consider a local frame
$\{e_1,...,e_r\}$, we have
$$\{e_i(x),e_j(x)\}=h_{ij}(x)$$
s.t. $(h_{ij}(x))$ is a positive definite Hermitian matrix depending smoothly on $x$.
\end{definition}

\begin{rem}
For any complex vector bundle, Hermitian structure always exists.
\end{rem}

证明与黎曼几何类似。(黎曼度量的存在性)

\begin{definition}(Hermitian connection)%connection compatible with Hermitian metric

A connection $D$ on $E$ is called Hermitian, if
$$\td\{e_i,e_j\}=\{De_i,e_j\}+\{e_i,De_j\}$$
\end{definition}

More generally, let $t\in C_p^{\infty}(X,E)$, $s\in C_q^\infty(X,Y)$,
$$\td\{s,t\}=\{\td t,s\}+(-1)^p\{t,Ds\}$$

\begin{prop}
$D$ is a Hermitian connection ,then the curvature
$$\Theta(D)^*=-\Theta(D)$$
(where $(-)^*$ is conjugate transpose of matrix)
\end{prop}

it means that, $i\Theta(D)\in C_2^\infty(X,\text{Herm}(E,E))$

\begin{proof}
$$0=\td^2\{e_i,e_j\}=\td\{De_i,e_j\}+\td\{e_i,De_j\}$$
$$=\{D^2e_i,e_j\}-\{De_i,De_j\}+\{De_i,De_j\}+\{e_i,D^2e_j\}=
\{(\Theta+\Theta^*)e_i,e_j\}$$
\end{proof}

\begin{rem}
$E$ is a Hermitian line bundle, $D$ is a Hermitian connection,
then $i\Theta(D)$ is a real $2$-form ,
$c_1(E)\in H^2(X,\bbR)$.
\end{rem}

(Chern connection)

\begin{definition}Let $X$ be a complex manifold.
$D'$ is called a connection of type $(1,0)$ on $E$,
if for any section $s\in C_{p,q}^\infty(X,E)$, we have
$D's\in C_{p+1,q}^\infty(X,E)$.

A connection $D''$ is called a connection of type $(0,1)$, if ...
$D''s\in C_{p,q+1}^\infty(X,E)$.
\end{definition}

\begin{rem}Let $E\to X$ be a %holomorphic  这里需要是全纯向量丛吗?
vector bundle.
Let $D$ be a connection on $E$, locally
$$Ds\xra{\sim}\td\sgm+A\wedge\sgm$$
$$\td\sgm=\p\sgm+\pbar\sgm$$
so,let $A'$ be the $(1,0)$-part of $A$,...,
$$Ds=\p\sgm+A'\wedge\sgm+(\pbar\sgm+A''\wedge\sgm)=:D's+D''s$$
\end{rem}

\begin{prop}
$E$:Hermitian vector bundle, $D$ is a Hermitian connection, locally,
take a $C^\infty$-frame $e_1,...,e_r$ which is orthonomal
(i.e. $\{e_i(x),e_j(x)\}=\delta_{ij}$), then the  connection coefficient
$A=A'+A''$ satisfies
$$(A')^*=-A''$$
($\iff \overline(iA)=iA$ )
\end{prop}

\begin{proof}
because
$$0=\td{e_i,e_j}=\{De_i,e_j\}+\{e_i,De_j\}
=\{a_i^ke_k,e_j\}+\{e_i,a_j^le_l\}
=a^j_i+\overline{a^i_j}$$
so, $A^*=-A$.
\end{proof}

\begin{cor}
$E\to X$ is a Hermitian vector bundle,
$D_0''$ is a connection of type $(0,1)$ on $E$.
Then exists a unique Hermitian connection $D$
such that $D''=D_0''$.
\end{cor}

\begin{proof}
Let $A''=A_0''$ and $A'=-(A_0'')^*\rightsquigarrow A=A'+A''$, and
$D$ is given by $A$.
\end{proof}

Let $E\to X$ is a holomorphic Hermitian vector bundle,
observe that $\pbar$ defines a connection of type $(0,1)$ on $E$(check!)

assume $E$ is a holomorphic line bundle, take a section
$s\in C_p^\infty(X,E)$, i.e. we have a family of $p$-forms $(s_\afa)$
such that $s_\afa=g_{\afa\beta}s_\beta$
where $g_{\afa,\beta}$ is the holomorphic transition matrix.
$$\pbar s\xra{\sim}\pbar s_\beta$$
then
$$\pbar s_\afa=g_{\afa,\beta}\pbar s_\beta$$

(so, $\pbar$ is a connection of $(0,1)$)

this connection is called the canonical connection of type $(0,1)$.

\begin{definition}
Let $E\to X$ holomorphic Hermitian vector bundle,
the connection $D$ on $E$ is called Chern connection if
$$D''=\pbar$$
\end{definition}

\textbf{Curvature of Chern connection}

$E\to X$ is holomorphic Hermite vector bundle , $D$ is the Chern connection,
Locally let $\{e_1,...,e_r\}$ be a holomorphic frame, and two local sections
$$s,t\in C^\infty(\Omg, E)$$
where
$$s=\sum_{i=1}^r\sgm_ie_i$$
$$t=\sum_{i=1}^r t_ie_i$$
Since $D$ is Hermitian ,
$$\td \{s,t\}=\td ((\sgm_1,...,\sgm_r)H
\begin{pmatrix}
t_1\\
\vdots\\
t_r
\end{pmatrix})
=(\td\sgm)^THt+\sgm^T(\td H)t+\sgm^TH\td(t)
$$
so, we have
$$\{Ds,t\}+\{s,Dt\}=(\td\sgm+\overline{H}^{-1}\p\overline{H}\wedge\sgm)^T\wedge H\overline{t}
+\sgm^T\wedge H\overline{(\td t+\overline{H}^{-1}\p\overline{H}\wedge t)}$$

so ,
$$Ds=\td\sgm+\overline{H}^{-1}\p\overline{H}\wedge\sgm$$
$$D's=\p\sgm+\overline{H}^{-1}\p\overline{H}\wedge\sgm
=\overline{H}^{-1}\p(\overline{H}\sgm)$$
$$D''s=\pbar\sgm$$
so,
$$(D')^2s=\overline{H}^{-1}\p(\overline{H}(\overline{H}^{-1}\p(\overline{H}\sgm)))
=\cdots=0$$

$$(D'')^2s=\cdots=0$$

So we have
$$\Theta(D)=(D'+D'')^2=D'D''+D''D'$$

Locally ,
$$\Theta s=D'D''s+D''D's=\overline{H}^{-1}\p(\overline{H}\pbar\sgm)
+\pbar(\overline{H}^{-1}\pbar(\overline{H}\sgm))
=\cdots=\overline{H}^{-1}\p\overline{H}\wedge\pbar\sgm
+\pbar(\overline{H}^{-1})\sgm
$$
$$
  = \pbar(\overline{H}^{-1}\p\overline{H})\sgm
$$

So, Chern curvature
$$\Theta_D=\pbar(\overline{H}^{-1}\p\overline{H})$$

%Next time: Hodge theory...
%%%%%%%%%%%%%%%%%%%%%%%%%%%%%%%%%%%%%%%%%
%%%%%%%%%%%%%2019.4.9周二第七周%%%%%%%%%%%%%%%%%

Last time: $E\to X$ is a holomorphic vector bundle
with a Hermitian metric $H$.
Then there is a unique connection $D_E$s.t. ... called Chern connection.

Curvature of Chern Connection:
$$\Theta(D_E)=\pbar(\overline{H}^{-1}\p\overline{H})$$

so,
$$i\Theta(D_E)\in C_{1,1}^\infty(X,\Hom(E,E))$$

\begin{example}(Special case: $E$ is  a holomorphic line bundle)

locally, let $e$ be ha holomorphic frame,
$\langle e,e\rangle=h$ is the metric.
then,
$$\Theta=\pbar(h^{-1}\p h)=\pbar\p\log h$$
so,
$$i\Theta(E)=-i\p\pbar\log h$$
\end{example}
if $h=e^{-2\fai}$ where $\fai$ is a smooth function, then
$$i\Theta(E)=2i\p\pbar\fai
=2\sqrt{-1}\sum_{k,l}\pmfrac{\fai}{z_k}{\overline{z_l}}
\td z_k\wedge\td\overline{z_l}
$$

\textbf{Question}: let $s$ be a local holomorphic section of $E$,
$$-i\p\pbar\log|s|_h^2=?$$
(Hint:$\frac{i}{\pi}\p\pbar\log z=?$单复变,
按分布意义下求导 .等于狄拉克测度2333333)
{\color{red}可能是期末题目?}

\begin{example}
$\mcalO(-1)$ on $\bbC P^n$, tautological line bundle.
(Recall: $\bbC P^n$ is a compact complex manifold with holomorphic charts
$$\Omg_j:=\{[z_0;z_1;...;z_n]|z_j\neq 0\}\to
\left(
\frac{z_0}{z_j},\cdots,\hat{1},\cdots
\frac{z_n}{z_j}
\right)\in\bbC^n$$
)
\end{example}

Let $V$ be a complex vector space, $\dim_\bbC V=n+1$.
Denote the projective space by
$$\bbP(V)=(V\setminus\{0\})/\bbC^*$$
Let $\underline{V}:=\bbP(V)\times V$ be the trivial vector bundle,
define
$$\mcalO(-1):=\{([x],\xi)|\xi\in\bbC\cdot x\}$$

\begin{prop}
$\mcalO(-1)$ is a holomorphic line bundle on $\bbP(V)$.
\end{prop}
\begin{proof}
$\mcalO(-1)|_{\Omg_j}$ has a non-vanishing holomorphic section $\mcalE_j$ defined by
$$\mcalE_j([x])=\frac{x}{x_j}$$
for $0\leq j\leq n$.
\end{proof}

Assume $V$ has a Hermitian inner product, then $\mcalO(-1)$ has an
Hermitian structure induced from $V$.

Let $e_0,...,e_n$ be an orthonormal basis of $V$, then
$\mcalO(-1)|_{\Omg_0}$ has a non-vanishing holomorphic section:
$$\mcalE_0(z_1,...,z_n)=e_0+z_1e_1+...+z_ne_n$$
where
$$\Omg_0=\{[1;z_1;...;z_n]|z_j\in\bbC\}\cong\bbC^n$$
then,
$$|\mcalE_0|^2_h=1+|z_1|^2+...+|z_n|^2$$
so the Chern curvature of $\mcalO(-1)$ on $\Omg_0$ is given by
$$\Theta=\pbar\p\log(1+|z_1|^2+\cdots+|z_n|^2)$$

Denote $\mcalO(1):=\mcalO(-1)^*$, then
$$\Theta(\mcalO(1))=-\pbar\p\log(1+|z_1|^2+...+|z_n|^2)$$
on $\Omg_0$.
$$i\Theta(\mcalO(1))=i\p\pbar\log(1+|z_0|^2+...+|z_n|^2)
=\sqrt{-1}\sum_{1\leq k,l\leq n}
c_{k,l}\td z_k\wedge\td\overline{z_l}$$

Exercise: $(c_{kl})$ is a positive definite Hermitian matrix.

"Fubini-Study metric" on $\bbP(V)$.$\mcalO(1)$ is
"hyperplane line bundle of $\bbP(V)$".

Exercise: calculate
$$\int_{\bbP(V)}
    \left(
      \frac{i}{2\pi}
      \Theta(\mcalO(1))
    \right)^{\wedge n}
=?
$$
(Hint: $\bbP(V)\setminus\Omg_0$ is a zero-measure set)

$E\to X$ : holomorphic line bundle, $D_E$ is a Chern connection.
$$c_1(E)=\{\frac{i}{2\pi}\Theta(D_E)\}\in H_{DR}^2(X,\bbR)$$

Exercise:
{\color{red} $60\%$的概率出现于期末试题}

Consider the sequence
$$0\to\bbZ\to\mcalO\xra{e^{2\pi i *}}\mcalO^*\to 0$$
it induces a long exact sequence
$$\cdots\to
H^1(X,\mcalO)\to H^1(X,\mcalO^*)\xra{\delta}H^2(X,\bbZ)\to H^2(X,\mcalO)\to\cdots
$$
prove: Consider $E$ as an element of $H^1(X,\mcalO^*)$, then the
image of $\delta(E)$ in $H^2(X,\bbR)\cong H^2_{DR}(X,\bbR)$ is $c_1(E)$.

Exercise: $E$ is a holomorphic line bundle, denote
$\theta:=\frac{i}{2\pi}\Theta(D_E)$ real $(1,1)-form$,
where $D_E$ is Chern connection
with a metric $h$.
Prove: for any smooth function $f\in C^\infty(X,\bbR)$,
there exists a Hermitian metric $h_f$ s.t.
$$\frac{i}{2\pi}\Theta_{E,h_f}=\theta+i\p\pbar f$$


