\chapter{Hermite向量丛}
\section{联络与曲率}

Recall: $X$ is a smooth manifold, $E$ is a vector bundle of rank $r$, if

(1)$\pi:E\to X$ is smooth map,

(2)for any $x\in X$, $E_x:=\pi^{-1}(x)$ is a vector space over $\bbK$
($\bbK=\bbR$ or $\bbC$) of dimension $r$.

(3)there an open covering $\mcalU=(\mcalU_{\alpha})_{\alpha\in I}$ and trivializations
$$\theta_\alpha: E|_{U_{\alpha}}\cong U_{\alpha}\times \bbK^r$$
and for any intersection $U_{\alpha}\cap U_{\beta}$, we have
%%%%%%%见笔记%%%%%%%%%5

\begin{rem}
$$g_{\alpha\beta}=g_{\beta\alpha}^{-1}$$
$$g_{\alpha\beta}g_{\beta\gamma}g_{\gamma\alpha}=1$$
(cocycle condition)
\end{rem}

\textbf{Special Case: line bundle}
rank $E$=1.

then $g_{\alpha\beta}\in C^{\infty}(U_{\alpha\beta},\bbK^*)=\mcalE^*(U_{\alpha\beta})$
invertible smooth function on $U_{\alpha\beta}$.
then, Cech cohomology,
$$(\delta g)_{\alpha\beta\gamma}=g_{\beta\gamma}g_{\alpha\gamma}^{-1}g_{\alpha\beta}=1$$
so,
$$(g_{\alpha,\beta})\in\mcalZ^1(\mcalU,\mcalE^*)
\surj H^1(\mcalU,\mcalE^*)\inj\check{H}^1(X,\mcalE^*)$$
we get a map
$$\{\text{line bundles}\}\to \check{H}^1(X,\mcalE^*)$$
actually, we have
$$\{\text{isomorphic classes of line bundles}\}
\longleftrightarrow H^1(X,\mcalE^*)$$
1-1 correspondence.

Now, $X$ be a complex manifold,
a complex vector bundle $E$ is called homomorphic,
if ... the transition matrix $g_{\alpha\beta}$ is holomorphic...

Holomorphic line bundles :
$$g_{\alpha\beta}\in\mcalO^*(U_{\alpha\beta})$$
$\mcalO^*$:sheaf of invertible holomorphic functions...

FACT: there is a map
$$\{\text{holomorphic line bundle}\}\to\check{H}^1(X,\mcalO^*)$$

\begin{example}trivial vector bundle $X\times\bbK^r$
\end{example}

\begin{example}Tangent bundle $TX$.
(transition matrix $g_{\alpha\beta}$ are given by Jacobi matrix..)
\end{example}

\begin{definition}(Local frame of vector bundles)

$$\theta_{\alpha}:E|_{U_{\alpha}}\xra{\sim} U_{\alpha}\times\bbK^r$$
be a trivialization, we define
$$
  e_{\lmd}(x):=
  \theta_{\alpha}^{-1}
  (x,\begin{pmatrix}
       0\\
       \ldots\\
       1(\leftarrow i\text{th})\\
       \ldots\\
       0
     \end{pmatrix})
$$
then, $\{e_1,...,e_r\}$ be a local smooth section
$s\in\Gamma(U_{\alpha},E)$ can be written as
$$s(x)=\sum\sigma_{\lmd}(x)$$
where $\sigma_{\lmd}\in C^{\infty}(U_\alpha,\bbK)$.
\end{definition}

\textbf{(Connection)}

\begin{notation}

For $X$ be a smooth manifold, $E$ is a vector bundle(real or complex), denote
$$C_p^k(\Omg,E):=C^k(\Omg,\wedgeform{p}T^*M\ten E)$$
is the space of $k$-differential $p$-forms with values in $E$.

Locally, consider a trivialization of $E$,
$$\theta_{\alpha}E|_{U_{\alpha}}\cong U_{\alpha}\times \bbK^{r}$$
($\rightsquigarrow$ frame $(e_1,...e_r)$)
$$s\in \sum\fai_{\lmd}(x)\ten e_{\lmd}(x)$$
where $\fai_{\lmd}$ is a $p$-form.
\end{notation}

\begin{definition}
a (linear) connection on $E$ is a linear differential operator of order $1$ acting on
$C^{\infty}\downdot(X,E)$:
$$D:C_{p}^{\infty}(X,E)\to C_{p+1}^{\infty}(X,E)$$
$$D(f\wedge x):= \td f\wedge s+(-1)^p f\wedge Ds$$
where $f\in C^{\infty}(X,\wedgeform{p}T^*M)$, $s\in C^{\infty}(X,E)$.
\end{definition}

Locally, consider a local trivialization
$$\theta:E|_{\Omg}\xra{\sim}\Omg\times\bbK^r$$
with a frame $\{e_1,...,e_r\}$. any section
$t\in C^{\infty}_p(\Omg,E)$ can be written as
$$t=\sum_{1\leq\lmd\leq r}\sgm_{\lmd}\ten e_{\lmd}$$
$$Ds=\sum_{\lmd=1}^r\td\sgm_{\lmd}\wedge e_{\lmd}+(-1)^p\sgm_{\lmd}\wedge De_{\lmd}$$
where
$$De_{\lmd}\in C_1^{\infty}(\Omg, E)$$
can be written as
$$De_{\lmd}=\sum_{\mu=1}^r
             a_{\mu\lmd}\ten e_{\mu}$$
where "$a_{\mu \lmd}$" is called the coefficients of $D$
 with respect to frame $\{e_1,...,e_r\}$ .

so,
$$D(t)=
\sum_{\lmd,\mu}
  \td\sgm_{\lmd}\wedge e_{\lmd}+(-1)^p\sgm_{\lmd}\wedge a_{\mu\lmd}\wedge e_{\mu}
=\sum_{\mu}
   \sum_{\lmd}
     \left(
       \td\sgm_{\mu}+
       a_{\mu\lmd}\wedge\sgm_{\lmd}
     \right)$$
%平凡化下,用矩阵再写一下,自己脑补%
$$Dt=\td\sgm+A\wedge\sgm$$
where $A=(a_{\mu\lmd})$.

RMK: connection always exists!

%%%%%%%2019.4.2第六周周二%%%%%%%%%%%%%%

Recall: for any (connected) smooth manifold,
$E\to X$ is a smooth vector bundle,

Connection:
$$D:C^\infty_p(X,E)\to C_{p+1}^\infty(X,E)$$
where $C^\infty_p(X,E):=C^\infty(X,\wedge^pT^*M\ten E)$

$$D(f\wedge s)=\td f\wedge s+(-1)^{\deg f}f\wedge Ds$$
Essentially,
$$D:C^\infty(X,E)\to C_1^\infty(X,E)$$

Locally, consider a trivialization
$\theta:E|_\Omg\xra{\sim}\Omg\times \bbK^r$, and a local frame
$(e_1,...,e_r)$ where $e_k(x)=\theta^{-1}(x,\begin{pmatrix}
0\\\vdots\\1 (k^{th})\\
\vdots\\0
\end{pmatrix})$.

Let $s\in C^\infty(\Omg,E)$, i.e.
$$s=\sum_{i=1}^r\sgm_ie_i$$
where $\sgm_i$ are smooth functions.
$$Ds=\td\sgm+A\wedge\sgm$$
where
$$
  \sgm=
  \begin{pmatrix}
  \sgm_1\\\vdots\\\sgm_r
  \end{pmatrix}
  \quad
  A={a_{ij}}
$$

consider another trivialization
$$\tilde\theta:E|_\Omg\xra{\sim}\Omg\times\bbK^r$$
$\rightsquigarrow$ a local frame $(\tilde{e_1},...,\tilde{e_r})$.
Then there exists a invertible linear transform s.t.
$$\tilde{e_k}=g_k^me_m$$
assume
$$De_k=a_k^le_l\qquad
D\tilde{e_k}=\tilde{a}_k^l\tilde{e}_l$$
we have
$$\td g_k^ne_n+g_k^ma_m^ne_n=\tilde{a}_k^lg_l^ne_n$$
$$\rightsquigarrow\quad
\tilde{a}_k^lg_l^n(g^{-1})^p_n=\td g_k^n(g^{-1})^p_n+g_k^ma_m^n(g^{-1})^p_n$$
$$\rightsquigarrow\quad
\tilde{a}_l^p=\td g_k^n(g^{-1})^p_n)+g_k^ma^n_m(g^{-1})^p_n
$$
$$\rightsquigarrow\quad
\tilde{A}=\td g\cdot g^{-1}+g\cdot A\cdot g^{-1}$$

\textbf{Curvature}

$$H_D:=D^2$$%H要加一个圈。。。
locally,
$$D^2s=D(\td\sgm+A\wedge\sgm)=
\td(\td\sgm+A\wedge\sgm)+A\wedge(\td\sgm+A\wedge\sgm)$$
$$
  =\td A\wedge\sgm-A\wedge\td\sgm+A\wedge\td\sgm+A\wedge A\wedge\sgm
  =(\td A+A\wedge A)\wedge\sgm
$$
so we have
$$H=\td A+A\wedge A$$

Similarly to $\tilde{A},A$ we have

Exercise:
$$\tilde{H}=gHg^{-1}$$
曲率在不同平凡化下的表达式。where
$$\tilde{e}=ge$$

$\rightsquigarrow H$ can be considered as a section of $C^\infty_2(X,\Hom(E,E))$.
because
$$\tilde{H}\tilde{e}=gHg^{-1}\tilde{e}=gHe$$
independent of the choice of local frames.

\section{向量丛的构造}
\begin{definition}(dual of vector bundles)
$E\to X$, and $g_{\alpha\beta}$ :transition matrix of $E$,
the dual is given by
$(g_{\alpha\beta})^{-1}$.
(用转移函数来定义向量丛)
\end{definition}

\begin{definition}
direct sum of two vector bundles $(E,F)\to E\oplus F$.
locally,
$$(g_{\alpha,\beta})\oplus(h_{\alpha\beta})$$
direct sum of transition matrices.
\end{definition}

\begin{definition}
tensor product of two vector bundles.

locally, tensor product of two transition matrices.
\end{definition}

fact: let $D_E$ be a connection on $E$,then it induces a connection $D_{E^*}$.
Let $u$ be a local section of $E^*$, $s$ local section of $E$,
then we define
$$\td\langle u,s\rangle=\langle D_{E^*}u,s\rangle+\langle u,D_{E}s\rangle$$

Exercise:
$$H(D_{E^*})=-H(D_E)^T$$

and for two vector bundles $E,F$, connections $D_E,D_F$, then
$$D_{E\oplus F}:=D_E\oplus D_F$$
$$H(E\oplus F)=H_E\oplus H_F$$

as for tensor product,
we define $D_{E\ten F}$ as follows:
$$D_{E\ten F}(s\ten t)=D_Es\ten t+s\ten D_Ft$$
check the curvature
$$H_{E\ten F}=H_E\ten id_F+id_E\ten H_F$$

\begin{rem}
we can also consider wedge product of vector bundles.
Consider vector bundles $E_1,...,E_k$,
with connections $D_{E_1},...,D_{E_k}$,
let $s_i\in C_{p_i}^\infty(X,E^i)$
then
$$D_{E_1\wedge,...,\wedge E_k}(s_1\wedge...\wedge s_k)
=\sum_{i=1}^k(-1)^{p_1+...+p_{i-1}}s_1
\wedge...\wedge D_{E_i}s_i\wedge...\wedge s_k$$
\end{rem}

Let $E$ be a vector bundle of rank $r$,
then $\wedgeform{r}E$ is a line bundle,
with transition matrix by $\det(g_{\alpha\beta})$.
this bundle is denoted by $\det E$.(Det-bundle)

Let $s_1,...,s_r$ be local sections of $E$,
then we have
$$D_{\det E}(s_1\wedge\cdots\wedge s_r)= tr(H_E)s_1\wedge\cdots\wedge s_r$$

\section{陈省身示性类}
chern classes (defined by curvature).

Let $E\to X$ be a smooth complex vector bundle of rank $r$,
where $X$ be a complex manifold.

(Chern-Weil theory)

$V$ be a complex vector space, $f:\underbrace{V\times\cdots\times V}_k\to \bbC$
be a symmetric multi-linear form of degree $k$.

$\rightsquigarrow f(v):=f(v,v,...,v)$ is a homogeneous polynomial of degree $k$.
\begin{definition}
assume $G$ is a group (left) acting on $V$, s.t.
$$f(g(v_1),...,g(v_k))=f(v_1,...,v_k)$$
for any $g\in G,v_i\in V$, then we say $f$ is $G$-invariant.
\end{definition}

Special case: $G=GL(r,\bbC)$ and $V=Lie G=\mfkgl{r,\bbC}$ be the Lie algebra of $G$.
the action is
$$(g,M)\mapsto gMg^{-1}$$

Consider
$$\det(I+\frac{i}{2\pi}tm)=I+tf_1(M)+t^2f_2(M)+\cdots t^rf_r(M)$$

$\rightsquigarrow\forall 1\leq k\leq r$, $f_k$ is $G$-invariant.

Let $E\to X$ complex vector bundle on a complex manifold,
let $D_E$ be a connection,
curvature $H_E\in C_2^\infty(X,\Hom(E,E))$.
 Let $f\in GL(r,\bbC)$- invariant "$k$-form",then

(1)Let $H_{\alpha},H_{\beta}$ be the curvature forms of $E$ in different trivialization,
then $f(H_\alpha)=f(H_{\beta})$ , so we get a globally defined $2k$-form.

assume $H_\alpha=gH_\beta g^{-1}$, then
$$f(H_\alpha)=f(g H_\beta g^{-1})=f(H_\beta)$$

(2) we also have
$$\td f(H)=0$$
locally , $H=H_\alpha=\td a_\alpha+A_\alpha\wedge A_\alpha$,then
$$\td f(H)=\td f(H_\alpha,H_\alpha,...,H_\alpha)
=\sum_{i=1}^kf(H_\alpha,...,\underbrace{\td H_\alpha}_{i},...,\H_\alpha)$$
$$
=\sum_{i=1}^kf(H_\alpha,...,\td A_\alpha\wedge A_\alpha-A_\alpha\wedge\td A_\alpha,...,H_\alpha)
$$

Fact:(in Riemannian geometry) For any $x\in X$,
we always can find a local frame s.t.
$A_\alpha(x)=0$.

so, choose this frame,
$$\td f(H)=0$$

So, $[f(H)]\in H^{2k}(X,\bbC)$

(3) Claim : the class $[f(H)]$ is independent of the choice of the connections $D_E$.

Let $D_0,D_1$ be two connections, consider
$$D_t=(1-t)D_0+tD_1$$
$t\in[0,1]$, curvature $H_t$
%%%%%%%%将以上所有的H都换成\Theta%%%%%%%%%

Fact: $\alpha:=A_1-A_0$ is globally defined, and in $C_1^\infty(X,\Hom(E,E))$.

Fact: $$\frac{\td}{\td t}f(H_t)=k\td f(\alpha,H_t,H_t,...,H_t)$$

So,
$$f(H_1)-f(H_0)=
\int_0^1\frac{\td}{\td t}f(H_t)\td t
=\td\int_0^1f(\alpha,H_t,H_t,...,H_t)\td t$$
So,
$$[f(H_1)]-[f(H_0)]$$

\begin{definition}
the $k$-th Chern class of $E$
$$c_k(E):=[f_k(\Theta_E)]\in H^{2k}(X,\bbC)$$
\end{definition}

%%%%%%%%%%2019.4.04第六周星期四;清明节前最后一节课%%%%%%%%%%%%%%%
Recall: Chern Class

$X$ complex manifold, $E\to X$ is a smooth complex vector bundle of rank $r$.
$D$ is a connection, curvature $\Theta(D)\in C_2^\infty(X,\Hom(E,E))$.

linear algebra:
$$\det(I+\frac{i}{2\pi}tM)=I+tf_1(M)+t^2f_2(M)+\cdots+t^rf_r(M)$$

Chern class $\{f_k(\Theta)\}\in H^{2k}_{DR}(X,\bbC)$ is independent of choice of connection.

Today:

Special case: $E$ is a complex line bundle.
Let $D_0$ be a connection on $E$, locally $D_0e=A_0e$, $A_0$ is $1$-form.
curvature
$$\Theta(D_0)=D_0^2=\td A_0+A_0\wedge A_0=\td A_0$$
so, curvature is $\td$-exact, so $\td\Theta(D_0)=0$.
$$\det(I+\frac{i}{2\pi}tM)=I+\frac{i}{2\pi}tM$$
so, the first Chern class of line bundle is
$$c_1(E)=\{\frac{i}{2\pi}\Theta(D_0)\}$$

Let $D_1$ be another connection, locally $D_1e=A_1e$, so
$\Theta(D_1)=\td A_1$.so,
$$\Theta(D_1)-\Theta(D_0)=\td(A_1-A_0)$$
where
$$A_1-A_0\in C_1^\infty(X,\Hom(E,E))$$
(when $E$ is line bundle ,$\Hom(E,E)\cong E^*\ten E$ is trivial bundle)

so, $A_1-A_0$ is a globally defined smooth function on $X$. So,
$$\{\Theta(D_1)\}=\{\Theta(D_0)\}\in H^2(X,\bbC)$$
independent of the choice of connection.

\section{Hermite向量丛}

\begin{definition}
a complex vector bundle $E\to X$ of rank $r$ is called a Hermitian vector bundle, if
we have an inner product on $E$, i.e. locally, consider a local frame
$\{e_1,...,e_r\}$, we have
$$\{e_i(x),e_j(x)\}=h_{ij}(x)$$
s.t. $(h_{ij}(x))$ is a positive definite Hermitian matrix depending smoothly on $x$.
\end{definition}

\begin{rem}
For any complex vector bundle, Hermitian structure always exists.
\end{rem}

证明与黎曼几何类似。(黎曼度量的存在性)

\begin{definition}(Hermitian connection)%connection compatible with Hermitian metric

A connection $D$ on $E$ is called Hermitian, if
$$\td\{e_i,e_j\}=\{De_i,e_j\}+\{e_i,De_j\}$$
\end{definition}

More generally, let $t\in C_p^{\infty}(X,E)$, $s\in C_q^\infty(X,Y)$,
$$\td\{s,t\}=\{\td t,s\}+(-1)^p\{t,Ds\}$$

\begin{prop}
$D$ is a Hermitian connection ,then the curvature
$$\Theta(D)^*=-\Theta(D)$$
(where $(-)^*$ is conjugate transpose of matrix)
\end{prop}

it means that, $i\Theta(D)\in C_2^\infty(X,\text{Herm}(E,E))$

\begin{proof}
$$0=\td^2\{e_i,e_j\}=\td\{De_i,e_j\}+\td\{e_i,De_j\}$$
$$=\{D^2e_i,e_j\}-\{De_i,De_j\}+\{De_i,De_j\}+\{e_i,D^2e_j\}=
\{(\Theta+\Theta^*)e_i,e_j\}$$
\end{proof}

\begin{rem}
$E$ is a Hermitian line bundle, $D$ is a Hermitian connection,
then $i\Theta(D)$ is a real $2$-form ,
$c_1(E)\in H^2(X,\bbR)$.
\end{rem}

(Chern connection)

\begin{definition}Let $X$ be a complex manifold.
$D'$ is called a connection of type $(1,0)$ on $E$,
if for any section $s\in C_{p,q}^\infty(X,E)$, we have
$D's\in C_{p+1,q}^\infty(X,E)$.

A connection $D''$ is called a connection of type $(0,1)$, if ...
$D''s\in C_{p,q+1}^\infty(X,E)$.
\end{definition}

\begin{rem}Let $E\to X$ be a %holomorphic  这里需要是全纯向量丛吗?
vector bundle.
Let $D$ be a connection on $E$, locally
$$Ds\xra{\sim}\td\sgm+A\wedge\sgm$$
$$\td\sgm=\p\sgm+\pbar\sgm$$
so,let $A'$ be the $(1,0)$-part of $A$,...,
$$Ds=\p\sgm+A'\wedge\sgm+(\pbar\sgm+A''\wedge\sgm)=:D's+D''s$$
\end{rem}

\begin{prop}
$E$:Hermitian vector bundle, $D$ is a Hermitian connection, locally,
take a $C^\infty$-frame $e_1,...,e_r$ which is orthonomal
(i.e. $\{e_i(x),e_j(x)\}=\delta_{ij}$), then the  connection coefficient
$A=A'+A''$ satisfies
$$(A')^*=-A''$$
($\iff \overline(iA)=iA$ )
\end{prop}

\begin{proof}
because
$$0=\td{e_i,e_j}=\{De_i,e_j\}+\{e_i,De_j\}
=\{a_i^ke_k,e_j\}+\{e_i,a_j^le_l\}
=a^j_i+\overline{a^i_j}$$
so, $A^*=-A$.
\end{proof}

\begin{cor}
$E\to X$ is a Hermitian vector bundle,
$D_0''$ is a connection of type $(0,1)$ on $E$.
Then exists a unique Hermitian connection $D$
such that $D''=D_0''$.
\end{cor}

\begin{proof}
Let $A''=A_0''$ and $A'=-(A_0'')^*\rightsquigarrow A=A'+A''$, and
$D$ is given by $A$.
\end{proof}

Let $E\to X$ is a holomorphic Hermitian vector bundle,
observe that $\pbar$ defines a connection of type $(0,1)$ on $E$(check!)

assume $E$ is a holomorphic line bundle, take a section
$s\in C_p^\infty(X,E)$, i.e. we have a family of $p$-forms $(s_\afa)$
such that $s_\afa=g_{\afa\beta}s_\beta$
where $g_{\afa,\beta}$ is the holomorphic transition matrix.
$$\pbar s\xra{\sim}\pbar s_\beta$$
then
$$\pbar s_\afa=g_{\afa,\beta}\pbar s_\beta$$

(so, $\pbar$ is a connection of $(0,1)$)

this connection is called the canonical connection of type $(0,1)$.

\begin{definition}
Let $E\to X$ holomorphic Hermitian vector bundle,
the connection $D$ on $E$ is called Chern connection if
$$D''=\pbar$$
\end{definition}

\textbf{Curvature of Chern connection}

$E\to X$ is holomorphic Hermite vector bundle , $D$ is the Chern connection,
Locally let $\{e_1,...,e_r\}$ be a holomorphic frame, and two local sections
$$s,t\in C^\infty(\Omg, E)$$
where
$$s=\sum_{i=1}^r\sgm_ie_i$$
$$t=\sum_{i=1}^r t_ie_i$$
Since $D$ is Hermitian ,
$$\td \{s,t\}=\td ((\sgm_1,...,\sgm_r)H
\begin{pmatrix}
t_1\\
\vdots\\
t_r
\end{pmatrix})
=(\td\sgm)^THt+\sgm^T(\td H)t+\sgm^TH\td(t)
$$
so, we have
$$\{Ds,t\}+\{s,Dt\}=(\td\sgm+\overline{H}^{-1}\p\overline{H}\wedge\sgm)^T\wedge H\overline{t}
+\sgm^T\wedge H\overline{(\td t+\overline{H}^{-1}\p\overline{H}\wedge t)}$$

so ,
$$Ds=\td\sgm+\overline{H}^{-1}\p\overline{H}\wedge\sgm$$
$$D's=\p\sgm+\overline{H}^{-1}\p\overline{H}\wedge\sgm
=\overline{H}^{-1}\p(\overline{H}\sgm)$$
$$D''s=\pbar\sgm$$
so,
$$(D')^2s=\overline{H}^{-1}\p(\overline{H}(\overline{H}^{-1}\p(\overline{H}\sgm)))
=\cdots=0$$

$$(D'')^2s=\cdots=0$$

So we have
$$\Theta(D)=(D'+D'')^2=D'D''+D''D'$$

Locally ,
$$\Theta s=D'D''s+D''D's=\overline{H}^{-1}\p(\overline{H}\pbar\sgm)
+\pbar(\overline{H}^{-1}\pbar(\overline{H}\sgm))
=\cdots=\overline{H}^{-1}\p\overline{H}\wedge\pbar\sgm
+\pbar(\overline{H}^{-1})\sgm
$$
$$
  = \pbar(\overline{H}^{-1}\p\overline{H})\sgm
$$

So, Chern curvature
$$\Theta_D=\pbar(\overline{H}^{-1}\p\overline{H})$$

%Next time: Hodge theory...
%%%%%%%%%%%%%%%%%%%%%%%%%%%%%%%%%%%%%%%%%
%%%%%%%%%%%%%2019.4.9周二第七周%%%%%%%%%%%%%%%%%

Last time: $E\to X$ is a holomorphic vector bundle
with a Hermitian metric $H$.
Then there is a unique connection $D_E$s.t. ... called Chern connection.

Curvature of Chern Connection:
$$\Theta(D_E)=\pbar(\overline{H}^{-1}\p\overline{H})$$

so,
$$i\Theta(D_E)\in C_{1,1}^\infty(X,\Hom(E,E))$$

\begin{example}(Special case: $E$ is  a holomorphic line bundle)

locally, let $e$ be ha holomorphic frame,
$\langle e,e\rangle=h$ is the metric.
then,
$$\Theta=\pbar(h^{-1}\p h)=\pbar\p\log h$$
so,
$$i\Theta(E)=-i\p\pbar\log h$$
\end{example}
if $h=e^{-2\fai}$ where $\fai$ is a smooth function, then
$$i\Theta(E)=2i\p\pbar\fai
=2\sqrt{-1}\sum_{k,l}\pmfrac{\fai}{z_k}{\overline{z_l}}
\td z_k\wedge\td\overline{z_l}
$$

\textbf{Question}: let $s$ be a local holomorphic section of $E$,
$$-i\p\pbar\log|s|_h^2=?$$
(Hint:$\frac{i}{\pi}\p\pbar\log z=?$单复变,
按分布意义下求导 .等于狄拉克测度2333333)
{\color{red}可能是期末题目?}

\begin{example}
$\mcalO(-1)$ on $\bbC P^n$, tautological line bundle.
(Recall: $\bbC P^n$ is a compact complex manifold with holomorphic charts
$$\Omg_j:=\{[z_0;z_1;...;z_n]|z_j\neq 0\}\to
\left(
\frac{z_0}{z_j},\cdots,\hat{1},\cdots
\frac{z_n}{z_j}
\right)\in\bbC^n$$
)
\end{example}

Let $V$ be a complex vector space, $\dim_\bbC V=n+1$.
Denote the projective space by
$$\bbP(V)=(V\setminus\{0\})/\bbC^*$$
Let $\underline{V}:=\bbP(V)\times V$ be the trivial vector bundle,
define
$$\mcalO(-1):=\{([x],\xi)|\xi\in\bbC\cdot x\}$$

\begin{prop}
$\mcalO(-1)$ is a holomorphic line bundle on $\bbP(V)$.
\end{prop}
\begin{proof}
$\mcalO(-1)|_{\Omg_j}$ has a non-vanishing holomorphic section $\mcalE_j$ defined by
$$\mcalE_j([x])=\frac{x}{x_j}$$
for $0\leq j\leq n$.
\end{proof}

Assume $V$ has a Hermitian inner product, then $\mcalO(-1)$ has an
Hermitian structure induced from $V$.

Let $e_0,...,e_n$ be an orthonormal basis of $V$, then
$\mcalO(-1)|_{\Omg_0}$ has a non-vanishing holomorphic section:
$$\mcalE_0(z_1,...,z_n)=e_0+z_1e_1+...+z_ne_n$$
where
$$\Omg_0=\{[1;z_1;...;z_n]|z_j\in\bbC\}\cong\bbC^n$$
then,
$$|\mcalE_0|^2_h=1+|z_1|^2+...+|z_n|^2$$
so the Chern curvature of $\mcalO(-1)$ on $\Omg_0$ is given by
$$\Theta=\pbar\p\log(1+|z_1|^2+\cdots+|z_n|^2)$$

Denote $\mcalO(1):=\mcalO(-1)^*$, then
$$\Theta(\mcalO(1))=-\pbar\p\log(1+|z_1|^2+...+|z_n|^2)$$
on $\Omg_0$.
$$i\Theta(\mcalO(1))=i\p\pbar\log(1+|z_0|^2+...+|z_n|^2)
=\sqrt{-1}\sum_{1\leq k,l\leq n}
c_{k,l}\td z_k\wedge\td\overline{z_l}$$

Exercise: $(c_{kl})$ is a positive definite Hermitian matrix.

"Fubini-Study metric" on $\bbP(V)$.$\mcalO(1)$ is
"hyperplane line bundle of $\bbP(V)$".

Exercise: calculate
$$\int_{\bbP(V)}
    \left(
      \frac{i}{2\pi}
      \Theta(\mcalO(1))
    \right)^{\wedge n}
=?
$$
(Hint: $\bbP(V)\setminus\Omg_0$ is a zero-measure set)

$E\to X$ : holomorphic line bundle, $D_E$ is a Chern connection.
$$c_1(E)=\{\frac{i}{2\pi}\Theta(D_E)\}\in H_{DR}^2(X,\bbR)$$

Exercise:
{\color{red} $60\%$的概率出现于期末试题}

Consider the sequence
$$0\to\bbZ\to\mcalO\xra{e^{2\pi i *}}\mcalO^*\to 0$$
it induces a long exact sequence
$$\cdots\to
H^1(X,\mcalO)\to H^1(X,\mcalO^*)\xra{\delta}H^2(X,\bbZ)\to H^2(X,\mcalO)\to\cdots
$$
prove: Consider $E$ as an element of $H^1(X,\mcalO^*)$, then the
image of $\delta(E)$ in $H^2(X,\bbR)\cong H^2_{DR}(X,\bbR)$ is $c_1(E)$.

Exercise: $E$ is a holomorphic line bundle, denote
$\theta:=\frac{i}{2\pi}\Theta(D_E)$ real $(1,1)-form$,
where $D_E$ is Chern connection
with a metric $h$.
Prove: for any smooth function $f\in C^\infty(X,\bbR)$,
there exists a Hermitian metric $h_f$ s.t.
$$\frac{i}{2\pi}\Theta_{E,h_f}=\theta+i\p\pbar f$$


