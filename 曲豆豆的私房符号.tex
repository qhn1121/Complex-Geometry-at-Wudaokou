%这是曲豆豆的红包

%%%%%%%%%%%%%%定理环境%%%%%%%%%%%%%%%%%%%%%%%

\newtheorem{mythm}{定理}[section]
\newtheorem{mylemma}[mythm]{引理}
\newtheorem{myprop}[mythm]{性质}
\newtheorem{myaxiom}[mythm]{公理}
\newtheorem{mydefinition}[mythm]{定义}
\newtheorem{mynotation}[mythm]{记号}
\newtheorem{example}[mythm]{例子}
\newtheorem{PreImportantExample}[mythm]{重要例子}
\newtheorem{rem}[mythm]{注记}
\newtheorem{mycor}[mythm]{推论}
\newtheorem{claim}[mythm]{断言}
\newtheorem{prob}{习题}[section]


\chapterbackgroundcolor


\newenvironment{thm}
  {\definecolor{shadecolor}{RGB}{\thmcolor}
    \begin{shaded}\begin{mythm}}
  {\end{mythm}\end{shaded}
  \chapterbackgroundcolor}

\newenvironment{lemma}
  {\definecolor{shadecolor}{RGB}{\lemmacolor}
    \begin{shaded}\begin{mylemma}}
  {\end{mylemma}\end{shaded}
  \chapterbackgroundcolor}

\newenvironment{prop}
  {\definecolor{shadecolor}{RGB}{\propcolor}
    \begin{shaded}\begin{myprop}}
  {\end{myprop}\end{shaded}
  \chapterbackgroundcolor}

\newenvironment{axiom}
  {\definecolor{shadecolor}{RGB}{\axiomcolor}
    \begin{shaded}\begin{myaxiom}}
  {\end{myaxiom}\end{shaded}
  \chapterbackgroundcolor}

\newenvironment{cor}
  {\definecolor{shadecolor}{RGB}{\lemmacolor}
    \begin{shaded}\begin{mycor}}
  {\end{mycor}\end{shaded}
  \chapterbackgroundcolor}

\newenvironment{definition}
  {\definecolor{shadecolor}{RGB}{\definitioncolor}
    \begin{shaded}\begin{mydefinition}}
  {\end{mydefinition}\end{shaded}
  \chapterbackgroundcolor}

\newenvironment{notation}
  {\definecolor{shadecolor}{RGB}{\notationcolor}
    \begin{shaded}\begin{mynotation}}
  {\end{mynotation}\end{shaded}
  \chapterbackgroundcolor}

\newenvironment{Example}
  {\definecolor{shadecolor}{RGB}{\examplecolor}
    \begin{shaded}\begin{PreImportantExample}}
  {\end{PreImportantExample}\end{shaded}
  \chapterbackgroundcolor}

%%%%%%%%%%%%%%%%%%%%%%%%%%%%%%%%%%%%%%%%%%%%%%%%%%%%%%%%%%%%
\newcommand*{\vs}{\vspace{5pt}}
\newcommand*{\vsp}{\vspace{10pt}}
\newcommand*{\vspp}{\vspace{20pt}}
\newcommand*{\kong}{$\,\,\,$}
\newcommand*{\fengexian}{
  \rule[0pt]{14.3cm}{0.01em}
\vs}
\newcommand*{\ffgexm}{          %ffgexm是“分割线”的微软双拼
  \rule[0pt]{14.3cm}{0.01em}
\vs}
%%%%%%%%%%%%%%%%%%%%%%%%%%%%%%%%%%%%%%%%%%%%%%%%%%%%%%%%%%%%
\newcommand*{\Langle}{\left\langle}
\newcommand*{\Rangle}{\right\rangle}
%%%%%%%%%%%%%%%%%%%%%%%%%%%%%%%%%%%%%%%%%%%%%%%%%%%%%%%%%%%%
\newcommand*{\bbA}{\mathbb{A}}
\newcommand*{\bbB}{\mathbb{B}}
\newcommand*{\bbC}{\mathbb{C}}
\newcommand*{\bbD}{\mathbb{D}}
\newcommand*{\bbE}{\mathbb{E}}
\newcommand*{\bbF}{\mathbb{F}}
\newcommand*{\bbG}{\mathbb{G}}
\newcommand*{\bbH}{\mathbb{H}}
\newcommand*{\bbI}{\mathbb{I}}
\newcommand*{\bbJ}{\mathbb{J}}
\newcommand*{\bbK}{\mathbb{K}}
\newcommand*{\bbL}{\mathbb{L}}
\newcommand*{\bbM}{\mathbb{M}}
\newcommand*{\bbN}{\mathbb{N}}
\newcommand*{\bbO}{\mathbb{O}}
\newcommand*{\bbP}{\mathbb{P}}
\newcommand*{\bbQ}{\mathbb{Q}}
\newcommand*{\bbR}{\mathbb{R}}
\newcommand*{\bbS}{\mathbb{S}}
\newcommand*{\bbT}{\mathbb{T}}
\newcommand*{\bbU}{\mathbb{U}}
\newcommand*{\bbV}{\mathbb{V}}
\newcommand*{\bbW}{\mathbb{W}}
\newcommand*{\bbX}{\mathbb{X}}
\newcommand*{\bbY}{\mathbb{Y}}
\newcommand*{\bbZ}{\mathbb{Z}}

\newcommand*{\mcalA}{\mathcal{A}}
\newcommand*{\mcalB}{\mathcal{B}}
\newcommand*{\mcalC}{\mathcal{C}}
\newcommand*{\mcalD}{\mathcal{D}}
\newcommand*{\mcalE}{\mathcal{E}}
\newcommand*{\mcalF}{\mathcal{F}}
\newcommand*{\mcalG}{\mathcal{G}}
\newcommand*{\mcalH}{\mathcal{H}}
\newcommand*{\mcalI}{\mathcal{I}}
\newcommand*{\mcalJ}{\mathcal{J}}
\newcommand*{\mcalK}{\mathcal{K}}
\newcommand*{\mcalL}{\mathcal{L}}
\newcommand*{\mcalM}{\mathcal{M}}
\newcommand*{\mcalN}{\mathcal{N}}
\newcommand*{\mcalO}{\mathcal{O}}
\newcommand*{\mcalP}{\mathcal{P}}
\newcommand*{\mcalQ}{\mathcal{Q}}
\newcommand*{\mcalR}{\mathcal{R}}
\newcommand*{\mcalS}{\mathcal{S}}
\newcommand*{\mcalT}{\mathcal{T}}
\newcommand*{\mcalU}{\mathcal{U}}
\newcommand*{\mcalV}{\mathcal{V}}
\newcommand*{\mcalW}{\mathcal{W}}
\newcommand*{\mcalX}{\mathcal{X}}
\newcommand*{\mcalY}{\mathcal{Y}}
\newcommand*{\mcalZ}{\mathcal{Z}}

\newcommand*{\mfkg}{\mathfrak{g}}    %Lie algebra
\newcommand*{\mfkh}{\mathfrak{h}}
\newcommand*{\mfkm}{\mathfrak{m}}    %maximal ideal
\newcommand*{\mfkp}{\mathfrak{p}}    %prime ideal
\newcommand*{\mfkq}{\mathfrak{q}}

\newcommand*{\mfkgl}{\mathfrak{gl}}
\newcommand*{\mfksl}{\mathfrak{sl}}
\newcommand*{\mfkso}{\mathfrak{so}}
\newcommand*{\mfksp}{\mathfrak{sp}}

\newcommand*{\bfk}{\boldsymbol{k}}
\newcommand*{\bfp}{\boldsymbol{p}}
\newcommand*{\bfq}{\boldsymbol{q}}
\newcommand*{\bfu}{\boldsymbol{u}}
\newcommand*{\bfv}{\boldsymbol{v}}
\newcommand*{\bfw}{\boldsymbol{w}}
\newcommand*{\bfx}{\boldsymbol{x}}
\newcommand*{\bfy}{\boldsymbol{y}}
\newcommand*{\bfz}{\boldsymbol{z}}

\newcommand*{\uhat}{\widehat{u}}
\newcommand*{\vhat}{\widehat{v}}
\newcommand*{\what}{\widehat{w}}             % w-hat, NOT what.....

\newcommand*{\Ahat}{\widehat{A}}
\newcommand*{\Bhat}{\widehat{B}}
\newcommand*{\Chat}{\widehat{C}}
\newcommand*{\Dhat}{\widehat{D}}
\newcommand*{\Ehat}{\widehat{E}}
\newcommand*{\Fhat}{\widehat{F}}
\newcommand*{\Ghat}{\widehat{G}}
\newcommand*{\Hhat}{\widehat{H}}
\newcommand*{\Xhat}{\widehat{X}}

\newcommand*{\zbar}{\overline{z}}            %complex number
\newcommand*{\wbar}{\overline{w}}
\newcommand*{\taubar}{\overline{\tau}}

\newcommand*{\ftil}{\widetilde{f}}
\newcommand*{\gtil}{\widetilde{g}}
\newcommand*{\htil}{\widetilde{h}}
\newcommand*{\itil}{\widetilde{i}}
\newcommand*{\stil}{\widetilde{s}}
\newcommand*{\ttil}{\widetilde{t}}
\newcommand*{\xtil}{\widetilde{x}}
\newcommand*{\ytil}{\widetilde{y}}
\newcommand*{\ztil}{\widetilde{z}}
\newcommand*{\wtil}{\widetilde{w}}

\newcommand*{\Ftil}{\widetilde{F}}
\newcommand*{\Gtil}{\widetilde{G}}
\newcommand*{\Htil}{\widetilde{H}}
\newcommand*{\Ntil}{\widetilde{N}}
\newcommand*{\Rtil}{\widetilde{R}}

\newcommand*{\mcalFtil}{\widetilde{\mathcal{F}}}
\newcommand*{\mcalGtil}{\widetilde{\mathcal{G}}}
\newcommand*{\mcalHtil}{\widetilde{\mathcal{H}}}

\newcommand*{\afatil}{\widetilde{\alpha}}
\newcommand*{\faitil}{\widetilde{\varphi}}
\newcommand*{\Omgtil}{\widetilde{\Omega}}

\newcommand*{\afa}{\alpha}
\newcommand*{\lmd}{\lambda}
\newcommand*{\Lmd}{\Lambda}
\newcommand*{\sgm}{\sigma}
\newcommand*{\Sgm}{\Sigma}
\newcommand*{\gma}{\gamma}
\newcommand*{\Gma}{\Gamma}
\newcommand*{\dta}{\delta}
\newcommand*{\Dta}{\Delta}
\newcommand*{\omg}{\omega}
\newcommand*{\Omg}{\Omega}
\newcommand*{\veps}{\varepsilon}
\newcommand*{\fai}{\varphi}
\newcommand*{\pai}{\varpi}

%%%%%%%%%%%%%%%%%%%%%%%%%%%%%%%%%%%%%%%%%%%%%%%%
\newcommand*{\Cech}{C\v{e}ch}
%%%%%%%%%%%%%%%%%%%%%%%%%%%%%%%%%%%%%%%%%%%%%%%%

\newcommand*{\ra}{\rightarrow}
\newcommand*{\inj}{\hookrightarrow}
\newcommand*{\surj}{\twoheadrightarrow}
\newcommand*{\xra}{\xrightarrow}
\newcommand*{\xla}{\xleftarrow}
\newcommand*{\xle}{\xlongequal}                 %等号上方文字
\newcommand*{\llra}{\longleftrightarrow}
\newcommand*{\Llra}{\Longleftrightarrow}
\newcommand*{\rlim}{\varinjlim}                 %colimit,余极限,归纳极限,正向极限
\newcommand*{\llim}{\varprojlim}                %limit,极限,投射极限,逆向极限
\newcommand*{\suobing}{\lrcorner\,}             %张量的缩并

\newcommand*{\xybigrow}{\xymatrixrowsep{5pc}}   %交换图排版专用
\newcommand*{\xybigcol}{\xymatrixcolsep{5pc}}   %增大行、列间距
\newcommand*{\tabularbigrow}{\specialrule{0em}{4pt}{4pt}}

\newcommand*{\pz}{\text{---}}                   %破折号
\newcommand*{\yc}{\triangle}                    %余代数的乘法,co-product
\newcommand*{\p}{\partial}                      %偏微分、边缘算子partial
\newcommand*{\td}{\mathrm{d}}                   %微分算子d
\newcommand*{\pbar}{\overline{\partial}}        %Dolbeaut算子partial-bar
\newcommand*{\ten}{\otimes}                     %张量积tensor product
\newcommand*{\bigten}{\bigotimes}
\newcommand*{\op}{^{\text{op}}}
\newcommand*{\downdot}{_{\bullet}}              %下方黑点,用于同调代数中的链复形
\newcommand*{\ddowndot}{_{\bullet\bullet}}
\newcommand*{\updot}{^{\bullet}}
\newcommand*{\wedgeform}{\bigwedge\nolimits^}
\newcommand*{\ssubset}{\subset\!\subset}        %紧包含
\newcommand*{\vkong}{\varnothing}               %空集

%%%%%%%%%%%%%%%%%%%%%%%%%%%%%%%%%%%%%%%%%%%%%%%%

\newcommand*{\lbar}[1]{\overline{#1}}    %比较长的上横线

\newcommand*{\fps}[1]{[\![#1]\!]}        %Formal power series形式幂级数
\newcommand*{\Ls}[1]{(\!(#1)\!)}         %Laurent series     洛朗级数

\newcommand*{\Bigset}[2]
  {
   \Big\{ #1 \Big| #2 \Big\}             %集合符号
  }

\newcommand*{\pair}[2]                   %普通的内积
  {
    \langle
      #1,#2
    \rangle
  }

\newcommand*{\Pair}[2]                   %大的的内积
  {
    \left\langle
      #1,#2
    \right\rangle
  }

\newcommand*{\ppair}[2]                   %双括号内积
  {
    \langle\!\!
      \langle
        #1,#2
      \rangle\!\!
    \rangle
  }

\newcommand*{\Ppair}[2]                   %大双括号内积
  {
    \left\langle\!\!
      \left\langle
        #1,#2
      \right\rangle\!\!
    \right\rangle
  }

\newcommand*{\norm}[1]                   %范数
  {
    |\!|#1|\!|
  }

\newcommand*{\Norm}[1]                   %大型范数
  {
    \left|\!
      \left|
        #1
      \right|\!
    \right|
  }

\newcommand*{\nnorm}[1]                   %三竖线范数
  {
    |\!|\!|
      #1
    |\!|\!|
  }

\newcommand*{\Nnorm}[1]                   %大型三竖线范数
  {
    \left|\!
      \left|\!
        \left|
          #1
        \right|\!
      \right|\!
    \right|
  }

\newcommand*{\Choose}[2]                  %方括号组合数
  {\left[{
      #1 \atop
      #2
  }\right]}

\newcommand*{\dd}[1]                       %常微分算子
  {\frac{\mathrm{d}   }
        {\mathrm{d} #1}
  }

\newcommand*{\ddfrac}[2]
  {\frac{\mathrm{d} #1}
        {\mathrm{d} #2}
  }                                       %一阶常微分

\newcommand*{\pp}[1]
  {\frac{\partial   }
        {\partial #1}
  }                                       %切向量场,一阶偏微分算子

\newcommand*{\pfrac}[2]
  {\frac{\partial #1}
        {\partial #2}
  }                                       %一阶偏微分

\newcommand*{\ppfrac}[2]
  {\frac{\partial^2{#1}}
        {\partial{#2}^2}}                 %二阶偏微分

\newcommand*{\pmfrac}[3]
  {\frac{\partial^2{#1}}
        {\partial{#2}\partial{#3}}}        %二阶混合偏微分

%amsmath 宏包输入矩阵%
%matrix pmatrix bmatrix Bmatrix vmatrix Vmatrix%
%分别为无括号,小括号,中括号,大括号,竖线,双竖线%
%%%%%%%%%%%%%%%%%%%%%%%%%%%%%%%%%%%%%%%%%%%%%%%%
\DeclareMathOperator{\ad}{ad}         %adjoint
\DeclareMathOperator{\Ass}{Ass}       %Associated Prime
\DeclareMathOperator{\Aut}{Aut}       %automorphism
\DeclareMathOperator{\Bl}{Bl}         %blow-up
\DeclareMathOperator{\CE}{CE}         %Chevalley-Eilenberg
\DeclareMathOperator{\ch}{char}
\DeclareMathOperator{\cl}{cl}         %classical
\DeclareMathOperator{\Coder}{Coder}   %Co-derivation  并不是“码农”的意思
\DeclareMathOperator{\coker}{coker}   %co-kernel      余核
\DeclareMathOperator{\Conf}{Conf}     %Kontsevich公式当中用到了;configuration of points
\DeclareMathOperator{\Crit}{Crit}     %Critical locus
\DeclareMathOperator{\Der}{Der}       %Derivation        导子
\DeclareMathOperator{\Div}{div}       %divergence     散度
\DeclareMathOperator{\DR}{DR}         %De-Rham
\DeclareMathOperator{\End}{End}       %Endomorphism
\DeclareMathOperator{\Ext}{Ext}
\DeclareMathOperator{\ev}{ev}         %evaluation
\DeclareMathOperator{\Free}{Free}
\DeclareMathOperator{\Gal}{Gal}
\DeclareMathOperator{\GL}{GL}
\DeclareMathOperator{\HC}{HC}         %Hochschild Cyclic (co)homology
\DeclareMathOperator{\HH}{HH}         %Hochschild (co)homology
\DeclareMathOperator{\Hom}{Hom}       %Homorphism
\DeclareMathOperator{\id}{id}         %identity
\DeclareMathOperator{\im}{Im}         %Image
\DeclareMathOperator{\Inn}{Inn}       %Inner Derivation  内导子
\DeclareMathOperator{\Inv}{Inv}       %Invariant
\DeclareMathOperator{\Lie}{Lie}       %Lie algebra or Lie derivation
\DeclareMathOperator{\Mor}{Mor}       %Morphism
\DeclareMathOperator{\Obj}{Obj}       %Object
\DeclareMathOperator{\Obs}{Obs}       %Observable
\DeclareMathOperator{\Ord}{Ord}       %Order
\DeclareMathOperator{\per}{per}       %periodic cyclic complex
\DeclareMathOperator{\PV}{PV}         %Polyvector field
\DeclareMathOperator{\ran}{ran}       %range
\DeclareMathOperator{\rank}{rank}     %rank 秩
\DeclareMathOperator{\Rep}{Rep}
\DeclareMathOperator{\sgn}{sgn}
\DeclareMathOperator{\Sh}{Sh}
\DeclareMathOperator{\Span}{span}
\DeclareMathOperator{\Spec}{Spec}     %素谱
\DeclareMathOperator{\supp}{supp}     %Support 支撑集
\DeclareMathOperator{\Sym}{Sym}       %Symmetry  对称张量积
\DeclareMathOperator{\Tor}{Tor}
\DeclareMathOperator{\Tot}{Tot}       %Total complex of a double-complex
\DeclareMathOperator{\YM}{YM}         %Yang-Mills   杨振宁-米尔斯

%%%%%%%%%%%%%%%%%%%%%%%%%%%%%%%%%%%%%%%%%%%%%%%%
%%%%%%%%%%%%以下是范畴论记号%%%%%%%%%%%%%%%%%%%%
\DeclareMathOperator{\Commu}{\mathsf{Commu}}
\DeclareMathOperator{\alg}{\mathsf{alg}}
\DeclareMathOperator{\Mod}{\mathsf{Mod}}
\DeclareMathOperator{\Setcat}{\mathsf{Set}}
\DeclareMathOperator{\Ringcat}{\mathsf{Ring}}
\DeclareMathOperator{\Abcat}{\mathsf{Ab}}
\DeclareMathOperator{\modcat}{\!\text{-}\mathsf{Mod}}
\DeclareMathOperator{\Shcat}{\mathsf{Sh}}
\DeclareMathOperator{\pShcat}{\mathsf{pSh}}

\newcommand*{\Modcat}[2]
    {\Mod^{#1}_{#2}}
\newcommand*{\Assalgcat}[2]
    {\mathsf{Ass}\!\text{-}\!\alg^{#1}_{#2}}
\newcommand*{\Commualgcat}[2]
    {\Commu\!\text{-}\!\alg^{#1}_{#2}} 